%Options > Configure Texmaker > Editor > Spelling Dictionary, per corrector en català

\documentclass[11pt, a4paper]{report}
\usepackage[a4paper,left=30mm,right=20mm,top=25mm,bottom=25mm]{geometry}
\sloppy %per forçar el canvi de línia si la paraula supera el marge dret
\usepackage[utf8]{inputenc}

% Per utilitzar la font Helvetica (Arial)
\renewcommand{\familydefault}{\sfdefault}
\usepackage[scaled=1]{helvet}
\usepackage[helvet]{sfmath}
\everymath={\sf}
%Equacions amb una font sans_serif, \mathrm{equació aquí, són les letres les que queden inclinades}
%\usepackage{arev} % sans-serif math font
%\usepackage{helvet} % sans-serif text font


% Per comptar imatges enlloc de mostrar 1.1, 1.2...
\usepackage{chngcntr}
\counterwithout{figure}{chapter}
\counterwithout{table}{chapter}
\counterwithout{equation}{chapter}

\usepackage{graphicx}
\graphicspath{{images/}} %directori amb les imatges que volem insertar
\usepackage{float} %per forçar imatges amb H
\usepackage[normalem]{ulem} %negreta múltiples línies
%\usepackage{soul}

\usepackage{caption}
\captionsetup[figure]{labelfont={},name={Figura},labelsep=period}
\captionsetup[table]{labelfont={},name={Taula},labelsep=period}


\usepackage{subcaption}
\usepackage{amsmath} %per fòrmules matemàtiques
\usepackage[table]{xcolor} %per colors a les taules
%\usepackage{circuitikz} %per circuits electrònics
\usepackage{siunitx} %per les labels dels components
\usepackage[american,cuteinductors,smartlabels]{circuitikz} %american/european
\usepackage{tikz} %quadrícula
\usepackage[a4paper, left=30mm, right=20mm, top=25mm, bottom=25mm]{geometry} %geometria de la pàgina, 25 però per ajustar bé
\setlength{\headsep}{20pt}
\setlength{\footskip}{25pt}
%\usepackage[a4paper, width=150mm, top=25mm, bottom=25mm]{geometry} %geometria de la pàgina
\usepackage{lipsum} %per generar dummy text
\usepackage{xpatch} %per la distància entre títol i top

%Capçaleres i peus de pàgina
\usepackage{fancyhdr}
%\pagestyle{fancy} %fancy, plain
\fancypagestyle{plain}{
  \fancyhf{}% Clear header/footer
  \fancyhead[L]{\footnotesize{Plaques solars fotovoltaiques sensoritzades per habitatge unifamiliar}}
  \fancyhead[R]{\footnotesize{Estat d'amidaments}}
  \fancyfoot[R]{\footnotesize{\thepage}}
}
\pagestyle{plain}% Set page style to plain.

%\fancyhead{}
%\fancyhead[LO,LE]{PROJECTES}
%\fancyfoot{}
%\fancyfoot[LE,RO]{\thepage} %número de la pàgina, a la dreta
%\fancyfoot[LO, CE]{Capítol \thechapter} %nom del capítol, a l'esquerra
%\fancyfoot[CO, CE]{\href{https://github.com/LFanals}{Llorenç Fanals Batllori}} %nom de l'autor, al centre
% \renewcommand{\headrulewidth}{0.4pt}
%\renewcommand{\footrulewidth}{0.4pt}

%Per tenir el nombre de pàgina a l'inici d'un capítol
%\fancypagestyle{plain}{
%\fancyhf{}
%\renewcommand\headrulewidth{0pt}
%\fancyfoot[R]{\thepage}
%}

%Per configurar el color dels links i referències
\usepackage{color}
\usepackage{hyperref}
\hypersetup{
    colorlinks=true, %true si es volen links de colors
    linkcolor=black,  %colors de les referències internes, blue
    filecolor=magenta,      %magenta
    urlcolor=[rgb]{0,0,0}, %Color dels links d'Internet, sobre 255=2^8-1=2^0+...+2^7, {0,0.5,1}
}

%Bibliografia
\usepackage[backend=bibtex]{biblatex}
\addbibresource{bibliography.bib}

%Canviem el nom que hi ha per defecte als índex i altres, per passar-ho al català
\renewcommand{\contentsname}{Índex}
\renewcommand{\listfigurename}{Índex de figures}
\renewcommand{\chaptername}{Capítol}
\renewcommand{\appendixname}{Annex}
\renewcommand{\listtablename}{Índex de taules}
% \renewcommand{\figurename}{Figura} % ho tinc amb caption
% \captionsetup[table]{name=Taula} % ho tinc amb caption

\definecolor{color_quadricula}{HTML}{0066ff} %color per la quadrícula

% Pels circuits
%\usepackage[american]{circuitikz}
\usetikzlibrary{calc}
\ctikzset{bipoles/thickness=1}
\ctikzset{bipoles/length=1.2cm}
\ctikzset{bipoles/diode/height=.375}
\ctikzset{bipoles/diode/width=.3}
\ctikzset{tripoles/thyristor/height=.8}
\ctikzset{tripoles/thyristor/width=1}
\ctikzset{bipoles/vsourceam/height/.initial=.7}
\ctikzset{bipoles/vsourceam/width/.initial=.7}
\tikzstyle{every node}=[font=\small]
\tikzstyle{every path}=[line width=0.8pt,line cap=round,line join=round]

%Per insertar codi
\usepackage{listings}
\usepackage{color}
\definecolor{dkgreen}{rgb}{0,0.6,0}
\definecolor{gray}{rgb}{0.5,0.5,0.5}
\definecolor{mauve}{rgb}{0.58,0,0.82}

\lstset{frame=none, %tb, none
  language=Python,
  aboveskip=2mm,
  belowskip=3mm,
  showstringspaces=false,
  columns=flexible,
  basicstyle={\scriptsize\ttfamily}, %small
  numbers=none, %left
  numberstyle=\tiny\color{gray},
  keywordstyle=\color{blue},
  commentstyle=\color{dkgreen},
  stringstyle=\color{mauve},
  breaklines=true,
  breakatwhitespace=true,
  tabsize=3
}


%Per tenir el format de capítol correcte
\usepackage{titlesec}

\usepackage{etoolbox}
%\usepackage{hyperref}

%Per chapter
\titlespacing*{\chapter}{0pt}{-25pt}{11pt} %Espaiat del títol de capítol amb els altres elements
\titleformat{\chapter}[hang] %Per seguir escrivint darrera el número
{\normalfont\fontsize{11}{15}\bfseries}{\thechapter.}{0.4em}{\MakeUppercase} %\fontsize{Tamany}{Espai múltiples línies}

%Per secció
\titlespacing*{\section}{0pt}{11pt}{11pt} %Espaiat del títol de capítol amb els altres elements
\titleformat{\section}[hang] %Per seguir escrivint darrera el número
{\normalfont\fontsize{11}{15}}{\thesection.}{0.4em}{\bfseries} %\fontsize{Tamany}{Espai múltiples línies}

%Per subsecció
\titlespacing*{\subsection}{0pt}{11pt}{11pt} %Espaiat del títol de capítol amb els altres elements
\titleformat{\subsection}[hang] %Per seguir escrivint darrera el número
{\normalfont\fontsize{11}{15}}{\thesubsection.}{0.4em}{} %\fontsize{Tamany}{Espai múltiples línies}

%Per paràgraf
\titlespacing*{\paragraph}{0pt}{0pt}{22pt} %Espaiat del títol de capítol amb els altres elements
\titleformat{\paragraph}[hang] %Per seguir escrivint darrera el número
{\normalfont\fontsize{11}{15}}{}{}{} %\fontsize{Tamany}{Espai múltiples línies}


%Interlineat, 1.2*1.25=1.5
\linespread{1.25}

%Espaiat entre paràgrafs
%\setlength{\parskip}{22pt}
 

\makeatletter
\def\tagform@#1{\maketag@@@{(\ignorespaces{Eq.~#1}\unskip)}}
\makeatother



%Per no tenir negreta a l'index
\usepackage{etoolbox}% http://ctan.org/pkg/etoolbox
\makeatletter
\patchcmd{\l@chapter}{\bfseries}{}{}{}% \patchcmd{<cmd>}{<search>}{<replace>}{<success>}{<failure>}
\makeatother

%Per tenir punts a l'índex
\makeatletter
\renewcommand*\l@chapter{\@dottedtocline{0}{0em}{1.5em}}
\makeatother

%Per taula que adapta bé els espais
\usepackage{tabularx}
\usepackage{tabu} % http://mirrors.ibiblio.org/CTAN/macros/latex/contrib/tabu/tabu.pdf
\tabulinesep = 1mm
\usepackage[font=footnotesize]{caption} %Captions de les figures més petites

%Appendix
\usepackage[]{appendix} %toc, page

%Alinear al separador decimal amb espais
\usepackage{setspace}
\renewcommand*{\arraystretch}{1.25}



%-------------------------------------------------------------------------------------------------------------
%-------------------------------------------------------------------------------------------------------------
%-------------------------------------------------------------------------------------------------------------
%-------------------------------------------------------------------------------------------------------------

\begin{document}
\pagenumbering{Roman}


%\begin{titlepage}
	\begin{center}
		\vspace*{1cm}
		
		\Huge
		\textbf{Document per Projectes}
		
		\vspace{0.5cm}
		\LARGE
		Adaptat a \LaTeX
	
		\vspace{1.5cm}
		
		\textbf{Llorenç Fanals Batllori}
		
		\vfill
		
		\small
		%\uppercase{Un treball lliurat a la Universitat - en compliment dels requisits pel grau en -}\\
		% TFG
		
		\vspace{1cm}
		
		%\includegraphics[scale=width=0.4\textwidth]{images/a_graph}
	\end{center}
	
	\begin{flushright}
	\large	
	Departament o grup de recerca\\
	UdG\\
	%País\\
	28/08/2019
	\end{flushright}
	


\end{titlepage}

%\thispagestyle{plain}

\begin{center}
	\large
	\textbf{Informe}
	
	\vspace{0.4cm}
	\large
	Descripció
	
	\vspace{0.4cm}
	\textbf{Llorenç Fanals Batllori}
	
	\vspace{0.9cm}
	\textbf{Abstract}
\end{center}
\lipsum[1]




%\chapter*{Dedicacions}
%Dedico aquest treball a -

%\chapter*{Agraïments}
%Vull agraïr a \\

\cleardoublepage\pagenumbering{arabic}

\begin{spacing}{2}
\tableofcontents
\end{spacing}
%\listoffigures %No fa falta crec
%\listoftables %No fa falta crec
\begin{spacing}{1.5}

\chapter{\uppercase{Instal·lació dels mòduls fotovoltaics}}

% Table generated by Excel2LaTeX from sheet 'Hoja1'
% Table generated by Excel2LaTeX from sheet 'Hoja1'
\begin{table}[H]
  \begin{center}
    \begin{tabularx} {\textwidth} {|X|r|} \hline
  \multicolumn{1}{|c|}{Descripció} &  \multicolumn{1}{c|}{Quantitat}\\ \hline \hline
    Perfil alumini 40x40 mm tipus B 1 m & 38 \\ \hline
    Grapa alumini 6 cm & 40 \\ \hline
    Escaire alumini en L & 40 \\ \hline
    Cargol martell M6 16 mm & 200 \\ \hline
    Cargol autoroscant M6 16 mm & 55 \\ \hline
    Volandera M6 & 33 \\ \hline
    Femella hexagonal M6 10 mm & 33 \\ \hline
    Tacs Fischer 072095 nylon 6x50 mm & 55 \\ \hline
        Hores enginyer & 1 \\ \hline
    Hores oficial de primera & 12 \\ \hline
    Hores oficial de segona & 12 \\ \hline
    \end{tabularx}%  
  \end{center}

  \label{tab:addlabel}%
\end{table}%



\clearpage

\chapter{\uppercase{Instal·lació elèctrica fotovoltaica}}



% Table generated by Excel2LaTeX from sheet 'Hoja1'
\begin{table}[H]
  \centering
    \begin{tabularx} {\textwidth} {|X|r|} \hline
  \multicolumn{1}{|c|}{Descripció} &  \multicolumn{1}{c|}{Quantitat}\\ \hline \hline
    Placa GLC 330 W & 10 \\ \hline
    Inversor FRONIUS Primo 3.0-1 Light 3 kW & 1 \\ \hline
    Metres cable Ethernet RJ-45 CAT 8 & 10 \\ \hline
    Metres cable 1,5 m$m^2$ recobriment contra el Sol & 100 \\ \hline
    Metres cable 4 m$m^2$ PVC recobriment contra el Sol & 30 \\ \hline
    Metres cable 4 m$m^2$ PVC & 15 \\ \hline
    Metres cable 6 m$m^2$ PVC recobriment contra el Sol & 60 \\ \hline
    Metres cable 10 m$m^2$ PVC recobriment contra el Sol & 38 \\ \hline
    Punteres Enghofer E 1.5-10 1,5 m$m^2$ 10 mm & 12 \\ \hline
    Punteres Enghofer E 4-10, 4 m$m^2$, 10 mm & 20 \\ \hline
    Punteres Enghofer E 6-10, 6 m$m^2$, 10 mm & 20 \\ \hline
    Punteres Enghofer E 10-10 10 m$m^2$ 10 mm & 12 \\ \hline
    Cinta aïllant 10 m 1,6 cm & 3 \\ \hline
    Caixa estanca Solera CONS 100x100x55 mm & 2 \\ \hline
    Canal Euroquint 25 16 mm 1,5 metres & 20 \\ \hline
    Corba canal VECAMCO & 10 \\ \hline
    Paquet de 50 brides 200x2,6  mm & 2 \\ \hline
    Regleta nylon 12 pols 16 mm & 4 \\ \hline
    Premsaestopes M12 & 10 \\ \hline
    Cargol autoroscant M6 16 mm & 12 \\ \hline
    Tacs Fischer 072095 nylon 6x50 mm & 50 \\ \hline
    Díode SM74611KTTR & 60 \\ \hline
    Hores enginyer & 1 \\ \hline
    Hores oficial de primera & 12 \\ \hline
    Hores oficial de segona & 12 \\ \hline
    \end{tabularx}%
  \label{tab:addlabel}%
\end{table}%


%$m^2$
\chapter{\uppercase{Quadre elèctric de proteccions}}



% Table generated by Excel2LaTeX from sheet 'Hoja1'
\begin{table}[H]
  \centering
    \begin{tabularx} {\textwidth} {|X|r|} \hline
  \multicolumn{1}{|c|}{Descripció} &  \multicolumn{1}{c|}{Quantitat}\\ \hline \hline
    Caixa quadre elèctric VE106F & 1 \\ \hline
    Protecció contra sobretensions Hager MZ240V II & 4 \\ \hline
    Diferencial O. Electric 30 mA 25 A classe A & 1 \\ \hline
    Magnetotèrmic Schneider 16 A II 6 kA & 3 \\ \hline
    Punteres Enghofer E 4-10 4 m$m^2$ 10 mm & 25 \\ \hline
    Metres guia DIN simètrica & 1 \\ \hline
    Pinta bipolar Schneider 100 A 8 mòduls & 1 \\ \hline
    Metres cable 4 m$m^2$ PVC & 1 \\ \hline
    Cargol autoroscant M6 16 mm & 8 \\ \hline
    Tacs Fischer 072095 nylon 6x50 mm & 8 \\ \hline
    Hores enginyer & 1 \\ \hline
    Hores oficial de primera & 5 \\ \hline

    \end{tabularx}%
  \label{tab:addlabel}%
\end{table}%

%\chapter{\uppercase{Posada en funcionament dels generadors fotovoltaics}}

\begin{table}[H]
  \begin{center}
    \begin{tabularx} {\textwidth} {|X|r|} \hline
  \multicolumn{1}{|c|}{Descripció} &  \multicolumn{1}{c|}{Quantitat}\\ \hline \hline
    Hores enginyer & 5 \\ \hline
    \end{tabularx}%  
  \end{center}

  \label{tab:addlabel}%
\end{table}%

\clearpage
\chapter{\uppercase{Circuit imprès}}

\begin{table}[H]
  \begin{center}
    \begin{tabularx} {\textwidth} {|X|r|} \hline
  \multicolumn{1}{|c|}{Descripció} &  \multicolumn{1}{c|}{Quantitat}\\ \hline \hline
    Centilitres aigua oxigenada 80 vol. & 10 \\ \hline
    Centilitres sosa càustica 2\% & 20 \\ \hline
    Centilitres salfumant & 10 \\ \hline
    Centilitres aigua & 3 \\ \hline
    Centilitres alcohol 96\% & 1 \\ \hline
    Placa PCB fotosensible positiva fibra doble cara 80x120 mm  & 1 \\ \hline
            Hores enginyer & 1 \\ \hline
    Hores oficial de segona & 1 \\ \hline
    \end{tabularx}%  
  \end{center}

  \label{tab:addlabel}%
\end{table}%

%\si{\ohm}
%\chapter{\uppercase{Circuit imprès}}

\begin{table}[H]
  \begin{center}
    \begin{tabularx} {\textwidth} {|X|r|} \hline
  \multicolumn{1}{|c|}{Descripció} &  \multicolumn{1}{c|}{Quantitat}\\ \hline \hline
    Litres aigua oxigenada 80 vol. & 0,10 \\ \hline
    Litres sosa càustica 2\% & 0,20 \\ \hline
    Litres salfumant & 0,10 \\ \hline
    Litres aigua & 3,00 \\ \hline
    Litres alcohol 96\% & 0,01 \\ \hline
    Hores oficial de segona & 1,00 \\ \hline
    \end{tabularx}%  
  \end{center}

  \label{tab:addlabel}%
\end{table}%
%\chapter{\uppercase{Caixa}}

\begin{table}[H]
  \begin{center}
    \begin{tabularx} {\textwidth} {|X|r|} \hline
  \multicolumn{1}{|c|}{Descripció} &  \multicolumn{1}{c|}{Quantitat}\\ \hline \hline
    Metres filament PLA 0,75mm & 30 \\ \hline
            Hores enginyer & 1 \\ \hline
    Hores oficial de segona & 2 \\ \hline
    \end{tabularx}%  
  \end{center}

  \label{tab:addlabel}%
\end{table}%
\chapter{\uppercase{Muntatge}} %  de l'electrònica i configuració

\begin{table}[H]
  \begin{center}
    \begin{tabularx} {\textwidth} {|X|r|} \hline
  \multicolumn{1}{|c|}{Descripció} &  \multicolumn{1}{c|}{Quantitat}\\ \hline \hline
    Grams estany 0,4 mm & 50 \\ \hline
    Grams decapant & 5 \\ \hline
    Torreta mascle femella M4 5 mm + 6 mm & 4 \\ \hline
    Mòdul de comunicació ESP-12E & 1 \\ \hline
    Oscil·lador SMD & 1 \\ \hline
    Amplificador operacional LM324 & 6 \\ \hline
    Multiplexor 16 entrades CD74HC4067M & 1 \\ \hline
    Conversor USB-TTL CH340G & 1 \\ \hline
    Polsadors 1206 SMD & 2 \\ \hline
    Resistència 100 k\si{\ohm} 1206 SMD 1/2 W & 13 \\ \hline
    Resistència 12 k\si{\ohm} 1206 SMD 1/2 W & 6 \\ \hline
    Resistència 8,2 k\si{\ohm} 1206 SMD 1/2 W & 20 \\ \hline
    Resistència 1,2 k\si{\ohm} 1206 SMD 1/2 W & 20 \\ \hline
    Resistència 1 k\si{\ohm} 1206 SMD 1/2 W & 12 \\ \hline
    Resistència 470 \si{\ohm} 1206 SMD 1/2 W & 4 \\ \hline
    Resistència 0 \si{\ohm} (NC) 1206 SMD 1/2 W & 1 \\ \hline
    Transistor NPN S8050 & 2 \\ \hline
    Díode LED 1206 SMD & 1 \\ \hline
    Díode 1N4007 1A & 1 \\ \hline
    Condensador 22 pF 10 V 1206 SMD & 2 \\ \hline
    Condensador 100 nF 10 V 1206 SMD & 10 \\ \hline
    Condensador 100 uF 10 V 1206 SMD & 1 \\ \hline
    Condensador 10 uF 10 V 1206 SMD & 2 \\ \hline
    Condensador 470 pF 10 V 1206 SMD & 3 \\ \hline
    Condensador 1 pF 10 V 1206 SMD & 1 \\ \hline
    Condensador 100 uF 10 V electrolític & 1 \\ \hline
    Connector regleta femella 2,54 mm 1x06 & 2 \\ \hline
    Connector mascle 2,54 mm 1x03 & 1 \\ \hline
    Connector USB Micro & 1 \\ \hline
    Metres filament PLA 0,75mm & 30 \\ \hline
    Hores enginyer & 2 \\ \hline
    Hores oficial de segona & 10 \\ \hline
    \end{tabularx}%  
  \end{center}

  \label{tab:addlabel}%
\end{table}%

%\si{\ohm}
\chapter{\uppercase{Comprovació}}

\begin{table}[H]
  \begin{center}
    \begin{tabularx} {\textwidth} {|X|r|} \hline
  \multicolumn{1}{|c|}{Descripció} &  \multicolumn{1}{c|}{Quantitat}\\ \hline \hline
    Hores enginyer & 5 \\ \hline
    \end{tabularx}%  
  \end{center}

  \label{tab:addlabel}%
\end{table}%





\clearpage
\chapter{\uppercase{Programació}}

\begin{table}[H]
  \begin{center}
    \begin{tabularx} {\textwidth} {|X|r|} \hline
  \multicolumn{1}{|c|}{Descripció} &  \multicolumn{1}{c|}{Quantitat}\\ \hline \hline
    Hores enginyer & 3 \\ \hline
    \end{tabularx}%  
  \end{center}

  \label{tab:addlabel}%
\end{table}%





\clearpage
\chapter{\uppercase{Posada en funcionament}}

\begin{table}[H]
  \begin{center}
    \begin{tabularx} {\textwidth} {|X|r|} \hline
  \multicolumn{1}{|c|}{Descripció} &  \multicolumn{1}{c|}{Quantitat}\\ \hline \hline
    Hores enginyer & 10 \\ \hline
    \end{tabularx}%  
  \end{center}

  \label{tab:addlabel}%
\end{table}%


\vspace*{\fill}
\noindent Llorenç Fanals Batllori\\
Graduat en Enginyeria Electrònica Industrial i Automàtica\\
\\
\\
Girona, 25 de novembre de 2019.



\clearpage

%\chapter{\uppercase{Conclusió}}
El present document té l'objectiu de legalitzar la instal·lació contra incendis, els materials i les evacuacions d'emergència d'un obrador de plats cuinats.\\
\newline
Per desenvolupar aquesta memòria s'ha seguit el Reial decret 2267/2004 del 3 de desembre, el qual aprova el Reglament de seguretat contra incendis en establiments industrials. Amb aquest reglament es pot dir que la càrrega de foc és baixa i els materials són correctes.\\
\newline Per validar la instal·lació contra incendis s'ha seguit el Reglament d'instal·lacions de protecció contra incendis aprovat pel Reial decret 513/2017 del 22 de maig. Gràcies a aquest document s'ha pogut avaluar la instal·lació contra incendis com a correcta.\\
\newline El REBT s'ha consultat per verificar que els nivells d'il·luminació d'emergència mesurats són correctes. La Norma Básica de l'Edificació, NBE-CPI/96, ha permès considerar correctes les dimensions de les portes d'emergència i els passadissos.\\
%
\newline Amb tot l'indicat en aquesta memòria es considera que la nau té un risc baix d'incendi i que la instal·lació contra incendis, els materials i els recorreguts d'evacuació són els adequats segons la legislació present fins a dia d'avui. 

\vspace*{\fill}
\noindent Llorenç Fanals Batllori\\
Graduat en Enginyeria Electrònica Industrial i Automàtica\\
\\
\\
Girona, 26 d'octubre de 2019.

\clearpage

\begin{appendices}
%\chapter{Títol de l'annex}

%\chapter{\uppercase{Càlculs}}
Pel càlcul de les seccions dels conductors cal tenir en compte els factors de simultaneïtat d'alguns elements i els factors que marca el REBT: 1,25 pel motor elèctric de més potència de la línia, tal com es detalla a la ITC-47; i 1,8 per les lluminàries amb descàrrega, tal com s'indica a la ITC-44. A l'obrador hi ha molts motors elèctric però cap llum amb descàrrega.\\
\newline
En algunes línies es considera que el factor de potència és unitari. A la realitat mai valdrà exactament 1, però sí que es preveu que tingui un valor molt semblant. Les màquines que s'han escollit tenen un factor de potència proper a l'unitari però diferent de 1.\\
\newline Per calcular la intensitat de les línies monofàsiques es fa servir la següent fórmula:
\begin{equation}
I_{linia} = \frac{P}{V*\cos(\phi)}
\end{equation}
V = 230 V\\
P és la potència que consumeixen els elements connectats a la línia\\
$\phi$ és el factor de potència\\
\newline En trifàsic, l'equació que s'utilitza és:
\begin{equation}
I_{linia} = \frac{P}{\sqrt3*V_{linia}*\cos(\phi)}
\end{equation}
$V_{linia}$ = 400 V\\
\newline És important calcular la caiguda de tensió a les línies per tal de veure si estan dimensionades correctament. La caiguda de tensió en línies d'enllumenat no pot ser superior al 3\% i en línies de força no pot ser superior al 5\% de la tensió de subministrament. La caiguda de tensió màxima a la derivació individual és de 1,5\%.\\
\newline En monofàsic:
\begin{equation}
e(\%)=\frac{P}{V}\frac{2*l}{k*S}
\end{equation}
l és la longitud ja sigui de la fase o el neutre des del comptador a l'element més llunyà\\
$k = 56 \frac{m}{mm^{2}\si{\ohm}}$\\
S és la secció del cable en m$m^2$\\
\newline
En trifàsic, l'equació que s'utilitza és:
\begin{equation}
e(\%)=\frac{P}{V}\frac{l}{k*S}
\end{equation}
\\
El dimensionament de les línies ha de permetre que les caigudes de tensió no superin els màxims indicats prèviament. Alhora, els cables han de poder admetre les intensitats calculades, per això ens guiem amb la taula de la ITC-19 del REBT. Finalment, cal comprovar que  l'interruptor magnetotèrmic té una intensitat nominal superior a la calculada per la línia i menor a l'admissible que marca la ITC-19.\\
\newline La instal·lació és trifàsica, per tant, hi ha 3 conductors de fase i un conductor de neutre. El conductor de terra transcorre per totes les línies i té una secció igual als conductors de les línies, tal com s'indica al plànol. El neutre, que arriba per l'escomesa, també és de la mateixa secció que els conductors de fase. Les màquines trifàsiques necessiten el neutre pels seus equips electrònics.\\
\newline
Per comprovar que el valor de secció de la derivació individual és correcte quan la línia va amb una terna de cables unipolars per tub cal tenir en compte un factor d'intensitat de 0,8.
\begin{equation}
I_{DI} < 0.8 * I_{max. admissible}
\end{equation}

\noindent A continuació es mostren les diferents línies de forma detallada. La secció s'ha comprovat tenint en compte les fórmules explicades i les seccions mínimes per intensitat segons marca el REBT. S'han verificat les línies pel cas més desfavorable. Les seccions dels tubs compleixen amb la ITC-21.\\
\newline Les tensions nominals són 230 V per les línies monofàsiques i 400 V per les trifàsiques. Tots els cables de les línies són de coure de 450/750 V d'aïllament. La derivació individual és de coure amb 0,6/1 kV d'aïllament. L'aïllament de la instal·lació és de 1.000 k$\si{\ohm}$.

\begin{table}[H]
\scriptsize
\begin{center}
 \begin{tabu} to \textwidth {|X[0.5, l]|X[2, l]|X[r]|X[0.6, r]|X[r]|X[r]|X[r]|X[r]|X[r]|X[r]|X[0.5,r]|}%{X | c c c} 
 \hline
 Línia& Descripció & Potència (W) & cos($\phi$) & Intensitat (A) & Distància màxima (m) & Seccions fase, neutre, terra ($mm^{2}$) & Diàmetre tub (mm) & Caiguda de tensió (\%) & Caiguda de tensió acum. (\%)\\
 \hline \hline 
DI & Derivació individual& 87.000 \ \ \ \ & 0,96 & 131,32 & 8 &3x35 + 35 + 16& 160 & 0,23 & 0,23 \\ \hline
L1 & Enllumenat habitacions i cambra& 1.370,5 & 1 & 5,96 & 55 &2,5 + 2,5 + 2,5& 20 & 2,04 & 2,27 \\ \hline
L2 & Enllumenat cuina & 1.476 \ \ \ \  & 1 & 6,42 & 47 &2,5 + 2,5 + 2,5& 20 & 1,87 & 2,10 \\ \hline 
L3 & Força oficina, menjador, màquines de buit & 7.775 \ \ \ \  & 1 & 33,80 & 51 &10 + 10 + 10& 25 & 2,68 & 2,91 \\ \hline 
L4 & Força cuina & 7.235 \ \ \ \  & 1 & 31,46 & 33 & 6 + 6 + 6& 25 & 2,67 & 2,90 \\ \hline
L5 & Rentaplats cuina & 36.000 \ \ \ \ & 0,95 & 54,92 & 54 &3x16 + 16 + 16& 32 & 1,44 & 1,67 \\ \hline 
L6 & Extractors i cambres de fred & 19.000 \ \ \ \ & 0,9 & 30,60 & 36 &3x10 + 10 + 10& 32 & 0,86 & 1,09 \\ \hline
L7 & Abatidor sala de preparació & 16.975 \ \ \ \ & 0,95 & 25,90 & 44 &3x10 + 10 + 10& 32 & 0,84 & 1,07 \\ \hline

 \end{tabu}
 \caption{Línies detallades}
\end{center}
\end{table}



%\chapter{\uppercase{Característiques}}
L'aïllament dels cables elèctrics de les línies és EPR de 450/750 V d'aïllament. Els cables transcorren en safata perforada pel passadís i dins de tubs corrugats en muntatge superficial (B2) a la resta de zones. El diàmetre d'aquests tubs s'indica en l'anterior annex. Els càlculs s'han efectuat considerant que tota la llargada dels cables va amb el muntatge B2, que és més restrictiu que la safata.\\
\newline Es fan servir els colors gris, marró i negre per les fases, el blau pel neutre i el conductor groc i verd pel terra.\\
\newline Hi ha instal·lades caixes de derivació al llarg de la instal·lació i l'enllumenat dels vestidors, l'oficina i el menjador es controla amb interruptors de paret. Els cables dels l'enllumenats que no estan en contacte amb la paret es passen pel fals sostre.\\
\newline Les màquines trifàsiques es connecten a la xarxa mitjançant una base CETAC.\\
\newline Els extractors de la cuina de l'obrador van controlats amb variadors de freqüència. La seva línia va amb un diferencial de 100 mA de classe B degut als alts corrents de fuga que poden donar-se. Aigües amunt de tots els agrupaments hi ha instal·lat un diferencial de 300 mA de sensibilitat per protegir tota la instal·lació i alhora tenir selectivitat amb el diferencials que té aigües avall.\\
\newline El maxímetre del conjunt de protecció i mesura garanteix el subministrament elèctric tot i sobrepassar la potència contractada. Si en un moment puntual es connectés alguna màquina més i pel marge donat no saltés cap interruptor magnetotèrmic però s'estigués superant la potència contractada, hi seguiria havent subministrament elèctric i l'empresa subministradora aplicaria un recàrrec a la factura.\\
\newline 
S'agrupen les línies tenint en compte si el subministrament és trifàsic o monofàsic. S'intenta, en la mesura del possible, que tots els grups tinguin potències similars. És per això que el diferencial que agrupa les 4 línies monofàsiques és de 4 pols: els 3 conductors de fase i el neutre passaran per aquest diferencial i s'alimentaran les diferents línies monofàsiques amb diferents fases. Així es pot aconseguir una instal·lació trifàsica bastant ben equilibrada.\\
\newline Els llums d'emergència són de tipus no permanent i es considera que tenen una potència de 3 W. Al disposar de bateria i només encendre's quan hi ha una emergència, no s'han tingut en compte per la previsió de càrregues.\\
\newline Per millorar el factor de potència de la derivació individual, o sigui, de tota la instal·lació, hi ha instal·lada una bateria de condensadors de 20 kVAr la qual dona una factor de potència de 0,998.




\end{appendices}


\end{spacing}
%\cite{einstein} % per fer una cita
%\printbibliography[title=Bibliografia] %ARA BÉ

\end{document}