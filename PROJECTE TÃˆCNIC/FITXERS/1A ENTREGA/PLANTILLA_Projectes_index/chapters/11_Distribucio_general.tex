\chapter{\uppercase{Distribució general}}
%Càlculs
\section{Coeficients de simultaneïtat i arrencada}
Pel càlcul de les seccions dels conductors cal tenir en compte els factors de simultaneïtat d'alguns elements i els factors que marca el REBT: 1,25 pel motor elèctric de més potència de la línia, tal com es detalla a la ITC-47; i 1,8 per les lluminàries amb descàrrega, tal com s'indica a la ITC-44.\\
\newline Màquines com els extractors, les màquines buit, les màquines de fred, les neveres i congeladors disposen de motors elèctrics. El rentaplats també té motor elèctric, però la majoria de potència que consumeix ho fa per escalfar aigua, així que no s'aplica cap factor.\\
\newline En la nostra instal·lació no hi ha cap enllumenat amb descàrrega, totes les lluminàries són de tipus LED.
%
%
%
%
\section{Agrupament de línies}
S'agrupen les línies tenint en compte si el subministrament és trifàsic o monofàsic. S'intenta, en la mesura del possible, que tots els grups tinguin potències similars. És per això que el diferencial que agrupa les 4 línies monofàsiques és de 4 pols: els 3 conductors de fase passaran per aquest diferencial i s'alimentaran les diferents línies monofàsiques amb diferents fases. Així es pot aconseguir una instal·lació trifàsica bastant ben equilibrada.\\
\newline Es preveu utilitzar safata foradada en certs trams i muntatge superficial en altres. Pel REBT s'escull el muntatge B2, considerant que només tenim aquest muntatge, és el cas més desfavorable. L'aïllament serà EPR.
%
% \small
\subsection{Agrupament 1}
Comprèn les línies L1, L2, L3 i L4. Són monofàsiques però com s'ha comentat s'alimenten amb fases diferents. L1 i L2 són línies d'enllumenat i L3 i L4 són línies de força per màquines monofàsiques. El factor de potència d'aquestes línies es considera 1.
%
\begin{table}[H]
\small
\begin{center}
 \begin{tabu} to \textwidth {|X[0.5, l]|X[2, l]|X[r]|X[r]|X[r]|X[r]|X[r]|X[r]|X[0.5,r]|}%{X | c c c} 
 \hline
 Línia& Descripció & Potència (W) & Intensitat (A) & Distància màxima (m) & Secció ($mm^{2}$) & Caiguda de Tensió (\%) & REBT (A)& PIA (A)\\
 \hline \hline 
L1 & Enllumenat habitacions i cambra& 1.370,5 & 5,96 & 55 &2x2,5 + 2,5& 2,04 & 25 & 10 \\ \hline
L2 & Enllumenat cuina & 1.476 \ \ \ \  & 6,42 & 47 &2x2,5 + 2,5& 1,87 & 25 & 10 \\ \hline 
L3 & Força oficina, menjador, màquines de buit & 7.775 \ \ \ \  & 33,80 & 51 &2x10 + 10& 2,68 & 60 & 40 \\ \hline 
L4 & Força cuina & 7.235 \ \ \ \  & 31,46 & 33 &2x6 + 6& 2,67 & 44 & 40 \\ \hline 
 \hline
 Total & & 8.900,5 & 38,35 & & & & & 80 \\
 \hline
 \end{tabu}
 \caption{Agrupament 1}
\end{center}
\end{table}


\subsection{Agrupament 2}
Aquest agrupament és purament per la línia que alimenta els rentaplats de la cuina, la L5. Aquests rentaplats consumeixen molta potència perquè escalfen aigua; el seu motor consumeix poca potència, per això no se'ls aplica cap factor d'arrencada. El factor de potència és de 0,95. 
\begin{table}[H]
\small
\begin{center}
 \begin{tabu} to \textwidth {|X[0.5, l]|X[2, l]|X[r]|X[r]|X[r]|X[r]|X[r]|X[r]|X[0.5,r]|}%{X | c c c} 
 \hline
 Línia& Descripció & Potència (W) & Intensitat (A) & Distància màxima (m) & Secció ($mm^{2}$) & Caiguda de Tensió (\%) & REBT (A)& PIA (A)\\
 \hline \hline 
L5 & Rentaplats cuina & 36.000 & 54,92 & 54 &4x16 + 16& 1,44 & 70 & 63 \\
 \hline
 \end{tabu}
 \caption{Agrupament 2}
\end{center}
\end{table}

\subsection{Agrupament 3}
Aquest agrupament alimenta els dos extractors, que van amb motor elèctric, i els equips de les cambres de fred. Per tant, s'aplica el factor d'arrencada de 1,25 a un dels extractors, ja que aquests consumeixen més que els equips de fred. El factor de potència és de 0,9.
\begin{table}[H]
\small
\begin{center}
 \begin{tabu} to \textwidth {|X[0.5, l]|X[2, l]|X[r]|X[r]|X[r]|X[r]|X[r]|X[r]|X[0.5,r]|}%{X | c c c} 
 \hline
 Línia& Descripció & Potència (W) & Intensitat (A) & Distància màxima (m) & Secció ($mm^{2}$) & Caiguda de Tensió (\%) & REBT (A)& PIA (A)\\
 \hline \hline 
L6 & Extractors i cambres de fred & 19.000 & 30,60 & 36 &4x10 + 10& 0,86 & 52 & 40 \\
 \hline
 \end{tabu}
 \caption{Agrupament 3}
\end{center}
\end{table}

\subsection{Agrupament 4}
Aquest agrupament alimenta l'abatidor de la sala de preparació. S'aplica el factor d'arrencada de 1,25. El seu factor de potència és de 0,95.
\begin{table}[H]
\small
\begin{center}
 \begin{tabu} to \textwidth {|X[0.5, l]|X[2, l]|X[r]|X[r]|X[r]|X[r]|X[r]|X[r]|X[0.5,r]|}%{X | c c c} 
 \hline
 Línia& Descripció & Potència (W) & Intensitat (A) & Distància màxima (m) & Secció ($mm^{2}$) & Caiguda de Tensió (\%) & REBT (A)& PIA (A)\\
 \hline \hline 
L7 & Abatidor sala de preparació & 16.975 & 25,90 & 44 &4x10 + 10& 0,84 & 52 & 40 \\
 \hline
 \end{tabu}
 \caption{Agrupament 4}
\end{center}
\end{table}
%
%

\begin{table}[H]
\small
\begin{center}
 \begin{tabu} to \textwidth {|X[3, l]|X[2, l]|X[r]|X[r]|X[r]|X[r]|}%{X | c c c} 
 \hline
 Línia & Descripció & Intensitat (A) & Intensitat PIA (A) & Classe & Nombre de pols \\
 \hline \hline 
L1 & Enllumenat habitacions i cambres & 5,61 & 10 & C & 2 \\ \hline
L2 & Enllumenat cuina & 6,42 & 10 & C & 2 \\ \hline
L3 & Força oficina, menjador, màquines de buit & 33,80 & 40 & C & 2 \\ \hline
L4 & Força cuina & 31,45 & 40 & C & 2 \\ \hline
L5 & Rentaplats cuina & 54,92 & 64 & C & 4 \\ \hline
L6 & Extractors i cambres de fred & 30,60 & 40 & D & 4 \\ \hline
L7 & Abatidor sala de preparació & 25,9 & 40 & C & 4 \\ \hline \hline
L1+L2+L3+L4+L5+L6+L7 & IGA & 130,08 & 160 & C & 4 \\ \hline

 \end{tabu}
 \caption{Proteccions magnetotèrmiques de les línies}
\end{center}
\end{table}
\noindent L'últim magnetotèrmic de la taula és un IGA de 160 A d'intensitat nominal, classe C i 4 pols; situat aigües amunt de les línies. Les màquines i l'enllumenat poden arribar a estirar 130,08 A. El Vademècum indica que cal protegir la instal·lació per corrents superiors a aquest, els quals podrien danyar la instal·lació.


\begin{table}[H]
\small
\begin{center}
 \begin{tabu} to \textwidth {|X[3, l]|X[2, l]|X[r]|X[r]|X[1,r]|X[1,r]|X[r]|}%{X | c c c} 
 \hline
 Línia & Descripció & Intensitat (A) & Intensitat nominal (A) & Sens. (mA) & Classe & Nombre de pols \\
 \hline \hline 
L1+L2+L3+L4 & Enllumenat i força monofàsics & 77,3 & 80 & 30 & A & 4 \\ \hline
L5 & Rentaplats cuina & 54,92 & 63 & 30 & A & 4 \\ \hline
L6 & Extractors i cambres de fred & 30,60 & 40 & 100 & B & 4 \\ \hline
L7 & Abatidor de la sala de preparació & 25,9 & 40 & 30 & A & 4 \\ \hline \hline
L1+L2+L3+L4+L5+L6+L7 & Totes les línies & 130,08 & 160 & 300 & A & 4 \\ \hline
 \end{tabu}
 \caption{Proteccions diferencials de la instal·lació}
\end{center}
\end{table}
\noindent L'últim diferencial de la taula és general, està en sèrie amb l'IGA i, tot i que no obligatori, decidim instal·lar-lo seguint el Vademècum com a recomanació.




\clearpage
