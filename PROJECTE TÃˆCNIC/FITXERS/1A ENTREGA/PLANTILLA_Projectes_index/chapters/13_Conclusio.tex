\chapter{\uppercase{Conclusió}}
L'objectiu d'aquesta memòria és legalitzar la instal·lació elèctrica d'una nau industrial de plats cuinats.
Per desenvolupar aquesta memòria s'ha seguit el Reglament Electrotècnic de Baixa Tensió. S'han consultat les instruccions tècniques que afecten a la instal·lació i les normes UNE adients.\\
\newline La ITC-07 detalla com han de ser les xarxes subterrànies per distribució en baixa tensió. L'escomesa i la derivació individual són subterrànies, cal tenir en compte la profunditat a la què van. Amb aquesta instrucció s'ha calculat la secció dels cables de fase i el neutre de la derivació individual.\\
\newline La ITC-18 tracta sobre les instal·lacions de posada a terra. És molt important tenir una bona instal·lació de posada a terra per protegir contra contactes indirectes i garantir que els interruptors diferencials funcionen correctament.\\
\newline La ITC-30 parla de locals amb característiques especials. L'obrador és un local humit. És per això que la tensió màxima de defecte, tal com marca la ITC-18, és de 24 V. Com que es fa servir un interruptor diferencial de 1 A de sensibilitat, cal tenir una posada a terra de menys de 24 $\si\ohm$.\\
\newline La ITC-15 descriu les derivacions individuals. Ha estat d'ajuda per saber quines són les caigudes de tensió permeses en aquest tram de la instal·lació, en el nostre cas un 1,5\%.\\
\newline La ITC-11 descriu l'escomesa. En el nostre cas és subterrània i ve donada per l'empresa distribuïdora, no l'hem de calcular nosaltres.\\
\newline La ITC-13 parla de les caixes generals de protecció. Cita les normes UNE-EN 60439-1, UNE-EN 60439-3, UNE 20324 i UNE-EN 50102.\\
\newline La ITC-17 comenta com són les quadres elèctrics i quins components tenen. S'ha tingut en compte per preveure tots els elements de protecció i comandament que han d'anar al quadre elèctric de l'obrador.\\
\newline La ITC-47 cita quin és el coeficient d'arrencada per una línia amb diversos motors. S'ha tingut en compte pel correcte dimensionament de les línies elèctriques.\\
\newline La ITC-28 ha estat consultada per poder dir que l'obrador no és un local de pública concurrència, tot i tenir una sala de venta al públic.\\
\newline La ITC-22 comenta com protegir la instal·lació contra sobreintensitats. Ho tenim en compte pel quadre general de protecció i comandament.\\
\newline La ITC-23 descriu les sobretensions, per què es poden originar i a quins equips poden afectar. Instal·lem una protecció contra sobretensions a la nostra instal·lació.\\
\newline La ITC-24 l'hem consultat per protegir la nostra instal·lació contra contactes indirectes mitjançant interruptors diferencials.\\
\newline S'ha consultat el Vademècum d'Endesa com a guia per dimensionar correctament el conjunt de protecció i mesura. A més, ens ha servir d'ajuda per escollir correctament alguns elements del quadre general de protecció i comandament.\\
\newline Amb tot l'indicat en aquesta memòria es considera que la instal·lació és legal i està llesta per utilitzar.

\vspace*{\fill}
\noindent Llorenç Fanals Batllori\\
Graduat en Enginyeria Electrònica Industrial i Automàtica\\
\\
\\
\\
Girona, 9 d'octubre de 2019

\clearpage