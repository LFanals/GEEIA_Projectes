%Options > Configure Texmaker > Editor > Spelling Dictionary, per corrector en català

\documentclass[11pt, a4paper]{report}
\usepackage[a4paper,left=30mm,right=20mm,top=25mm,bottom=25mm]{geometry}
\sloppy %per forçar el canvi de línia si la paraula supera el marge dret
\usepackage[utf8]{inputenc}



% Per utilitzar la font Helvetica (Arial)
\renewcommand{\familydefault}{\sfdefault}
\usepackage[scaled=1]{helvet}
%\usepackage[helvet]{sfmath}
%\everymath={\sf}
%Equacions amb una font sans_serif, \mathrm{equació aquí, són les letres les que queden inclinades}
%\usepackage{arev} % sans-serif math font
%\usepackage{helvet} % sans-serif text font


% Per comptar imatges enlloc de mostrar 1.1, 1.2...
\usepackage{chngcntr}
\counterwithout{figure}{chapter}
\counterwithout{table}{chapter}
\counterwithout{equation}{chapter}

\usepackage{graphicx}
\graphicspath{{images/}} %directori amb les imatges que volem insertar
\usepackage{float} %per forçar imatges amb H
\usepackage[normalem]{ulem} %negreta múltiples línies
%\usepackage{soul}

\usepackage{caption}
\captionsetup[figure]{labelfont={},name={Figura},labelsep=period}
\captionsetup[table]{labelfont={},name={Taula},labelsep=period}


\usepackage{subcaption}
\usepackage{amsmath} %per fòrmules matemàtiques
\usepackage[table]{xcolor} %per colors a les taules
%\usepackage{circuitikz} %per circuits electrònics
\usepackage{siunitx} %per les labels dels components
\usepackage[american,cuteinductors,smartlabels]{circuitikz} %american/european
\usepackage{tikz} %quadrícula
\usepackage[a4paper, left=30mm, right=20mm, top=25mm, bottom=25mm]{geometry} %geometria de la pàgina, 25 però per ajustar bé
\setlength{\headsep}{20pt}
\setlength{\footskip}{25pt}
%\usepackage[a4paper, width=150mm, top=25mm, bottom=25mm]{geometry} %geometria de la pàgina
\usepackage{lipsum} %per generar dummy text
\usepackage{xpatch} %per la distància entre títol i top

%Capçaleres i peus de pàgina
\usepackage{fancyhdr}
%\pagestyle{fancy} %fancy, plain
\fancypagestyle{plain}{
  \fancyhf{}% Clear header/footer
  \fancyhead[L]{\footnotesize{Plaques solars fotovoltaiques sensoritzades per habitatge unifamiliar}}
  \fancyhead[R]{\footnotesize{Resum}}
  \fancyfoot[R]{\footnotesize{\thepage}}
}
\pagestyle{plain}% Set page style to plain.

%\fancyhead{}
%\fancyhead[LO,LE]{PROJECTES}
%\fancyfoot{}
%\fancyfoot[LE,RO]{\thepage} %número de la pàgina, a la dreta
%\fancyfoot[LO, CE]{Capítol \thechapter} %nom del capítol, a l'esquerra
%\fancyfoot[CO, CE]{\href{https://github.com/LFanals}{Llorenç Fanals Batllori}} %nom de l'autor, al centre
% \renewcommand{\headrulewidth}{0.4pt}
%\renewcommand{\footrulewidth}{0.4pt}

%Per tenir el nombre de pàgina a l'inici d'un capítol
%\fancypagestyle{plain}{
%\fancyhf{}
%\renewcommand\headrulewidth{0pt}
%\fancyfoot[R]{\thepage}
%}

%Per configurar el color dels links i referències
\usepackage{color}
\usepackage{hyperref}
\hypersetup{
    colorlinks=true, %true si es volen links de colors
    linkcolor=black,  %colors de les referències internes, blue
    filecolor=magenta,      %magenta
    urlcolor=[rgb]{0,0,0}, %Color dels links d'Internet, sobre 255=2^8-1=2^0+...+2^7, {0,0.5,1}
}

%Bibliografia
\usepackage[backend=bibtex]{biblatex}
\addbibresource{bibliography.bib}

%Canviem el nom que hi ha per defecte als índex i altres, per passar-ho al català
\renewcommand{\contentsname}{Índex}
\renewcommand{\listfigurename}{Índex de figures}
\renewcommand{\chaptername}{Capítol}
\renewcommand{\appendixname}{Annex}
\renewcommand{\listtablename}{Índex de taules}
% \renewcommand{\figurename}{Figura} % ho tinc amb caption
% \captionsetup[table]{name=Taula} % ho tinc amb caption

\definecolor{color_quadricula}{HTML}{0066ff} %color per la quadrícula

% Pels circuits
%\usepackage[american]{circuitikz}
\usetikzlibrary{calc}
\ctikzset{bipoles/thickness=1}
\ctikzset{bipoles/length=1.2cm}
\ctikzset{bipoles/diode/height=.375}
\ctikzset{bipoles/diode/width=.3}
\ctikzset{tripoles/thyristor/height=.8}
\ctikzset{tripoles/thyristor/width=1}
\ctikzset{bipoles/vsourceam/height/.initial=.7}
\ctikzset{bipoles/vsourceam/width/.initial=.7}
\tikzstyle{every node}=[font=\small]
\tikzstyle{every path}=[line width=0.8pt,line cap=round,line join=round]

%Per insertar codi
\usepackage{listings}
\usepackage{color}
\definecolor{dkgreen}{rgb}{0,0.6,0}
\definecolor{gray}{rgb}{0.5,0.5,0.5}
\definecolor{mauve}{rgb}{0.58,0,0.82}

\lstset{frame=none, %tb, none
  language=Python,
  aboveskip=5mm,
  belowskip=5mm,
  showstringspaces=false,
  columns=flexible,
  basicstyle={\normalsize\ttfamily}, %small
  numbers=none, %left
  numberstyle=\tiny\color{gray},
  keywordstyle=\color{blue},
  commentstyle=\color{dkgreen},
  stringstyle=\color{mauve},
  breaklines=true,
  breakatwhitespace=true,
  tabsize=3,
  literate={á}{{\'a}}1 {é}{{\'e}}1 {ó}{{\'o}}1 {í}{{\'i}}1 {ú}{{\'u}}1  {à}{\`{a}}1 {è}{\`{e}}1  {ò}{\`{o}}1  {ï}{\"\i}1  {ç}{\c{c}}1 ,  
}


%Per tenir el format de capítol correcte
\usepackage{titlesec}

\usepackage{etoolbox}
%\usepackage{hyperref}

%Per chapter
\titlespacing*{\chapter}{0pt}{-25pt}{11pt} %Espaiat del títol de capítol amb els altres elements
\titleformat{\chapter}[hang] %Per seguir escrivint darrera el número
{\normalfont\fontsize{11}{15}\bfseries}{\thechapter.}{0.4em}{\MakeUppercase} %\fontsize{Tamany}{Espai múltiples línies}

%Per secció
\titlespacing*{\section}{0pt}{11pt}{11pt} %Espaiat del títol de capítol amb els altres elements
\titleformat{\section}[hang] %Per seguir escrivint darrera el número
{\normalfont\fontsize{11}{15}}{\thesection.}{0.4em}{\bfseries} %\fontsize{Tamany}{Espai múltiples línies}

%Per subsecció
\titlespacing*{\subsection}{0pt}{11pt}{11pt} %Espaiat del títol de capítol amb els altres elements
\titleformat{\subsection}[hang] %Per seguir escrivint darrera el número
{\normalfont\fontsize{11}{15}}{\thesubsection.}{0.4em}{} %\fontsize{Tamany}{Espai múltiples línies}

%Per paràgraf
\titlespacing*{\paragraph}{0pt}{0pt}{22pt} %Espaiat del títol de capítol amb els altres elements
\titleformat{\paragraph}[hang] %Per seguir escrivint darrera el número
{\normalfont\fontsize{11}{15}}{}{}{} %\fontsize{Tamany}{Espai múltiples línies}


%Interlineat, 1.2*1.25=1.5
\linespread{1.25}

%Espaiat entre paràgrafs
%\setlength{\parskip}{22pt}
 

\makeatletter
\def\tagform@#1{\maketag@@@{(\ignorespaces{Eq.~#1}\unskip)}}
\makeatother



%Per no tenir negreta a l'index
\usepackage{etoolbox}% http://ctan.org/pkg/etoolbox
\makeatletter
\patchcmd{\l@chapter}{\bfseries}{}{}{}% \patchcmd{<cmd>}{<search>}{<replace>}{<success>}{<failure>}
\makeatother

%Per tenir punts a l'índex
\makeatletter
\renewcommand*\l@chapter{\@dottedtocline{0}{0em}{1.5em}}
\makeatother

%Per taula que adapta bé els espais
\usepackage{tabularx}
\usepackage{tabu} % http://mirrors.ibiblio.org/CTAN/macros/latex/contrib/tabu/tabu.pdf
\tabulinesep = 1mm
\usepackage[font=footnotesize]{caption} %Captions de les figures més petites

%Appendix
\usepackage[]{appendix} %toc, page

%Alinear al separador decimal amb espais
\usepackage{setspace}
\renewcommand*{\arraystretch}{1.25}

%Per fer el símbol de grau celsius amb \textcelsius{}
\usepackage{textcomp}


%%%%%%%%%%%%%%%%%%%%%%%%%%%%%%%%%%%%%%%%%%%%%%%%%%%%%%%%%%%%%%%%%%%%%%%%%%%%%%%% 
%%% ~ Arduino Language - Arduino IDE Colors ~                                  %%%
%%%                                                                            %%%
%%% Kyle Rocha-Brownell | 10/2/2017 | No Licence                               %%%
%%% -------------------------------------------------------------------------- %%%
%%%                                                                            %%%
%%% Place this file in your working directory (next to the latex file you're   %%%
%%% working on).  To add it to your project, place:                            %%%
%%%    %%%%%%%%%%%%%%%%%%%%%%%%%%%%%%%%%%%%%%%%%%%%%%%%%%%%%%%%%%%%%%%%%%%%%%%%%%%%%%%% 
%%% ~ Arduino Language - Arduino IDE Colors ~                                  %%%
%%%                                                                            %%%
%%% Kyle Rocha-Brownell | 10/2/2017 | No Licence                               %%%
%%% -------------------------------------------------------------------------- %%%
%%%                                                                            %%%
%%% Place this file in your working directory (next to the latex file you're   %%%
%%% working on).  To add it to your project, place:                            %%%
%%%    %%%%%%%%%%%%%%%%%%%%%%%%%%%%%%%%%%%%%%%%%%%%%%%%%%%%%%%%%%%%%%%%%%%%%%%%%%%%%%%% 
%%% ~ Arduino Language - Arduino IDE Colors ~                                  %%%
%%%                                                                            %%%
%%% Kyle Rocha-Brownell | 10/2/2017 | No Licence                               %%%
%%% -------------------------------------------------------------------------- %%%
%%%                                                                            %%%
%%% Place this file in your working directory (next to the latex file you're   %%%
%%% working on).  To add it to your project, place:                            %%%
%%%    \input{arduinoLanguage.tex}                                             %%%
%%% somewhere before \begin{document} in your latex file.                      %%%
%%%                                                                            %%%
%%% In your document, place your arduino code between:                         %%%
%%%   \begin{lstlisting}[language=Arduino]                                     %%%
%%% and:                                                                       %%%
%%%   \end{lstlisting}                                                         %%%
%%%                                                                            %%%
%%% Or create your own style to add non-built-in functions and variables.      %%%
%%%                                                                            %%%
 %%%%%%%%%%%%%%%%%%%%%%%%%%%%%%%%%%%%%%%%%%%%%%%%%%%%%%%%%%%%%%%%%%%%%%%%%%%%%%%% 

\usepackage{color}
\usepackage{listings}    
\usepackage{courier}

%%% Define Custom IDE Colors %%%
\definecolor{arduinoGreen}    {rgb} {0.17, 0.43, 0.01}
\definecolor{arduinoGrey}     {rgb} {0.47, 0.47, 0.33}
\definecolor{arduinoOrange}   {rgb} {0.8 , 0.4 , 0   }
\definecolor{arduinoBlue}     {rgb} {0.01, 0.61, 0.98}
\definecolor{arduinoDarkBlue} {rgb} {0.0 , 0.2 , 0.5 }

%%% Define Arduino Language %%%
\lstdefinelanguage{Arduino}{
  language=C++, % begin with default C++ settings 
%
%
  %%% Keyword Color Group 1 %%%  (called KEYWORD3 by arduino)
  keywordstyle=\color{arduinoGreen},   
  deletekeywords={  % remove all arduino keywords that might be in c++
                break, case, override, final, continue, default, do, else, for, 
                if, return, goto, switch, throw, try, while, setup, loop, export, 
                not, or, and, xor, include, define, elif, else, error, if, ifdef, 
                ifndef, pragma, warning,
                HIGH, LOW, INPUT, INPUT_PULLUP, OUTPUT, DEC, BIN, HEX, OCT, PI, 
                HALF_PI, TWO_PI, LSBFIRST, MSBFIRST, CHANGE, FALLING, RISING, 
                DEFAULT, EXTERNAL, INTERNAL, INTERNAL1V1, INTERNAL2V56, LED_BUILTIN, 
                LED_BUILTIN_RX, LED_BUILTIN_TX, DIGITAL_MESSAGE, FIRMATA_STRING, 
                ANALOG_MESSAGE, REPORT_DIGITAL, REPORT_ANALOG, SET_PIN_MODE, 
                SYSTEM_RESET, SYSEX_START, auto, int8_t, int16_t, int32_t, int64_t, 
                uint8_t, uint16_t, uint32_t, uint64_t, char16_t, char32_t, operator, 
                enum, delete, bool, boolean, byte, char, const, false, float, double, 
                null, NULL, int, long, new, private, protected, public, short, 
                signed, static, volatile, String, void, true, unsigned, word, array, 
                sizeof, dynamic_cast, typedef, const_cast, struct, static_cast, union, 
                friend, extern, class, reinterpret_cast, register, explicit, inline, 
                _Bool, complex, _Complex, _Imaginary, atomic_bool, atomic_char, 
                atomic_schar, atomic_uchar, atomic_short, atomic_ushort, atomic_int, 
                atomic_uint, atomic_long, atomic_ulong, atomic_llong, atomic_ullong, 
                virtual, PROGMEM,
                Serial, Serial1, Serial2, Serial3, SerialUSB, Keyboard, Mouse,
                abs, acos, asin, atan, atan2, ceil, constrain, cos, degrees, exp, 
                floor, log, map, max, min, radians, random, randomSeed, round, sin, 
                sq, sqrt, tan, pow, bitRead, bitWrite, bitSet, bitClear, bit, 
                highByte, lowByte, analogReference, analogRead, 
                analogReadResolution, analogWrite, analogWriteResolution, 
                attachInterrupt, detachInterrupt, digitalPinToInterrupt, delay, 
                delayMicroseconds, digitalWrite, digitalRead, interrupts, millis, 
                micros, noInterrupts, noTone, pinMode, pulseIn, pulseInLong, shiftIn, 
                shiftOut, tone, yield, Stream, begin, end, peek, read, print, 
                println, available, availableForWrite, flush, setTimeout, find, 
                findUntil, parseInt, parseFloat, readBytes, readBytesUntil, readString, 
                readStringUntil, trim, toUpperCase, toLowerCase, charAt, compareTo, 
                concat, endsWith, startsWith, equals, equalsIgnoreCase, getBytes, 
                indexOf, lastIndexOf, length, replace, setCharAt, substring, 
                toCharArray, toInt, press, release, releaseAll, accept, click, move, 
                isPressed, isAlphaNumeric, isAlpha, isAscii, isWhitespace, isControl, 
                isDigit, isGraph, isLowerCase, isPrintable, isPunct, isSpace, 
                isUpperCase, isHexadecimalDigit, 
                }, 
  morekeywords={   % add arduino structures to group 1
                break, case, override, final, continue, default, do, else, for, 
                if, return, goto, switch, throw, try, while, setup, loop, export, 
                not, or, and, xor, include, define, elif, else, error, if, ifdef, 
                ifndef, pragma, warning,
                }, 
% 
%
  %%% Keyword Color Group 2 %%%  (called LITERAL1 by arduino)
  keywordstyle=[2]\color{arduinoBlue},   
  keywords=[2]{   % add variables and dataTypes as 2nd group  
                HIGH, LOW, INPUT, INPUT_PULLUP, OUTPUT, DEC, BIN, HEX, OCT, PI, 
                HALF_PI, TWO_PI, LSBFIRST, MSBFIRST, CHANGE, FALLING, RISING, 
                DEFAULT, EXTERNAL, INTERNAL, INTERNAL1V1, INTERNAL2V56, LED_BUILTIN, 
                LED_BUILTIN_RX, LED_BUILTIN_TX, DIGITAL_MESSAGE, FIRMATA_STRING, 
                ANALOG_MESSAGE, REPORT_DIGITAL, REPORT_ANALOG, SET_PIN_MODE, 
                SYSTEM_RESET, SYSEX_START, auto, int8_t, int16_t, int32_t, int64_t, 
                uint8_t, uint16_t, uint32_t, uint64_t, char16_t, char32_t, operator, 
                enum, delete, bool, boolean, byte, char, const, false, float, double, 
                null, NULL, int, long, new, private, protected, public, short, 
                signed, static, volatile, String, void, true, unsigned, word, array, 
                sizeof, dynamic_cast, typedef, const_cast, struct, static_cast, union, 
                friend, extern, class, reinterpret_cast, register, explicit, inline, 
                _Bool, complex, _Complex, _Imaginary, atomic_bool, atomic_char, 
                atomic_schar, atomic_uchar, atomic_short, atomic_ushort, atomic_int, 
                atomic_uint, atomic_long, atomic_ulong, atomic_llong, atomic_ullong, 
                virtual, PROGMEM,
                },  
% 
%
  %%% Keyword Color Group 3 %%%  (called KEYWORD1 by arduino)
  keywordstyle=[3]\bfseries\color{arduinoOrange},
  keywords=[3]{  % add built-in functions as a 3rd group
                Serial, Serial1, Serial2, Serial3, SerialUSB, Keyboard, Mouse,
                },      
%
%
  %%% Keyword Color Group 4 %%%  (called KEYWORD2 by arduino)
  keywordstyle=[4]\color{arduinoOrange},
  keywords=[4]{  % add more built-in functions as a 4th group
                abs, acos, asin, atan, atan2, ceil, constrain, cos, degrees, exp, 
                floor, log, map, max, min, radians, random, randomSeed, round, sin, 
                sq, sqrt, tan, pow, bitRead, bitWrite, bitSet, bitClear, bit, 
                highByte, lowByte, analogReference, analogRead, 
                analogReadResolution, analogWrite, analogWriteResolution, 
                attachInterrupt, detachInterrupt, digitalPinToInterrupt, delay, 
                delayMicroseconds, digitalWrite, digitalRead, interrupts, millis, 
                micros, noInterrupts, noTone, pinMode, pulseIn, pulseInLong, shiftIn, 
                shiftOut, tone, yield, Stream, begin, end, peek, read, print, 
                println, available, availableForWrite, flush, setTimeout, find, 
                findUntil, parseInt, parseFloat, readBytes, readBytesUntil, readString, 
                readStringUntil, trim, toUpperCase, toLowerCase, charAt, compareTo, 
                concat, endsWith, startsWith, equals, equalsIgnoreCase, getBytes, 
                indexOf, lastIndexOf, length, replace, setCharAt, substring, 
                toCharArray, toInt, press, release, releaseAll, accept, click, move, 
                isPressed, isAlphaNumeric, isAlpha, isAscii, isWhitespace, isControl, 
                isDigit, isGraph, isLowerCase, isPrintable, isPunct, isSpace, 
                isUpperCase, isHexadecimalDigit, 
                },      
%
%
  %%% Set Other Colors %%%
  stringstyle=\color{arduinoDarkBlue},    
  commentstyle=\color{arduinoGrey},    
%          
%   
  %%%% Line Numbering %%%%
%   numbers=left,                    
  numbersep=5pt,                   
  numberstyle=\color{arduinoGrey},    
  %stepnumber=2,                      % show every 2 line numbers
%
%
  %%%% Code Box Style %%%%
  breaklines=true,                    % wordwrapping
  tabsize=2,         
  basicstyle=\ttfamily  
}                                             %%%
%%% somewhere before \begin{document} in your latex file.                      %%%
%%%                                                                            %%%
%%% In your document, place your arduino code between:                         %%%
%%%   \begin{lstlisting}[language=Arduino]                                     %%%
%%% and:                                                                       %%%
%%%   \end{lstlisting}                                                         %%%
%%%                                                                            %%%
%%% Or create your own style to add non-built-in functions and variables.      %%%
%%%                                                                            %%%
 %%%%%%%%%%%%%%%%%%%%%%%%%%%%%%%%%%%%%%%%%%%%%%%%%%%%%%%%%%%%%%%%%%%%%%%%%%%%%%%% 

\usepackage{color}
\usepackage{listings}    
\usepackage{courier}

%%% Define Custom IDE Colors %%%
\definecolor{arduinoGreen}    {rgb} {0.17, 0.43, 0.01}
\definecolor{arduinoGrey}     {rgb} {0.47, 0.47, 0.33}
\definecolor{arduinoOrange}   {rgb} {0.8 , 0.4 , 0   }
\definecolor{arduinoBlue}     {rgb} {0.01, 0.61, 0.98}
\definecolor{arduinoDarkBlue} {rgb} {0.0 , 0.2 , 0.5 }

%%% Define Arduino Language %%%
\lstdefinelanguage{Arduino}{
  language=C++, % begin with default C++ settings 
%
%
  %%% Keyword Color Group 1 %%%  (called KEYWORD3 by arduino)
  keywordstyle=\color{arduinoGreen},   
  deletekeywords={  % remove all arduino keywords that might be in c++
                break, case, override, final, continue, default, do, else, for, 
                if, return, goto, switch, throw, try, while, setup, loop, export, 
                not, or, and, xor, include, define, elif, else, error, if, ifdef, 
                ifndef, pragma, warning,
                HIGH, LOW, INPUT, INPUT_PULLUP, OUTPUT, DEC, BIN, HEX, OCT, PI, 
                HALF_PI, TWO_PI, LSBFIRST, MSBFIRST, CHANGE, FALLING, RISING, 
                DEFAULT, EXTERNAL, INTERNAL, INTERNAL1V1, INTERNAL2V56, LED_BUILTIN, 
                LED_BUILTIN_RX, LED_BUILTIN_TX, DIGITAL_MESSAGE, FIRMATA_STRING, 
                ANALOG_MESSAGE, REPORT_DIGITAL, REPORT_ANALOG, SET_PIN_MODE, 
                SYSTEM_RESET, SYSEX_START, auto, int8_t, int16_t, int32_t, int64_t, 
                uint8_t, uint16_t, uint32_t, uint64_t, char16_t, char32_t, operator, 
                enum, delete, bool, boolean, byte, char, const, false, float, double, 
                null, NULL, int, long, new, private, protected, public, short, 
                signed, static, volatile, String, void, true, unsigned, word, array, 
                sizeof, dynamic_cast, typedef, const_cast, struct, static_cast, union, 
                friend, extern, class, reinterpret_cast, register, explicit, inline, 
                _Bool, complex, _Complex, _Imaginary, atomic_bool, atomic_char, 
                atomic_schar, atomic_uchar, atomic_short, atomic_ushort, atomic_int, 
                atomic_uint, atomic_long, atomic_ulong, atomic_llong, atomic_ullong, 
                virtual, PROGMEM,
                Serial, Serial1, Serial2, Serial3, SerialUSB, Keyboard, Mouse,
                abs, acos, asin, atan, atan2, ceil, constrain, cos, degrees, exp, 
                floor, log, map, max, min, radians, random, randomSeed, round, sin, 
                sq, sqrt, tan, pow, bitRead, bitWrite, bitSet, bitClear, bit, 
                highByte, lowByte, analogReference, analogRead, 
                analogReadResolution, analogWrite, analogWriteResolution, 
                attachInterrupt, detachInterrupt, digitalPinToInterrupt, delay, 
                delayMicroseconds, digitalWrite, digitalRead, interrupts, millis, 
                micros, noInterrupts, noTone, pinMode, pulseIn, pulseInLong, shiftIn, 
                shiftOut, tone, yield, Stream, begin, end, peek, read, print, 
                println, available, availableForWrite, flush, setTimeout, find, 
                findUntil, parseInt, parseFloat, readBytes, readBytesUntil, readString, 
                readStringUntil, trim, toUpperCase, toLowerCase, charAt, compareTo, 
                concat, endsWith, startsWith, equals, equalsIgnoreCase, getBytes, 
                indexOf, lastIndexOf, length, replace, setCharAt, substring, 
                toCharArray, toInt, press, release, releaseAll, accept, click, move, 
                isPressed, isAlphaNumeric, isAlpha, isAscii, isWhitespace, isControl, 
                isDigit, isGraph, isLowerCase, isPrintable, isPunct, isSpace, 
                isUpperCase, isHexadecimalDigit, 
                }, 
  morekeywords={   % add arduino structures to group 1
                break, case, override, final, continue, default, do, else, for, 
                if, return, goto, switch, throw, try, while, setup, loop, export, 
                not, or, and, xor, include, define, elif, else, error, if, ifdef, 
                ifndef, pragma, warning,
                }, 
% 
%
  %%% Keyword Color Group 2 %%%  (called LITERAL1 by arduino)
  keywordstyle=[2]\color{arduinoBlue},   
  keywords=[2]{   % add variables and dataTypes as 2nd group  
                HIGH, LOW, INPUT, INPUT_PULLUP, OUTPUT, DEC, BIN, HEX, OCT, PI, 
                HALF_PI, TWO_PI, LSBFIRST, MSBFIRST, CHANGE, FALLING, RISING, 
                DEFAULT, EXTERNAL, INTERNAL, INTERNAL1V1, INTERNAL2V56, LED_BUILTIN, 
                LED_BUILTIN_RX, LED_BUILTIN_TX, DIGITAL_MESSAGE, FIRMATA_STRING, 
                ANALOG_MESSAGE, REPORT_DIGITAL, REPORT_ANALOG, SET_PIN_MODE, 
                SYSTEM_RESET, SYSEX_START, auto, int8_t, int16_t, int32_t, int64_t, 
                uint8_t, uint16_t, uint32_t, uint64_t, char16_t, char32_t, operator, 
                enum, delete, bool, boolean, byte, char, const, false, float, double, 
                null, NULL, int, long, new, private, protected, public, short, 
                signed, static, volatile, String, void, true, unsigned, word, array, 
                sizeof, dynamic_cast, typedef, const_cast, struct, static_cast, union, 
                friend, extern, class, reinterpret_cast, register, explicit, inline, 
                _Bool, complex, _Complex, _Imaginary, atomic_bool, atomic_char, 
                atomic_schar, atomic_uchar, atomic_short, atomic_ushort, atomic_int, 
                atomic_uint, atomic_long, atomic_ulong, atomic_llong, atomic_ullong, 
                virtual, PROGMEM,
                },  
% 
%
  %%% Keyword Color Group 3 %%%  (called KEYWORD1 by arduino)
  keywordstyle=[3]\bfseries\color{arduinoOrange},
  keywords=[3]{  % add built-in functions as a 3rd group
                Serial, Serial1, Serial2, Serial3, SerialUSB, Keyboard, Mouse,
                },      
%
%
  %%% Keyword Color Group 4 %%%  (called KEYWORD2 by arduino)
  keywordstyle=[4]\color{arduinoOrange},
  keywords=[4]{  % add more built-in functions as a 4th group
                abs, acos, asin, atan, atan2, ceil, constrain, cos, degrees, exp, 
                floor, log, map, max, min, radians, random, randomSeed, round, sin, 
                sq, sqrt, tan, pow, bitRead, bitWrite, bitSet, bitClear, bit, 
                highByte, lowByte, analogReference, analogRead, 
                analogReadResolution, analogWrite, analogWriteResolution, 
                attachInterrupt, detachInterrupt, digitalPinToInterrupt, delay, 
                delayMicroseconds, digitalWrite, digitalRead, interrupts, millis, 
                micros, noInterrupts, noTone, pinMode, pulseIn, pulseInLong, shiftIn, 
                shiftOut, tone, yield, Stream, begin, end, peek, read, print, 
                println, available, availableForWrite, flush, setTimeout, find, 
                findUntil, parseInt, parseFloat, readBytes, readBytesUntil, readString, 
                readStringUntil, trim, toUpperCase, toLowerCase, charAt, compareTo, 
                concat, endsWith, startsWith, equals, equalsIgnoreCase, getBytes, 
                indexOf, lastIndexOf, length, replace, setCharAt, substring, 
                toCharArray, toInt, press, release, releaseAll, accept, click, move, 
                isPressed, isAlphaNumeric, isAlpha, isAscii, isWhitespace, isControl, 
                isDigit, isGraph, isLowerCase, isPrintable, isPunct, isSpace, 
                isUpperCase, isHexadecimalDigit, 
                },      
%
%
  %%% Set Other Colors %%%
  stringstyle=\color{arduinoDarkBlue},    
  commentstyle=\color{arduinoGrey},    
%          
%   
  %%%% Line Numbering %%%%
%   numbers=left,                    
  numbersep=5pt,                   
  numberstyle=\color{arduinoGrey},    
  %stepnumber=2,                      % show every 2 line numbers
%
%
  %%%% Code Box Style %%%%
  breaklines=true,                    % wordwrapping
  tabsize=2,         
  basicstyle=\ttfamily  
}                                             %%%
%%% somewhere before \begin{document} in your latex file.                      %%%
%%%                                                                            %%%
%%% In your document, place your arduino code between:                         %%%
%%%   \begin{lstlisting}[language=Arduino]                                     %%%
%%% and:                                                                       %%%
%%%   \end{lstlisting}                                                         %%%
%%%                                                                            %%%
%%% Or create your own style to add non-built-in functions and variables.      %%%
%%%                                                                            %%%
 %%%%%%%%%%%%%%%%%%%%%%%%%%%%%%%%%%%%%%%%%%%%%%%%%%%%%%%%%%%%%%%%%%%%%%%%%%%%%%%% 

\usepackage{color}
\usepackage{listings}    
\usepackage{courier}

%%% Define Custom IDE Colors %%%
\definecolor{arduinoGreen}    {rgb} {0, 0, 0}
\definecolor{arduinoGrey}     {rgb} {0, 0, 0}
\definecolor{arduinoOrange}   {rgb} {0 , 0 , 0}
\definecolor{arduinoBlue}     {rgb} {0, 0, 0}
\definecolor{arduinoDarkBlue} {rgb} {0 , 0 , 0}

%%% Define Arduino Language %%%
\lstdefinelanguage{Arduino_negre}{
%  language=C++, % begin with default C++ settings 
%
%
  %%% Keyword Color Group 1 %%%  (called KEYWORD3 by arduino)
  keywordstyle=\color{arduinoGreen},   
  deletekeywords={  % remove all arduino keywords that might be in c++
                break, case, override, final, continue, default, do, else, for, 
                if, return, goto, switch, throw, try, while, setup, loop, export, 
                not, or, and, xor, include, define, elif, else, error, if, ifdef, 
                ifndef, pragma, warning,
                HIGH, LOW, INPUT, INPUT_PULLUP, OUTPUT, DEC, BIN, HEX, OCT, PI, 
                HALF_PI, TWO_PI, LSBFIRST, MSBFIRST, CHANGE, FALLING, RISING, 
                DEFAULT, EXTERNAL, INTERNAL, INTERNAL1V1, INTERNAL2V56, LED_BUILTIN, 
                LED_BUILTIN_RX, LED_BUILTIN_TX, DIGITAL_MESSAGE, FIRMATA_STRING, 
                ANALOG_MESSAGE, REPORT_DIGITAL, REPORT_ANALOG, SET_PIN_MODE, 
                SYSTEM_RESET, SYSEX_START, auto, int8_t, int16_t, int32_t, int64_t, 
                uint8_t, uint16_t, uint32_t, uint64_t, char16_t, char32_t, operator, 
                enum, delete, bool, boolean, byte, char, const, false, float, double, 
                null, NULL, int, long, new, private, protected, public, short, 
                signed, static, volatile, String, void, true, unsigned, word, array, 
                sizeof, dynamic_cast, typedef, const_cast, struct, static_cast, union, 
                friend, extern, class, reinterpret_cast, register, explicit, inline, 
                _Bool, complex, _Complex, _Imaginary, atomic_bool, atomic_char, 
                atomic_schar, atomic_uchar, atomic_short, atomic_ushort, atomic_int, 
                atomic_uint, atomic_long, atomic_ulong, atomic_llong, atomic_ullong, 
                virtual, PROGMEM,
                Serial, Serial1, Serial2, Serial3, SerialUSB, Keyboard, Mouse,
                abs, acos, asin, atan, atan2, ceil, constrain, cos, degrees, exp, 
                floor, log, map, max, min, radians, random, randomSeed, round, sin, 
                sq, sqrt, tan, pow, bitRead, bitWrite, bitSet, bitClear, bit, 
                highByte, lowByte, analogReference, analogRead, 
                analogReadResolution, analogWrite, analogWriteResolution, 
                attachInterrupt, detachInterrupt, digitalPinToInterrupt, delay, 
                delayMicroseconds, digitalWrite, digitalRead, interrupts, millis, 
                micros, noInterrupts, noTone, pinMode, pulseIn, pulseInLong, shiftIn, 
                shiftOut, tone, yield, Stream, begin, end, peek, read, print, 
                println, available, availableForWrite, flush, setTimeout, find, 
                findUntil, parseInt, parseFloat, readBytes, readBytesUntil, readString, 
                readStringUntil, trim, toUpperCase, toLowerCase, charAt, compareTo, 
                concat, endsWith, startsWith, equals, equalsIgnoreCase, getBytes, 
                indexOf, lastIndexOf, length, replace, setCharAt, substring, 
                toCharArray, toInt, press, release, releaseAll, accept, click, move, 
                isPressed, isAlphaNumeric, isAlpha, isAscii, isWhitespace, isControl, 
                isDigit, isGraph, isLowerCase, isPrintable, isPunct, isSpace, 
                isUpperCase, isHexadecimalDigit, 
                }, 
  morekeywords={   % add arduino structures to group 1
                break, case, override, final, continue, default, do, else, for, 
                if, return, goto, switch, throw, try, while, setup, loop, export, 
                not, or, and, xor, include, define, elif, else, error, if, ifdef, 
                ifndef, pragma, warning,
                }, 
% 
%
  %%% Keyword Color Group 2 %%%  (called LITERAL1 by arduino)
  keywordstyle=[2]\color{arduinoBlue},   
  keywords=[2]{   % add variables and dataTypes as 2nd group  
                HIGH, LOW, INPUT, INPUT_PULLUP, OUTPUT, DEC, BIN, HEX, OCT, PI, 
                HALF_PI, TWO_PI, LSBFIRST, MSBFIRST, CHANGE, FALLING, RISING, 
                DEFAULT, EXTERNAL, INTERNAL, INTERNAL1V1, INTERNAL2V56, LED_BUILTIN, 
                LED_BUILTIN_RX, LED_BUILTIN_TX, DIGITAL_MESSAGE, FIRMATA_STRING, 
                ANALOG_MESSAGE, REPORT_DIGITAL, REPORT_ANALOG, SET_PIN_MODE, 
                SYSTEM_RESET, SYSEX_START, auto, int8_t, int16_t, int32_t, int64_t, 
                uint8_t, uint16_t, uint32_t, uint64_t, char16_t, char32_t, operator, 
                enum, delete, bool, boolean, byte, char, const, false, float, double, 
                null, NULL, int, long, new, private, protected, public, short, 
                signed, static, volatile, String, void, true, unsigned, word, array, 
                sizeof, dynamic_cast, typedef, const_cast, struct, static_cast, union, 
                friend, extern, class, reinterpret_cast, register, explicit, inline, 
                _Bool, complex, _Complex, _Imaginary, atomic_bool, atomic_char, 
                atomic_schar, atomic_uchar, atomic_short, atomic_ushort, atomic_int, 
                atomic_uint, atomic_long, atomic_ulong, atomic_llong, atomic_ullong, 
                virtual, PROGMEM,
                },  
% 
%
  %%% Keyword Color Group 3 %%%  (called KEYWORD1 by arduino)
  keywordstyle=[3]\bfseries\color{arduinoOrange},
  keywords=[3]{  % add built-in functions as a 3rd group
                Serial, Serial1, Serial2, Serial3, SerialUSB, Keyboard, Mouse,
                },      
%
%
  %%% Keyword Color Group 4 %%%  (called KEYWORD2 by arduino)
  keywordstyle=[4]\color{arduinoOrange},
  keywords=[4]{  % add more built-in functions as a 4th group
                abs, acos, asin, atan, atan2, ceil, constrain, cos, degrees, exp, 
                floor, log, map, max, min, radians, random, randomSeed, round, sin, 
                sq, sqrt, tan, pow, bitRead, bitWrite, bitSet, bitClear, bit, 
                highByte, lowByte, analogReference, analogRead, 
                analogReadResolution, analogWrite, analogWriteResolution, 
                attachInterrupt, detachInterrupt, digitalPinToInterrupt, delay, 
                delayMicroseconds, digitalWrite, digitalRead, interrupts, millis, 
                micros, noInterrupts, noTone, pinMode, pulseIn, pulseInLong, shiftIn, 
                shiftOut, tone, yield, Stream, begin, end, peek, read, print, 
                println, available, availableForWrite, flush, setTimeout, find, 
                findUntil, parseInt, parseFloat, readBytes, readBytesUntil, readString, 
                readStringUntil, trim, toUpperCase, toLowerCase, charAt, compareTo, 
                concat, endsWith, startsWith, equals, equalsIgnoreCase, getBytes, 
                indexOf, lastIndexOf, length, replace, setCharAt, substring, 
                toCharArray, toInt, press, release, releaseAll, accept, click, move, 
                isPressed, isAlphaNumeric, isAlpha, isAscii, isWhitespace, isControl, 
                isDigit, isGraph, isLowerCase, isPrintable, isPunct, isSpace, 
                isUpperCase, isHexadecimalDigit, 
                },      
%
%
  %%% Set Other Colors %%%
  stringstyle=\color{arduinoDarkBlue},    
  commentstyle=\color{arduinoGrey},    
%          
%   
  %%%% Line Numbering %%%%
%   numbers=left,                    
  numbersep=5pt,                   
  numberstyle=\color{arduinoGrey},    
  %stepnumber=2,                      % show every 2 line numbers
%
%
  %%%% Code Box Style %%%%
  breaklines=true,                    % wordwrapping
  tabsize=2,         
  basicstyle=\ttfamily  
} 
\lstdefinestyle{myArduino_negre}{
  language=Arduino_negre,
  basicstyle={\small\ttfamily}, %small
%% make listing changes here %%
}
%%%%%%%%%%%%%%%%%%%%%%%%%%%%%%%%%%%%%%%%%%%%%%%%%%%%%%%%%%%%%%%%%%%%%%%%%%%%%%%% 
%%% ~ Arduino Language - Arduino IDE Colors ~                                  %%%
%%%                                                                            %%%
%%% Kyle Rocha-Brownell | 10/2/2017 | No Licence                               %%%
%%% -------------------------------------------------------------------------- %%%
%%%                                                                            %%%
%%% Place this file in your working directory (next to the latex file you're   %%%
%%% working on).  To add it to your project, place:                            %%%
%%%    %%%%%%%%%%%%%%%%%%%%%%%%%%%%%%%%%%%%%%%%%%%%%%%%%%%%%%%%%%%%%%%%%%%%%%%%%%%%%%%% 
%%% ~ Arduino Language - Arduino IDE Colors ~                                  %%%
%%%                                                                            %%%
%%% Kyle Rocha-Brownell | 10/2/2017 | No Licence                               %%%
%%% -------------------------------------------------------------------------- %%%
%%%                                                                            %%%
%%% Place this file in your working directory (next to the latex file you're   %%%
%%% working on).  To add it to your project, place:                            %%%
%%%    %%%%%%%%%%%%%%%%%%%%%%%%%%%%%%%%%%%%%%%%%%%%%%%%%%%%%%%%%%%%%%%%%%%%%%%%%%%%%%%% 
%%% ~ Arduino Language - Arduino IDE Colors ~                                  %%%
%%%                                                                            %%%
%%% Kyle Rocha-Brownell | 10/2/2017 | No Licence                               %%%
%%% -------------------------------------------------------------------------- %%%
%%%                                                                            %%%
%%% Place this file in your working directory (next to the latex file you're   %%%
%%% working on).  To add it to your project, place:                            %%%
%%%    \input{arduinoLanguage.tex}                                             %%%
%%% somewhere before \begin{document} in your latex file.                      %%%
%%%                                                                            %%%
%%% In your document, place your arduino code between:                         %%%
%%%   \begin{lstlisting}[language=Arduino]                                     %%%
%%% and:                                                                       %%%
%%%   \end{lstlisting}                                                         %%%
%%%                                                                            %%%
%%% Or create your own style to add non-built-in functions and variables.      %%%
%%%                                                                            %%%
 %%%%%%%%%%%%%%%%%%%%%%%%%%%%%%%%%%%%%%%%%%%%%%%%%%%%%%%%%%%%%%%%%%%%%%%%%%%%%%%% 

\usepackage{color}
\usepackage{listings}    
\usepackage{courier}

%%% Define Custom IDE Colors %%%
\definecolor{arduinoGreen}    {rgb} {0.17, 0.43, 0.01}
\definecolor{arduinoGrey}     {rgb} {0.47, 0.47, 0.33}
\definecolor{arduinoOrange}   {rgb} {0.8 , 0.4 , 0   }
\definecolor{arduinoBlue}     {rgb} {0.01, 0.61, 0.98}
\definecolor{arduinoDarkBlue} {rgb} {0.0 , 0.2 , 0.5 }

%%% Define Arduino Language %%%
\lstdefinelanguage{Arduino}{
  language=C++, % begin with default C++ settings 
%
%
  %%% Keyword Color Group 1 %%%  (called KEYWORD3 by arduino)
  keywordstyle=\color{arduinoGreen},   
  deletekeywords={  % remove all arduino keywords that might be in c++
                break, case, override, final, continue, default, do, else, for, 
                if, return, goto, switch, throw, try, while, setup, loop, export, 
                not, or, and, xor, include, define, elif, else, error, if, ifdef, 
                ifndef, pragma, warning,
                HIGH, LOW, INPUT, INPUT_PULLUP, OUTPUT, DEC, BIN, HEX, OCT, PI, 
                HALF_PI, TWO_PI, LSBFIRST, MSBFIRST, CHANGE, FALLING, RISING, 
                DEFAULT, EXTERNAL, INTERNAL, INTERNAL1V1, INTERNAL2V56, LED_BUILTIN, 
                LED_BUILTIN_RX, LED_BUILTIN_TX, DIGITAL_MESSAGE, FIRMATA_STRING, 
                ANALOG_MESSAGE, REPORT_DIGITAL, REPORT_ANALOG, SET_PIN_MODE, 
                SYSTEM_RESET, SYSEX_START, auto, int8_t, int16_t, int32_t, int64_t, 
                uint8_t, uint16_t, uint32_t, uint64_t, char16_t, char32_t, operator, 
                enum, delete, bool, boolean, byte, char, const, false, float, double, 
                null, NULL, int, long, new, private, protected, public, short, 
                signed, static, volatile, String, void, true, unsigned, word, array, 
                sizeof, dynamic_cast, typedef, const_cast, struct, static_cast, union, 
                friend, extern, class, reinterpret_cast, register, explicit, inline, 
                _Bool, complex, _Complex, _Imaginary, atomic_bool, atomic_char, 
                atomic_schar, atomic_uchar, atomic_short, atomic_ushort, atomic_int, 
                atomic_uint, atomic_long, atomic_ulong, atomic_llong, atomic_ullong, 
                virtual, PROGMEM,
                Serial, Serial1, Serial2, Serial3, SerialUSB, Keyboard, Mouse,
                abs, acos, asin, atan, atan2, ceil, constrain, cos, degrees, exp, 
                floor, log, map, max, min, radians, random, randomSeed, round, sin, 
                sq, sqrt, tan, pow, bitRead, bitWrite, bitSet, bitClear, bit, 
                highByte, lowByte, analogReference, analogRead, 
                analogReadResolution, analogWrite, analogWriteResolution, 
                attachInterrupt, detachInterrupt, digitalPinToInterrupt, delay, 
                delayMicroseconds, digitalWrite, digitalRead, interrupts, millis, 
                micros, noInterrupts, noTone, pinMode, pulseIn, pulseInLong, shiftIn, 
                shiftOut, tone, yield, Stream, begin, end, peek, read, print, 
                println, available, availableForWrite, flush, setTimeout, find, 
                findUntil, parseInt, parseFloat, readBytes, readBytesUntil, readString, 
                readStringUntil, trim, toUpperCase, toLowerCase, charAt, compareTo, 
                concat, endsWith, startsWith, equals, equalsIgnoreCase, getBytes, 
                indexOf, lastIndexOf, length, replace, setCharAt, substring, 
                toCharArray, toInt, press, release, releaseAll, accept, click, move, 
                isPressed, isAlphaNumeric, isAlpha, isAscii, isWhitespace, isControl, 
                isDigit, isGraph, isLowerCase, isPrintable, isPunct, isSpace, 
                isUpperCase, isHexadecimalDigit, 
                }, 
  morekeywords={   % add arduino structures to group 1
                break, case, override, final, continue, default, do, else, for, 
                if, return, goto, switch, throw, try, while, setup, loop, export, 
                not, or, and, xor, include, define, elif, else, error, if, ifdef, 
                ifndef, pragma, warning,
                }, 
% 
%
  %%% Keyword Color Group 2 %%%  (called LITERAL1 by arduino)
  keywordstyle=[2]\color{arduinoBlue},   
  keywords=[2]{   % add variables and dataTypes as 2nd group  
                HIGH, LOW, INPUT, INPUT_PULLUP, OUTPUT, DEC, BIN, HEX, OCT, PI, 
                HALF_PI, TWO_PI, LSBFIRST, MSBFIRST, CHANGE, FALLING, RISING, 
                DEFAULT, EXTERNAL, INTERNAL, INTERNAL1V1, INTERNAL2V56, LED_BUILTIN, 
                LED_BUILTIN_RX, LED_BUILTIN_TX, DIGITAL_MESSAGE, FIRMATA_STRING, 
                ANALOG_MESSAGE, REPORT_DIGITAL, REPORT_ANALOG, SET_PIN_MODE, 
                SYSTEM_RESET, SYSEX_START, auto, int8_t, int16_t, int32_t, int64_t, 
                uint8_t, uint16_t, uint32_t, uint64_t, char16_t, char32_t, operator, 
                enum, delete, bool, boolean, byte, char, const, false, float, double, 
                null, NULL, int, long, new, private, protected, public, short, 
                signed, static, volatile, String, void, true, unsigned, word, array, 
                sizeof, dynamic_cast, typedef, const_cast, struct, static_cast, union, 
                friend, extern, class, reinterpret_cast, register, explicit, inline, 
                _Bool, complex, _Complex, _Imaginary, atomic_bool, atomic_char, 
                atomic_schar, atomic_uchar, atomic_short, atomic_ushort, atomic_int, 
                atomic_uint, atomic_long, atomic_ulong, atomic_llong, atomic_ullong, 
                virtual, PROGMEM,
                },  
% 
%
  %%% Keyword Color Group 3 %%%  (called KEYWORD1 by arduino)
  keywordstyle=[3]\bfseries\color{arduinoOrange},
  keywords=[3]{  % add built-in functions as a 3rd group
                Serial, Serial1, Serial2, Serial3, SerialUSB, Keyboard, Mouse,
                },      
%
%
  %%% Keyword Color Group 4 %%%  (called KEYWORD2 by arduino)
  keywordstyle=[4]\color{arduinoOrange},
  keywords=[4]{  % add more built-in functions as a 4th group
                abs, acos, asin, atan, atan2, ceil, constrain, cos, degrees, exp, 
                floor, log, map, max, min, radians, random, randomSeed, round, sin, 
                sq, sqrt, tan, pow, bitRead, bitWrite, bitSet, bitClear, bit, 
                highByte, lowByte, analogReference, analogRead, 
                analogReadResolution, analogWrite, analogWriteResolution, 
                attachInterrupt, detachInterrupt, digitalPinToInterrupt, delay, 
                delayMicroseconds, digitalWrite, digitalRead, interrupts, millis, 
                micros, noInterrupts, noTone, pinMode, pulseIn, pulseInLong, shiftIn, 
                shiftOut, tone, yield, Stream, begin, end, peek, read, print, 
                println, available, availableForWrite, flush, setTimeout, find, 
                findUntil, parseInt, parseFloat, readBytes, readBytesUntil, readString, 
                readStringUntil, trim, toUpperCase, toLowerCase, charAt, compareTo, 
                concat, endsWith, startsWith, equals, equalsIgnoreCase, getBytes, 
                indexOf, lastIndexOf, length, replace, setCharAt, substring, 
                toCharArray, toInt, press, release, releaseAll, accept, click, move, 
                isPressed, isAlphaNumeric, isAlpha, isAscii, isWhitespace, isControl, 
                isDigit, isGraph, isLowerCase, isPrintable, isPunct, isSpace, 
                isUpperCase, isHexadecimalDigit, 
                },      
%
%
  %%% Set Other Colors %%%
  stringstyle=\color{arduinoDarkBlue},    
  commentstyle=\color{arduinoGrey},    
%          
%   
  %%%% Line Numbering %%%%
%   numbers=left,                    
  numbersep=5pt,                   
  numberstyle=\color{arduinoGrey},    
  %stepnumber=2,                      % show every 2 line numbers
%
%
  %%%% Code Box Style %%%%
  breaklines=true,                    % wordwrapping
  tabsize=2,         
  basicstyle=\ttfamily  
}                                             %%%
%%% somewhere before \begin{document} in your latex file.                      %%%
%%%                                                                            %%%
%%% In your document, place your arduino code between:                         %%%
%%%   \begin{lstlisting}[language=Arduino]                                     %%%
%%% and:                                                                       %%%
%%%   \end{lstlisting}                                                         %%%
%%%                                                                            %%%
%%% Or create your own style to add non-built-in functions and variables.      %%%
%%%                                                                            %%%
 %%%%%%%%%%%%%%%%%%%%%%%%%%%%%%%%%%%%%%%%%%%%%%%%%%%%%%%%%%%%%%%%%%%%%%%%%%%%%%%% 

\usepackage{color}
\usepackage{listings}    
\usepackage{courier}

%%% Define Custom IDE Colors %%%
\definecolor{arduinoGreen}    {rgb} {0.17, 0.43, 0.01}
\definecolor{arduinoGrey}     {rgb} {0.47, 0.47, 0.33}
\definecolor{arduinoOrange}   {rgb} {0.8 , 0.4 , 0   }
\definecolor{arduinoBlue}     {rgb} {0.01, 0.61, 0.98}
\definecolor{arduinoDarkBlue} {rgb} {0.0 , 0.2 , 0.5 }

%%% Define Arduino Language %%%
\lstdefinelanguage{Arduino}{
  language=C++, % begin with default C++ settings 
%
%
  %%% Keyword Color Group 1 %%%  (called KEYWORD3 by arduino)
  keywordstyle=\color{arduinoGreen},   
  deletekeywords={  % remove all arduino keywords that might be in c++
                break, case, override, final, continue, default, do, else, for, 
                if, return, goto, switch, throw, try, while, setup, loop, export, 
                not, or, and, xor, include, define, elif, else, error, if, ifdef, 
                ifndef, pragma, warning,
                HIGH, LOW, INPUT, INPUT_PULLUP, OUTPUT, DEC, BIN, HEX, OCT, PI, 
                HALF_PI, TWO_PI, LSBFIRST, MSBFIRST, CHANGE, FALLING, RISING, 
                DEFAULT, EXTERNAL, INTERNAL, INTERNAL1V1, INTERNAL2V56, LED_BUILTIN, 
                LED_BUILTIN_RX, LED_BUILTIN_TX, DIGITAL_MESSAGE, FIRMATA_STRING, 
                ANALOG_MESSAGE, REPORT_DIGITAL, REPORT_ANALOG, SET_PIN_MODE, 
                SYSTEM_RESET, SYSEX_START, auto, int8_t, int16_t, int32_t, int64_t, 
                uint8_t, uint16_t, uint32_t, uint64_t, char16_t, char32_t, operator, 
                enum, delete, bool, boolean, byte, char, const, false, float, double, 
                null, NULL, int, long, new, private, protected, public, short, 
                signed, static, volatile, String, void, true, unsigned, word, array, 
                sizeof, dynamic_cast, typedef, const_cast, struct, static_cast, union, 
                friend, extern, class, reinterpret_cast, register, explicit, inline, 
                _Bool, complex, _Complex, _Imaginary, atomic_bool, atomic_char, 
                atomic_schar, atomic_uchar, atomic_short, atomic_ushort, atomic_int, 
                atomic_uint, atomic_long, atomic_ulong, atomic_llong, atomic_ullong, 
                virtual, PROGMEM,
                Serial, Serial1, Serial2, Serial3, SerialUSB, Keyboard, Mouse,
                abs, acos, asin, atan, atan2, ceil, constrain, cos, degrees, exp, 
                floor, log, map, max, min, radians, random, randomSeed, round, sin, 
                sq, sqrt, tan, pow, bitRead, bitWrite, bitSet, bitClear, bit, 
                highByte, lowByte, analogReference, analogRead, 
                analogReadResolution, analogWrite, analogWriteResolution, 
                attachInterrupt, detachInterrupt, digitalPinToInterrupt, delay, 
                delayMicroseconds, digitalWrite, digitalRead, interrupts, millis, 
                micros, noInterrupts, noTone, pinMode, pulseIn, pulseInLong, shiftIn, 
                shiftOut, tone, yield, Stream, begin, end, peek, read, print, 
                println, available, availableForWrite, flush, setTimeout, find, 
                findUntil, parseInt, parseFloat, readBytes, readBytesUntil, readString, 
                readStringUntil, trim, toUpperCase, toLowerCase, charAt, compareTo, 
                concat, endsWith, startsWith, equals, equalsIgnoreCase, getBytes, 
                indexOf, lastIndexOf, length, replace, setCharAt, substring, 
                toCharArray, toInt, press, release, releaseAll, accept, click, move, 
                isPressed, isAlphaNumeric, isAlpha, isAscii, isWhitespace, isControl, 
                isDigit, isGraph, isLowerCase, isPrintable, isPunct, isSpace, 
                isUpperCase, isHexadecimalDigit, 
                }, 
  morekeywords={   % add arduino structures to group 1
                break, case, override, final, continue, default, do, else, for, 
                if, return, goto, switch, throw, try, while, setup, loop, export, 
                not, or, and, xor, include, define, elif, else, error, if, ifdef, 
                ifndef, pragma, warning,
                }, 
% 
%
  %%% Keyword Color Group 2 %%%  (called LITERAL1 by arduino)
  keywordstyle=[2]\color{arduinoBlue},   
  keywords=[2]{   % add variables and dataTypes as 2nd group  
                HIGH, LOW, INPUT, INPUT_PULLUP, OUTPUT, DEC, BIN, HEX, OCT, PI, 
                HALF_PI, TWO_PI, LSBFIRST, MSBFIRST, CHANGE, FALLING, RISING, 
                DEFAULT, EXTERNAL, INTERNAL, INTERNAL1V1, INTERNAL2V56, LED_BUILTIN, 
                LED_BUILTIN_RX, LED_BUILTIN_TX, DIGITAL_MESSAGE, FIRMATA_STRING, 
                ANALOG_MESSAGE, REPORT_DIGITAL, REPORT_ANALOG, SET_PIN_MODE, 
                SYSTEM_RESET, SYSEX_START, auto, int8_t, int16_t, int32_t, int64_t, 
                uint8_t, uint16_t, uint32_t, uint64_t, char16_t, char32_t, operator, 
                enum, delete, bool, boolean, byte, char, const, false, float, double, 
                null, NULL, int, long, new, private, protected, public, short, 
                signed, static, volatile, String, void, true, unsigned, word, array, 
                sizeof, dynamic_cast, typedef, const_cast, struct, static_cast, union, 
                friend, extern, class, reinterpret_cast, register, explicit, inline, 
                _Bool, complex, _Complex, _Imaginary, atomic_bool, atomic_char, 
                atomic_schar, atomic_uchar, atomic_short, atomic_ushort, atomic_int, 
                atomic_uint, atomic_long, atomic_ulong, atomic_llong, atomic_ullong, 
                virtual, PROGMEM,
                },  
% 
%
  %%% Keyword Color Group 3 %%%  (called KEYWORD1 by arduino)
  keywordstyle=[3]\bfseries\color{arduinoOrange},
  keywords=[3]{  % add built-in functions as a 3rd group
                Serial, Serial1, Serial2, Serial3, SerialUSB, Keyboard, Mouse,
                },      
%
%
  %%% Keyword Color Group 4 %%%  (called KEYWORD2 by arduino)
  keywordstyle=[4]\color{arduinoOrange},
  keywords=[4]{  % add more built-in functions as a 4th group
                abs, acos, asin, atan, atan2, ceil, constrain, cos, degrees, exp, 
                floor, log, map, max, min, radians, random, randomSeed, round, sin, 
                sq, sqrt, tan, pow, bitRead, bitWrite, bitSet, bitClear, bit, 
                highByte, lowByte, analogReference, analogRead, 
                analogReadResolution, analogWrite, analogWriteResolution, 
                attachInterrupt, detachInterrupt, digitalPinToInterrupt, delay, 
                delayMicroseconds, digitalWrite, digitalRead, interrupts, millis, 
                micros, noInterrupts, noTone, pinMode, pulseIn, pulseInLong, shiftIn, 
                shiftOut, tone, yield, Stream, begin, end, peek, read, print, 
                println, available, availableForWrite, flush, setTimeout, find, 
                findUntil, parseInt, parseFloat, readBytes, readBytesUntil, readString, 
                readStringUntil, trim, toUpperCase, toLowerCase, charAt, compareTo, 
                concat, endsWith, startsWith, equals, equalsIgnoreCase, getBytes, 
                indexOf, lastIndexOf, length, replace, setCharAt, substring, 
                toCharArray, toInt, press, release, releaseAll, accept, click, move, 
                isPressed, isAlphaNumeric, isAlpha, isAscii, isWhitespace, isControl, 
                isDigit, isGraph, isLowerCase, isPrintable, isPunct, isSpace, 
                isUpperCase, isHexadecimalDigit, 
                },      
%
%
  %%% Set Other Colors %%%
  stringstyle=\color{arduinoDarkBlue},    
  commentstyle=\color{arduinoGrey},    
%          
%   
  %%%% Line Numbering %%%%
%   numbers=left,                    
  numbersep=5pt,                   
  numberstyle=\color{arduinoGrey},    
  %stepnumber=2,                      % show every 2 line numbers
%
%
  %%%% Code Box Style %%%%
  breaklines=true,                    % wordwrapping
  tabsize=2,         
  basicstyle=\ttfamily  
}                                             %%%
%%% somewhere before \begin{document} in your latex file.                      %%%
%%%                                                                            %%%
%%% In your document, place your arduino code between:                         %%%
%%%   \begin{lstlisting}[language=Arduino]                                     %%%
%%% and:                                                                       %%%
%%%   \end{lstlisting}                                                         %%%
%%%                                                                            %%%
%%% Or create your own style to add non-built-in functions and variables.      %%%
%%%                                                                            %%%
 %%%%%%%%%%%%%%%%%%%%%%%%%%%%%%%%%%%%%%%%%%%%%%%%%%%%%%%%%%%%%%%%%%%%%%%%%%%%%%%% 

\usepackage{color}
\usepackage{listings}    
\usepackage{courier}

%%% Define Custom IDE Colors %%%
\definecolor{arduinoGreen}    {rgb} {0.17, 0.43, 0.01}
\definecolor{arduinoGrey}     {rgb} {0.47, 0.47, 0.33}
\definecolor{arduinoOrange}   {rgb} {0.8 , 0.4 , 0   }
\definecolor{arduinoBlue}     {rgb} {0.01, 0.61, 0.98}
\definecolor{arduinoDarkBlue} {rgb} {0.0 , 0.2 , 0.5 }

%%% Define Arduino Language %%%
\lstdefinelanguage{Arduino}{
  language=C++, % begin with default C++ settings 
%
%
  %%% Keyword Color Group 1 %%%  (called KEYWORD3 by arduino)
  keywordstyle=\color{arduinoGreen},   
  deletekeywords={  % remove all arduino keywords that might be in c++
                break, case, override, final, continue, default, do, else, for, 
                if, return, goto, switch, throw, try, while, setup, loop, export, 
                not, or, and, xor, include, define, elif, else, error, if, ifdef, 
                ifndef, pragma, warning,
                HIGH, LOW, INPUT, INPUT_PULLUP, OUTPUT, DEC, BIN, HEX, OCT, PI, 
                HALF_PI, TWO_PI, LSBFIRST, MSBFIRST, CHANGE, FALLING, RISING, 
                DEFAULT, EXTERNAL, INTERNAL, INTERNAL1V1, INTERNAL2V56, LED_BUILTIN, 
                LED_BUILTIN_RX, LED_BUILTIN_TX, DIGITAL_MESSAGE, FIRMATA_STRING, 
                ANALOG_MESSAGE, REPORT_DIGITAL, REPORT_ANALOG, SET_PIN_MODE, 
                SYSTEM_RESET, SYSEX_START, auto, int8_t, int16_t, int32_t, int64_t, 
                uint8_t, uint16_t, uint32_t, uint64_t, char16_t, char32_t, operator, 
                enum, delete, bool, boolean, byte, char, const, false, float, double, 
                null, NULL, int, long, new, private, protected, public, short, 
                signed, static, volatile, String, void, true, unsigned, word, array, 
                sizeof, dynamic_cast, typedef, const_cast, struct, static_cast, union, 
                friend, extern, class, reinterpret_cast, register, explicit, inline, 
                _Bool, complex, _Complex, _Imaginary, atomic_bool, atomic_char, 
                atomic_schar, atomic_uchar, atomic_short, atomic_ushort, atomic_int, 
                atomic_uint, atomic_long, atomic_ulong, atomic_llong, atomic_ullong, 
                virtual, PROGMEM,
                Serial, Serial1, Serial2, Serial3, SerialUSB, Keyboard, Mouse,
                abs, acos, asin, atan, atan2, ceil, constrain, cos, degrees, exp, 
                floor, log, map, max, min, radians, random, randomSeed, round, sin, 
                sq, sqrt, tan, pow, bitRead, bitWrite, bitSet, bitClear, bit, 
                highByte, lowByte, analogReference, analogRead, 
                analogReadResolution, analogWrite, analogWriteResolution, 
                attachInterrupt, detachInterrupt, digitalPinToInterrupt, delay, 
                delayMicroseconds, digitalWrite, digitalRead, interrupts, millis, 
                micros, noInterrupts, noTone, pinMode, pulseIn, pulseInLong, shiftIn, 
                shiftOut, tone, yield, Stream, begin, end, peek, read, print, 
                println, available, availableForWrite, flush, setTimeout, find, 
                findUntil, parseInt, parseFloat, readBytes, readBytesUntil, readString, 
                readStringUntil, trim, toUpperCase, toLowerCase, charAt, compareTo, 
                concat, endsWith, startsWith, equals, equalsIgnoreCase, getBytes, 
                indexOf, lastIndexOf, length, replace, setCharAt, substring, 
                toCharArray, toInt, press, release, releaseAll, accept, click, move, 
                isPressed, isAlphaNumeric, isAlpha, isAscii, isWhitespace, isControl, 
                isDigit, isGraph, isLowerCase, isPrintable, isPunct, isSpace, 
                isUpperCase, isHexadecimalDigit, 
                }, 
  morekeywords={   % add arduino structures to group 1
                break, case, override, final, continue, default, do, else, for, 
                if, return, goto, switch, throw, try, while, setup, loop, export, 
                not, or, and, xor, include, define, elif, else, error, if, ifdef, 
                ifndef, pragma, warning,
                }, 
% 
%
  %%% Keyword Color Group 2 %%%  (called LITERAL1 by arduino)
  keywordstyle=[2]\color{arduinoBlue},   
  keywords=[2]{   % add variables and dataTypes as 2nd group  
                HIGH, LOW, INPUT, INPUT_PULLUP, OUTPUT, DEC, BIN, HEX, OCT, PI, 
                HALF_PI, TWO_PI, LSBFIRST, MSBFIRST, CHANGE, FALLING, RISING, 
                DEFAULT, EXTERNAL, INTERNAL, INTERNAL1V1, INTERNAL2V56, LED_BUILTIN, 
                LED_BUILTIN_RX, LED_BUILTIN_TX, DIGITAL_MESSAGE, FIRMATA_STRING, 
                ANALOG_MESSAGE, REPORT_DIGITAL, REPORT_ANALOG, SET_PIN_MODE, 
                SYSTEM_RESET, SYSEX_START, auto, int8_t, int16_t, int32_t, int64_t, 
                uint8_t, uint16_t, uint32_t, uint64_t, char16_t, char32_t, operator, 
                enum, delete, bool, boolean, byte, char, const, false, float, double, 
                null, NULL, int, long, new, private, protected, public, short, 
                signed, static, volatile, String, void, true, unsigned, word, array, 
                sizeof, dynamic_cast, typedef, const_cast, struct, static_cast, union, 
                friend, extern, class, reinterpret_cast, register, explicit, inline, 
                _Bool, complex, _Complex, _Imaginary, atomic_bool, atomic_char, 
                atomic_schar, atomic_uchar, atomic_short, atomic_ushort, atomic_int, 
                atomic_uint, atomic_long, atomic_ulong, atomic_llong, atomic_ullong, 
                virtual, PROGMEM,
                },  
% 
%
  %%% Keyword Color Group 3 %%%  (called KEYWORD1 by arduino)
  keywordstyle=[3]\bfseries\color{arduinoOrange},
  keywords=[3]{  % add built-in functions as a 3rd group
                Serial, Serial1, Serial2, Serial3, SerialUSB, Keyboard, Mouse,
                },      
%
%
  %%% Keyword Color Group 4 %%%  (called KEYWORD2 by arduino)
  keywordstyle=[4]\color{arduinoOrange},
  keywords=[4]{  % add more built-in functions as a 4th group
                abs, acos, asin, atan, atan2, ceil, constrain, cos, degrees, exp, 
                floor, log, map, max, min, radians, random, randomSeed, round, sin, 
                sq, sqrt, tan, pow, bitRead, bitWrite, bitSet, bitClear, bit, 
                highByte, lowByte, analogReference, analogRead, 
                analogReadResolution, analogWrite, analogWriteResolution, 
                attachInterrupt, detachInterrupt, digitalPinToInterrupt, delay, 
                delayMicroseconds, digitalWrite, digitalRead, interrupts, millis, 
                micros, noInterrupts, noTone, pinMode, pulseIn, pulseInLong, shiftIn, 
                shiftOut, tone, yield, Stream, begin, end, peek, read, print, 
                println, available, availableForWrite, flush, setTimeout, find, 
                findUntil, parseInt, parseFloat, readBytes, readBytesUntil, readString, 
                readStringUntil, trim, toUpperCase, toLowerCase, charAt, compareTo, 
                concat, endsWith, startsWith, equals, equalsIgnoreCase, getBytes, 
                indexOf, lastIndexOf, length, replace, setCharAt, substring, 
                toCharArray, toInt, press, release, releaseAll, accept, click, move, 
                isPressed, isAlphaNumeric, isAlpha, isAscii, isWhitespace, isControl, 
                isDigit, isGraph, isLowerCase, isPrintable, isPunct, isSpace, 
                isUpperCase, isHexadecimalDigit, 
                },      
%
%
  %%% Set Other Colors %%%
  stringstyle=\color{arduinoDarkBlue},    
  commentstyle=\color{arduinoGrey},    
%          
%   
  %%%% Line Numbering %%%%
%   numbers=left,                    
  numbersep=5pt,                   
  numberstyle=\color{arduinoGrey},    
  %stepnumber=2,                      % show every 2 line numbers
%
%
  %%%% Code Box Style %%%%
  breaklines=true,                    % wordwrapping
  tabsize=2,         
  basicstyle=\ttfamily  
}
\lstdefinestyle{myArduino}{
  language=Arduino,
  basicstyle={\small\ttfamily}, %small
%% make listing changes here %%
}

\lstdefinestyle{myArduino_x}{
%  language=Arduino_x,
  basicstyle={\small\ttfamily}, %small
%% make listing changes here %%
}




% flowchart
\usepackage{pstricks}
\usepackage{pst-node,pst-3d}
\usepackage{pst-blur}
\definecolor{Pink}{rgb}{1.,0.75,0.8}


\usepackage[LGRgreek]{mathastext} % per evitar tenir les equacions en cursiva

%\usepackage[latin1]{inputenc}
\usepackage{tikz}
\usetikzlibrary{shapes,arrows}

%-------------------------------------------------------------------------------------------------------------
%-------------------------------------------------------------------------------------------------------------
%-------------------------------------------------------------------------------------------------------------
%-------------------------------------------------------------------------------------------------------------

\begin{document}
\pagenumbering{Roman}


%\begin{titlepage}
	\begin{center}
		\vspace*{1cm}
		
		\Huge
		\textbf{Document per Projectes}
		
		\vspace{0.5cm}
		\LARGE
		Adaptat a \LaTeX
	
		\vspace{1.5cm}
		
		\textbf{Llorenç Fanals Batllori}
		
		\vfill
		
		\small
		%\uppercase{Un treball lliurat a la Universitat - en compliment dels requisits pel grau en -}\\
		% TFG
		
		\vspace{1cm}
		
		%\includegraphics[scale=width=0.4\textwidth]{images/a_graph}
	\end{center}
	
	\begin{flushright}
	\large	
	Departament o grup de recerca\\
	UdG\\
	%País\\
	28/08/2019
	\end{flushright}
	


\end{titlepage}

%\thispagestyle{plain}

\begin{center}
	\large
	\textbf{Informe}
	
	\vspace{0.4cm}
	\large
	Descripció
	
	\vspace{0.4cm}
	\textbf{Llorenç Fanals Batllori}
	
	\vspace{0.9cm}
	\textbf{Abstract}
\end{center}
\lipsum[1]




%\chapter*{Dedicacions}
%Dedico aquest treball a -

%\chapter*{Agraïments}
%Vull agraïr a \\

\cleardoublepage\pagenumbering{arabic}

\begin{spacing}{2}
\tableofcontents
\end{spacing}
%\listoffigures %No fa falta crec
%\listoftables %No fa falta crec
\begin{spacing}{1.5}




%\chapter{\uppercase{Introducció}}

\section*{1. Portada}
Breu presentació personal de mi i del títol del treball.

\section*{2. Introducció}
L'ordre escollit és lògic, primer vull exposar el dimensionament de potència, amb això quins panells i inversor escullo i finalment els aspectes que s'han tingut en compte pels càlculs de la instal·lació elèctrica. Per altra banda, per l'electrònica, primer parlar de hardware i després de la programació i els resultats que podria visualitzar el client.

\section*{3. Característiques de l'habitatge}
Citar que la casa unifamiliar està situada a Vulpellac amb latitud molt propera als 42 graus. Coneixent la latitud es pot calcular l'angle òptim d'inclinació dels panells i amb aquest la irradiació global.\\
\newline El criteri utilitzat és el d'aconseguir una generació semblant però menor al consum, anualment parlant. Modalitat d'autoconsum, connectat a la xarxa amb excedents.

\section*{4. Generador fotovoltaic i inversor}
Un cop se sap la potència dir que s'ha escollit un panell amb bones prestacions i quants se'n posaran.\\
\newline Explicar molt breument que un panell pot treballar com a receptor quan té altres panells en sèrie que generen més. Per això es decideix col·locar díodes en paral·lel amb cada fila de cel·les. Avançar que això farà disminuir la tensió als terminals del panell.\\
\newline Citar que s'escull un inversor amb 2 entrades i correcte per l'aplicació.

\section*{5. Instal·lació elèctrica}
Anomenar els factors d'escalfament, agrupament, radiació solar, el 125\% de la ITC-BT-40, el fet de què un panell solar té la intensitat limitada, que el magnetotèrmic que poso a l'entrada de l'inversor només serveix com a interruptor i no com a protecció.\\
\newline Diferenciar que per cable s'han tingut en compte tots els factors, en canvi per dimensionar els magnetotèrmics, que només fan de simple interruptor, es té en compte la intensitat de curtcircuit, que és la màxima que pot passar.\\
\newline A la sortida de l'inversor comentar que hi ha un magnetotèrmic que deixés passar la intensitat equivalent als 3.300 W i un diferencial.

\section*{6. Placa electrònica d'adquisició de dades i comunicació}
Explicar que la placa s'alimenta amb 5 V que es converteixen a 3,3 V per alguns integrats. Citar que s'han mirat intensitats i que la que ens pot donar el regulador és suficient.\\
\newline L'atenuador s'encarrega de baixar els nivells de tensions per treballar amb operacionals, que s'alimenten a 5 V, i després es fa la resta de tensions. Deixar clar que tenim 10 diferències de tensions que s'entren al multiplexor. Es programa l'ESP-12E. Hi ha adaptació correcta d'impedàncies. 

\section*{7. Programació i web}
Explicar que es programa l'ESP-12E. Remarcar que s'ha de configurar en el programa la xarxa Wi-Fi i la contrasenya d'aquesta. Comentar de forma superficial l'organigrama del programa, les condicions que hi ha i quines accions pot fer.\\
\newline Citar que és al programa on es defineix la web i que s'incorporen gràfiques.

\section*{8. Conclusió}
Concloure que amb aquest treball s'ha dimensionat una instal·lació fotovoltaica tenint en compte les particularitats d'aquests generadors. S'ha dissenyat una placa electrònica per adquirir dades i comunicar-les tan a nivell de hardware com a nivell de software. Es publiquen les dades en una pàgina web.

\section*{9. Preguntes}
Torn de preguntes. No tenir excessiva pressa al contestar. Si es dona el cas, citar que la resposta es troba a la memòria o a algun altre document. Mostrar confiança i argumentar les respostes.





















\clearpage




\chapter{\uppercase{Instal·lació fotovoltaica}}


Com que es coneix la localització de la casa és directe saber la latitud i es pot calcular l'angle òptim d'inclinació dels panells solars, que és de 32,66$^{\circ}$. S'instal·len els panells cara sud per maximitzar la generació anual d'energia. Amb aquestes dades és possible calcular la irradiació anual per metre quadrat, que és de 1.569,92 kWh/$m^2$. \\
\newline Es disposa del consum elèctric anual a la casa. Es dimensiona la instal·lació fotovoltaica per donar un valor semblant d'energia elèctrica anual a la consumida, sense superar-la, però. Es determina que un bon dimensionament consisteix en instal·lar 10 panells del 17\% d'eficiència i d'un màxim de 330 W de potència.\\
\newline Es realitzen càlculs d'ombres tal i com marca l'IDAE i es determina que l'energia generada al cap de l'any seria de 4.768 kWh. L'energia consumida anualment és d'uns 5.504 kWh. D'aquesta manera es preveu que es compensi la majoria del consum, ja que la casa està connectada a la xarxa i amb la modalitat d'autoconsum amb excedents acollits a compensació.\\
\newline S'ha estudiat un model generalitzat de panell solar, el qual facilita entendre quan és que un panell fotovoltaic pot actuar de receptor d'energia enlloc de generador. Per evitar això es decideix instal·lar díodes amb una caiguda de tensió de 26 mV en paral·lel amb cada fila de cel·les.\\
\newline El fabricant dels panells indica alguns paràmetres màxims com la tensió de circuit obert i la intensitat de curtcircuit, gràcies als quals es pot dimensionar l'inversor. S'ha escollit un Fronius Primo 3.0-1. Com que l'inversor té dues entrades s'ha optat per connectar 5 panells en sèrie a cada una de les entrades.\\
\newline Les dades dels panells també han estat utilitzades per dimensionar la instal·lació elèctrica. Pels cables s'han tingut en compte diversos factors com són l'indicat a la ITC-BT-40, un factor de temperatura, un factor d'agrupament i un factor de radiació solar. Per dimensionar els interruptors magnetotèrmics, alguns dels quals només s'utilitzen per obrir el circuit i no com a protecció, s'ha utilitzat la intensitat de curtcircuit de les plaques i la de sortida de l'inversor.




\clearpage



\chapter{\uppercase{Placa electrònica d'adquisició de dades i comunicació}}

L'inversor es pot connectar a Internet de manera que el client pot conèixer la generació d'energia, però no pot conèixer com està funcionant cada placa per separat. Per això es decideix dissenyar una placa electrònica\\
\newline És la placa electrònica l'encarregada d'adquirir la tensió de cada panell i així complementar la informació que dona l'inversor. Si alguna fila de cel·les intenta passar a treballar com a receptora d'energia el díode en paral·lel actua i això es tradueix en una disminució de la tensió als terminals del panell.\\
\newline Aquest equip electrònic dona la informació al client que li ha de permetre no només conèixer si alguna placa s'ombreja més sovint de l'esperat, sinó també saber si s'ha malmès alguna cel·la.\\
\newline La placa treballa amb tensions d'alimentació de 5 V a i 3,3 V. Un convertidor de tensió anomenat SPX3819M5-L-3-3 juntament amb condensadors a l'entrada i a la sortida assegura una bona reducció de la tensió. Alguns components com l'ESP-12E s'alimenten a 3,3 V ja que és el nivell recomanat pel fabricant. Altres, com els amplificadors operacionals s'alimenten a 5 V degut a què aquests no són Rail-to-Rail.\\
\newline El circuit d'instrumentació s'encarrega de reduir les tensions a nivells amb què es pugui treballar amb els amplificadors operacionals. Aquests fan la resta entre les tensions als terminals de cada placa. El resultat de la resta s'adequa al nivell de tensió permès de l'entrada del convertidor ADC del component ESP-12E.\\
\newline Per tal de poder llegir les diferències de tensions de les 10 plaques, després de passar les senyals pel circuit d'instrumentació es decideix multiplexar les senyals amb un CD74HC4067M. Les 4 entrades d'aquest component van connectades a 4 pins de propòsit general de l'ESP-12E. El multiplexor té baixa impedància entre l'entrada seleccionada i la sortida, tot i això es fa servir un seguidor de tensió per aïllar l'etapa.\\
\newline S'ha dissenyat una placa de circuit imprès de 8,4 cm x 10 cm aproximadament. La placa té dues capes i la majoria de components són SMD.



\clearpage



\chapter{\uppercase{Programació i pàgina web}}

La placa es programa per tal d'adquirir dades de tensió actuant sobre el multiplexor cada hora. Es guarden les dades en memòria i es mostra la informació a una pàgina web cada vegada que un client s'hi connecta.\\
\newline Després d'iniciar el programa la placa intenta connectar-se a la xarxa Wi-Fi que se li ha dit. En el programa s'indica el nom de la xarxa Wi-Fi i la contrasenya.\\
\newline Un cop s'hi connecta porta un comptatge de temps per tal de determinar si cal fer una mesura o no. En cas que faci falta fer-ne es llegiran les 10 senyals analògiques disponibles, es faran els càlculs per traduir les lectures a tensió i s'enregistraran.\\
\newline Contínuament s'analitza si un client s'ha connectat a l'adreça IP de la placa. En cas afirmatiu se li mostrarà la web amb les dades més recents.\\
\newline La web consta d'un títol, un subtítol, dues gràfiques que mostren les dades de les últimes 24 hores i una imatge per identificar on està situat cada panell.



\clearpage




\chapter{\uppercase{Conclusió}}
El present document té l'objectiu de legalitzar la instal·lació contra incendis, els materials i les evacuacions d'emergència d'un obrador de plats cuinats.\\
\newline
Per desenvolupar aquesta memòria s'ha seguit el Reial decret 2267/2004 del 3 de desembre, el qual aprova el Reglament de seguretat contra incendis en establiments industrials. Amb aquest reglament es pot dir que la càrrega de foc és baixa i els materials són correctes.\\
\newline Per validar la instal·lació contra incendis s'ha seguit el Reglament d'instal·lacions de protecció contra incendis aprovat pel Reial decret 513/2017 del 22 de maig. Gràcies a aquest document s'ha pogut avaluar la instal·lació contra incendis com a correcta.\\
\newline El REBT s'ha consultat per verificar que els nivells d'il·luminació d'emergència mesurats són correctes. La Norma Básica de l'Edificació, NBE-CPI/96, ha permès considerar correctes les dimensions de les portes d'emergència i els passadissos.\\
%
\newline Amb tot l'indicat en aquesta memòria es considera que la nau té un risc baix d'incendi i que la instal·lació contra incendis, els materials i els recorreguts d'evacuació són els adequats segons la legislació present fins a dia d'avui. 

\vspace*{\fill}
\noindent Llorenç Fanals Batllori\\
Graduat en Enginyeria Electrònica Industrial i Automàtica\\
\\
\\
Girona, 26 d'octubre de 2019.

\clearpage


\appendix

%\chapter{\uppercase{Càlculs}}
%\section{Càlculs línies elèctriques}
Per tal de calcular les seccions mínimes de les línies es té en compte la caiguda de tensió i la intensitat màxima admissible dels cables. El REBT indica a la ITC-BT-40 que els cables de connexió hauran d'estar dimensionats per una intensitat no inferior al 125\% de la intensitat màxima del generador. A més, la caiguda de tensió màxima permesa entre el generador i el punt d'interconnexió amb la xarxa és de 1,5\% respecte la tensió nominal.\\
\newline Recordar que les línies de connexió entre els panells solars, amb l'inversor i a la placa electrònica es té una senyal elèctrica de tipus continu. La línia que connecta la sortida de l'inversor amb la instal·lació interior és de senyal alterna, i es considera que amb un factor de potència unitari. El fabricant indica que el factor de potència de l'inversor escollit és sempre molt proper a la unitat.\\
\newline Per calcular la intensitat de les línies monofàsiques es fa servir l'Equació \ref{eq:il}:
\begin{equation} \label{eq:il}
I_{linia} = \frac{P}{V*\cos(\phi)}
\end{equation}
V = 230 V.\\
P: potència que consumeixen els elements connectats a la línia (W).\\
$\phi$: factor de potència.\\
\newline En monofàsic cal fer servir l'Equació \ref{eq:e}:
\begin{equation}\label{eq:e}
e(\%)=\frac{P}{V}\frac{2*l}{k*S}
\end{equation}
l: longitud ja sigui de la fase o el neutre des del comptador a l'element més llunyà (m).\\
$k = 56 m/mm^{2}\si{\ohm}$.\\
S:secció del cable (m$m^2$).\\
%
\newline El dimensionament de les línies ha de permetre que les caigudes de tensió no superin els màxims indicats prèviament. Alhora, els cables han de poder admetre les intensitats calculades, per això ens guiem amb la taula de la ITC-19 del REBT. Finalment, cal comprovar que  l'interruptor magnetotèrmic té una intensitat nominal superior a la calculada per la línia i menor a l'admissible que marca la ITC-19.\\
\newline S'ha de tenir en compte el factor de correcció per radiació directa del Sol ($F_{sol}$), tal i com es comenta a la ITC-BT-06. S'escull un factor per radiació directa del Sol de 0,9.\\
\newline També s'ha de tenir en compte el factor de correcció per agrupament de cables ($F_{grup}$) de 0,89 per parelles de cables i de 0,75 per l'agrupament de cables que va a la placa electrònica, tal com marca la ITC-BT-06 que tracta sobre instal·lacions aèries.\\
\newline Finalment, el tercer factor és el factor de correcció per temperatura ($F_{temp}$), que s'escull de 0,9, que és el factor que s'ha d'agafar segons la ITC-BT-06 per temperatures de 50 $^\circ$C.\\
%
%
%
%
\newline Per totes les línies menys per la de connexionat la placa electrònica es té en compte un factor de 1,25, tal com marca la ITC-BT-19 per generadors. La caiguda de tensió no pot ser major de l'1,5\% respecte la tensió nominal.\\
\newline Es decideix instal·lar cables de recobriment de PVC sobre paret, muntatge C6.\\
%
\newline Amb aquests factors coneguts es pot calcular la intensitat de càlcul de les diferents línies, a partir de la qual es determinen les seccions.\\
\newline La intensitat de curtcircuit augmenta amb la temperatura un 0,055\% per grau centígrad. Es calcula a 70 C$^{\circ}$ la intensitat de curtcircuit en condicions de 1.000 W/$m^2$. La que facilita el fabricant és de 9,33 A a 25 C$^{\circ}$.\\
\newline Per conèixer la intensitat màxima de curtcircuit, en el cas més desfavorable, cal seguir l'Equació \ref{eq:isc}:
\begin{equation} \label{eq:isc}
I_{sc}(T=70\  C ^{\circ})= I_{sc} (1 + \alpha (T-25 \ C^{\circ}))
\end{equation}

\noindent $I_{sc}$: intensitat de curtcircuit del panell a 25 C$^{\circ}$ (A).\\
$\alpha$: factor lineal d'increment d'intensitat de curtcircuit per efecte de la temperatura (\%).\\
T: temperatura, en aquest cas 70 C$^{\circ}$.\\
\noindent El resultat és de 9,56 A.\\
%
\newline Per calcular la tensió màxima de circuit obert s'ha fet servir l'Equació \ref{eq:isc}:
\begin{equation} \label{eq:isc}
V_{oc}(T=-10\  C ^{\circ})= V_{oc} (1 + \beta (T-25 \ C^{\circ}))
\end{equation}
\noindent  $V_{oc}$: Tensió de circuit obert a 25 C$^{\circ}$ (V).\\
$\beta$: factor lineal de decrement de tensió de circuit obert per efecte de la temperatura (\%).\\
%
\newline La tensió de circuit obert a 25 C$^{\circ}$ és de 46,2 V. El factor lineal que indica el fabricant és de -0,32\%/C$^{\circ}$, que dona un total de 51,37 V. S'han tingut en compte pel dimensionament de la placa electrònica.\\
\newline Dit això, per calcular les seccions dels cables cal destacar que el producte de la intensitat de curtcircuit calculada per la temperatura més desfavorable multiplicada per 1,25 ha de ser igual a la secció del REBT multiplicada pels factors de radiació solar, agrupament i temperatura.\\
\newline Si s'aïlla, es pot dir que la intensitat de curtcircuit calculada multiplicada per 1,25 i dividida pels factors mencionats ha de ser admesa per les seccions dels cables del REBT.\\
\newline A la Taula \ref{tab:taulax} es detalla la intensitat de cada línia i els factors que cal tenir en compte.

\begin{table}[H]
\scriptsize
  \centering
    \begin{tabu} to \textwidth  {|X[0.3, l]|X[2, l]|X[0.5, r]|X[0.6, r]|X[0.5, r]|X[0.5, r]|X[0.5, r]|} \hline
Línia &  Descripció & Intensitat (A) & Factor generador & Factor radiació solar & Factor agrupament  & Factor temperatura\\ \hline \hline
L1 & Connexionat entre els panells solars & 9,56 & 1,25 & 0,9 & 0,89 & 0,9 \\ \hline
L2 & Connexionat de la branca 1 a l'inversor & 9,56 & 1,25 & 0,9 & 0,89 & 0,9\\ \hline
L3 & Connexionat de la branca 2 a l'inversor & 9,56 & 1,25 & 0,9 & 0,89 & 0,9\\ \hline
L4 & Connexionat de l'inversor al QGPC & 14,3 & 1,25 & 1,0 & 1,00 & 1,0 \\ \hline \hline
L5 & Línies de connexionat dels panells fotovoltaics a la placa electrònica & 0,0023 & 1,00 & 0,9 & 0,75 & 0,9 \\ \hline
	
    \end{tabu}%
  \caption{Intensitat de càlcul pel dimensionament de les línies}
    \label{tab:taulax}%
 \end{table}%

\noindent Amb aquesta intensitat de càlcul podem dimensionar les línies. Es calculen les seccions per tal d'evitar tenir un percentatge de caiguda de tensió entre els generadors i el punt de connexió a la xarxa superior de 1,5\%, tal com marca la ITC-BT-40 de generadors.\\
\newline A la Taula \ref{tab:t2} es mostren dades de les diverses línies detallades.\\
\newline L'inversor sempre intenta donar els 230 V a la seva sortida, sempre que el valor de l'entrada superi els 80 V mínims que marca el fabricant. S'ha acumulat la caiguda de tensió de la línia 1 amb la línia 2 i la 3; i la més desfavorable entre la 2 i la 3 amb la 4. La ITC-BT-19 verifica que per intensitats admissibles les seccions són correctes.
%
\begin{table}[H]
\scriptsize
\begin{center}
 \begin{tabu} to 0.985\textwidth {|X[0.5, l]|X[1.5, l]|X[0.8, r]|X[0.6, r]|X[r]|X[r]|X[r]|X[r]|X[r]|X[r]|X[0.5,r]|}%{X | c c c} 
 \hline
 Línia& Descripció & Potència (W) & cos($\phi$) & Intensitat nominal (A) & Distància màxima (m) & Seccions ($mm^{2}$) & Diàmetre tub (mm) & Caiguda de tensió (\%) & Caiguda de tensió acum. (\%)\\
 \hline \hline 

L1 & Connexionat entre els panells solars & 1.650 & 1 & 9,56 & 10 & 2x4 & 20 & 0,42 & 0,42 \\ \hline
L2 & Connexionat de la branca 1 a l'inversor & 1.650  & 1 & 9,56  & 30 & 2x10 & 20 & 0,50 & 0,92 \\ \hline 
L3 & Connexionat de la branca 2 a l'inversor & 1.650  & 1 & 9,56  & 19 & 2x6 & 25 & 0,53 & 0,95 \\ \hline 
L4 & Connexionat de l'inversor al QGPC & 3.300  & 1 & 14,35 & 6 & 2x4 + 4 & 25 & 0,38 & 1,33 \\ \hline \hline
L5 & Línies de connexionat dels panells fotovoltaics a la placa electrònica & 0,529 & 1 & 0,0038& 21 & 2x1,5 & 32 & 0,000575 & 0,000575 \\ \hline 


 \end{tabu}
 \caption{Línies detallades}
 \label{tab:t2}%
\end{center}
\end{table}

 
 
%
%




%\section{Càlculs placa electrònica}
% o hauria de dir instrumentació?



\clearpage


% Table generated by Excel2LaTeX from sheet 'Hoja1'
%\begin{table}[H]
%  \centering
%    \begin{tabularx} {\textwidth} {|X|r|} \hline
%  \multicolumn{1}{|c|}{Descripció} &  \multicolumn{1}{c|}{Quantitat}\\ \hline \hline
%
 %   Placa GLC 330 W & 10 \\ \hline
%    Inversor FRONIUS Primo 3.0-1 Light 3kW & 1 \\ \hline
%    Metres cable Ethernet RJ-45 CAT 8 & 10 \\ \hline
%    Metres cable 4 m$m^2$ PVC & 45 \\ \hline
 %   Metres cable 1,5 m$m^2$ PVC & 100 \\ \hline
 %   Punteres Enghofer E 4-10, 4 m$m^2$, 10 mm & 20 \\ \hline
 %   Punteres Enghofer E 1.5-10 1,5 m$m^2$ 10 mm & 12 \\ \hline
 %   Cinta aïllant 10 m 1,6 cm & 3 \\ \hline
 %   Caixa estanca Solera CONS 100x100x55 mm & 2 \\ \hline
  %  Canal Euroquint 25,16 mm 1,5 metres & 20 \\ \hline
%    Curva canal VECAMCO & 10 \\ \hline
%    Paquet de 50 brides 200x2,6  mm & 2 \\ \hline
%    Regleta nylon 12 pols 16 mm & 4 \\ \hline
%    Premsaestopes M12 & 10 \\ \hline
%    Cargol autoroscant M4 16 mm & 12 \\ \hline
%    Tacs Fischer 072095 nylon 6x50 mm & 50 \\ \hline
%    Díode SM74611KTTR & 10 \\ \hline
%            Hores enginyer & 1 \\ \hline
%    Hores oficial de primera & 12 \\ \hline
%    Hores oficial de segona & 12 \\ \hline
%    \end{tabularx}%
%  \label{tab:addlabel}%
% \end{table}%


\end{spacing}
\begin{spacing}{1}

%\chapter{\uppercase{Programa}}
\begin{lstlisting}[style=myArduino]
/*********
Llorenç Fanals Batllori
Graduat en Enginyeria Electrònica Industrial i Automàtica
20/11/2019
*********/

#include <ESP8266WiFi.h> // Es carrega la llibreria Wi-Fi

// Credencials de la xarxa Wi-Fi a què ens volem connectar
const char* ssid     = "DESKTOP-E5M4HBA 4049";
const char* password = "E^1w1736";

// Port que volem utilitzar. El 80 és el port per defecte, així que teclejant la IP a un navegador en farem prou. Si fos un altre port la IP acabaria en ":número_port".
WiFiServer server(80);


unsigned long TempsActual = millis(); // Current time
unsigned long TempsAnterior = 0; // Previous time
const long TempsConnectat = 20000; // Define timeout time in milliseconds (example: 2000ms = 2s)


#define files 24
#define columnes 5

float vector[files][columnes]; // vector de dades
float vector2[files][columnes]; // vector de dades
int i = 0; // iterador per files
int j = 0; // iterador per columnes

#define D0 16
#define D1 5
#define D2 4
#define D3 12 // 0

#define ENTRADA_ANALOGICA A0

unsigned int hores_posada_marxa = 10; // l'hora en què es fa la posada en marxa
unsigned int minuts_posada_marxa = 23; // a les 10:23 es fa la posada en marxa

unsigned int hora_actual;
unsigned int minuts_actual;
unsigned int millis_anteriors;

void inicialitza_vectors(){ // Emplena els vectors de dades fictícies. A còpia d'hores s'aniran reemplaçant per dades reals
  
  vector[0][0]=0; vector[0][1]=0; vector[0][2]=0; vector[0][3]=0; vector[0][4]=0;
  vector[1][0]=0; vector[1][1]=0; vector[1][2]=0; vector[1][3]=0; vector[1][4]=0;
  vector[2][0]=0; vector[2][1]=0; vector[2][2]=0; vector[2][3]=0; vector[2][4]=0;
  vector[3][0]=0; vector[3][1]=0; vector[3][2]=0; vector[3][3]=0; vector[3][4]=0;
  vector[4][0]=0; vector[4][1]=0; vector[4][2]=0; vector[4][3]=0; vector[4][4]=0;
  vector[5][0]=0; vector[5][1]=0; vector[5][2]=0; vector[5][3]=0; vector[5][4]=0;
  vector[6][0]=0; vector[6][1]=0; vector[6][2]=0; vector[6][3]=0; vector[6][4]=0;
  vector[7][0]=3; vector[7][1]=10; vector[7][2]=14.3; vector[7][3]=17.2; vector[7][4]=21.3;
  vector[8][0]=29.3; vector[8][1]=31.2; vector[8][2]=32.1; vector[8][3]=29.4; vector[8][4]=25.5;
  vector[9][0]=31.6; vector[9][1]=31.5; vector[9][2]=27.5; vector[9][3]=27.2; vector[9][4]=22.2;
  vector[10][0]=32.0; vector[10][1]=31.6; vector[10][2]=28; vector[10][3]=22.7; vector[10][4]=26.5;
  vector[11][0]=28.1; vector[11][1]=27.1; vector[11][2]=31.7; vector[11][3]=27.9; vector[11][4]=32.5; // central, pic
  vector[12][0]=26.4; vector[12][1]=21.5; vector[12][2]=29.6; vector[12][3]=31.1; vector[12][4]=31.6;
  vector[13][0]=25.2; vector[13][1]=30.5; vector[13][2]=29.1; vector[13][3]=30.4; vector[13][4]=30.3;
  vector[14][0]=29.3; vector[14][1]=29.4; vector[14][2]=28.6; vector[14][3]=29.5; vector[14][4]=29.4;
  vector[15][0]=15.6; vector[15][1]=15.3; vector[15][2]=14.2; vector[15][3]=18.3; vector[15][4]=21.2;
  vector[16][0]=0; vector[16][1]=0; vector[16][2]=0; vector[16][3]=0; vector[16][4]=0;
  vector[17][0]=0; vector[17][1]=0; vector[17][2]=0; vector[17][3]=0; vector[17][4]=0;
  vector[18][0]=0; vector[18][1]=0; vector[18][2]=0; vector[18][3]=0; vector[18][4]=0;
  vector[19][0]=0; vector[19][1]=0; vector[19][2]=0; vector[19][3]=0; vector[19][4]=0;
  vector[20][0]=0; vector[20][1]=0; vector[20][2]=0; vector[20][3]=0; vector[20][4]=0;
  vector[21][0]=0; vector[21][1]=0; vector[21][2]=0; vector[21][3]=0; vector[21][4]=0;
  vector[22][0]=0; vector[22][1]=0; vector[22][2]=0; vector[22][3]=0; vector[22][4]=0;
  vector[23][0]=0; vector[23][1]=0; vector[23][2]=0; vector[23][3]=0; vector[23][4]=0;

  vector2[0][0]=0; vector2[0][1]=0; vector2[0][2]=0; vector2[0][3]=0; vector2[0][4]=0;
  vector2[1][0]=0; vector2[1][1]=0; vector2[1][2]=0; vector2[1][3]=0; vector2[1][4]=0;
  vector2[2][0]=0; vector2[2][1]=0; vector2[2][2]=0; vector2[2][3]=0; vector2[2][4]=0;
  vector2[3][0]=0; vector2[3][1]=0; vector2[3][2]=0; vector2[3][3]=0; vector2[3][4]=0;
  vector2[4][0]=0; vector2[4][1]=0; vector2[4][2]=0; vector2[4][3]=0; vector2[4][4]=0;
  vector2[5][0]=0; vector2[5][1]=0; vector2[5][2]=0; vector2[5][3]=0; vector2[5][4]=0;
  vector2[6][0]=0; vector2[6][1]=0; vector2[6][2]=0; vector2[6][3]=0; vector2[6][4]=0;
  vector2[7][0]=12; vector2[7][1]=10; vector2[7][2]=14.3; vector2[7][3]=17.2; vector2[7][4]=21.3;
  vector2[8][0]=29.3; vector2[8][1]=31.2; vector2[8][2]=32.1; vector2[8][3]=29.4; vector2[8][4]=32.5;
  vector2[9][0]=31.6; vector2[9][1]=31.5; vector2[9][2]=27.5; vector2[9][3]=27.2; vector2[9][4]=32.2;
  vector2[10][0]=32.0; vector2[10][1]=31.6; vector2[10][2]=28; vector2[10][3]=28.7; vector2[10][4]=31.5;
  vector2[11][0]=31.1; vector2[11][1]=27.1; vector2[11][2]=26.7; vector2[11][3]=27.9; vector2[11][4]=31.5; // central, pic
  vector2[12][0]=31.4; vector2[12][1]=26.5; vector2[12][2]=29.6; vector2[12][3]=26.1; vector2[12][4]=31.6;
  vector2[13][0]=30.2; vector2[13][1]=30.5; vector2[13][2]=29.1; vector2[13][3]=21.4; vector2[13][4]=25.3;
  vector2[14][0]=29.3; vector2[14][1]=29.4; vector2[14][2]=28.6; vector2[14][3]=29.5; vector2[14][4]=21.4;
  vector2[15][0]=15.6; vector2[15][1]=15.3; vector2[15][2]=14.2; vector2[15][3]=18.3; vector2[15][4]=15.2;
  vector2[16][0]=0; vector2[16][1]=0; vector2[16][2]=0; vector2[16][3]=0; vector2[16][4]=0;
  vector2[17][0]=0; vector2[17][1]=0; vector2[17][2]=0; vector2[17][3]=0; vector2[17][4]=0;
  vector2[18][0]=0; vector2[18][1]=0; vector2[18][2]=0; vector2[18][3]=0; vector2[18][4]=0;
  vector2[19][0]=0; vector2[19][1]=0; vector2[19][2]=0; vector2[19][3]=0; vector2[19][4]=0;
  vector2[20][0]=0; vector2[20][1]=0; vector2[20][2]=0; vector2[20][3]=0; vector2[20][4]=0;
  vector2[21][0]=0; vector2[21][1]=0; vector2[21][2]=0; vector2[21][3]=0; vector2[21][4]=0;
  vector2[22][0]=0; vector2[22][1]=0; vector2[22][2]=0; vector2[22][3]=0; vector2[22][4]=0;
  vector2[23][0]=0; vector2[23][1]=0; vector2[23][2]=0; vector2[23][3]=0; vector2[23][4]=0;
}


void setup() {
  hora_actual = hores_posada_marxa;
  minuts_actual = minuts_posada_marxa;

  // Configurem els pins digital com a sortides per actuar sobre el multiplexor
  pinMode(D0, OUTPUT);
  pinMode(D1, OUTPUT);
  pinMode(D2, OUTPUT);
  pinMode(D3, OUTPUT);

  // Dades temporals dels vectors. Serveixen per mostrar com queden representades les gràfiques. S'aniran borrant les dades més antigues.
  inicialitza_vectors();
  
  Serial.begin(115200); // Habilitem el port sèrie a 115200 de baud rate 

  // Ens connectem al Wi-Fi amb l'adreça i la contrasenya definits
  Serial.print("Connectant a: ");
  Serial.println(ssid); // Mostrem l'adreça del Wi-Fi
  WiFi.begin(ssid, password); // Iniciem la comunicació
  
  while (WiFi.status() != WL_CONNECTED) {
    delay(500);
    Serial.print("."); // Cada 0,5 s que passin sense connectar-se mostra un punt
  }
  
  // S'ha connectat
  Serial.println("");
  Serial.println("WiFi connectat");
  Serial.println("Adreça IP: ");
  Serial.println(WiFi.localIP());
  server.begin();
  
}


void loop(){
  WiFiClient client = server.available();   // Escolta si hi ha clients

  if (client) {                             // Si es connecta un nou client,
    Serial.println("Nou client.");          
    String LiniaActual = "";                // una cadena memoritza la informació enviada pel client
    TempsActual = millis();
    TempsAnterior = TempsActual;
    while (client.connected() && TempsActual - TempsAnterior <= TempsConnectat) { // Si estem connectats i no han passat els milisegons que indica TempsConnectat,
      TempsActual = millis();         
      if (client.available()) {             // Si el client ens passa informació,
        char c = client.read();             // llegim un caràcters ascii (un byte)
        Serial.write(c);                    // i el mostrem per pantalla
        if (c == '\n') {                    // Si rebem un canvi de línia com a caràcter,
          // és el final de la petició HTTP
          if (LiniaActual.length() == 0) {
            // Ara responem donant un OK i indicant el content type, volem una pàgina html. Finalment una línia en blanc, és el protocol
            client.println("HTTP/1.1 200 OK");
            client.println("Content-type:text/html");
            client.println("Tancant connexió");
            client.println();
      
            // Al navegador volem veure una web normal i corrent que es programa amb etiquetes HTML, alguna classe CSS i serveis JavaScript
            
            client.println("<!DOCTYPE html><html>");
            client.println("<head><meta name=\"viewport\" content=\"width=device-width, initial-scale=1\">");
            client.println("<link rel=\"icon\" href=\"data:,\">");

            // Definim la gràfica de la primera branca
            client.println("<script type=\"text/javascript\" src=\"https://www.gstatic.com/charts/loader.js\"></script>\n    <script type=\"text/javascript\">\n      google.charts.load('current', {'packages':['line']});\n      google.charts.setOnLoadCallback(drawChart);\n\n    function drawChart() {\n\n      var data = new google.visualization.DataTable();\n      data.addColumn('number', 'Hora');\n      data.addColumn('number', 'Panell 1.1');\n      data.addColumn('number', 'Panell 1.2');\n      data.addColumn('number', 'Panell 1.3');\n      data.addColumn('number', 'Panell 1.4');\n      data.addColumn('number', 'Panell 1.5');\n\n");
            client.println("data.addRows([\n");
            for (i=0; i<files; i++){
                client.println("[");
                client.println(String(i+1)); 
                client.println(",");
                client.println(String(vector[i][0]));
                for (j=1; j<columnes; j++){
                  client.println(","); client.println(String(vector[i][j]));
                }  
                client.println("]"); client.println(","); client.println("\n");
            }
            client.println("]);\n\n\n      var options = {\n        chart: {\n          title: 'Tensions a la branca 1 (V)',\n          // subtitle: 'in millions of dollars (USD)'\n        },\n     //   width: 900,\n     //   height: 500\n      };\n\n      var chart = new google.charts.Line(document.getElementById('linechart_material'
            ));\n\n      chart.draw(data, google.charts.Line.convertOptions(options));\n    }\n    </script>\n");

            // Definim la gràfica de la segona branca
            client.println("    <script type=\"text/javascript\" src=\"https://www.gstatic.com/charts/loader.js\"></script>\n    <script type=\"text/javascript\">\n    google.charts.load('current', {'packages':['line']});\n    google.charts.setOnLoadCallback(drawChart);\n    \n\n    function drawChart() {\n\n    var data = new google.visualization.DataTable();\n    data.addColumn('number', 'Hora');\n    data.addColumn('number', 'Panell 2.1');\n      data.addColumn('number', 'Panell 2.2');\n      data.addColumn('number', 'Panell 2.3');\n      data.addColumn('number', 'Panell 2.4');\n      data.addColumn('number', 'Panell 2.5');\n\n");
            client.println("data.addRows([\n");
            for (i=0; i<files; i++){
                client.println("[");
                client.println(String(i+1)); 
                client.println(",");
                client.println(String(vector2[i][0]));
                for (j=1; j<columnes; j++){
                  client.println(","); client.println(String(vector2[i][j]));
                }  
                client.println("]"); client.println(","); client.println("\n");
            }
            client.println(" ]); \n\n\n\n    var options = {\n        chart: {\n        title: 'Tensions a la branca 2 (V)',\n       // is3D: true\n        // subtitle: 'in millions of dollars (USD)'\n        },\n     //   width: 700,\n     //   height: 400\n    };\n\n    var chart = new google.charts.Line(document.getElementById('linechart_material2'
            ));\n\n    chart.draw(data, google.charts.Line.convertOptions(options));\n    }\n    </script>");

           
            // Definim els títols de la pàgina, el que en HTML es coneix com a headings. Alguns caràcters en català no són ben representats, cal corregir-ho
            client.println("<body><h1 align=\"left\">Instal&middotlaci&oacute fotovoltaica sensoritzada per habitatge unifamiliar</h1>"); // &middot =  , &oacute = ó
            client.println("<h2 align=\"left\">Lloren&ccedil Fanals Batllori</h2>"); // &ccedil = ç

            client.println("<div id=\"linechart_material\" style=\"width: 800px; height: 400px; padding: 25px\"></div>  \n"); // Inserim les gràfiques del primer grup de plaques
            client.println("<div id=\"linechart_material2\" style=\"width: 800px; height: 400px; padding: 25px\"></div> "); // Inserim les gràfiques del segon grup de plaques
          
            client.println("<img src=\"https://drive.google.com/uc?export=view&id=15-EkLWMhYaR
            sv-dtbyrlKOrbD7dY71B2\"\n   align=\"left\"      style=\"width: 700px; height: 700px;  padding: 25px\" alt=\"Croquis de les plaques a la teulada\">"); // Inserim imatge del nom de cada panell
 
            client.println("</body></html>"); // Pàgina finalitzada
            
            client.println(); // Línia en blanc per finalitzar la comunicació
            
            break; // Sortim del while()
          } 
          
          else { // si tens una nova línia, neteja LiniaActual
            LiniaActual = "";
          }
          
        } 
       
        else if (c != '\r') {  // Si tens algun caràcter afegiex-lo al final de LiniaActual
          LiniaActual += c;   
        }
        
      }
    }

    // Tanquem la connexió, esperant un nou client o que el client existent refresqui la pàgina
    client.stop();
    Serial.println("Client desconnectat.");
    Serial.println("");
  }

    // Mirem si cal actualitzar els minuts i les hores i si cal fer una lectura de tensions
    comprova_temps();

  //
//  Serial.println(analogRead(ENTRADA_ANALOGICA));
//  Serial.println(millis());
}




// Encapcem amb funcions

void comprova_temps(){
    if ((millis() - millis_anteriors) >= 60000){ // ha passat un minut
        minuts_actual++;
        millis_anteriors = millis(); // memoritzem el moment en què això ha passat
        if (minuts_actual >= 60){ // si portem 60 minuts, diem que en portem 0 i incrementem l'hora
            minuts_actual = 0;
            hora_actual++;
            lectura_tensions(); // cridem la funció que llegeix les tensions
        }
        if (hora_actual >= 24){ // si l'hora és 24, la passem a 0
            hora_actual = 0;  
        }
    }
}

void lectura_tensions(){
    float tensio_superior = 0;
    float tensio_inferior = 0;
    float guany_bit_tensio = 0.0476288*3.3; // relació entre volts i bits llegits
    int memoria_ms = 0;
    int ms_delay = 20;
    
    digitalWrite(D3, LOW); digitalWrite(D2, LOW); digitalWrite(D1, LOW); digitalWrite(D0, LOW);
    memoria_ms = millis();
    while ((millis() - memoria_ms) < ms_delay){}
    vector[hora_actual][0] = analogRead(ENTRADA_ANALOGICA) * guany_bit_tensio;

    digitalWrite(D3, LOW); digitalWrite(D2, LOW); digitalWrite(D1, LOW); digitalWrite(D0, HIGH);
    memoria_ms = millis();
    while ((millis() - memoria_ms) < ms_delay){}
    vector[hora_actual][1] = analogRead(ENTRADA_ANALOGICA) * guany_bit_tensio;

    digitalWrite(D3, LOW); digitalWrite(D2, LOW); digitalWrite(D1, HIGH); digitalWrite(D0, LOW);
    memoria_ms = millis();
    while ((millis() - memoria_ms) < ms_delay){}
    vector[hora_actual][2] = analogRead(ENTRADA_ANALOGICA) * guany_bit_tensio;

    digitalWrite(D3, LOW); digitalWrite(D2, LOW); digitalWrite(D1, HIGH); digitalWrite(D0, HIGH);
    memoria_ms = millis();
    while ((millis() - memoria_ms) < ms_delay){}
    vector[hora_actual][3] = analogRead(ENTRADA_ANALOGICA) * guany_bit_tensio;

    digitalWrite(D3, LOW); digitalWrite(D2, HIGH); digitalWrite(D1, LOW); digitalWrite(D0, LOW);
    memoria_ms = millis();
    while ((millis() - memoria_ms) < ms_delay){}
    vector[hora_actual][4] = analogRead(ENTRADA_ANALOGICA) * guany_bit_tensio;

// Ara la mateixa idea però pel vector 2

    digitalWrite(D3, LOW); digitalWrite(D2, HIGH); digitalWrite(D1, LOW); digitalWrite(D0, HIGH);
    memoria_ms = millis();
    while ((millis() - memoria_ms) < ms_delay){}
    vector2[hora_actual][0] = analogRead(ENTRADA_ANALOGICA) * guany_bit_tensio;

    digitalWrite(D3, LOW); digitalWrite(D2, HIGH); digitalWrite(D1, HIGH); digitalWrite(D0, LOW);
    memoria_ms = millis();
    while ((millis() - memoria_ms) < ms_delay){}
    vector2[hora_actual][1] = analogRead(ENTRADA_ANALOGICA) * guany_bit_tensio;

    digitalWrite(D3, LOW); digitalWrite(D2, HIGH); digitalWrite(D1, HIGH); digitalWrite(D0, HIGH);
    memoria_ms = millis();
    while ((millis() - memoria_ms) < ms_delay){}
    vector2[hora_actual][2] = analogRead(ENTRADA_ANALOGICA) * guany_bit_tensio;

    digitalWrite(D3, HIGH); digitalWrite(D2, LOW); digitalWrite(D1, LOW); digitalWrite(D0, LOW);
    memoria_ms = millis();
    while ((millis() - memoria_ms) < ms_delay){}
    vector2[hora_actual][3] = analogRead(ENTRADA_ANALOGICA) * guany_bit_tensio;

    digitalWrite(D3, HIGH); digitalWrite(D2, LOW); digitalWrite(D1, LOW); digitalWrite(D0, HIGH);
    memoria_ms = millis();
    while ((millis() - memoria_ms) < ms_delay){}
    vector2[hora_actual][4] = analogRead(ENTRADA_ANALOGICA) * guany_bit_tensio;
  
}

\end{lstlisting}




\clearpage


% Table generated by Excel2LaTeX from sheet 'Hoja1'
%\begin{table}[H]
%  \centering
%    \begin{tabularx} {\textwidth} {|X|r|} \hline
%  \multicolumn{1}{|c|}{Descripció} &  \multicolumn{1}{c|}{Quantitat}\\ \hline \hline
%
 %   Placa GLC 330 W & 10 \\ \hline
%    Inversor FRONIUS Primo 3.0-1 Light 3kW & 1 \\ \hline
%    Metres cable Ethernet RJ-45 CAT 8 & 10 \\ \hline
%    Metres cable 4 m$m^2$ PVC & 45 \\ \hline
 %   Metres cable 1,5 m$m^2$ PVC & 100 \\ \hline
 %   Punteres Enghofer E 4-10, 4 m$m^2$, 10 mm & 20 \\ \hline
 %   Punteres Enghofer E 1.5-10 1,5 m$m^2$ 10 mm & 12 \\ \hline
 %   Cinta aïllant 10 m 1,6 cm & 3 \\ \hline
 %   Caixa estanca Solera CONS 100x100x55 mm & 2 \\ \hline
  %  Canal Euroquint 25,16 mm 1,5 metres & 20 \\ \hline
%    Curva canal VECAMCO & 10 \\ \hline
%    Paquet de 50 brides 200x2,6  mm & 2 \\ \hline
%    Regleta nylon 12 pols 16 mm & 4 \\ \hline
%    Premsaestopes M12 & 10 \\ \hline
%    Cargol autoroscant M4 16 mm & 12 \\ \hline
%    Tacs Fischer 072095 nylon 6x50 mm & 50 \\ \hline
%    Díode SM74611KTTR & 10 \\ \hline
%            Hores enginyer & 1 \\ \hline
%    Hores oficial de primera & 12 \\ \hline
%    Hores oficial de segona & 12 \\ \hline
%    \end{tabularx}%
%  \label{tab:addlabel}%
% \end{table}%


\begin{appendices}
%\chapter{Títol de l'annex}

%\chapter{\uppercase{Càlculs}}
Pel càlcul de les seccions dels conductors cal tenir en compte els factors de simultaneïtat d'alguns elements i els factors que marca el REBT: 1,25 pel motor elèctric de més potència de la línia, tal com es detalla a la ITC-47; i 1,8 per les lluminàries amb descàrrega, tal com s'indica a la ITC-44. A l'obrador hi ha molts motors elèctric però cap llum amb descàrrega.\\
\newline
En algunes línies es considera que el factor de potència és unitari. A la realitat mai valdrà exactament 1, però sí que es preveu que tingui un valor molt semblant. Les màquines que s'han escollit tenen un factor de potència proper a l'unitari però diferent de 1.\\
\newline Per calcular la intensitat de les línies monofàsiques es fa servir la següent fórmula:
\begin{equation}
I_{linia} = \frac{P}{V*\cos(\phi)}
\end{equation}
V = 230 V\\
P és la potència que consumeixen els elements connectats a la línia\\
$\phi$ és el factor de potència\\
\newline En trifàsic, l'equació que s'utilitza és:
\begin{equation}
I_{linia} = \frac{P}{\sqrt3*V_{linia}*\cos(\phi)}
\end{equation}
$V_{linia}$ = 400 V\\
\newline És important calcular la caiguda de tensió a les línies per tal de veure si estan dimensionades correctament. La caiguda de tensió en línies d'enllumenat no pot ser superior al 3\% i en línies de força no pot ser superior al 5\% de la tensió de subministrament. La caiguda de tensió màxima a la derivació individual és de 1,5\%.\\
\newline En monofàsic:
\begin{equation}
e(\%)=\frac{P}{V}\frac{2*l}{k*S}
\end{equation}
l és la longitud ja sigui de la fase o el neutre des del comptador a l'element més llunyà\\
$k = 56 \frac{m}{mm^{2}\si{\ohm}}$\\
S és la secció del cable en m$m^2$\\
\newline
En trifàsic, l'equació que s'utilitza és:
\begin{equation}
e(\%)=\frac{P}{V}\frac{l}{k*S}
\end{equation}
\\
El dimensionament de les línies ha de permetre que les caigudes de tensió no superin els màxims indicats prèviament. Alhora, els cables han de poder admetre les intensitats calculades, per això ens guiem amb la taula de la ITC-19 del REBT. Finalment, cal comprovar que  l'interruptor magnetotèrmic té una intensitat nominal superior a la calculada per la línia i menor a l'admissible que marca la ITC-19.\\
\newline La instal·lació és trifàsica, per tant, hi ha 3 conductors de fase i un conductor de neutre. El conductor de terra transcorre per totes les línies i té una secció igual als conductors de les línies, tal com s'indica al plànol. El neutre, que arriba per l'escomesa, també és de la mateixa secció que els conductors de fase. Les màquines trifàsiques necessiten el neutre pels seus equips electrònics.\\
\newline
Per comprovar que el valor de secció de la derivació individual és correcte quan la línia va amb una terna de cables unipolars per tub cal tenir en compte un factor d'intensitat de 0,8.
\begin{equation}
I_{DI} < 0.8 * I_{max. admissible}
\end{equation}

\noindent A continuació es mostren les diferents línies de forma detallada. La secció s'ha comprovat tenint en compte les fórmules explicades i les seccions mínimes per intensitat segons marca el REBT. S'han verificat les línies pel cas més desfavorable. Les seccions dels tubs compleixen amb la ITC-21.\\
\newline Les tensions nominals són 230 V per les línies monofàsiques i 400 V per les trifàsiques. Tots els cables de les línies són de coure de 450/750 V d'aïllament. La derivació individual és de coure amb 0,6/1 kV d'aïllament. L'aïllament de la instal·lació és de 1.000 k$\si{\ohm}$.

\begin{table}[H]
\scriptsize
\begin{center}
 \begin{tabu} to \textwidth {|X[0.5, l]|X[2, l]|X[r]|X[0.6, r]|X[r]|X[r]|X[r]|X[r]|X[r]|X[r]|X[0.5,r]|}%{X | c c c} 
 \hline
 Línia& Descripció & Potència (W) & cos($\phi$) & Intensitat (A) & Distància màxima (m) & Seccions fase, neutre, terra ($mm^{2}$) & Diàmetre tub (mm) & Caiguda de tensió (\%) & Caiguda de tensió acum. (\%)\\
 \hline \hline 
DI & Derivació individual& 87.000 \ \ \ \ & 0,96 & 131,32 & 8 &3x35 + 35 + 16& 160 & 0,23 & 0,23 \\ \hline
L1 & Enllumenat habitacions i cambra& 1.370,5 & 1 & 5,96 & 55 &2,5 + 2,5 + 2,5& 20 & 2,04 & 2,27 \\ \hline
L2 & Enllumenat cuina & 1.476 \ \ \ \  & 1 & 6,42 & 47 &2,5 + 2,5 + 2,5& 20 & 1,87 & 2,10 \\ \hline 
L3 & Força oficina, menjador, màquines de buit & 7.775 \ \ \ \  & 1 & 33,80 & 51 &10 + 10 + 10& 25 & 2,68 & 2,91 \\ \hline 
L4 & Força cuina & 7.235 \ \ \ \  & 1 & 31,46 & 33 & 6 + 6 + 6& 25 & 2,67 & 2,90 \\ \hline
L5 & Rentaplats cuina & 36.000 \ \ \ \ & 0,95 & 54,92 & 54 &3x16 + 16 + 16& 32 & 1,44 & 1,67 \\ \hline 
L6 & Extractors i cambres de fred & 19.000 \ \ \ \ & 0,9 & 30,60 & 36 &3x10 + 10 + 10& 32 & 0,86 & 1,09 \\ \hline
L7 & Abatidor sala de preparació & 16.975 \ \ \ \ & 0,95 & 25,90 & 44 &3x10 + 10 + 10& 32 & 0,84 & 1,07 \\ \hline

 \end{tabu}
 \caption{Línies detallades}
\end{center}
\end{table}



%\chapter{\uppercase{Característiques}}
L'aïllament dels cables elèctrics de les línies és EPR de 450/750 V d'aïllament. Els cables transcorren en safata perforada pel passadís i dins de tubs corrugats en muntatge superficial (B2) a la resta de zones. El diàmetre d'aquests tubs s'indica en l'anterior annex. Els càlculs s'han efectuat considerant que tota la llargada dels cables va amb el muntatge B2, que és més restrictiu que la safata.\\
\newline Es fan servir els colors gris, marró i negre per les fases, el blau pel neutre i el conductor groc i verd pel terra.\\
\newline Hi ha instal·lades caixes de derivació al llarg de la instal·lació i l'enllumenat dels vestidors, l'oficina i el menjador es controla amb interruptors de paret. Els cables dels l'enllumenats que no estan en contacte amb la paret es passen pel fals sostre.\\
\newline Les màquines trifàsiques es connecten a la xarxa mitjançant una base CETAC.\\
\newline Els extractors de la cuina de l'obrador van controlats amb variadors de freqüència. La seva línia va amb un diferencial de 100 mA de classe B degut als alts corrents de fuga que poden donar-se. Aigües amunt de tots els agrupaments hi ha instal·lat un diferencial de 300 mA de sensibilitat per protegir tota la instal·lació i alhora tenir selectivitat amb el diferencials que té aigües avall.\\
\newline El maxímetre del conjunt de protecció i mesura garanteix el subministrament elèctric tot i sobrepassar la potència contractada. Si en un moment puntual es connectés alguna màquina més i pel marge donat no saltés cap interruptor magnetotèrmic però s'estigués superant la potència contractada, hi seguiria havent subministrament elèctric i l'empresa subministradora aplicaria un recàrrec a la factura.\\
\newline 
S'agrupen les línies tenint en compte si el subministrament és trifàsic o monofàsic. S'intenta, en la mesura del possible, que tots els grups tinguin potències similars. És per això que el diferencial que agrupa les 4 línies monofàsiques és de 4 pols: els 3 conductors de fase i el neutre passaran per aquest diferencial i s'alimentaran les diferents línies monofàsiques amb diferents fases. Així es pot aconseguir una instal·lació trifàsica bastant ben equilibrada.\\
\newline Els llums d'emergència són de tipus no permanent i es considera que tenen una potència de 3 W. Al disposar de bateria i només encendre's quan hi ha una emergència, no s'han tingut en compte per la previsió de càrregues.\\
\newline Per millorar el factor de potència de la derivació individual, o sigui, de tota la instal·lació, hi ha instal·lada una bateria de condensadors de 20 kVAr la qual dona una factor de potència de 0,998.




\end{appendices}


\end{spacing}
%\cite{einstein} % per fer una cita
%\printbibliography[title=Bibliografia] %ARA BÉ

\end{document}