\chapter{\uppercase{Disposicions tècniques}}
En aquest capítol es detalla la legislació i normativa que cal complir per assegurar el correcte funcionament de la instal·lació fotovoltaica i de l'electrònica. 

\section{Reglaments}
La legislació i reglamentació que cal complir s'exposa a continuació.\\
\newline Reglament Electrotècnic de Baixa Tensió (REBT) aprovat el 2 d'agost de 2002. Per aquest projecte és imprescindible consultar i complir la ITC-BT-40.\\
\newline Reial Decret 244/2019 de 5 d'abril, el qual regula les condicions administratives, tècniques i econòmiques de l'autoconsum d'energia elèctrica.\\
\newline Guia IDAE 021, que és la Guia Professional de Tramitació d'Autoconsum, novembre de 2019.\\
\newline Reial Decret 1663/2000 de 29 de setembre, que tracta sobre la connexió d'instal·lacions fotovoltaiques a la xarxa de baixa tensió.\\
\newline Resolució de 31 de maig de 2001 que estableix models de contracte tipus i models de factures per les instal·lacions solars fotovoltaiques connectades a la xarxa elèctrica.\\
\newline Reial Decret 436/2004 de 12 de març que estableix la metodologia per a l'actualització del règim jurídic i econòmic de l'activitat de producció d'energia elèctrica en règim especial.\\
\newline Reial Decret 1955/2000 d'1 de desembre que regula el transport, distribució, comercialització, subministrament i procediments d'autorització d'instal·lacions d'energia elèctrica.\\
\newline Reial Decret 208/2005 de 25 de febrer que tracta sobre aparells elèctrics i electrònics i la gestió dels seus residus.\\
\newline Reial Decret 186/2016 de 6 de maig que regula la compatibilitat electromagnètica dels equips elèctrics i electrònics.\\
\newline Reial Decret 187/2016 de 6 de maig que regula les exigències del material elèctric destinat a ser utilitzat en límits de tensió.\\
\newline Reial Decret 188/2016 de 6 de maig que estableix els requisits per la comercialització, posada en funcionament i ús d'equips radioelèctrics. Regula el procediment d'avaluació de la conformitat, la vigilància del mercat i el règim sancionador dels equips de telecomunicació.\\
\newline Llei 31/1995 de 8 de novembre sobre la prevenció de riscos laborals.\\
\newline Reial Decret 1627/1997 de 24 d'octubre de 1997 sobre disposicions mínimes de seguretat i salut en les obres.\\
\newline Reial decret 486/1997 de 14 d'abril de 1997 sobre Disposicions mínimes de seguretat i salut en els llocs de treball.\\
\newline Reial decret 485/1997 de 14 d'abril de 1997 sobre disposicions mínimes en matèria de senyalització de seguretat i salut en el treball.\\
\newline Reial decret 1215/1997 de 18 de juliol de 1997 sobre disposicions mínimes de seguretat i salut per a la utilització pels treballadors dels equips de treball.\\
\newline Reial decret 773/1997 de 30 de maig de 1997 sobre disposicions mínimes de seguretat i salut relatives a la utilització pels treballadors d'equips de protecció individual.


\section{Normes}
Són d'obligat compliment les normes que les anteriors lleis i reglaments citen. També ho són les que es mostren a continuació.\\
\newline Norma UNEIX EN 61453 sobre assaig ultraviolat per a mòduls fotovoltaics.\\
\newline Norma UNE-EN 50380 sobre informacions de les fitxes de característiques i de les plaques característiques dels mòduls fotovoltaics.\\
\newline Norma UNEIX EN 60891 sobre el procediment de correcció amb la temperatura i la irradiància característica I-V dels dispositius fotovoltaics de silici cristal·lí.\\
\newline Norma UNEIX EN 60904 sobre dispositius fotovoltaics i els requisits pels mòduls solars de referència.\\
\newline Norma UNEIX EN 61173 sobre protecció contra sobretensiones dels sistemes fotovoltaics productors d'energia. \\
\newline Norma UNEIX EN 61194 sobre paràmetres característics de sistemes fotovoltaics autònoms.\\
\newline Norma UNEIX EN 61646:1997 sobre mòduls fotovoltaics de làmina prima per a aplicació terrestre.\\
\newline Norma UNEIX EN 61277 sobre sistemes fotovoltaics terrestres generadors de potència.\\
\newline Norma UNEIX EN 61727 sobre sistemes fotovoltaics i característiques de la interfície de connexió a la xarxa elèctrica.\\
\newline Norma UNEIX EN 61721 sobre susceptibilitat d'un mòdul fotovoltaic al dany per impacte accidental.\\
\newline Norma UNEIX EN 61646:1997 sobre mòduls fotovoltaics de làmina prima per a aplicació terrestre.\\
\newline Norma UNEIX EN 61683 sobre sistemes fotovoltaics i condicionadors de potència.\\
\newline Norma UNEIX 206001 sobre mòduls fotovoltaics i criteris ecològics.\\
\newline Norma UNEIX 61215 sobre mòduls fotovoltaics de silici cristal·lí per a aplicació terrestre.\\
\newline Norma UNEIX EN 61701 sobre assaig de corrosió per boira salina de mòduls fotovoltaics.\\
\newline Norma UNEIX EN 61724 sobre monitoratge de sistemes fotovoltaics. \\
\newline Norma UNEIX EN 61725 sobre expressió analítica per als perfils solars diaris.\\
\newline Norma UNEIX EN 61829 sobre camps fotovoltaics de silici cristal·lí, que tracta la mesura en el lloc de característiques I-V.\\
\newline UNE-EN IEC 61204-3:2018 sobre fonts d'alimentació de baixa tensió amb sortida en corrent continu. La part 3 parla sobre compatibilitat electromagnètica.\\
\newline UNE-EN 61508:2011 sobre seguretat funcional dels sistemes elèctrics i electrònics, siguin o no programables. La part 1 tracta sobre requisits generals, la 3 sobre requisits del software.
