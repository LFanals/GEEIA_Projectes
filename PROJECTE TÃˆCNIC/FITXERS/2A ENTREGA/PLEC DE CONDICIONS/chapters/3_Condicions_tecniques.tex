\chapter{\uppercase{Condicions tècniques}}
El projectista demana que es compleixin les condicions tècniques referents a seguretat i salut, materials, fabricació, muntatge i programació i comprovació i posada en funcionament. Si es compleixen aquestes condicions el projectista preveu que tècnicament el projecte s'ha desenvolupat correctament i que es minimitzen els problemes que es puguin donar.

\section{Seguretat i salut}
És molt important assegurar unes bones condicions de seguretat en el treball i salut als treballadors involucrats en el projecte. Per això s'han de complir les lleis que s'indiquen a continuació en aquest apartat. En cas de no fer-ho l'autor del projecte s'eximeix de tota responsabilitat.\\
\newline Es tindrà especialment en compte la Llei 31/1995 de 8 de novembre sobre la prevenció de riscos laborals. A l'apartat de Normes figuren altres normatives que també s'han de complir.\\
\newline El Director d'obra podrà exigir al Contractista els documents que acreditin que el personal involucrat en la instal·lació ha formalitzat els tràmits necessaris amb la Seguretat Social.\\
\newline El Director d'obra pot exigir el cessament de l'obra en qualsevol moment per tal de garantir la integritat física dels treballadors.\\
\newline El personal que treballi en la instal·lació ha d'usar les mesures de protecció adequades per la feina que realitza, com per exemple usar guants, ulleres, casc... En moments de treballar amb equips sota tensió el personal usarà roba sense accessoris metàl·lics i evitaran utilitzar eines metàl·liques. El calçat serà aïllant.\\
\newline El Contractista mantindrà en regla una pòlissa d'Assegurança que el protegeixi a ell i els seus empleats enfront tot tipus de responsabilitats.

\section{Materials}
Els materials utilitzats per dur a terme la instal·lació fotovoltaica i per crear i instal·lar la placa electrònica estan detallats a l'Estat d'amidaments, document present en aquest projecte.\\
\newline Tots els materials i equips estan homologats segons normes UNE o similars, i són vigents per la CE. Pel muntatge es té en compte el REBT.\\
\newline El Contractista és responsable de la vigilància dels materials durant l'emmagatzematge i desenvolupament de la instal·lació.\\
\newline El Contractista haurà de complir el que indiquin els plànols, que són de caràcter contractual. Si té dubtes haurà de consultar-los amb l'autor del projecte, i comunicar allò que no pugui fer.\\
%contractista qui és?
\newline Els components electrònics han d'estar en perfecte estat i en cas de detectar alguna anomalia s'ha de comunicar al proveïdor i aconseguir components electrònics en bon estat.\\
\newline Les resistències tindran una tolerància màxima del 5\% i els condensadors tindran una tolerància màxima del 20\%. Les resistències han de poder dissipar la potència indicada pel fabricant en condicions de temperatura ambient. Els condensadors han de ser capaços de suportar la tensió que indica el fabricant.\\
\newline El plàstic per la caixa impresa amb una impressora 3D serà PLA amb 0,75 mm de diàmetre i una tolerància de diàmetre màxima de 0,01 mm.

\section{Fabricació}
En cas de detectar anomalies en el material elèctric se n'informarà al proveïdor i es prendran les mesures adequades per tal de garantir que el material utilitzat és de primera qualitat.\\
\newline El circuit imprès de les plaques electròniques es pot subcontractar. S'ha de garantir una tolerància màxima de pistes de 0,1 mm. Els diàmetres dels forats han de tenir el diàmetre indicat amb una tolerància màxima de 0,2 mm. S'ha de garantir que el circuit imprès sigui per les dues cares i les dimensions siguin les que s'indiquen al document Plànols.\\
\newline Durant la fabricació de la caixa s'ha de garantir que la impressora 3D estigui plana en una superfície estàtica. La impressora ha de disposar de blocs antivibratoris. S'ha de realitzar la impressió 3D en un espai amb una temperatura no inferior a 20 graus centígrads i no superior a 28 graus centígrads. La base de la peça no ha d'aixecar-se i desenganxar-se de la taula.\\
\newline La impressió ha de donar lloc a una peça com la dissenyada, amb una tolerància de 0,2 mm com a màxim en tot cas. En cas de no complir amb l'indicat, s'ha de tornar a imprimir la peça.


\section{Muntatge i programació}
%plaques
%soldadures
El muntatge dels panells solars fotovoltaics l'ha de fer personal qualificat i apte, amb experiència prèvies d'aquest tipus d'instal·lació. El personal ha de seguir les normes de seguretat i salut i ha de ser capaç d'entendre l'esquema unifilar que es troba al document Plànols per tal de col·locar els cables amb la secció necessària i les proteccions indicades. Els dispositius s'han de connectar després d'haver accionat l'IGA i no tenir tensió a la instal·lació.\\
\newline El muntatge mecànic s'ha de fer amb el material que marca l'Estat d'amidaments i seguint les indicacions de Plànols. S'ha d'assegurar que es munta una estructura rígida que no es deteriori per l'acció del vent. L'orientació de les plaques solars ha de ser l'adequada.\\
\newline El muntatge de la placa electrònica ha de desenvolupat per personal competent i amb experiència muntant components electrònics SMD. S'exigeix disposar d'algun tipus de lupa o microscopi per facilitar el muntatge i alhora assegurar unes bones soldadures. Una pistola d'aire calent pot facilitar el muntatge.\\
\newline Els components s'han de situar a les posicions que indiquen els Plànols. Les soldadures s'han de realitzar de forma acurada i ràpida, ajudant-se de decapant. S'ha d'evitar escalfar directament els components durant més de 10 segons de forma contínua per evitar el deteriorament.\\
\newline És d'especial importància no situar cap component ni element conductor prop de l'antena del mòdul ESP-12E, per tal d'evitar interferències i males comunicacions.\\
\newline S'ha de programar el dispositiu amb el programa facilitat.

\section{Comprovació i posada en funcionament}
Abans de la posada en funcionament cal fer una revisió i comprovar amb un multímetre o similar que la placa electrònica estigui ben soldada. Hi ha d'haver tots els components. S'ha de verificar que el mòdul és capaç de connectar-se a una xarxa Wi-Fi i que les lectures de tensió són correctes.\\
\newline A la posada en funcionament un tècnic competent ha de revisar que les connexions entre les plaques solars siguin les correctes. Els panells solars, els díodes, les seccions dels cables i els dispositius de protecció i comandament han de ser els indicats en el projecte. S'ha de comprovar que les plaques lliuren energia i que la placa electrònica es comunica correctament amb la xarxa Wi-Fi de l'habitatge, així com que les seves lectures són correctes.

\clearpage


% Table generated by Excel2LaTeX from sheet 'Hoja1'
%\begin{table}[H]
%  \centering
%    \begin{tabularx} {\textwidth} {|X|r|} \hline
%  \multicolumn{1}{|c|}{Descripció} &  \multicolumn{1}{c|}{Quantitat}\\ \hline \hline
%
 %   Placa GLC 330 W & 10 \\ \hline
%    Inversor FRONIUS Primo 3.0-1 Light 3kW & 1 \\ \hline
%    Metres cable Ethernet RJ-45 CAT 8 & 10 \\ \hline
%    Metres cable 4 m$m^2$ PVC & 45 \\ \hline
 %   Metres cable 1,5 m$m^2$ PVC & 100 \\ \hline
 %   Punteres Enghofer E 4-10, 4 m$m^2$, 10 mm & 20 \\ \hline
 %   Punteres Enghofer E 1.5-10 1,5 m$m^2$ 10 mm & 12 \\ \hline
 %   Cinta aïllant 10 m 1,6 cm & 3 \\ \hline
 %   Caixa estanca Solera CONS 100x100x55 mm & 2 \\ \hline
  %  Canal Euroquint 25,16 mm 1,5 metres & 20 \\ \hline
%    Curva canal VECAMCO & 10 \\ \hline
%    Paquet de 50 brides 200x2,6  mm & 2 \\ \hline
%    Regleta nylon 12 pols 16 mm & 4 \\ \hline
%    Premsaestopes M12 & 10 \\ \hline
%    Cargol autoroscant M4 16 mm & 12 \\ \hline
%    Tacs Fischer 072095 nylon 6x50 mm & 50 \\ \hline
%    Díode SM74611KTTR & 10 \\ \hline
%            Hores enginyer & 1 \\ \hline
%    Hores oficial de primera & 12 \\ \hline
%    Hores oficial de segona & 12 \\ \hline
%    \end{tabularx}%
%  \label{tab:addlabel}%
% \end{table}%
