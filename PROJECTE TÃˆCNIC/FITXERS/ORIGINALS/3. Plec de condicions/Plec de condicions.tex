%Options > Configure Texmaker > Editor > Spelling Dictionary, per corrector en català

\documentclass[11pt, a4paper]{report}
\usepackage[a4paper,left=30mm,right=20mm,top=25mm,bottom=25mm]{geometry}
\sloppy %per forçar el canvi de línia si la paraula supera el marge dret
\usepackage[utf8]{inputenc}

% Per utilitzar la font Helvetica (Arial)
\renewcommand{\familydefault}{\sfdefault}
\usepackage[scaled=1]{helvet}
\usepackage[helvet]{sfmath}
\everymath={\sf}
%Equacions amb una font sans_serif, \mathrm{equació aquí, són les letres les que queden inclinades}
%\usepackage{arev} % sans-serif math font
%\usepackage{helvet} % sans-serif text font


% Per comptar imatges enlloc de mostrar 1.1, 1.2...
\usepackage{chngcntr}
\counterwithout{figure}{chapter}
\counterwithout{table}{chapter}
\counterwithout{equation}{chapter}

\usepackage{graphicx}
\graphicspath{{images/}} %directori amb les imatges que volem insertar
\usepackage{float} %per forçar imatges amb H
\usepackage[normalem]{ulem} %negreta múltiples línies
%\usepackage{soul}

\usepackage{caption}
\captionsetup[figure]{labelfont={},name={Figura},labelsep=period}
\captionsetup[table]{labelfont={},name={Taula},labelsep=period}


\usepackage{subcaption}
\usepackage{amsmath} %per fòrmules matemàtiques
\usepackage[table]{xcolor} %per colors a les taules
%\usepackage{circuitikz} %per circuits electrònics
\usepackage{siunitx} %per les labels dels components
\usepackage[american,cuteinductors,smartlabels]{circuitikz} %american/european
\usepackage{tikz} %quadrícula
\usepackage[a4paper, left=30mm, right=20mm, top=25mm, bottom=25mm]{geometry} %geometria de la pàgina, 25 però per ajustar bé
\setlength{\headsep}{20pt}
\setlength{\footskip}{25pt}
%\usepackage[a4paper, width=150mm, top=25mm, bottom=25mm]{geometry} %geometria de la pàgina
\usepackage{lipsum} %per generar dummy text
\usepackage{xpatch} %per la distància entre títol i top

%Capçaleres i peus de pàgina
\usepackage{fancyhdr}
%\pagestyle{fancy} %fancy, plain
\fancypagestyle{plain}{
  \fancyhf{}% Clear header/footer
  \fancyhead[L]{\footnotesize{Plaques solars fotovoltaiques sensoritzades per habitatge unifamiliar}}
  \fancyhead[R]{\footnotesize{Plec de condicions}}
  \fancyfoot[R]{\footnotesize{\thepage}}
}
\pagestyle{plain}% Set page style to plain.

%\fancyhead{}
%\fancyhead[LO,LE]{PROJECTES}
%\fancyfoot{}
%\fancyfoot[LE,RO]{\thepage} %número de la pàgina, a la dreta
%\fancyfoot[LO, CE]{Capítol \thechapter} %nom del capítol, a l'esquerra
%\fancyfoot[CO, CE]{\href{https://github.com/LFanals}{Llorenç Fanals Batllori}} %nom de l'autor, al centre
% \renewcommand{\headrulewidth}{0.4pt}
%\renewcommand{\footrulewidth}{0.4pt}

%Per tenir el nombre de pàgina a l'inici d'un capítol
%\fancypagestyle{plain}{
%\fancyhf{}
%\renewcommand\headrulewidth{0pt}
%\fancyfoot[R]{\thepage}
%}

%Per configurar el color dels links i referències
\usepackage{color}
\usepackage{hyperref}
\hypersetup{
    colorlinks=true, %true si es volen links de colors
    linkcolor=black,  %colors de les referències internes, blue
    filecolor=magenta,      %magenta
    urlcolor=[rgb]{0,0,0}, %Color dels links d'Internet, sobre 255=2^8-1=2^0+...+2^7, {0,0.5,1}
}

%Bibliografia
\usepackage[backend=bibtex]{biblatex}
\addbibresource{bibliography.bib}

%Canviem el nom que hi ha per defecte als índex i altres, per passar-ho al català
\renewcommand{\contentsname}{Índex}
\renewcommand{\listfigurename}{Índex de figures}
\renewcommand{\chaptername}{Capítol}
\renewcommand{\appendixname}{Annex}
\renewcommand{\listtablename}{Índex de taules}
% \renewcommand{\figurename}{Figura} % ho tinc amb caption
% \captionsetup[table]{name=Taula} % ho tinc amb caption

\definecolor{color_quadricula}{HTML}{0066ff} %color per la quadrícula

% Pels circuits
%\usepackage[american]{circuitikz}
\usetikzlibrary{calc}
\ctikzset{bipoles/thickness=1}
\ctikzset{bipoles/length=1.2cm}
\ctikzset{bipoles/diode/height=.375}
\ctikzset{bipoles/diode/width=.3}
\ctikzset{tripoles/thyristor/height=.8}
\ctikzset{tripoles/thyristor/width=1}
\ctikzset{bipoles/vsourceam/height/.initial=.7}
\ctikzset{bipoles/vsourceam/width/.initial=.7}
\tikzstyle{every node}=[font=\small]
\tikzstyle{every path}=[line width=0.8pt,line cap=round,line join=round]

%Per insertar codi
\usepackage{listings}
\usepackage{color}
\definecolor{dkgreen}{rgb}{0,0.6,0}
\definecolor{gray}{rgb}{0.5,0.5,0.5}
\definecolor{mauve}{rgb}{0.58,0,0.82}

\lstset{frame=none, %tb, none
  language=Python,
  aboveskip=2mm,
  belowskip=3mm,
  showstringspaces=false,
  columns=flexible,
  basicstyle={\scriptsize\ttfamily}, %small
  numbers=none, %left
  numberstyle=\tiny\color{gray},
  keywordstyle=\color{blue},
  commentstyle=\color{dkgreen},
  stringstyle=\color{mauve},
  breaklines=true,
  breakatwhitespace=true,
  tabsize=3
}


%Per tenir el format de capítol correcte
\usepackage{titlesec}

\usepackage{etoolbox}
%\usepackage{hyperref}

%Per chapter
\titlespacing*{\chapter}{0pt}{-25pt}{11pt} %Espaiat del títol de capítol amb els altres elements
\titleformat{\chapter}[hang] %Per seguir escrivint darrera el número
{\normalfont\fontsize{11}{15}\bfseries}{\thechapter.}{0.4em}{\MakeUppercase} %\fontsize{Tamany}{Espai múltiples línies}

%Per secció
\titlespacing*{\section}{0pt}{11pt}{11pt} %Espaiat del títol de capítol amb els altres elements
\titleformat{\section}[hang] %Per seguir escrivint darrera el número
{\normalfont\fontsize{11}{15}}{\thesection.}{0.4em}{\bfseries} %\fontsize{Tamany}{Espai múltiples línies}

%Per subsecció
\titlespacing*{\subsection}{0pt}{11pt}{11pt} %Espaiat del títol de capítol amb els altres elements
\titleformat{\subsection}[hang] %Per seguir escrivint darrera el número
{\normalfont\fontsize{11}{15}}{\thesubsection.}{0.4em}{} %\fontsize{Tamany}{Espai múltiples línies}

%Per paràgraf
\titlespacing*{\paragraph}{0pt}{0pt}{22pt} %Espaiat del títol de capítol amb els altres elements
\titleformat{\paragraph}[hang] %Per seguir escrivint darrera el número
{\normalfont\fontsize{11}{15}}{}{}{} %\fontsize{Tamany}{Espai múltiples línies}


%Interlineat, 1.2*1.25=1.5
\linespread{1.25}

%Espaiat entre paràgrafs
%\setlength{\parskip}{22pt}
 

\makeatletter
\def\tagform@#1{\maketag@@@{(\ignorespaces{Eq.~#1}\unskip)}}
\makeatother



%Per no tenir negreta a l'index
\usepackage{etoolbox}% http://ctan.org/pkg/etoolbox
\makeatletter
\patchcmd{\l@chapter}{\bfseries}{}{}{}% \patchcmd{<cmd>}{<search>}{<replace>}{<success>}{<failure>}
\makeatother

%Per tenir punts a l'índex
\makeatletter
\renewcommand*\l@chapter{\@dottedtocline{0}{0em}{1.5em}}
\makeatother

%Per taula que adapta bé els espais
\usepackage{tabularx}
\usepackage{tabu} % http://mirrors.ibiblio.org/CTAN/macros/latex/contrib/tabu/tabu.pdf
\tabulinesep = 1mm
\usepackage[font=footnotesize]{caption} %Captions de les figures més petites

%Appendix
\usepackage[]{appendix} %toc, page

%Alinear al separador decimal amb espais
\usepackage{setspace}
\renewcommand*{\arraystretch}{1.25}

%Per fer el símbol de grau celsius amb \textcelsius{}
\usepackage{textcomp}

%-------------------------------------------------------------------------------------------------------------
%-------------------------------------------------------------------------------------------------------------
%-------------------------------------------------------------------------------------------------------------
%-------------------------------------------------------------------------------------------------------------

\begin{document}
\pagenumbering{Roman}


%\begin{titlepage}
	\begin{center}
		\vspace*{1cm}
		
		\Huge
		\textbf{Document per Projectes}
		
		\vspace{0.5cm}
		\LARGE
		Adaptat a \LaTeX
	
		\vspace{1.5cm}
		
		\textbf{Llorenç Fanals Batllori}
		
		\vfill
		
		\small
		%\uppercase{Un treball lliurat a la Universitat - en compliment dels requisits pel grau en -}\\
		% TFG
		
		\vspace{1cm}
		
		%\includegraphics[scale=width=0.4\textwidth]{images/a_graph}
	\end{center}
	
	\begin{flushright}
	\large	
	Departament o grup de recerca\\
	UdG\\
	%País\\
	28/08/2019
	\end{flushright}
	


\end{titlepage}

%\thispagestyle{plain}

\begin{center}
	\large
	\textbf{Informe}
	
	\vspace{0.4cm}
	\large
	Descripció
	
	\vspace{0.4cm}
	\textbf{Llorenç Fanals Batllori}
	
	\vspace{0.9cm}
	\textbf{Abstract}
\end{center}
\lipsum[1]




%\chapter*{Dedicacions}
%Dedico aquest treball a -

%\chapter*{Agraïments}
%Vull agraïr a \\

\cleardoublepage\pagenumbering{arabic}

\begin{spacing}{2}
\tableofcontents
\end{spacing}
%\listoffigures %No fa falta crec
%\listoftables %No fa falta crec
\begin{spacing}{1.5}

\chapter{\uppercase{Introducció}}
El present document determina els requisits i les condicions a què estan subjectes la instal·lació fotovoltaica i la placa electrònica encarregada d'adquirir dades. Per l'execució del projecte s'ha de complir tot el que determina aquest document, en cas contrari el projectista no pot garantir un correcte funcionament i no es fa responsable dels accidents que puguin ocórrer.

\section{Objecte del plec}
L'objectiu del document és detallar i recollir per escrit les condicions obligatòries a què estan sotmesos el personal instal·lador, el personal proveïdor, i, en general, qualsevol persona física o jurídica involucrada en el projecte. Si es compleix aquest document s'assegura un funcionament correcte.


\section{Documents contractuals i informatius}
Tots els documents d'aquest projecte, que són Memòria, Plànols, Plec de condicions, Estat d'amidaments i Pressupost són contractuals. 


\section{Compatibilitat entre documents} %passar model i tal a annex?
El projectista ha estat el més coherent possible en els documents del present projecte. En l'improbable cas de discrepància entre dos o més documents, l'ordre de prioritat a seguir és l'ordre en què s'ordenen els documents del projecte: Memòria, Plànols, Plec de condicions, Estat d'amidaments i Pressupost.

%http://www.ti.com/lit/ds/symlink/sm74611.pdf

\clearpage


% Table generated by Excel2LaTeX from sheet 'Hoja1'
%\begin{table}[H]
%  \centering
%    \begin{tabularx} {\textwidth} {|X|r|} \hline
%  \multicolumn{1}{|c|}{Descripció} &  \multicolumn{1}{c|}{Quantitat}\\ \hline \hline
%
 %   Placa GLC 330 W & 10 \\ \hline
%    Inversor FRONIUS Primo 3.0-1 Light 3kW & 1 \\ \hline
%    Metres cable Ethernet RJ-45 CAT 8 & 10 \\ \hline
%    Metres cable 4 m$m^2$ PVC & 45 \\ \hline
 %   Metres cable 1,5 m$m^2$ PVC & 100 \\ \hline
 %   Punteres Enghofer E 4-10, 4 m$m^2$, 10 mm & 20 \\ \hline
 %   Punteres Enghofer E 1.5-10 1,5 m$m^2$ 10 mm & 12 \\ \hline
 %   Cinta aïllant 10 m 1,6 cm & 3 \\ \hline
 %   Caixa estanca Solera CONS 100x100x55 mm & 2 \\ \hline
  %  Canal Euroquint 25,16 mm 1,5 metres & 20 \\ \hline
%    Curva canal VECAMCO & 10 \\ \hline
%    Paquet de 50 brides 200x2,6  mm & 2 \\ \hline
%    Regleta nylon 12 pols 16 mm & 4 \\ \hline
%    Premsaestopes M12 & 10 \\ \hline
%    Cargol autoroscant M4 16 mm & 12 \\ \hline
%    Tacs Fischer 072095 nylon 6x50 mm & 50 \\ \hline
%    Díode SM74611KTTR & 10 \\ \hline
%            Hores enginyer & 1 \\ \hline
%    Hores oficial de primera & 12 \\ \hline
%    Hores oficial de segona & 12 \\ \hline
%    \end{tabularx}%
%  \label{tab:addlabel}%
% \end{table}%

\chapter{\uppercase{Disposicions tècniques}}
En aquest capítol es detalla la legislació i normativa que cal complir per assegurar el correcte funcionament de la instal·lació fotovoltaica i de l'electrònica. 

\section{Reglaments}
La legislació i reglamentació que cal complir s'exposa a continuació.\\
\newline Reglament Electrotècnic de Baixa Tensió (REBT) aprovat el 2 d'agost de 2002. Per aquest projecte és imprescindible consultar i complir la ITC-BT-40.\\
\newline Reial Decret 244/2019 de 5 d'abril, el qual regula les condicions administratives, tècniques i econòmiques de l'autoconsum d'energia elèctrica.\\
\newline Guia IDAE 021, que és la Guia Professional de Tramitació d'Autoconsum, novembre de 2019.\\
\newline Reial Decret 1663/2000 de 29 de setembre, que tracta sobre la connexió d'instal·lacions fotovoltaiques a la xarxa de baixa tensió.\\
\newline Resolució de 31 de maig de 2001 que estableix models de contracte tipus i models de factures per les instal·lacions solars fotovoltaiques connectades a la xarxa elèctrica.\\
\newline Reial Decret 436/2004 de 12 de març que estableix la metodologia per a l'actualització del règim jurídic i econòmic de l'activitat de producció d'energia elèctrica en règim especial.\\
\newline Reial Decret 1955/2000 d'1 de desembre que regula el transport, distribució, comercialització, subministrament i procediments d'autorització d'instal·lacions d'energia elèctrica.\\
\newline Reial Decret 208/2005 de 25 de febrer que tracta sobre aparells elèctrics i electrònics i la gestió dels seus residus.\\
\newline Reial Decret 186/2016 de 6 de maig que regula la compatibilitat electromagnètica dels equips elèctrics i electrònics.\\
\newline Reial Decret 187/2016 de 6 de maig que regula les exigències del material elèctric destinat a ser utilitzat en límits de tensió.\\
\newline Reial Decret 188/2016 de 6 de maig que estableix els requisits per la comercialització, posada en funcionament i ús d'equips radioelèctrics. Regula el procediment d'avaluació de la conformitat, la vigilància del mercat i el règim sancionador dels equips de telecomunicació.\\
\newline Llei 31/1995 de 8 de novembre sobre la prevenció de riscos laborals.\\
\newline Reial Decret 1627/1997 de 24 d'octubre de 1997 sobre disposicions mínimes de seguretat i salut en les obres.\\
\newline Reial decret 486/1997 de 14 d'abril de 1997 sobre Disposicions mínimes de seguretat i salut en els llocs de treball.\\
\newline Reial decret 485/1997 de 14 d'abril de 1997 sobre disposicions mínimes en matèria de senyalització de seguretat i salut en el treball.\\
\newline Reial decret 1215/1997 de 18 de juliol de 1997 sobre disposicions mínimes de seguretat i salut per a la utilització pels treballadors dels equips de treball.\\
\newline Reial decret 773/1997 de 30 de maig de 1997 sobre disposicions mínimes de seguretat i salut relatives a la utilització pels treballadors d'equips de protecció individual.


\section{Normes}
Són d'obligat compliment les normes que les anteriors lleis i reglaments citen. També ho són les que es mostren a continuació.\\
\newline Norma UNEIX EN 61453 sobre assaig ultraviolat per a mòduls fotovoltaics.\\
\newline Norma UNE-EN 50380 sobre informacions de les fitxes de característiques i de les plaques característiques dels mòduls fotovoltaics.\\
\newline Norma UNEIX EN 60891 sobre el procediment de correcció amb la temperatura i la irradiància característica I-V dels dispositius fotovoltaics de silici cristal·lí.\\
\newline Norma UNEIX EN 60904 sobre dispositius fotovoltaics i els requisits pels mòduls solars de referència.\\
\newline Norma UNEIX EN 61173 sobre protecció contra sobretensiones dels sistemes fotovoltaics productors d'energia. \\
\newline Norma UNEIX EN 61194 sobre paràmetres característics de sistemes fotovoltaics autònoms.\\
\newline Norma UNEIX EN 61646:1997 sobre mòduls fotovoltaics de làmina prima per a aplicació terrestre.\\
\newline Norma UNEIX EN 61277 sobre sistemes fotovoltaics terrestres generadors de potència.\\
\newline Norma UNEIX EN 61727 sobre sistemes fotovoltaics i característiques de la interfície de connexió a la xarxa elèctrica.\\
\newline Norma UNEIX EN 61721 sobre susceptibilitat d'un mòdul fotovoltaic al dany per impacte accidental.\\
\newline Norma UNEIX EN 61646:1997 sobre mòduls fotovoltaics de làmina prima per a aplicació terrestre.\\
\newline Norma UNEIX EN 61683 sobre sistemes fotovoltaics i condicionadors de potència.\\
\newline Norma UNEIX 206001 sobre mòduls fotovoltaics i criteris ecològics.\\
\newline Norma UNEIX 61215 sobre mòduls fotovoltaics de silici cristal·lí per a aplicació terrestre.\\
\newline Norma UNEIX EN 61701 sobre assaig de corrosió per boira salina de mòduls fotovoltaics.\\
\newline Norma UNEIX EN 61724 sobre monitoratge de sistemes fotovoltaics. \\
\newline Norma UNEIX EN 61725 sobre expressió analítica per als perfils solars diaris.\\
\newline Norma UNEIX EN 61829 sobre camps fotovoltaics de silici cristal·lí, que tracta la mesura en el lloc de característiques I-V.\\
\newline UNE-EN IEC 61204-3:2018 sobre fonts d'alimentació de baixa tensió amb sortida en corrent continu. La part 3 parla sobre compatibilitat electromagnètica.\\
\newline UNE-EN 61508:2011 sobre seguretat funcional dels sistemes elèctrics i electrònics, siguin o no programables. La part 1 tracta sobre requisits generals, la 3 sobre requisits del software.

\chapter{\uppercase{Condicions tècniques}}
El projectista demana que es compleixin les condicions tècniques referents a seguretat i salut, materials, fabricació, muntatge i programació i comprovació i posada en funcionament. Si es compleixen aquestes condicions el projectista preveu que tècnicament el projecte s'ha desenvolupat correctament i que es minimitzen els problemes que es puguin donar.

\section{Seguretat i salut}
És molt important assegurar unes bones condicions de seguretat en el treball i salut als treballadors involucrats en el projecte. Per això s'han de complir les lleis que s'indiquen a continuació en aquest apartat. En cas de no fer-ho l'autor del projecte s'eximeix de tota responsabilitat.\\
\newline Es tindrà especialment en compte la Llei 31/1995 de 8 de novembre sobre la prevenció de riscos laborals. A l'apartat de Normes figuren altres normatives que també s'han de complir.\\
\newline El Director d'obra podrà exigir al Contractista els documents que acreditin que el personal involucrat en la instal·lació ha formalitzat els tràmits necessaris amb la Seguretat Social.\\
\newline El Director d'obra pot exigir el cessament de l'obra en qualsevol moment per tal de garantir la integritat física dels treballadors.\\
\newline El personal que treballi en la instal·lació ha d'usar les mesures de protecció adequades per la feina que realitza, com per exemple usar guants, ulleres, casc... En moments de treballar amb equips sota tensió el personal usarà roba sense accessoris metàl·lics i evitaran utilitzar eines metàl·liques. El calçat serà aïllant.\\
\newline El Contractista mantindrà en regla una pòlissa d'Assegurança que el protegeixi a ell i els seus empleats enfront tot tipus de responsabilitats.

\section{Materials}
Els materials utilitzats per dur a terme la instal·lació fotovoltaica i per crear i instal·lar la placa electrònica estan detallats a l'Estat d'amidaments, document present en aquest projecte.\\
\newline Tots els materials i equips estan homologats segons normes UNE o similars, i són vigents per la CE. Pel muntatge es té en compte el REBT.\\
\newline El Contractista és responsable de la vigilància dels materials durant l'emmagatzematge i desenvolupament de la instal·lació.\\
\newline El Contractista haurà de complir el que indiquin els plànols, que són de caràcter contractual. Si té dubtes haurà de consultar-los amb l'autor del projecte, i comunicar allò que no pugui fer.\\
%contractista qui és?
\newline Els components electrònics han d'estar en perfecte estat i en cas de detectar alguna anomalia s'ha de comunicar al proveïdor i aconseguir components electrònics en bon estat.\\
\newline Les resistències tindran una tolerància màxima del 5\% i els condensadors tindran una tolerància màxima del 20\%. Les resistències han de poder dissipar la potència indicada pel fabricant en condicions de temperatura ambient. Els condensadors han de ser capaços de suportar la tensió que indica el fabricant.\\
\newline El plàstic per la caixa impresa amb una impressora 3D serà PLA amb 0,75 mm de diàmetre i una tolerància de diàmetre màxima de 0,01 mm.

\section{Fabricació}
En cas de detectar anomalies en el material elèctric se n'informarà al proveïdor i es prendran les mesures adequades per tal de garantir que el material utilitzat és de primera qualitat.\\
\newline El circuit imprès de les plaques electròniques es pot subcontractar. S'ha de garantir una tolerància màxima de pistes de 0,1 mm. Els diàmetres dels forats han de tenir el diàmetre indicat amb una tolerància màxima de 0,2 mm. S'ha de garantir que el circuit imprès sigui per les dues cares i les dimensions siguin les que s'indiquen al document Plànols.\\
\newline Durant la fabricació de la caixa s'ha de garantir que la impressora 3D estigui plana en una superfície estàtica. La impressora ha de disposar de blocs antivibratoris. S'ha de realitzar la impressió 3D en un espai amb una temperatura no inferior a 20 graus centígrads i no superior a 28 graus centígrads. La base de la peça no ha d'aixecar-se i desenganxar-se de la taula.\\
\newline La impressió ha de donar lloc a una peça com la dissenyada, amb una tolerància de 0,2 mm com a màxim en tot cas. En cas de no complir amb l'indicat, s'ha de tornar a imprimir la peça.


\section{Muntatge i programació}
%plaques
%soldadures
El muntatge dels panells solars fotovoltaics l'ha de fer personal qualificat i apte, amb experiència prèvies d'aquest tipus d'instal·lació. El personal ha de seguir les normes de seguretat i salut i ha de ser capaç d'entendre l'esquema unifilar que es troba al document Plànols per tal de col·locar els cables amb la secció necessària i les proteccions indicades. Els dispositius s'han de connectar després d'haver accionat l'IGA i no tenir tensió a la instal·lació.\\
\newline El muntatge mecànic s'ha de fer amb el material que marca l'Estat d'amidaments i seguint les indicacions de Plànols. S'ha d'assegurar que es munta una estructura rígida que no es deteriori per l'acció del vent. L'orientació de les plaques solars ha de ser l'adequada.\\
\newline El muntatge de la placa electrònica ha de desenvolupat per personal competent i amb experiència muntant components electrònics SMD. S'exigeix disposar d'algun tipus de lupa o microscopi per facilitar el muntatge i alhora assegurar unes bones soldadures. Una pistola d'aire calent pot facilitar el muntatge.\\
\newline Els components s'han de situar a les posicions que indiquen els Plànols. Les soldadures s'han de realitzar de forma acurada i ràpida, ajudant-se de decapant. S'ha d'evitar escalfar directament els components durant més de 10 segons de forma contínua per evitar el deteriorament.\\
\newline És d'especial importància no situar cap component ni element conductor prop de l'antena del mòdul ESP-12E, per tal d'evitar interferències i males comunicacions.\\
\newline S'ha de programar el dispositiu amb el programa facilitat.

\section{Comprovació i posada en funcionament}
Abans de la posada en funcionament cal fer una revisió i comprovar amb un multímetre o similar que la placa electrònica estigui ben soldada. Hi ha d'haver tots els components. S'ha de verificar que el mòdul és capaç de connectar-se a una xarxa Wi-Fi i que les lectures de tensió són correctes.\\
\newline A la posada en funcionament un tècnic competent ha de revisar que les connexions entre les plaques solars siguin les correctes. Els panells solars, els díodes, les seccions dels cables i els dispositius de protecció i comandament han de ser els indicats en el projecte. S'ha de comprovar que les plaques lliuren energia i que la placa electrònica es comunica correctament amb la xarxa Wi-Fi de l'habitatge, així com que les seves lectures són correctes.

\clearpage


% Table generated by Excel2LaTeX from sheet 'Hoja1'
%\begin{table}[H]
%  \centering
%    \begin{tabularx} {\textwidth} {|X|r|} \hline
%  \multicolumn{1}{|c|}{Descripció} &  \multicolumn{1}{c|}{Quantitat}\\ \hline \hline
%
 %   Placa GLC 330 W & 10 \\ \hline
%    Inversor FRONIUS Primo 3.0-1 Light 3kW & 1 \\ \hline
%    Metres cable Ethernet RJ-45 CAT 8 & 10 \\ \hline
%    Metres cable 4 m$m^2$ PVC & 45 \\ \hline
 %   Metres cable 1,5 m$m^2$ PVC & 100 \\ \hline
 %   Punteres Enghofer E 4-10, 4 m$m^2$, 10 mm & 20 \\ \hline
 %   Punteres Enghofer E 1.5-10 1,5 m$m^2$ 10 mm & 12 \\ \hline
 %   Cinta aïllant 10 m 1,6 cm & 3 \\ \hline
 %   Caixa estanca Solera CONS 100x100x55 mm & 2 \\ \hline
  %  Canal Euroquint 25,16 mm 1,5 metres & 20 \\ \hline
%    Curva canal VECAMCO & 10 \\ \hline
%    Paquet de 50 brides 200x2,6  mm & 2 \\ \hline
%    Regleta nylon 12 pols 16 mm & 4 \\ \hline
%    Premsaestopes M12 & 10 \\ \hline
%    Cargol autoroscant M4 16 mm & 12 \\ \hline
%    Tacs Fischer 072095 nylon 6x50 mm & 50 \\ \hline
%    Díode SM74611KTTR & 10 \\ \hline
%            Hores enginyer & 1 \\ \hline
%    Hores oficial de primera & 12 \\ \hline
%    Hores oficial de segona & 12 \\ \hline
%    \end{tabularx}%
%  \label{tab:addlabel}%
% \end{table}%

\chapter{\uppercase{Disposicions generals}}
A continuació es detallen aspectes de l'execució i la posada en marxa del projecte. El projectista promet complir amb els terminis exposats sota certes condicions temporals i econòmiques. A més ofereix certa garantia.

\section{Terminis d'execució}
El projectista es compromet amb què el muntatge dels panells solars, la seva instal·lació elèctrica i el seu quadre de proteccions, la realització del circuit imprès, el muntatge de la placa electrònica i la posada en funcionament es faran en un màxim de 15 dies laborables sempre i quan els distribuïdors compleixin amb el termini que marquen. El personal instal·lador ha de tenir accés a l'habitatge de 8:00 a 18:00 de dilluns a divendres.\\
\newline Si els distribuïdors han entregat el material a temps i el projecte s'ha acabat al cap de més de 15 dies laborables es pagarà al client el 10\% de l'import total del projecte un cop s'hagi acabat l'execució d'aquest.

\section{Garantia}
La garantia del projecte és de 5 anys des de la data de posada en funcionament. Aquesta garantia cobreix la substitució dels panells fotovoltaics malmesos per causes alienes al propietari. Si el rendiment d'algun dels panells és de menys d'un 90\% respecte l'inicial, se substituirà el panell en qüestió. Si algun element de protecció es malmet o es malmet la placa electrònica el projectista se'n fa responsable i resoldrà el problema cobrint els costos ell mateix.\\
\newline Un cop passats els 5 anys el propietari assumeix tota la responsabilitat i qualsevol perjudici que pugui passar a la seva instal·lació no serà cobert per la garantia.\\
\newline Els panells solars tenen més de 5 anys de garantia segons el fabricant. El client seguirà tenint la garantia del fabricant fins que hagi passat el termini que aquest indica.

\section{Forma de pagament}
El pagament del projecte i la posada en funcionament l'efectuarà el client cap a l'autor del projecte mitjançant una transferència bancària. S'ha cobrat un 50\% de l'import total del projecte abans de realitzar-lo. Un cop finalitzat el muntatge i instal·lació de les plaques solars i la placa electrònica el client haurà d'efectuar un pagament del 25\% de l'import total del projecte. El 25\% restant es pagarà al projectista després de la posada en funcionament.\\
\newline Tots els pagaments s'han d'efectuar amb un màxim de 48 hores respecte l'acabament d'aquella fase.
%es cobrarà o ja s'ha cobrat

\section{Disposicions legals}
En cas de denúncia i d'haver d'anar a judici per desacords entre client, contractista i/o projectista s'anirà als Jutjats de Girona.



\vspace*{\fill}
\noindent Llorenç Fanals Batllori\\
Graduat en Enginyeria Electrònica Industrial i Automàtica\\
%
\vspace*{16pt}

%
\noindent Girona, 25 de novembre de 2019.

\clearpage


\begin{appendices}
%\chapter{Títol de l'annex}

%\chapter{\uppercase{Càlculs}}
Pel càlcul de les seccions dels conductors cal tenir en compte els factors de simultaneïtat d'alguns elements i els factors que marca el REBT: 1,25 pel motor elèctric de més potència de la línia, tal com es detalla a la ITC-47; i 1,8 per les lluminàries amb descàrrega, tal com s'indica a la ITC-44. A l'obrador hi ha molts motors elèctric però cap llum amb descàrrega.\\
\newline
En algunes línies es considera que el factor de potència és unitari. A la realitat mai valdrà exactament 1, però sí que es preveu que tingui un valor molt semblant. Les màquines que s'han escollit tenen un factor de potència proper a l'unitari però diferent de 1.\\
\newline Per calcular la intensitat de les línies monofàsiques es fa servir la següent fórmula:
\begin{equation}
I_{linia} = \frac{P}{V*\cos(\phi)}
\end{equation}
V = 230 V\\
P és la potència que consumeixen els elements connectats a la línia\\
$\phi$ és el factor de potència\\
\newline En trifàsic, l'equació que s'utilitza és:
\begin{equation}
I_{linia} = \frac{P}{\sqrt3*V_{linia}*\cos(\phi)}
\end{equation}
$V_{linia}$ = 400 V\\
\newline És important calcular la caiguda de tensió a les línies per tal de veure si estan dimensionades correctament. La caiguda de tensió en línies d'enllumenat no pot ser superior al 3\% i en línies de força no pot ser superior al 5\% de la tensió de subministrament. La caiguda de tensió màxima a la derivació individual és de 1,5\%.\\
\newline En monofàsic:
\begin{equation}
e(\%)=\frac{P}{V}\frac{2*l}{k*S}
\end{equation}
l és la longitud ja sigui de la fase o el neutre des del comptador a l'element més llunyà\\
$k = 56 \frac{m}{mm^{2}\si{\ohm}}$\\
S és la secció del cable en m$m^2$\\
\newline
En trifàsic, l'equació que s'utilitza és:
\begin{equation}
e(\%)=\frac{P}{V}\frac{l}{k*S}
\end{equation}
\\
El dimensionament de les línies ha de permetre que les caigudes de tensió no superin els màxims indicats prèviament. Alhora, els cables han de poder admetre les intensitats calculades, per això ens guiem amb la taula de la ITC-19 del REBT. Finalment, cal comprovar que  l'interruptor magnetotèrmic té una intensitat nominal superior a la calculada per la línia i menor a l'admissible que marca la ITC-19.\\
\newline La instal·lació és trifàsica, per tant, hi ha 3 conductors de fase i un conductor de neutre. El conductor de terra transcorre per totes les línies i té una secció igual als conductors de les línies, tal com s'indica al plànol. El neutre, que arriba per l'escomesa, també és de la mateixa secció que els conductors de fase. Les màquines trifàsiques necessiten el neutre pels seus equips electrònics.\\
\newline
Per comprovar que el valor de secció de la derivació individual és correcte quan la línia va amb una terna de cables unipolars per tub cal tenir en compte un factor d'intensitat de 0,8.
\begin{equation}
I_{DI} < 0.8 * I_{max. admissible}
\end{equation}

\noindent A continuació es mostren les diferents línies de forma detallada. La secció s'ha comprovat tenint en compte les fórmules explicades i les seccions mínimes per intensitat segons marca el REBT. S'han verificat les línies pel cas més desfavorable. Les seccions dels tubs compleixen amb la ITC-21.\\
\newline Les tensions nominals són 230 V per les línies monofàsiques i 400 V per les trifàsiques. Tots els cables de les línies són de coure de 450/750 V d'aïllament. La derivació individual és de coure amb 0,6/1 kV d'aïllament. L'aïllament de la instal·lació és de 1.000 k$\si{\ohm}$.

\begin{table}[H]
\scriptsize
\begin{center}
 \begin{tabu} to \textwidth {|X[0.5, l]|X[2, l]|X[r]|X[0.6, r]|X[r]|X[r]|X[r]|X[r]|X[r]|X[r]|X[0.5,r]|}%{X | c c c} 
 \hline
 Línia& Descripció & Potència (W) & cos($\phi$) & Intensitat (A) & Distància màxima (m) & Seccions fase, neutre, terra ($mm^{2}$) & Diàmetre tub (mm) & Caiguda de tensió (\%) & Caiguda de tensió acum. (\%)\\
 \hline \hline 
DI & Derivació individual& 87.000 \ \ \ \ & 0,96 & 131,32 & 8 &3x35 + 35 + 16& 160 & 0,23 & 0,23 \\ \hline
L1 & Enllumenat habitacions i cambra& 1.370,5 & 1 & 5,96 & 55 &2,5 + 2,5 + 2,5& 20 & 2,04 & 2,27 \\ \hline
L2 & Enllumenat cuina & 1.476 \ \ \ \  & 1 & 6,42 & 47 &2,5 + 2,5 + 2,5& 20 & 1,87 & 2,10 \\ \hline 
L3 & Força oficina, menjador, màquines de buit & 7.775 \ \ \ \  & 1 & 33,80 & 51 &10 + 10 + 10& 25 & 2,68 & 2,91 \\ \hline 
L4 & Força cuina & 7.235 \ \ \ \  & 1 & 31,46 & 33 & 6 + 6 + 6& 25 & 2,67 & 2,90 \\ \hline
L5 & Rentaplats cuina & 36.000 \ \ \ \ & 0,95 & 54,92 & 54 &3x16 + 16 + 16& 32 & 1,44 & 1,67 \\ \hline 
L6 & Extractors i cambres de fred & 19.000 \ \ \ \ & 0,9 & 30,60 & 36 &3x10 + 10 + 10& 32 & 0,86 & 1,09 \\ \hline
L7 & Abatidor sala de preparació & 16.975 \ \ \ \ & 0,95 & 25,90 & 44 &3x10 + 10 + 10& 32 & 0,84 & 1,07 \\ \hline

 \end{tabu}
 \caption{Línies detallades}
\end{center}
\end{table}



%\chapter{\uppercase{Característiques}}
L'aïllament dels cables elèctrics de les línies és EPR de 450/750 V d'aïllament. Els cables transcorren en safata perforada pel passadís i dins de tubs corrugats en muntatge superficial (B2) a la resta de zones. El diàmetre d'aquests tubs s'indica en l'anterior annex. Els càlculs s'han efectuat considerant que tota la llargada dels cables va amb el muntatge B2, que és més restrictiu que la safata.\\
\newline Es fan servir els colors gris, marró i negre per les fases, el blau pel neutre i el conductor groc i verd pel terra.\\
\newline Hi ha instal·lades caixes de derivació al llarg de la instal·lació i l'enllumenat dels vestidors, l'oficina i el menjador es controla amb interruptors de paret. Els cables dels l'enllumenats que no estan en contacte amb la paret es passen pel fals sostre.\\
\newline Les màquines trifàsiques es connecten a la xarxa mitjançant una base CETAC.\\
\newline Els extractors de la cuina de l'obrador van controlats amb variadors de freqüència. La seva línia va amb un diferencial de 100 mA de classe B degut als alts corrents de fuga que poden donar-se. Aigües amunt de tots els agrupaments hi ha instal·lat un diferencial de 300 mA de sensibilitat per protegir tota la instal·lació i alhora tenir selectivitat amb el diferencials que té aigües avall.\\
\newline El maxímetre del conjunt de protecció i mesura garanteix el subministrament elèctric tot i sobrepassar la potència contractada. Si en un moment puntual es connectés alguna màquina més i pel marge donat no saltés cap interruptor magnetotèrmic però s'estigués superant la potència contractada, hi seguiria havent subministrament elèctric i l'empresa subministradora aplicaria un recàrrec a la factura.\\
\newline 
S'agrupen les línies tenint en compte si el subministrament és trifàsic o monofàsic. S'intenta, en la mesura del possible, que tots els grups tinguin potències similars. És per això que el diferencial que agrupa les 4 línies monofàsiques és de 4 pols: els 3 conductors de fase i el neutre passaran per aquest diferencial i s'alimentaran les diferents línies monofàsiques amb diferents fases. Així es pot aconseguir una instal·lació trifàsica bastant ben equilibrada.\\
\newline Els llums d'emergència són de tipus no permanent i es considera que tenen una potència de 3 W. Al disposar de bateria i només encendre's quan hi ha una emergència, no s'han tingut en compte per la previsió de càrregues.\\
\newline Per millorar el factor de potència de la derivació individual, o sigui, de tota la instal·lació, hi ha instal·lada una bateria de condensadors de 20 kVAr la qual dona una factor de potència de 0,998.




\end{appendices}


\end{spacing}
%\cite{einstein} % per fer una cita
%\printbibliography[title=Bibliografia] %ARA BÉ

\end{document}