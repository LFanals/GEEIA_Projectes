\chapter{\uppercase{Característiques}}
L'aïllament dels cables elèctrics de les línies és EPR de 450/750 V d'aïllament. Els cables transcorren en safata perforada pel passadís i dins de tubs corrugats en muntatge superficial (B2) a la resta de zones. El diàmetre d'aquests tubs s'indica en l'anterior annex. Els càlculs s'han efectuat considerant que tota la llargada dels cables va amb el muntatge B2, que és més restrictiu que la safata.\\
\newline Es fan servir els colors gris, marró i negre per les fases, el blau pel neutre i el conductor groc i verd pel terra.\\
\newline Hi ha instal·lades caixes de derivació al llarg de la instal·lació i l'enllumenat dels vestidors, l'oficina i el menjador es controla amb interruptors de paret. Els cables dels l'enllumenats que no estan en contacte amb la paret es passen pel fals sostre.\\
\newline Les màquines trifàsiques es connecten a la xarxa mitjançant una base CETAC.\\
\newline Els extractors de la cuina de l'obrador van controlats amb variadors de freqüència. La seva línia va amb un diferencial de 100 mA de classe B degut als alts corrents de fuga que poden donar-se. Aigües amunt de tots els agrupaments hi ha instal·lat un diferencial de 300 mA de sensibilitat per protegir tota la instal·lació i alhora tenir selectivitat amb el diferencials que té aigües avall.\\
\newline El maxímetre del conjunt de protecció i mesura garanteix el subministrament elèctric tot i sobrepassar la potència contractada. Si en un moment puntual es connectés alguna màquina més i pel marge donat no saltés cap interruptor magnetotèrmic però s'estigués superant la potència contractada, hi seguiria havent subministrament elèctric i l'empresa subministradora aplicaria un recàrrec a la factura.\\
\newline 
S'agrupen les línies tenint en compte si el subministrament és trifàsic o monofàsic. S'intenta, en la mesura del possible, que tots els grups tinguin potències similars. És per això que el diferencial que agrupa les 4 línies monofàsiques és de 4 pols: els 3 conductors de fase i el neutre passaran per aquest diferencial i s'alimentaran les diferents línies monofàsiques amb diferents fases. Així es pot aconseguir una instal·lació trifàsica bastant ben equilibrada.\\
\newline Els llums d'emergència són de tipus no permanent i es considera que tenen una potència de 3 W. Al disposar de bateria i només encendre's quan hi ha una emergència, no s'han tingut en compte per la previsió de càrregues.\\
\newline Per millorar el factor de potència de la derivació individual, o sigui, de tota la instal·lació, hi ha instal·lada una bateria de condensadors de 20 kVAr la qual dona una factor de potència de 0,998.

