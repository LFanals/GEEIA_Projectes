\chapter{\uppercase{Conclusió}}
L'objectiu d'aquesta memòria és legalitzar la instal·lació elèctrica d'una nau industrial de plats cuinats.
Per desenvolupar aquesta memòria s'ha seguit el Reglament Electrotècnic de Baixa Tensió. S'han consultat les instruccions tècniques que afecten a la instal·lació i les normes UNE adients.\\
\newline La ITC-30 parla de locals amb característiques especials. L'obrador és un local humit. És per això que la tensió màxima de defecte, tal com marca la ITC-18, és de 24 V. Com que es fa servir un interruptor diferencial de 300 mA de sensibilitat, cal tenir una posada a terra de menys de 80 $\si\ohm$.\\
\newline Les instruccions ITC-17, ITC-22, ITC-23 i ITC-24 s'han consultat per verificar el correcte estat dels elements del quadre elèctric.\\
\newline La ITC-28 ha estat consultada per poder determinar que l'obrador no és un local de pública concurrència, tot i tenir una sala de venta al públic.\\
\newline S'ha consultat el Vademècum d'Endesa com a guia pel conjunt de protecció i mesura. A més, ens ha servit d'ajuda per verificar alguns elements del quadre general de protecció i comandament.\\
\newline Amb tot l'indicat en aquesta memòria es considera que la instal·lació és legal i està llesta per utilitzar.

\vspace*{\fill}
\noindent Llorenç Fanals Batllori\\
Graduat en Enginyeria Electrònica Industrial i Automàtica\\
\\
\\
\\
Girona, 13 d'octubre de 2019.

\clearpage