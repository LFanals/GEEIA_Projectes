%Options > Configure Texmaker > Editor > Spelling Dictionary, per corrector en català

\documentclass[11pt, a4paper]{report}
\sloppy %per forçar el canvi de línia si la paraula supera el marge dret
\usepackage[utf8]{inputenc}

% Per utilitzar la font Helvetica (Arial)
\renewcommand{\familydefault}{\sfdefault}
\usepackage[scaled=1]{helvet}
\usepackage[helvet]{sfmath}
\everymath={\sf}
%Equacions amb una font sans_serif, \mathrm{equació aquí, són les letres les que queden inclinades}
%\usepackage{arev} % sans-serif math font
%\usepackage{helvet} % sans-serif text font


% Per comptar imatges enlloc de mostrar 1.1, 1.2...
\usepackage{chngcntr}
\counterwithout{figure}{chapter}
\counterwithout{table}{chapter}
\counterwithout{equation}{chapter}

\usepackage{graphicx}
\graphicspath{{images/}} %directori amb les imatges que volem insertar
\usepackage{float} %per forçar imatges amb H
\usepackage[normalem]{ulem} %negreta múltiples línies
%\usepackage{soul}

\usepackage{caption}
\captionsetup[figure]{labelfont={},name={Figura},labelsep=period}
\captionsetup[table]{labelfont={},name={Taula},labelsep=period}


\usepackage{subcaption}
\usepackage{amsmath} %per fòrmules matemàtiques
\usepackage[table]{xcolor} %per colors a les taules
%\usepackage{circuitikz} %per circuits electrònics
\usepackage{siunitx} %per les labels dels components
\usepackage[american,cuteinductors,smartlabels]{circuitikz} %american/european
\usepackage{tikz} %quadrícula
\usepackage[a4paper, left=30mm, right=20mm, top=25mm, bottom=25mm]{geometry} %geometria de la pàgina, 25 però per ajustar bé
\setlength{\headsep}{20pt}
\setlength{\footskip}{25pt}
%\usepackage[a4paper, width=150mm, top=25mm, bottom=25mm]{geometry} %geometria de la pàgina
\usepackage{lipsum} %per generar dummy text
\usepackage{xpatch} %per la distància entre títol i top

%Capçaleres i peus de pàgina
\usepackage{fancyhdr}
%\pagestyle{fancy} %fancy, plain
\fancypagestyle{plain}{
  \fancyhf{}% Clear header/footer
  \fancyhead[L]{\footnotesize{Obrador de plats cuinats}}
  \fancyhead[R]{\footnotesize{Instal·lació elèctrica}}
  \fancyfoot[R]{\footnotesize{\thepage}}
}
\pagestyle{plain}% Set page style to plain.

%\fancyhead{}
%\fancyhead[LO,LE]{PROJECTES}
%\fancyfoot{}
%\fancyfoot[LE,RO]{\thepage} %número de la pàgina, a la dreta
%\fancyfoot[LO, CE]{Capítol \thechapter} %nom del capítol, a l'esquerra
%\fancyfoot[CO, CE]{\href{https://github.com/LFanals}{Llorenç Fanals Batllori}} %nom de l'autor, al centre
% \renewcommand{\headrulewidth}{0.4pt}
%\renewcommand{\footrulewidth}{0.4pt}

%Per tenir el nombre de pàgina a l'inici d'un capítol
%\fancypagestyle{plain}{
%\fancyhf{}
%\renewcommand\headrulewidth{0pt}
%\fancyfoot[R]{\thepage}
%}

%Per configurar el color dels links i referències
\usepackage{color}
\usepackage{hyperref}
\hypersetup{
    colorlinks=true, %true si es volen links de colors
    linkcolor=black,  %colors de les referències internes, blue
    filecolor=magenta,      %magenta
    urlcolor=[rgb]{0,0,0}, %Color dels links d'Internet, sobre 255=2^8-1=2^0+...+2^7, {0,0.5,1}
}

%Bibliografia
\usepackage[backend=bibtex]{biblatex}
\addbibresource{bibliography.bib}

%Canviem el nom que hi ha per defecte als índex i altres, per passar-ho al català
\renewcommand{\contentsname}{Índex}
\renewcommand{\listfigurename}{Índex de figures}
\renewcommand{\chaptername}{Capítol}
\renewcommand{\appendixname}{Annex}
\renewcommand{\listtablename}{Índex de taules}
% \renewcommand{\figurename}{Figura} % ho tinc amb caption
% \captionsetup[table]{name=Taula} % ho tinc amb caption

\definecolor{color_quadricula}{HTML}{0066ff} %color per la quadrícula

% Pels circuits
%\usepackage[american]{circuitikz}
\usetikzlibrary{calc}
\ctikzset{bipoles/thickness=1}
\ctikzset{bipoles/length=1.2cm}
\ctikzset{bipoles/diode/height=.375}
\ctikzset{bipoles/diode/width=.3}
\ctikzset{tripoles/thyristor/height=.8}
\ctikzset{tripoles/thyristor/width=1}
\ctikzset{bipoles/vsourceam/height/.initial=.7}
\ctikzset{bipoles/vsourceam/width/.initial=.7}
\tikzstyle{every node}=[font=\small]
\tikzstyle{every path}=[line width=0.8pt,line cap=round,line join=round]

%Per insertar codi
\usepackage{listings}
\usepackage{color}
\definecolor{dkgreen}{rgb}{0,0.6,0}
\definecolor{gray}{rgb}{0.5,0.5,0.5}
\definecolor{mauve}{rgb}{0.58,0,0.82}

\lstset{frame=none, %tb, none
  language=Python,
  aboveskip=2mm,
  belowskip=3mm,
  showstringspaces=false,
  columns=flexible,
  basicstyle={\scriptsize\ttfamily}, %small
  numbers=none, %left
  numberstyle=\tiny\color{gray},
  keywordstyle=\color{blue},
  commentstyle=\color{dkgreen},
  stringstyle=\color{mauve},
  breaklines=true,
  breakatwhitespace=true,
  tabsize=3
}


%Per tenir el format de capítol correcte
\usepackage{titlesec}

\usepackage{etoolbox}
%\usepackage{hyperref}

%Per chapter
\titlespacing*{\chapter}{0pt}{-25pt}{11pt} %Espaiat del títol de capítol amb els altres elements
\titleformat{\chapter}[hang] %Per seguir escrivint darrera el número
{\normalfont\fontsize{11}{15}\bfseries}{\thechapter.}{0.4em}{\MakeUppercase} %\fontsize{Tamany}{Espai múltiples línies}

%Per secció
\titlespacing*{\section}{0pt}{11pt}{11pt} %Espaiat del títol de capítol amb els altres elements
\titleformat{\section}[hang] %Per seguir escrivint darrera el número
{\normalfont\fontsize{11}{15}}{\thesection.}{0.4em}{\bfseries} %\fontsize{Tamany}{Espai múltiples línies}

%Per subsecció
\titlespacing*{\subsection}{0pt}{11pt}{11pt} %Espaiat del títol de capítol amb els altres elements
\titleformat{\subsection}[hang] %Per seguir escrivint darrera el número
{\normalfont\fontsize{11}{15}}{\thesubsection.}{0.4em}{} %\fontsize{Tamany}{Espai múltiples línies}

%Per paràgraf
\titlespacing*{\paragraph}{0pt}{0pt}{22pt} %Espaiat del títol de capítol amb els altres elements
\titleformat{\paragraph}[hang] %Per seguir escrivint darrera el número
{\normalfont\fontsize{11}{15}}{}{}{} %\fontsize{Tamany}{Espai múltiples línies}


%Interlineat, 1.2*1.25=1.5
\linespread{1.25}

%Espaiat entre paràgrafs
%\setlength{\parskip}{22pt}
 

\makeatletter
\def\tagform@#1{\maketag@@@{(\ignorespaces{Eq.~#1}\unskip)}}
\makeatother



%Per no tenir negreta a l'index
\usepackage{etoolbox}% http://ctan.org/pkg/etoolbox
\makeatletter
\patchcmd{\l@chapter}{\bfseries}{}{}{}% \patchcmd{<cmd>}{<search>}{<replace>}{<success>}{<failure>}
\makeatother

%Per tenir punts a l'índex
\makeatletter
\renewcommand*\l@chapter{\@dottedtocline{0}{0em}{1.5em}}
\makeatother

%Per taula que adapta bé els espais
\usepackage{tabularx}
\usepackage{tabu} % http://mirrors.ibiblio.org/CTAN/macros/latex/contrib/tabu/tabu.pdf
\tabulinesep = 1mm
\usepackage[font=footnotesize]{caption} %Captions de les figures més petites

%Appendix
\usepackage[]{appendix} %toc, page

%Alinear al separador decimal amb espais
\usepackage{setspace}
\renewcommand*{\arraystretch}{1.25}


%-------------------------------------------------------------------------------------------------------------
%-------------------------------------------------------------------------------------------------------------
%-------------------------------------------------------------------------------------------------------------
%-------------------------------------------------------------------------------------------------------------

\begin{document}
\pagenumbering{Roman}


%\begin{titlepage}
	\begin{center}
		\vspace*{1cm}
		
		\Huge
		\textbf{Document per Projectes}
		
		\vspace{0.5cm}
		\LARGE
		Adaptat a \LaTeX
	
		\vspace{1.5cm}
		
		\textbf{Llorenç Fanals Batllori}
		
		\vfill
		
		\small
		%\uppercase{Un treball lliurat a la Universitat - en compliment dels requisits pel grau en -}\\
		% TFG
		
		\vspace{1cm}
		
		%\includegraphics[scale=width=0.4\textwidth]{images/a_graph}
	\end{center}
	
	\begin{flushright}
	\large	
	Departament o grup de recerca\\
	UdG\\
	%País\\
	28/08/2019
	\end{flushright}
	


\end{titlepage}

%\thispagestyle{plain}

\begin{center}
	\large
	\textbf{Informe}
	
	\vspace{0.4cm}
	\large
	Descripció
	
	\vspace{0.4cm}
	\textbf{Llorenç Fanals Batllori}
	
	\vspace{0.9cm}
	\textbf{Abstract}
\end{center}
\lipsum[1]




%\chapter*{Dedicacions}
%Dedico aquest treball a -

%\chapter*{Agraïments}
%Vull agraïr a \\

\cleardoublepage\pagenumbering{arabic}

\begin{spacing}{2}
\tableofcontents
\end{spacing}
%\listoffigures %No fa falta crec
%\listoftables %No fa falta crec
\begin{spacing}{1.5}

\chapter{\uppercase{Generalitats}}

Amb aquest projectes es vol legalitzar la instal·lació elèctrica d'un obrador de menjar preparat situat en una nau industrial de la Bisbal d'Empordà, al Carrer Ramon Serradell número 27. Un gran volum de la producció es destina a càterings tot i que també es ven al detall a la sala de venta al públic. Es preveu que hi hagi uns 10 treballadors alhora durant la jornada laboral.\\
\newline
El local dins el qual es preveu que es desenvolupi l'activitat és una nau industrial de 40 m x 20 m, o sigui 800 $m^{2}$ de superfície. S'utilitza únicament la planta baixa, es cobreix totalment amb un fals sostre de 3,5 m en algunes zones i 2,5 m en altres. La nau disposa d'una cuina, cambres de fred per entrada d'aliments i sortida de menjar, una oficina, una recepció, uns vestidors amb dutxes i lavabo i una sala de venta al públic, principalment.\\
\newline
La venda al detall de menjar és en una zona de la qual el públic només té accés a 19.43 $m^{2}$. El Reglament Electrotècnic de Baixa Tensió, a la ITC-28, diu que es comptabilitza una persona per 0,8 $m^{2}$, i que a partir de 50 persones el local es considera de pública concurrència. Com que en l'activitat la superfície que es preveu accessible al públic general és de menys de 40 $m^{2}$, el local no és de pública concurrència.\\
\newline La cuina de l'obrador es considera una zona amb presència d'aigua. Les cambres de fred, on es produeixen condensacions de vapor d'aigua, també es poden considerar zones humides. La ITC-30 detalla els punts que s'han de tenir en compte per aquestes zones.\\
\newline
Per verificar la instal·lació es seguirà el Reglament Electrotècnic de Baixa Tensió, aprovat el 2 d'agost pel Real Decreto 842/2002, amb les corresponents modificacions i ampliacions aplicades fins a la data d'aquest projecte. Es tenen en compte les normes UNE que afecten a la instal·lació. S'ha consultat el Vademècum d'Endesa per dimensionar correctament la caixa de protecció i mesura.

%\section{Característiques específiques}



\clearpage
\chapter{\uppercase{Previsió de càrregues}}
En aquest capítol es calcula la potència instal·lada, la potència útil i la potència a contractar. Per fer-ho agrupem les càrregues elèctriques segons siguin d'il·luminació o força.
\section{Enllumenat}
L'enllumenat que s'utilitza són bombetes LED, fluorescents LED i llums d'emergència tipus LED. 
\newline Els llums d'emergència són del tipus no permanent, és a dir, disposen d'una bateria i només s'encenen quan la tensió baixa del 70\% respecte la tensió nominal.  Per la previsió de càrregues no es tenen en compte perquè un cop carregats el seu consum és nul.
\begin{table}[H]
\small
\begin{center}
 \begin{tabularx}{\textwidth}{|l|X|r|r|}%{c | c c c} 
 \hline
 Zona & Elements i potència & Quantitat & Potència \\
 \hline
 Recepció & Fluorescent 18 W & 4 & 72 \ \ \ W\\
 Oficina & Fluorescents 18 W & 8 & 144 \ \ \ W\\
 Menjador & Fluorescents 18 W & 8 & 144 \ \ \ W\\
 Vestidors & Fluorescent 18 W, bombeta 5,5 W & 4, 4 & 94 \ \ \ W\\
 Caldera & Bombeta 5,5 W & 1 & 5,5 W\\
 Sortida de cuinats & Fluorescent 18 W & 6 & 108 \ \ \ W\\
 Cambres i magatzem & Fluorescent 18 W & 20 & 360 \ \ \ W\\
 Entrada d'aliments & Fluorescent 18 W & 8 & 144 \ \ \ W\\
 Arxiu & Fluorescent 18 W & 1 & 18 \ \ \ W\\
 Sala de preparació & Fluorescent 18 W & 8 & 144 \ \ \ W\\
 Sala de venta al públic & Fluorescent 18 W & 6 & 108 \ \ \ W\\
 Sala de residus & Fluorescent 18 W & 4 & 72 \ \ \ W\\
 Sala de productes de neteja & Fluorescent 18 W & 1 & 18 \ \ \ W\\
 Cuina & Fluorescent 18 W & 64 & 1.152 \ \ \ W\\
 
  

 \hline
 Total enllumenat & & & 2.583,5 W\\
 \hline
 \end{tabularx}
 \caption{Previsió de càrregues d'enllumenats}
\end{center}
\end{table}
%
%
%
%
%
\section{Força}
La potència necessària per les màquines és molt més elevada que la dels enllumenats. Les màquines de fred per les cambres, les neveres i els congeladors consumeixen potències considerables. Els rentaplats i els extractors de la cuina són les màquines que tenen consums més elevats.\\
%
\begin{table}[H]
\small
\begin{center}
 \begin{tabularx}{\textwidth}{|X|r|r|r|}%{c | c c c} 
 \hline
 Descripció & Potència unitària & Quantitat & Potència \\
 \hline
 Rentaplats & 12.000 W & 3 &  36.000 W \\
 Màquina de buit & 1.100 W & 3 &  3.300 W \\
 Forn & 2.400 W & 4 &  9.600 W\\ 
 Congelador & 1.262 W & 2 &  2.524 W \\
 Nevera & 1.000 W & 3 &  3.000 W \\
 Fregidora & 200 W & 4 &  800 W \\
 Abatidor & 13.580 W & 1 &  13.580 W\\ 
 Caldera & 100 W & 1 &  100 W \\
 Màquines de fred per congelador & 3.200 W & 2 & 6.400 W \\
 Màquines de fred per nevera & 1.800 W & 2 &  3.600 W \\
 Extractor & 4.000 W & 2 & 8.000 W \\
 Ordinador & 500 W & 3 &  1.500 W \\
 Microones & 800 W & 5 & 4.000 W \\
 Extractors lavabo & 20 W & 4 & 80 W\\
 \hline
 Total força & & &  92.484 W\\
 \hline
 \end{tabularx}
 \caption{Previsió de càrregues de força}
\end{center}
\end{table}
%
%
%


\section{Resum de previsió de càrregues}
Podem sumar les potències dels apartats anteriors per tenir la potència instal·lada.
\begin{table}[H]
\small
\begin{center}
 \begin{tabular}{|l|r|}%{c | c c c} [table-format = 10.5]
 \hline
 Descripció  & Potència \\
 \hline
  Enllumenats & 2.583,5 W \\
  Força &  92.484 \ \ \ W \\ 
 \hline
 Potència instal·lada & 95.067,5 W\\
 \hline
 \end{tabular}
 \caption{Potència instal·lada}
\end{center}
\end{table}

\noindent En aquesta instal·lació hi ha bastants equips de fred que en un règim normal consumeixen una potència menor a la indicada pel fabricant. A la sala de preparació hi ha 3 màquines de buit que funcionen en pocs períodes de curta duració durant la jornada laboral. Els rentaplats, que tenen consums d'electricitat molt alts, només s'engeguen al final del dia, moment pel qual els forns, que també tenen consums elevats, no estan en funcionament. \\
\newline Amb això explicat, es decideix aplicar un factor de simultaneïtat a tota la instal·lació de 0,9.
%
\begin{table}[H]
\begin{center}
 \begin{tabular}{|l|r|}%{c | c c c} 
 \hline
  Potència instal·lada & 95.067,5 \ \ W \\
  Coeficient de simultaneïtat &  0,9 \ \ \ \ \ \  \\ 
 \hline
 Potència útil & 85.560,3 \ \ W\\
 \hline
 \end{tabular}
 \caption{Potència contractada}
\end{center}
\end{table}
%
\noindent La potència útil és de 85.560,3 W. La potència a contractar és de 87 kW.

\clearpage
\chapter{\uppercase{Condicions del subministrament}}
Es preveu una tensió de servei trifàsica de 230 V de fase. Es disposarà del conductor de neutre i la freqüència serà de 50 Hz.\\
\newline
3 x 230/400 V a 50 Hz.\\
\newline
La potència màxima a contractar és de 89 kW.\\
\newline S'opta per escollir una tarifa 3.0 que és un tipus de tarifa que es pot contractar en subministres de baixa tensió i més de 15 kW de potència contractada. Aquesta tarifa té 3 períodes diferenciats: P1 (Punta), P2 (Pla) i P3 (Vall). Es poden contractar potències diferents a cada un dels períodes, només cal que un dels 3 períodes superi els 15 kW. Per fer servir aquesta tarifa cal instal·lar un maxímetre, que registrarà la potència consumida. Una característica molt interessant és que no es talla el subministrament si sobrepassem la potència contractada en el període en què estiguem.
\clearpage
\chapter{\uppercase{Empresa subministradora}}
L'empresa subministradora que dona servei a l'obrador és FECSA ENDESA, S.A., amb oficina a Carrer del Riu Güell, número 12, a Girona, amb codi postal 17001. El telèfon d'atenció al públic és 800760909.
\clearpage
\chapter{\uppercase{Escomesa i caixa general de protecció}}
La ITC-11 del REBT defineix l'escomesa com la part de la instal·lació de la xarxa de distribució que alimenta la caixa general de protecció o una unitat funcional equivalent.\\
\newline L'escomesa és subterrània i està en derivació. Prové d'una caixa de seccionament que instal·la l'empresa distribuïdora. La caiguda de tensió a l'escomesa és la que l'empresa subministradora té establerta en el seu repartiment de caigudes de tensió, hem mesurat que és de 0,1\%. L'escomesa transcorre a 70 centímetres de profunditat, compleix la ITC-07 per la instal·lació de cables aïllats directament enterrats. La seva longitud és de 1 m. No es creua amb altres cables o canonades. La secció dels conductors de fase i el neutre l'escomesa són de 50 m$m^2$. El tub, de 160 mm de diàmetre, compleix amb l'apartat de cables enterrats de la ITC-21.\\
\newline
El REBT defineix, a la ITC-13, les caixes generals de protecció com aquells elements de protecció de les línies generals d'alimentació. A la ITC-11 es cita que d'aquesta instrucció en endavant el Reglament utilitza la paraula "caixa general de protecció" o "CGP" per fer referència, també, a les unitats funcionals equivalents.\\
\newline 
La instal·lació compta amb un TMF10 com a conjunt de protecció i mesura. Una part d'aquest equip conté fusibles que fan la mateixa funció que una caixa general de protecció, són una unitat funcional equivalent. En aquesta instal·lació, doncs, no hi ha CGP com a tal.\\
\newline La potència contractada és de 87 kW. Els fusibles són del tipus gG, de 250 A i amb bases de tamany BUC 1.

\clearpage
\chapter{\uppercase{Línia repartidora}}
La línia repartidora, també anomenada Línia General d'Alimentació, és el conductor entre els fusibles de l'entrada de la Caixa General de Protecció i els comptadors, tal com indica la ITC-14.\\
\newline En el nostre cas, en què optem pel TMF10 i es dóna servei a un sol usuari, aquesta línia no existeix.
\clearpage
\chapter{\uppercase{Conjunt de protecció i mesura}}
El conjunt de protecció i mesura té la funció de protegir tota la instal·lació que té aigües avall i mesurar els consums d'aquesta, segons indica la ITC-16. Pot contenir fusibles i sempre conté un interruptor de protecció i corrent regulable, anomenat ICP-M.\\
\newline El conjunt de protecció i mesura utilitzat a la instal·lació és el TMF10, la versió amb un corrent entre 80 A i 160 A. La potència contractada de la nostra instal·lació és de 87 kW. Al Vademècum, mirem aquesta columna de 87 kW.\\
\newline
El conjunt de protecció i mesura està col·locat fora de la nau, encastat en obra, en un nínxol. La porta del nínxol és metàl·lica i té 20 cm de separació per costat i costat. El conjunt de protecció i mesura en sí està situat 60 cm sobre el nivell del terra. Els tubs pels quals li arriba l'escomesa subterrània són de polietilè de 160 mm de diàmetre.\\
\newline
A l'entrada del conjunt de protecció hi trobem l'escomesa. Aquesta escomesa té en sèrie uns fusibles, un per cada una de les 3 fases, i seguidament es troben uns embarrats. Sobre aquests embarrats hi ha els transformadors d'intensitat que permetran conèixer el consum. La sortida del conjunt de protecció i mesura és la derivació individual, la qual dimensionem en el següent capítol.\\
\newline
Com a element de protecció s'instal·la ICP-M de 160 A de corrent assignat, 10 kA de poder de tall, 125 A per la part tèrmica i 625 A per la part magnètica.\\
\newline Els fusibles són de tipus gG, de 250 A i les seves bases del tipus BUC 1.\\
\newline Els demés elements de protecció que apareixen a la taula del Vademècum es detallen més endavant, formen part del quadre general de protecció i comandament.\\
\newline Els tranformadors d'intensitat, que es col·loquen sobre els embarrats i serveixen per disminuir la intensitat i així mesurar-la més fàcilment, tenen una relació de transformació de corrent de 100/5.\\
\newline El cablatge és de coure, de 20x5 + 15x5.\\
\newline El mòdul TMF10 té un maxímetre. Aquest equip és un semblant d'un comptador digital. Llegeix les intensitats que li arriben dels transformadors d'intensitat i així és capaç de determinar la potència. El maxímetre, en cap cas, talla el subministrament elèctric.


\clearpage
\chapter{\uppercase{Derivació individual}}
Tal com marca el Reglament Electrotècnic de Baixa Tensió a la ITC-15, la derivació individual és aquella part de la instal·lació elèctrica que comença al final de la línia general d'alimentació i subministra energia elèctrica a la instal·lació de l'usuari. No tenim LGA, la derivació individual comença al conjunt de protecció i mesura.\\
\newline Pels conductors de fase s'utilitza recobriment EPR. Aquests conductors tenen una secció de 35 $mm^2$, compleixen amb la ITC-07 del REBT, i s'agrupen en un sol cable tetrapolar, amb tensió d'aïllament 0,6/1 kV i no propagador de flama. La secció del conductor de neutre de la instal·lació és de 16 $mm^2$, compleix amb el reglament. A l'annex de càlculs es donen més detalls.\\
\newline  Tal com marca la ITC-15, la caiguda de tensió màxima en derivacions individuals per subministres a un únic usuari és de 1,5\% respecte la tensió de servei. La intensitat que passa per fase pot arribar a ser de 130 A i s'utilitzen conductors de coure. A la derivació individual, de 8 m, la caiguda de tensió és del 0,23\%: compleix amb la caiguda de tensió permesa. El tub conductor té una diàmetre de 160 mm, és correcte segons la ITC-21. L'aïllament de la instal·lació de la derivació individual és de 1000 k$\si{\ohm}$.




\clearpage
\chapter{\uppercase{Interruptor de control de potència i interruptor general automàtic.}}
La ITC-17 del REBT contempla la utilització d'un interruptor de control de potència en vivendes i locals comercials, situat dins una caixa, la qual pot estar dins el mateix quadre elèctric en què es troben els demés elements de protecció i comandament. Actualment, degut als comptadors digitals, estan en desús.\\
\newline L'interruptor de control de potència és inexistent a la instal·lació elèctrica proposada. El càlcul de potència el fa el maxímetre i en cap cas talla el subministrament elèctric.\\
\newline En la nostra instal·lació disposem de l'ICP-M, situat al CPM, per tallar el subministrament, com a mesura de protecció. Aquest ICP-M té una intensitat nominal de 160 A, un poder de tall de 10 kA, una intensitat de tall tèrmic de 125 A i 4 pols. La intensitat de tall magnètic de l'ICP-M és de 625 A. El tall magnètic pot actuar en menys de 0,02 segons.\\
\newline El quadre general de protecció i comandament conté un IGA de 125 A que també té la funció de protegir la instal·lació. L'IGA té 4 pols i un poder de tall de 10 kA, compleix amb la ITC-17, que estableix un mínim de 4500 A de poder de tall.

\clearpage
\chapter{\uppercase{Quadre general de protecció i comandament}}
La ITC-17 explica les característiques del quadre general de protecció i comandament, el qual conté els elements de protecció i comandament necessaris per la instal·lació. És obligatori instal·lar un interruptor general automàtic (IGA), interruptors de tall per sobrecàrregues i curtcircuits per cada circuit interior, un dispositiu de protecció contra sobretensions si així ho indica la ITC-23 i un diferencial per cada grup de línies. És opcional instal·lar un interruptor diferencial general si tots els circuits estan protegits amb algun diferencial aigües amunt.\\
\newline El quadre general de protecció i comandament ha de tenir, com a mínim, un grau IP30 segons marca la norma UNE 20 324 i un grau IK07 segons la norma UNE-EN 50102.\\
\newline El Vademècum dona informació del diferencial general, l'IGA i l'interruptor contra sobretensions.

\section{Protecció contra sobreintensitats}
La ITC-22 del REBT indica que tots els circuits han d'estar protegits contra els efectes de les sobreintensitats que poden presentar-se. La interrupció del circuit s'ha de realitzar en un temps convenient. Alguns factors causants de sobreintensitats són les sobrecàrregues, els curtcircuits i les descàrregues elèctriques atmosfèriques.\\
\newline A la nostra instal·lació es protegeix cada línia contra sobreintensitats amb el magnetotèrmic normalitzat més proper a la intensitat prevista i que la superi. Els interruptors magnetotèrmics s'escullen monofàsics o trifàsics segons com sigui la línia que han de protegir.



\section{Protecció contra contactes indirectes}
Tal com comenta la ITC-24, el tall automàtic de l'alimentació després de l'aparició d'una fallada té la funció d'impedir que una tensió de contacte de valor suficient es mantingui durant un temps tal que podria donar lloc a riscos.\\
\newline La protecció contra contactes indirectes es fa amb interruptors diferencials de 30 mA de sensibilitat per la majoria de línies. La línia dels extractors i algunes màquines de fred van amb un diferencial 100 mA de sensibilitat per evitar obrir el circuit quan hi ha fugues de corrent generades pels equips electrònics i els seus filtres. Aigües amunt de les línies hi ha un diferencial de 300 mA de sensibilitat, tal com marca el Vademècum.\\
\newline Tots els diferencials són capaços de suportar la intensitat màxima prevista de les càrregues tenen aigües avall. El paràmetre que indica la intensitat màxima per la qual el fabricant garanteix que el diferencial actuarà és la intensitat nominal ($I_{N}$).
\newline Els interruptors diferencials s'escullen tots trifàsics degut a què 3 protegeixen línies trifàsiques i un protegeix línies monofàsiques que s'alimenten amb diferents fases.


\section{Protecció contra sobretensions}
La ITC-23 tracta sobre la protecció de les instal·lacions elèctriques contra les sobretensions transitòries que es transmeten a les xarxes de distribució i s'originen, fonamentalment, com a conseqüència de descàrregues atmosfèriques, commutacions de les xarxes i defectes d'aquestes.
S'instal·la un interruptor contra sobretensions permanents i transitòries per protegir el circuit de la possibilitat de què la tensió subministrada per l'empresa distribuïdora fos excessiva. També quedarem protegits contra pics de tensió de curta durada provocats per descàrregues elèctriques.

\clearpage
\chapter{\uppercase{Distribució general}}
%Càlculs
\section{Coeficients de simultaneïtat i arrencada}
Pel càlcul de les seccions dels conductors cal tenir en compte els factors de simultaneïtat d'alguns elements i els factors que marca el REBT: 1,25 pel motor elèctric de més potència de la línia, tal com es detalla a la ITC-47; i 1,8 per les lluminàries amb descàrrega, tal com s'indica a la ITC-44.\\
\newline Màquines com els extractors, les màquines buit, les màquines de fred, les neveres i congeladors disposen de motors elèctrics. El rentaplats també té motor elèctric, però la majoria de potència que consumeix ho fa per escalfar aigua, així que no s'aplica cap factor.\\
\newline En la nostra instal·lació no hi ha cap enllumenat amb descàrrega, totes les lluminàries són de tipus LED.
%
%
%
%
\section{Agrupament de línies}
S'agrupen les línies tenint en compte si el subministrament és trifàsic o monofàsic. S'intenta, en la mesura del possible, que tots els grups tinguin potències similars. És per això que el diferencial que agrupa les 4 línies monofàsiques és de 4 pols: els 3 conductors de fase passaran per aquest diferencial i s'alimentaran les diferents línies monofàsiques amb diferents fases. Així es pot aconseguir una instal·lació trifàsica bastant ben equilibrada.\\
\newline Es preveu utilitzar safata foradada en certs trams i muntatge superficial en altres. Pel REBT s'escull el muntatge B2, considerant que només tenim aquest muntatge, és el cas més desfavorable. L'aïllament serà EPR.
%
% \small
\subsection{Agrupament 1}
Comprèn les línies L1, L2, L3 i L4. Són monofàsiques però com s'ha comentat s'alimenten amb fases diferents. L1 i L2 són línies d'enllumenat i L3 i L4 són línies de força per màquines monofàsiques. El factor de potència d'aquestes línies es considera 1.
%
\begin{table}[H]
\small
\begin{center}
 \begin{tabu} to \textwidth {|X[0.5, l]|X[2, l]|X[r]|X[r]|X[r]|X[r]|X[r]|X[r]|X[0.5,r]|}%{X | c c c} 
 \hline
 Línia& Descripció & Potència (W) & Intensitat (A) & Distància màxima (m) & Secció ($mm^{2}$) & Caiguda de Tensió (\%) & REBT (A)& PIA (A)\\
 \hline \hline 
L1 & Enllumenat habitacions i cambra& 1.370,5 & 5,96 & 55 &2x2,5 + 2,5& 2,04 & 25 & 10 \\ \hline
L2 & Enllumenat cuina & 1.476 \ \ \ \  & 6,42 & 47 &2x2,5 + 2,5& 1,87 & 25 & 10 \\ \hline 
L3 & Força oficina, menjador, màquines de buit & 7.775 \ \ \ \  & 33,80 & 51 &2x10 + 10& 2,68 & 60 & 40 \\ \hline 
L4 & Força cuina & 7.235 \ \ \ \  & 31,46 & 33 &2x6 + 6& 2,67 & 44 & 40 \\ \hline 
 \hline
 Total & & 8.900,5 & 38,35 & & & & & 80 \\
 \hline
 \end{tabu}
 \caption{Agrupament 1}
\end{center}
\end{table}


\subsection{Agrupament 2}
Aquest agrupament és purament per la línia que alimenta els rentaplats de la cuina, la L5. Aquests rentaplats consumeixen molta potència perquè escalfen aigua; el seu motor consumeix poca potència, per això no se'ls aplica cap factor d'arrencada. El factor de potència és de 0,95. 
\begin{table}[H]
\small
\begin{center}
 \begin{tabu} to \textwidth {|X[0.5, l]|X[2, l]|X[r]|X[r]|X[r]|X[r]|X[r]|X[r]|X[0.5,r]|}%{X | c c c} 
 \hline
 Línia& Descripció & Potència (W) & Intensitat (A) & Distància màxima (m) & Secció ($mm^{2}$) & Caiguda de Tensió (\%) & REBT (A)& PIA (A)\\
 \hline \hline 
L5 & Rentaplats cuina & 36.000 & 54,92 & 54 &4x16 + 16& 1,44 & 70 & 63 \\
 \hline
 \end{tabu}
 \caption{Agrupament 2}
\end{center}
\end{table}

\subsection{Agrupament 3}
Aquest agrupament alimenta els dos extractors, que van amb motor elèctric, i els equips de les cambres de fred. Per tant, s'aplica el factor d'arrencada de 1,25 a un dels extractors, ja que aquests consumeixen més que els equips de fred. El factor de potència és de 0,9.
\begin{table}[H]
\small
\begin{center}
 \begin{tabu} to \textwidth {|X[0.5, l]|X[2, l]|X[r]|X[r]|X[r]|X[r]|X[r]|X[r]|X[0.5,r]|}%{X | c c c} 
 \hline
 Línia& Descripció & Potència (W) & Intensitat (A) & Distància màxima (m) & Secció ($mm^{2}$) & Caiguda de Tensió (\%) & REBT (A)& PIA (A)\\
 \hline \hline 
L6 & Extractors i cambres de fred & 19.000 & 30,60 & 36 &4x10 + 10& 0,86 & 52 & 40 \\
 \hline
 \end{tabu}
 \caption{Agrupament 3}
\end{center}
\end{table}

\subsection{Agrupament 4}
Aquest agrupament alimenta l'abatidor de la sala de preparació. S'aplica el factor d'arrencada de 1,25. El seu factor de potència és de 0,95.
\begin{table}[H]
\small
\begin{center}
 \begin{tabu} to \textwidth {|X[0.5, l]|X[2, l]|X[r]|X[r]|X[r]|X[r]|X[r]|X[r]|X[0.5,r]|}%{X | c c c} 
 \hline
 Línia& Descripció & Potència (W) & Intensitat (A) & Distància màxima (m) & Secció ($mm^{2}$) & Caiguda de Tensió (\%) & REBT (A)& PIA (A)\\
 \hline \hline 
L7 & Abatidor sala de preparació & 16.975 & 25,90 & 44 &4x10 + 10& 0,84 & 52 & 40 \\
 \hline
 \end{tabu}
 \caption{Agrupament 4}
\end{center}
\end{table}
%
%

\begin{table}[H]
\small
\begin{center}
 \begin{tabu} to \textwidth {|X[3, l]|X[2, l]|X[r]|X[r]|X[r]|X[r]|}%{X | c c c} 
 \hline
 Línia & Descripció & Intensitat (A) & Intensitat PIA (A) & Classe & Nombre de pols \\
 \hline \hline 
L1 & Enllumenat habitacions i cambres & 5,61 & 10 & C & 2 \\ \hline
L2 & Enllumenat cuina & 6,42 & 10 & C & 2 \\ \hline
L3 & Força oficina, menjador, màquines de buit & 33,80 & 40 & C & 2 \\ \hline
L4 & Força cuina & 31,45 & 40 & C & 2 \\ \hline
L5 & Rentaplats cuina & 54,92 & 64 & C & 4 \\ \hline
L6 & Extractors i cambres de fred & 30,60 & 40 & D & 4 \\ \hline
L7 & Abatidor sala de preparació & 25,9 & 40 & C & 4 \\ \hline \hline
L1+L2+L3+L4+L5+L6+L7 & IGA & 130,08 & 160 & C & 4 \\ \hline

 \end{tabu}
 \caption{Proteccions magnetotèrmiques de les línies}
\end{center}
\end{table}
\noindent L'últim magnetotèrmic de la taula és un IGA de 160 A d'intensitat nominal, classe C i 4 pols; situat aigües amunt de les línies. Les màquines i l'enllumenat poden arribar a estirar 130,08 A. El Vademècum indica que cal protegir la instal·lació per corrents superiors a aquest, els quals podrien danyar la instal·lació.


\begin{table}[H]
\small
\begin{center}
 \begin{tabu} to \textwidth {|X[3, l]|X[2, l]|X[r]|X[r]|X[1,r]|X[1,r]|X[r]|}%{X | c c c} 
 \hline
 Línia & Descripció & Intensitat (A) & Intensitat nominal (A) & Sens. (mA) & Classe & Nombre de pols \\
 \hline \hline 
L1+L2+L3+L4 & Enllumenat i força monofàsics & 77,3 & 80 & 30 & A & 4 \\ \hline
L5 & Rentaplats cuina & 54,92 & 63 & 30 & A & 4 \\ \hline
L6 & Extractors i cambres de fred & 30,60 & 40 & 100 & B & 4 \\ \hline
L7 & Abatidor de la sala de preparació & 25,9 & 40 & 30 & A & 4 \\ \hline \hline
L1+L2+L3+L4+L5+L6+L7 & Totes les línies & 130,08 & 160 & 300 & A & 4 \\ \hline
 \end{tabu}
 \caption{Proteccions diferencials de la instal·lació}
\end{center}
\end{table}
\noindent L'últim diferencial de la taula és general, està en sèrie amb l'IGA i, tot i que no obligatori, decidim instal·lar-lo seguint el Vademècum com a recomanació.




\clearpage

\chapter{\uppercase{Posada a terra}}
La ITC-18 del REBT comenta que les posades a terra s'estableixen, principalment, per limitar la tensió que, respecte el terra, poden presentar les carcasses metàl·liques dels equips. Això fa disminuir el risc que poden suposar aquest tipus d'avaries i amb l'actuació de les proteccions adients es pot eliminar aquest risc.\\
\newline Seguint el criteri de la ITC-30, la cuina de l'obrador és una zona humida, on, a part de l'aigua utilitzada per cuinar, serà molt habitual netejar amb aigua abundant. No totes les zones de l'obrador seran humides, però només que una ho sigui, ja es pot dir que la tensió màxima de defecte és de 24 V, tal com marca la ITC-18.\\
\newline Tenim tres diferencials de sensibilitats diferents en la nostra instal·lació: 30 mA, 100 mA i 300 mA. El més restrictiu és el de 300 mA, per tant, el que escollim per fel el càlcul. 
\begin{equation}
R_{terra} < 80 \ \si{\ohm}
\end{equation}
El règim de neutre de la instal·lació és TT.\\
\newline La instal·lació de posada a terra està feta amb 15 planxes metàl·liques de 1 m per 0,5 m enterrades sota els terrenys de la parcel·la. \\
\newline La ITC-18 diu que la posada a terra es revisarà com a mínim un cop a l'any en l'època en què el terreny estigui més sec per tal d'eliminar els possibles defectes que hi hagi. L'última mesura efectuada de la posada a terra es va fer el dia 2 d'agost de 2019 (02/08/2019) i indicava un valor de resistència de $15$ \si{\ohm}.

\clearpage
\chapter{\uppercase{Conclusió}}
L'objectiu d'aquesta memòria és legalitzar la instal·lació elèctrica d'una nau industrial de plats cuinats.
Per desenvolupar aquesta memòria s'ha seguit el Reglament Electrotècnic de Baixa Tensió. S'han consultat les instruccions tècniques que afecten a la instal·lació i les normes UNE adients.\\
\newline La ITC-30 parla de locals amb característiques especials. L'obrador és un local humit. És per això que la tensió màxima de defecte, tal com marca la ITC-18, és de 24 V. Com que es fa servir un interruptor diferencial de 300 mA de sensibilitat, cal tenir una posada a terra de menys de 80 $\si\ohm$.\\
\newline Les instruccions ITC-17, ITC-22, ITC-23 i ITC-24 s'han consultat per verificar el correcte estat dels elements del quadre elèctric.\\
\newline La ITC-28 ha estat consultada per poder determinar que l'obrador no és un local de pública concurrència, tot i tenir una sala de venta al públic.\\
\newline S'ha consultat el Vademècum d'Endesa com a guia pel conjunt de protecció i mesura. A més, ens ha servit d'ajuda per verificar alguns elements del quadre general de protecció i comandament.\\
\newline Amb tot l'indicat en aquesta memòria es considera que la instal·lació és legal i està llesta per utilitzar.

\vspace*{\fill}
\noindent Llorenç Fanals Batllori\\
Graduat en Enginyeria Electrònica Industrial i Automàtica\\
\\
\\
\\
Girona, 13 d'octubre de 2019.

\clearpage

\begin{appendices}
%\chapter{Títol de l'annex}

\chapter{\uppercase{Càlculs}}
Pel càlcul de les seccions dels conductors cal tenir en compte els factors de simultaneïtat d'alguns elements i els factors que marca el REBT: 1,25 pel motor elèctric de més potència de la línia, tal com es detalla a la ITC-47; i 1,8 per les lluminàries amb descàrrega, tal com s'indica a la ITC-44. A l'obrador hi ha molts motors elèctric però cap llum amb descàrrega.\\
\newline
En algunes línies es considera que el factor de potència és unitari. A la realitat mai valdrà exactament 1, però sí que es preveu que tingui un valor molt semblant. Les màquines que s'han escollit tenen un factor de potència proper a l'unitari però diferent de 1.\\
\newline Per calcular la intensitat de les línies monofàsiques es fa servir la següent fórmula:
\begin{equation}
I_{linia} = \frac{P}{V*\cos(\phi)}
\end{equation}
V = 230 V\\
P és la potència que consumeixen els elements connectats a la línia\\
$\phi$ és el factor de potència\\
\newline En trifàsic, l'equació que s'utilitza és:
\begin{equation}
I_{linia} = \frac{P}{\sqrt3*V_{linia}*\cos(\phi)}
\end{equation}
$V_{linia}$ = 400 V\\
\newline És important calcular la caiguda de tensió a les línies per tal de veure si estan dimensionades correctament. La caiguda de tensió en línies d'enllumenat no pot ser superior al 3\% i en línies de força no pot ser superior al 5\% de la tensió de subministrament. La caiguda de tensió màxima a la derivació individual és de 1,5\%.\\
\newline En monofàsic:
\begin{equation}
e(\%)=\frac{P}{V}\frac{2*l}{k*S}
\end{equation}
l és la longitud ja sigui de la fase o el neutre des del comptador a l'element més llunyà\\
$k = 56 \frac{m}{mm^{2}\si{\ohm}}$\\
S és la secció del cable en m$m^2$\\
\newline
En trifàsic, l'equació que s'utilitza és:
\begin{equation}
e(\%)=\frac{P}{V}\frac{l}{k*S}
\end{equation}
\\
El dimensionament de les línies ha de permetre que les caigudes de tensió no superin els màxims indicats prèviament. Alhora, els cables han de poder admetre les intensitats calculades, per això ens guiem amb la taula de la ITC-19 del REBT. Finalment, cal comprovar que  l'interruptor magnetotèrmic té una intensitat nominal superior a la calculada per la línia i menor a l'admissible que marca la ITC-19.\\
\newline La instal·lació és trifàsica, per tant, hi ha 3 conductors de fase i un conductor de neutre. El conductor de terra transcorre per totes les línies i té una secció igual als conductors de les línies, tal com s'indica al plànol. El neutre, que arriba per l'escomesa, també és de la mateixa secció que els conductors de fase. Les màquines trifàsiques necessiten el neutre pels seus equips electrònics.\\
\newline
Per comprovar que el valor de secció de la derivació individual és correcte quan la línia va amb una terna de cables unipolars per tub cal tenir en compte un factor d'intensitat de 0,8.
\begin{equation}
I_{DI} < 0.8 * I_{max. admissible}
\end{equation}

\noindent A continuació es mostren les diferents línies de forma detallada. La secció s'ha comprovat tenint en compte les fórmules explicades i les seccions mínimes per intensitat segons marca el REBT. S'han verificat les línies pel cas més desfavorable. Les seccions dels tubs compleixen amb la ITC-21.\\
\newline Les tensions nominals són 230 V per les línies monofàsiques i 400 V per les trifàsiques. Tots els cables de les línies són de coure de 450/750 V d'aïllament. La derivació individual és de coure amb 0,6/1 kV d'aïllament. L'aïllament de la instal·lació és de 1.000 k$\si{\ohm}$.

\begin{table}[H]
\scriptsize
\begin{center}
 \begin{tabu} to \textwidth {|X[0.5, l]|X[2, l]|X[r]|X[0.6, r]|X[r]|X[r]|X[r]|X[r]|X[r]|X[r]|X[0.5,r]|}%{X | c c c} 
 \hline
 Línia& Descripció & Potència (W) & cos($\phi$) & Intensitat (A) & Distància màxima (m) & Seccions fase, neutre, terra ($mm^{2}$) & Diàmetre tub (mm) & Caiguda de tensió (\%) & Caiguda de tensió acum. (\%)\\
 \hline \hline 
DI & Derivació individual& 87.000 \ \ \ \ & 0,96 & 131,32 & 8 &3x35 + 35 + 16& 160 & 0,23 & 0,23 \\ \hline
L1 & Enllumenat habitacions i cambra& 1.370,5 & 1 & 5,96 & 55 &2,5 + 2,5 + 2,5& 20 & 2,04 & 2,27 \\ \hline
L2 & Enllumenat cuina & 1.476 \ \ \ \  & 1 & 6,42 & 47 &2,5 + 2,5 + 2,5& 20 & 1,87 & 2,10 \\ \hline 
L3 & Força oficina, menjador, màquines de buit & 7.775 \ \ \ \  & 1 & 33,80 & 51 &10 + 10 + 10& 25 & 2,68 & 2,91 \\ \hline 
L4 & Força cuina & 7.235 \ \ \ \  & 1 & 31,46 & 33 & 6 + 6 + 6& 25 & 2,67 & 2,90 \\ \hline
L5 & Rentaplats cuina & 36.000 \ \ \ \ & 0,95 & 54,92 & 54 &3x16 + 16 + 16& 32 & 1,44 & 1,67 \\ \hline 
L6 & Extractors i cambres de fred & 19.000 \ \ \ \ & 0,9 & 30,60 & 36 &3x10 + 10 + 10& 32 & 0,86 & 1,09 \\ \hline
L7 & Abatidor sala de preparació & 16.975 \ \ \ \ & 0,95 & 25,90 & 44 &3x10 + 10 + 10& 32 & 0,84 & 1,07 \\ \hline

 \end{tabu}
 \caption{Línies detallades}
\end{center}
\end{table}



\chapter{\uppercase{Característiques}}
L'aïllament dels cables elèctrics de les línies és EPR de 450/750 V d'aïllament. Els cables transcorren en safata perforada pel passadís i dins de tubs corrugats en muntatge superficial (B2) a la resta de zones. El diàmetre d'aquests tubs s'indica en l'anterior annex. Els càlculs s'han efectuat considerant que tota la llargada dels cables va amb el muntatge B2, que és més restrictiu que la safata.\\
\newline Es fan servir els colors gris, marró i negre per les fases, el blau pel neutre i el conductor groc i verd pel terra.\\
\newline Hi ha instal·lades caixes de derivació al llarg de la instal·lació i l'enllumenat dels vestidors, l'oficina i el menjador es controla amb interruptors de paret. Els cables dels l'enllumenats que no estan en contacte amb la paret es passen pel fals sostre.\\
\newline Les màquines trifàsiques es connecten a la xarxa mitjançant una base CETAC.\\
\newline Els extractors de la cuina de l'obrador van controlats amb variadors de freqüència. La seva línia va amb un diferencial de 100 mA de classe B degut als alts corrents de fuga que poden donar-se. Aigües amunt de tots els agrupaments hi ha instal·lat un diferencial de 300 mA de sensibilitat per protegir tota la instal·lació i alhora tenir selectivitat amb el diferencials que té aigües avall.\\
\newline El maxímetre del conjunt de protecció i mesura garanteix el subministrament elèctric tot i sobrepassar la potència contractada. Si en un moment puntual es connectés alguna màquina més i pel marge donat no saltés cap interruptor magnetotèrmic però s'estigués superant la potència contractada, hi seguiria havent subministrament elèctric i l'empresa subministradora aplicaria un recàrrec a la factura.\\
\newline 
S'agrupen les línies tenint en compte si el subministrament és trifàsic o monofàsic. S'intenta, en la mesura del possible, que tots els grups tinguin potències similars. És per això que el diferencial que agrupa les 4 línies monofàsiques és de 4 pols: els 3 conductors de fase i el neutre passaran per aquest diferencial i s'alimentaran les diferents línies monofàsiques amb diferents fases. Així es pot aconseguir una instal·lació trifàsica bastant ben equilibrada.\\
\newline Els llums d'emergència són de tipus no permanent i es considera que tenen una potència de 3 W. Al disposar de bateria i només encendre's quan hi ha una emergència, no s'han tingut en compte per la previsió de càrregues.\\
\newline Per millorar el factor de potència de la derivació individual, o sigui, de tota la instal·lació, hi ha instal·lada una bateria de condensadors de 20 kVAr la qual dona una factor de potència de 0,998.




\end{appendices}


\end{spacing}
%\cite{einstein} % per fer una cita
%\printbibliography[title=Bibliografia] %ARA BÉ

\end{document}