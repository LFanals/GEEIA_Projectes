\chapter{\uppercase{Característiques}}
La instal·lació compleix amb el REBT.\\
\newline S'utilitzen interruptors magnetotèrmics amb corba C amb la intensitat indicada a les taules i a l'esquema unifilar. S'han dimensionat aquests magnetotèrmics per protegir la instal·lació de curtcircuits i sobreintensitats. El valor d'intensitat pel qual el magnetotèrmic pot començar a actuar és major que el valor d'intensitat previst a la línia i alhora menor que la intensitat admissible pel cable.\\
\newline Es preveu utilitzar interruptors diferencials de tipus A amb valors de 30 mA, 100 mA i 300 mA aigües amunt de cada agrupament de línies. Els rentaplats i els extractors, que es preveu que disposaran de variadors de freqüència, aniran amb diferencials de 100 mA degut als alts corrents de fuga que poden donar-se en aquestes línies. Aigües amunt de tots els agrupaments s'instal·la un diferencial de 300 mA de sensibilitat per protegir tota la instal·lació. Tots els interruptors diferencials tenen una intensitat nominal suficient per poder actuar quan la instal·lació està funcionant correctament. Es garanteix amb la posada a terra que cap contacte indirecte superi els 24 V que marca el REBT per locals humits, com és el cas. Així, es garanteix el correcte funcionament de la instal·lació.\\
\newline El QGPM també disposa d'un interruptor contra sobretensions permanents i transitòries, el qual ha d'actuar quan la tensió de servei excedeix de forma considerable la tensió de servei indicada al punt "Condicions del subministrament".\\
\newline La instal·lació disposa d'un model comercial de Conjunt de Protecció i Mesura (CPM) anomenat TMF10 el qual té l'escomesa com a entrada i la derivació individual com a sortida. El TMF10 disposa de maxímetre, mòdul de comunicacions, fusibles, embarrat i interruptor general regulable (ICP-M).\\
\newline No s'instal·la cap CGP perquè s'entén que va incorporada al TMF10. Com que el TMF10 té un maxímetre, no s'instal·la cap ICP. La instal·lació segueix estant protegida contra sobreintensitats a aigües amunt gràcies a l'interruptor ICP-M i al magnetotèrmic general del Quadre General de Protecció i Comandament (QGPM). El maxímetre garanteix el subministrament elèctric tot i sobrepassar la potència contractada. Si en un moment puntual es connectés alguna màquina més i pel marge donat no saltés cap interruptor magnetotèrmic però s'estigués superant la potència contractada, hi seguiria havent subministrament elèctric i l'empresa subministradora aplicaria un recàrrec a la factura.\\
