\chapter{\uppercase{Conjunt de protecció i mesura}}
El conjunt de protecció i mesura té la funció de protegir tota la instal·lació que té aigües avall i mesurar els consums d'aquesta. Pot contenir fusibles i sempre conté un interruptor de protrecció i corrent regulable, anomenat ICP-M.\\
\newline El conjunt de protecció i mesura (CPM) utilitzat a la instal·lació és el TMF10 amb un corrent entre 80 A i 160 A. La potència contractada de la nostra instal·lació és de 87 kW. Al Vademècum, mirem aquesta columna de 87 kW.\\
\newline
El conjunt de protecció i mesura està col·locat fora de la nau, encastat en obra, en un nínxol. La porta del nínxol és metàl·lica i té 20 cm de separació per costat i costat. El conjunt de protecció i mesura en sí està situat 60 cm sobre el nivell del terra. Els tubs pels quals li arriba l'escomesa subterrània són de polietilè de 160 mm de diàmetre.\\
\newline
A l'entrada del conjunt de protecció hi trobem l'escomesa. El seu dimensionament és feina de la companyia i el desconeixem. Aquesta escomesa té en sèrie uns fusibles, un per cada una de les 3 fases, i seguidament es troben uns embarrats. Sobre aquests embarrats hi ha els transformadors d'intensitat que permetran conèixer el consum. La sortida del conjunt de protecció i mesura és la derivació individual, la qual dimensionem en el següent capítol.\\
\newline
Com a element de protecció s'instal·la ICP-M de 160 A de corrent assignat, 10 kA de poder de tall, 125 A per la part tèrmica i 625 A per la part magnètica.\\
\newline Els demés elements de protecció que apareixen a la taula del Vademècum es detallen més endavant, formen part del quadre general de protecció i comandament.\\
\newline
S'escull el conjunt de mesura TMF10 Multifunció.\\
\newline Els tranformadors d'intensitat, que es col·loquen sobre els embarrats i serveixen per disminuir la intensitat i així mesurar-la més fàcilment, tenen una relació de transformació de corrent de 100/5.\\
\newline El cablatge és de coure, de 20x5 + 15x5.\\
\newline Els fusibles són de 250 A i les seves bases del tipus BUC 1.
\newline El mòdul TMF10 té un maxímetre. Aquest equip és un semblant d'un comptador digital. Llegeix les intensitats que li arriben dels transformadors d'intensitat i així és capaç de determinar la potència. Sol anar acompanyat d'un mòdul de comunicacions, el qual envia els registres a la companyia. Si hem escollit una tarifa 3.0 i consumim més potència de la contractada, se'ns cobrarà un recàrrec, però no se'ns tallarà el subministrament elèctric (a menys que el corrent fos molt alt i l'IGA o els fusibles actuessin). El maxímetre, en cap cas, talla el subministrament elèctric.


\clearpage