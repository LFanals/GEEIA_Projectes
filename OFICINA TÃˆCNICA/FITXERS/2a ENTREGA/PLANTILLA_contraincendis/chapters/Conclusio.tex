\chapter{\uppercase{Conclusió}}
El present document té l'objectiu de legalitzar la instal·lació contra incendis, els materials i les evacuacions d'emergència d'un obrador de plats cuinats.\\
\newline
Per desenvolupar aquesta memòria s'ha seguit el Reial decret 2267/2004 del 3 de desembre, el qual aprova el Reglament de seguretat contra incendis en establiments industrials. Amb aquest reglament es pot dir que la càrrega de foc és baixa i els materials són correctes.\\
\newline Per validar la instal·lació contra incendis s'ha seguit el Reglament d'instal·lacions de protecció contra incendis aprovat pel Reial decret 513/2017 del 22 de maig. Gràcies a aquest document s'ha pogut avaluar la instal·lació contra incendis com a correcta.\\
\newline El REBT s'ha consultat per verificar que els nivells d'il·luminació d'emergència mesurats són correctes. La Norma Básica de l'Edificació, NBE-CPI/96, ha permès considerar correctes les dimensions de les portes d'emergència i els passadissos.\\
%
\newline Amb tot l'indicat en aquesta memòria es considera que la nau té un risc baix d'incendi i que la instal·lació contra incendis, els materials i els recorreguts d'evacuació són els adequats segons la legislació present fins a dia d'avui. 

\vspace*{\fill}
\noindent Llorenç Fanals Batllori\\
Graduat en Enginyeria Electrònica Industrial i Automàtica\\
\\
\\
Girona, 26 d'octubre de 2019.

\clearpage