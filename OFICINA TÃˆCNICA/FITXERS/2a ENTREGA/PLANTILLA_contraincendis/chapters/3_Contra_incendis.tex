\section{\uppercase{Insta\Lgem ació contra incendis}}
L'obrador ha de disposar de les mesures correctes per extingir els incendis que es poden causar. Extintors, BIEs i hidrants són algunes d'aquestes mesures.\\ 
\newline Segons el reglament de seguretat contra incendis hi ha diferents tipus de focs. Els focs de classe A són focs de materials sòlids, generalment de naturalesa orgànica, la combinació dels quals s'efectua normalment amb la formació de brases. Els focs de classe B són focs de líquids o de sòlids liquables. Els focs de classe C són focs de gasos. Els focs de classe D són focs de metalls. Finalment, els focs de classe F són focs derivats de la utilització d'ingredients per cuinar (olis i greixos vegetals o animals) ens els aparells de cuina.\\
\newline A l'obrador poden haver-hi focs de classe A si per exemple es crema alguna taula o moble de fusta. Els focs de classe B es podrien donar amb algun líquid com alcohol. Un foc de classe C es podria donar per la combustió del gas que es fa servir per cuinar. Si es provoqués un incendi, el metall seria el més difícil d'encendre i es preveu que es pogués apagar amb un extintor de pols. Clarament també es poden donar focs de tipus F a la cuina. \\
\newline La instal·lació contra incendis de l'obrador està formada, majoritàriament, per extintors de pols. Aquests extintors permeten apagar els focs de tipus A, B i C i fins i tot els tipus D depenent del metall en qüestió. Com que l'element extintor és pols no hi ha risc de conduir l'electricitat. L'eficàcia d'aquests extintors és 21A, que és el mínim que marca la normativa.\\
\newline Disposem d'un extintor de $CO_2$ a tocar del quadre elèctric. Aquest extintor fa baixar la temperatura fins als -80 graus centígrads i es caracteritza per no conduir l'electricitat. Pot apagar focs dels tipus A, B i C. La seva eficàcia és de 98B.\\
\newline A la cuina els extintors són de tipus F, per poder apagar correctament el foc en olis i greixos. Aquests extintors descarreguen una fina boira a base d'acetat de potassi. Així creen una espuma que refreda l'oli i la grassa i la separen de l'aire. Aquests extintors són d'eficàcia 233B.\\
\newline Tal com marca la normativa, des de qualsevol punt de la instal·lació s'ha de poder arribar a un extintor recorrent, com a màxim, 15 m. S'han col·locat els extintors en punts estratègics, la majoria en zones comunes com els passadissos. Tots els extintors tenen marcatge CE i estan situats a una altura de 1,10 m. \\
\newline També es disposa de dues BIEs per tal de facilitar la feina als bombers en cas d'incendi, tot i que al tractar-se de tipus C i nivell baix no és obligat tenir-ne. Aquestes estan col·locades a 1 i 4 metres de les respectives entrades a l'obrador. La normativa marca un màxim de 5 metres respecte les entrades de l'obrador. Una està al costat del quadre elèctric i l'altre està al lloc per on surten els cuinats.\\
\newline Les mànegues de les BIEs són semirígides de 30 m de longitud. Al radi d'acció de cada BIE se li ha de sumar 5 metres, que és la longitud del raig d'aigua. Així, les BIEs de la instal·lació tenen un radi d'acció de 35 metres cada una. Cada zona de l'obrador està coberta per almenys una BIE.\\
\newline Les BIEs, de tipus DN 25 mm, poden donar 5 kg/$cm^2$ durant una hora i 20 minuts, compleixen amb el que marca el BOE. Amb una pressió de 10 kg/$cm^2$ durant 2 hores no experimenten fugues.\\
\newline La normativa marca que no és obligatori instal·lar alarmes de detecció automàtica d'incendis si l'edifici és de tipus C i nivell de risc intrínsec baix. Ara bé, si no s'instal·len sistemes de detecció automàtica d'incendis sí que calen sistemes manuals d'alarma d'incendi. A l'obrador, hi ha un polsador a cada sortida d'emergència així com a altres llocs de la nau. Es garanteix que la distància màxima des de qualsevol punt de la nau a un polsador és menor de 25 m.\\
\newline Es disposa de dos hidrants propers a la nau: un a 10 m i l'altre a 5, situats als laterals de la nau. Considerant que tenen un radi d'acció de 40 m, tal com marca la normativa, cobreixen tota la superfície de la nau. La pressió que poden donar és de 6 bar quan estan descarregant els cabals indicats anteriorment. Compleixen amb la normativa. Els dos tenen una boca de sortida de 100 mm.\\
\newline Ens han indicat que els hidrants poden donar fins a 1000 L/min durant 45 minuts. Els mínims que ens exigeixen són de 500 L/min durant 30 minuts.

