%Options > Configure Texmaker > Editor > Spelling Dictionary, per corrector en català

\documentclass[11pt, a4paper]{report}
\usepackage[a4paper,left=30mm,right=20mm,top=25mm,bottom=25mm]{geometry}
\usepackage{blindtext}
\sloppy %per forçar el canvi de línia si la paraula supera el marge dret
\usepackage[utf8]{inputenc}

% Per utilitzar la font Helvetica (Arial)
\renewcommand{\familydefault}{\sfdefault}
\usepackage[scaled=1]{helvet}
\usepackage[helvet]{sfmath}
\everymath={\sf}
%Equacions amb una font sans_serif, \mathrm{equació aquí, són les letres les que queden inclinades}
%\usepackage{arev} % sans-serif math font
%\usepackage{helvet} % sans-serif text font
\usepackage[T1]{fontenc}
\usepackage[utf8]{inputenc}
\usepackage[catalan]{babel}
\usepackage{newunicodechar}
\newunicodechar{Ŀ}{\L.}
\newunicodechar{ŀ}{\l.}


% Per comptar imatges enlloc de mostrar 1.1, 1.2...
\usepackage{chngcntr}
\counterwithout{figure}{chapter}
\counterwithout{table}{chapter}
\counterwithout{equation}{chapter}

\usepackage{graphicx}
\graphicspath{{images/}} %directori amb les imatges que volem insertar
\usepackage{float} %per forçar imatges amb H
\usepackage[normalem]{ulem} %negreta múltiples línies
%\usepackage{soul}

\usepackage{caption}
\captionsetup[figure]{labelfont={},name={Figura},labelsep=period}
\captionsetup[table]{labelfont={},name={Taula},labelsep=period}


\usepackage{subcaption}
\usepackage{amsmath} %per fòrmules matemàtiques
\usepackage[table]{xcolor} %per colors a les taules
%\usepackage{circuitikz} %per circuits electrònics
\usepackage{siunitx} %per les labels dels components
\usepackage[american,cuteinductors,smartlabels]{circuitikz} %american/european
\usepackage{tikz} %quadrícula
%\usepackage[a4paper, left=29mm, right=20mm, top=25mm, bottom=25mm]{geometry} %geometria de la pàgina, 25 però per ajustar bé
\setlength{\headsep}{20pt}
\setlength{\footskip}{25pt}
%\usepackage[a4paper, width=150mm, top=25mm, bottom=25mm]{geometry} %geometria de la pàgina
\usepackage{lipsum} %per generar dummy text
\usepackage{xpatch} %per la distància entre títol i top

%Capçaleres i peus de pàgina
\usepackage{fancyhdr}
%\pagestyle{fancy} %fancy, plain
\fancypagestyle{plain}{
  \fancyhf{}% Clear header/footer
  \fancyhead[L]{\footnotesize{Obrador de plats cuinats}}
  \fancyhead[R]{\footnotesize{Protecció contra incendis}}
  \fancyfoot[R]{\footnotesize{\thepage}}
}
\pagestyle{plain}% Set page style to plain.

%\fancyhead{}
%\fancyhead[LO,LE]{PROJECTES}
%\fancyfoot{}
%\fancyfoot[LE,RO]{\thepage} %número de la pàgina, a la dreta
%\fancyfoot[LO, CE]{Capítol \thechapter} %nom del capítol, a l'esquerra
%\fancyfoot[CO, CE]{\href{https://github.com/LFanals}{Llorenç Fanals Batllori}} %nom de l'autor, al centre
% \renewcommand{\headrulewidth}{0.4pt}
%\renewcommand{\footrulewidth}{0.4pt}

%Per tenir el nombre de pàgina a l'inici d'un capítol
%\fancypagestyle{plain}{
%\fancyhf{}
%\renewcommand\headrulewidth{0pt}
%\fancyfoot[R]{\thepage}
%}

%Per configurar el color dels links i referències
\usepackage{color}
\usepackage{hyperref}
\hypersetup{
    colorlinks=true, %true si es volen links de colors
    linkcolor=black,  %colors de les referències internes, blue
    filecolor=magenta,      %magenta
    urlcolor=[rgb]{0,0,0}, %Color dels links d'Internet, sobre 255=2^8-1=2^0+...+2^7, {0,0.5,1}
}

%Bibliografia
\usepackage[backend=bibtex]{biblatex}
\addbibresource{bibliography.bib}

%Canviem el nom que hi ha per defecte als índex i altres, per passar-ho al català
\renewcommand{\contentsname}{Índex}
\renewcommand{\listfigurename}{Índex de figures}
\renewcommand{\chaptername}{Capítol}
\renewcommand{\appendixname}{Annex}
\renewcommand{\listtablename}{Índex de taules}
% \renewcommand{\figurename}{Figura} % ho tinc amb caption
% \captionsetup[table]{name=Taula} % ho tinc amb caption

\definecolor{color_quadricula}{HTML}{0066ff} %color per la quadrícula

% Pels circuits
%\usepackage[american]{circuitikz}
\usetikzlibrary{calc}
\ctikzset{bipoles/thickness=1}
\ctikzset{bipoles/length=1.2cm}
\ctikzset{bipoles/diode/height=.375}
\ctikzset{bipoles/diode/width=.3}
\ctikzset{tripoles/thyristor/height=.8}
\ctikzset{tripoles/thyristor/width=1}
\ctikzset{bipoles/vsourceam/height/.initial=.7}
\ctikzset{bipoles/vsourceam/width/.initial=.7}
\tikzstyle{every node}=[font=\small]
\tikzstyle{every path}=[line width=0.8pt,line cap=round,line join=round]

%Per insertar codi
\usepackage{listings}
\usepackage{color}
\definecolor{dkgreen}{rgb}{0,0.6,0}
\definecolor{gray}{rgb}{0.5,0.5,0.5}
\definecolor{mauve}{rgb}{0.58,0,0.82}

\lstset{frame=none, %tb, none
  language=Python,
  aboveskip=2mm,
  belowskip=3mm,
  showstringspaces=false,
  columns=flexible,
  basicstyle={\scriptsize\ttfamily}, %small
  numbers=none, %left
  numberstyle=\tiny\color{gray},
  keywordstyle=\color{blue},
  commentstyle=\color{dkgreen},
  stringstyle=\color{mauve},
  breaklines=true,
  breakatwhitespace=true,
  tabsize=3
}


%Per tenir el format de capítol correcte
\usepackage{titlesec}

\usepackage{etoolbox}
%\usepackage{hyperref}

%Per chapter
\titlespacing*{\chapter}{0pt}{-25pt}{11pt} %Espaiat del títol de capítol amb els altres elements
\titleformat{\chapter}[hang] %Per seguir escrivint darrera el número
{\normalfont\fontsize{11}{15}\bfseries}{\thechapter.}{0.4em}{\MakeUppercase} %\fontsize{Tamany}{Espai múltiples línies}

%Per secció
\titlespacing*{\section}{0pt}{11pt}{11pt} %Espaiat del títol de capítol amb els altres elements
\titleformat{\section}[hang] %Per seguir escrivint darrera el número
{\normalfont\fontsize{11}{15}}{\thesection.}{0.4em}{\bfseries} %\fontsize{Tamany}{Espai múltiples línies}

%Per subsecció
\titlespacing*{\subsection}{0pt}{11pt}{11pt} %Espaiat del títol de capítol amb els altres elements
\titleformat{\subsection}[hang] %Per seguir escrivint darrera el número
{\normalfont\fontsize{11}{15}}{\thesubsection.}{0.4em}{} %\fontsize{Tamany}{Espai múltiples línies}

%Per paràgraf
\titlespacing*{\paragraph}{0pt}{0pt}{22pt} %Espaiat del títol de capítol amb els altres elements
\titleformat{\paragraph}[hang] %Per seguir escrivint darrera el número
{\normalfont\fontsize{11}{15}}{}{}{} %\fontsize{Tamany}{Espai múltiples línies}


%Interlineat, 1.2*1.25=1.5
\linespread{1.25}

%Espaiat entre paràgrafs
%\setlength{\parskip}{22pt}
 

\makeatletter
\def\tagform@#1{\maketag@@@{(\ignorespaces{Eq.~#1}\unskip)}}
\makeatother



%Per no tenir negreta a l'index
\usepackage{etoolbox}% http://ctan.org/pkg/etoolbox
\makeatletter
\patchcmd{\l@chapter}{\bfseries}{}{}{}% \patchcmd{<cmd>}{<search>}{<replace>}{<success>}{<failure>}
\makeatother

%Per tenir punts a l'índex
\makeatletter
\renewcommand*\l@chapter{\@dottedtocline{0}{0em}{1.5em}}
\makeatother

%Per taula que adapta bé els espais
\usepackage{tabularx}
\usepackage{tabu} % http://mirrors.ibiblio.org/CTAN/macros/latex/contrib/tabu/tabu.pdf
\tabulinesep = 1mm
\usepackage[font=footnotesize]{caption} %Captions de les figures més petites

%Appendix
\usepackage[]{appendix} %toc, page

%Alinear al separador decimal amb espais
\usepackage{setspace}
\renewcommand*{\arraystretch}{1.25}


%-------------------------------------------------------------------------------------------------------------
%-------------------------------------------------------------------------------------------------------------
%-------------------------------------------------------------------------------------------------------------
%-------------------------------------------------------------------------------------------------------------

\begin{document}
\pagenumbering{Roman}


%\begin{titlepage}
	\begin{center}
		\vspace*{1cm}
		
		\Huge
		\textbf{Document per Projectes}
		
		\vspace{0.5cm}
		\LARGE
		Adaptat a \LaTeX
	
		\vspace{1.5cm}
		
		\textbf{Llorenç Fanals Batllori}
		
		\vfill
		
		\small
		%\uppercase{Un treball lliurat a la Universitat - en compliment dels requisits pel grau en -}\\
		% TFG
		
		\vspace{1cm}
		
		%\includegraphics[scale=width=0.4\textwidth]{images/a_graph}
	\end{center}
	
	\begin{flushright}
	\large	
	Departament o grup de recerca\\
	UdG\\
	%País\\
	28/08/2019
	\end{flushright}
	


\end{titlepage}

%\thispagestyle{plain}

\begin{center}
	\large
	\textbf{Informe}
	
	\vspace{0.4cm}
	\large
	Descripció
	
	\vspace{0.4cm}
	\textbf{Llorenç Fanals Batllori}
	
	\vspace{0.9cm}
	\textbf{Abstract}
\end{center}
\lipsum[1]




%\chapter*{Dedicacions}
%Dedico aquest treball a -

%\chapter*{Agraïments}
%Vull agraïr a \\

\cleardoublepage\pagenumbering{arabic}

\begin{spacing}{2}
\tableofcontents
\end{spacing}
%\listoffigures %No fa falta crec
%\listoftables %No fa falta crec
\begin{spacing}{1.5}

\chapter{\uppercase{Generalitats}}
L'objectiu d'aquest document és justificar que la immissió acústica de l'obrador es manté dins els nivells màxims que indiquen les normatives. Així mateix, es vol verificar que les vibracions prenen valors permesos. L'activitat està situada al carrer Ramon Serradell, número 27, a La Bisbal d'Empordà, en un polígon industrial\\
\newline Per justificar-ho s'ha consultat la Llei 16/2002 de 28 de juny que tracta sobre Protecció contra la Contaminació Acústica. Aquesta llei comenta com s'han de mesurar les immissions de sorolls i contempla una gran quantitat de casos o elements que poden causar contaminació acústica. També detalla com han de ser les mesures de vibracions i els seus valors permesos.\\
\newline S'ha tingut en compte el mapa de capacitat acústica de l'Ajuntament de la Bisbal d'Empordà. En ell es marquen clarament diferents zones segons la sensibilitat acústica màxima que admeten.\\
\newline L'Ajuntament de la Bisbal d'Empordà no té una ordenança de sorolls i vibracions, per tant, es segueix el Decret 176/2009 de 10 de novembre de Protecció contra la Contaminació Acústica, el qual s'aplica a Catalunya. Aquest decret modifica els annexos de la Llei de Protecció contra la Contaminació Acústica de la Llei 16/2002 de 28 de juny, que s'aplica a nivell nacional.


\clearpage
\chapter{\uppercase{Protecció contra incendis}}
A continuació s'exposen els càlculs de càrrega de foc per determinar el nivell intrínsec d'incendi de l'obrador, amb el qual es pot classificar la instal·lació. Seguidament es detalla la instal·lació contra incendis existent a l'obrador i es consultat si és correcta; el mateix es fa pels materials. Finalment s'avaluen els recorreguts d'evacuació i la senyalització d'aquests. Es consulten les normatives indicades anteriorment.

\section{\uppercase{Càrrega de foc}}
%capítol de generalitats? i conclusions?
La càrrega de foc, com indica el seu nom, és la manera de comptabilitzar el rics d'incendi d'un espai o activitat. Depenent de la càrrega de foc calculada una activitat es pot legalitzar o no. Per calcular la càrrega de foc hi ha dues maneres: una és calculant la càrrega de foc de cada dependència o espai i ponderant-les; l'altra és calculant la càrrega de foc de cada massa de l'activitat.\\
\newline La fórmula per calcular la càrrega de foc d'un conjunt de superfícies, en $MJ/m^2$ és:
\begin{equation}
Q_s = \frac{\sum_{1}^{i}q_{si}S_iC_i}{A}R_a 
\end{equation}
$Q_s$: densitat de la càrrega de foc, ponderada i corregida, del sector o àrea d'incendi, en $MJ/m^2$.\\
$q_{si}$: densitat de càrrega de foc tabulada segons els diferents processos, en $MJ/m^2$.\\
$C_i$: coeficient adimensional que pondera el grau de perillositat per combustió dels combustibles.\\ 
$S_i$: superfície de cada zona amb un procés diferent i densitat de càrrega de foc diferent, en $m^2$.\\
$R_a$: coeficient adimensional que corregeix el grau de perillositat per l'activació, inherent a l'activitat industrial.\\
$A$: superfície construïda del sector d'incendi, en $m^2$.\\
\newline Per calcular la carrega de foc d'un magatzem, en $MJ/m^2$:
\begin{equation}
Q_s = \frac{\sum_{1}^{i}q_{vi}C_ih_is_i}{A}R_a 
\end{equation}
$q_{vi}$:càrrega de foc aportada per cada $m^3$ de cada zona amb diferent emmagatzematge, en $MJ/m^3$.\\
$h$: altura d'emmagatzematge de combustibles, en $m$.\\
$s_i$: superfície ocupada en planta per cada zona amb diferent emmagatzematge, en $m^2$.\\
\newline Per calcular la càrrega de foc es decideix dividir l'obrador en quatre zones: dues de magatzems i cambres de fred, producció, i oficina i altres. A continuació es donen més detalls d'aquests agrupaments. El factor $R_a$ és el màxim valor que tinguem si aquella activitat ocupa el 10\% o més del total de superfície de l'agrupament que hem escollit.\\
\newline Un agrupament és el de la cambra de congelació de sortida i el de la cambra frigorífica de sortida. Per calcular la seva càrrega de foc es fa servir la fórmula de magatzems.
%
\begin{table}[H]
\footnotesize
\begin{center}
 \begin{tabu} to \textwidth {|X[1.5, l]|X[r]|X[2, l]|X[1, r]|X[0.5, r]|X[r]|X[0.5, r]|X[r]|}%{X | c c c} 
 \hline
Estances & Superfície ($m^2$)& Descripció & $q_i$ (MJ/$m^2$) & $C_i$ & Altura de càlcul (m) & $R_a$ & $q_{i}S_ih_iC_i$ (MJ) \\
 \hline \hline 
Fred sortida & 17,53 & Armaris frigorífics & 300 & 1,0 & 1,0 & 1,0 & 5.259,0 \\ \hline
Congelador sortida & 17,53 & Congelats & 372 & 1,0 & 1,0 & 1,0 & 6.521,2 \\ \hline
\hline
Total & 35,06 & & & & & 1,0 & 11.780,2 \\ \hline
Càrrega de foc & \multicolumn{6}{c|}{} & 336,0 MJ/$m^2$ \\ \hline
 \end{tabu}
 \caption{Càrrega de foc calculada de cambres de sortida}
\end{center}
\end{table}
\noindent El següent agrupament comprèn el magatzem, la cambra frigorífica de l'entrada i la cambra de congelació de l'entrada. També es fa servir la fórmula de magatzem per calcular la seva càrrega de foc.
%
\begin{table}[H]
\footnotesize
\begin{center}
 \begin{tabu} to \textwidth {|X[1.5, l]|X[r]|X[2, l]|X[1, r]|X[0.5, r]|X[r]|X[0.5, r]|X[r]|}%{X | c c c} 
 \hline
Estances & Superfície ($m^2$)& Descripció & $q_i$ (MJ/$m^2$) & $C_i$ & Altura de càlcul (m) & $R_a$ & $q_{i}S_ih_iC_i$ (MJ) \\
 \hline \hline 
Magatzem & 17,36 & Alimentació, matèries primes & 3.400 & 1,0 & 0,2 & 2,0 & 11.804,8 \\ \hline
Fred entrada & 19,76 & Congelats & 300 & 1,0 & 0,8 & 1,0 & 4.742,4 \\ \hline
Congelador entrada & 19,79 & Congelats & 372 & 0,8 & 1,0 & 1,0 & 5.889,5 \\ \hline
\hline
Total & 56,91 & & & & & 2,0 & 22.436,7 \\ \hline
Càrrega de foc & \multicolumn{6}{c|}{} & 788,5 MJ/$m^2$ \\ \hline
 \end{tabu}
 \caption{Càrrega de foc calculada de magatzem i cambres d'entrada}
\end{center}
\end{table}
\noindent La zona de producció té la cuina, la sala de preparació, l'entrada, la venda al públic, la sala de residus i la sala amb els productes de neteja.
%
\begin{table}[H]
\footnotesize
\begin{center}
 \begin{tabu} to \textwidth {|X[1.5, l]|X[r]|X[2, l]|X[1, r]|X[0.5, r]|X[0.5, r]|X[r]|}%{X | c c c} 
 \hline
Estances & Superfície ($m^2$)& Descripció & $q_i$ (MJ/$m^2$ & $C_i$ &  $R_a$ & $q_{i}S_iC_i$ (MJ) \\
 \hline \hline 
Preparació & 48,26 & Embalatge de productes alimentaris & 800 & 1,0  & 1,5 & 38.608,0 \\ \hline
Cuina & 265,75 & Alimentació, plats precuinats & 200 & 1,0 & 1,0 & 53.510,0 \\ \hline
Entrada & 35,53 & Alimentació, embalatge & 800 & 1,0 & 1,5 & 28.424,0 \\ \hline
Venda al públic & 37,38 & Expedició de productes alimentaris & 800 & 1,0 & 1,5 & 29.904,0 \\ \hline
Sala de residus & 16,08 & Alimentació, embalatge & 800 & 1,0 & 1,5 & 12.846,0 \\ \hline
Neteja & 3,96 & Neteja química & 300 & 1,3 & 1,5 & 1.544,4 \\ \hline
\hline
Total & 406,96 & & & & 1,5 & 164.494,4 \\ \hline
Càrrega de foc & \multicolumn{5}{c|}{} & 606,3 MJ/$m^2$ \\ \hline
 \end{tabu}
 \caption{Càrrega de foc calculada de producció}
\end{center}
\end{table}
\noindent Finalment, l'últim agrupament és per la zona amb l'oficina, els vestidors, la recepció, el menjador, els arxius i la sala de la caldera.
%
\begin{table}[H]
\footnotesize
\begin{center}
 \begin{tabu} to \textwidth {|X[1.5, l]|X[r]|X[2, l]|X[1, r]|X[0.5, r]|X[0.5, r]|X[r]|}%{X | c c c} 
 \hline
Estances & Superfície ($m^2$)& Descripció & $q_i$ (MJ/$m^2$ & $C_i$ &  $R_a$ & $q_{i}S_iC_i$ (MJ) \\
 \hline \hline 
Caldera & 4,48 & Calderes, edificis de & 200 & 1,6  & 1,0 & 1.433,6 \\ \hline
Vestidors 1 & 25,33 & Guarda roba, armaris de fusta & 400 & 1,0 &  1,0 & 10.132,0 \\ \hline
Vestidors 2 & 25,33 & Guarda roba, armaris de fusta & 400 & 1,0 &  1,0 & 10.132,0 \\ \hline
Menjador & 31,44 & Alimentació, plats precuinats & 200 & 1,0 &  1,0 & 6.288,0 \\ \hline
Oficina & 19,32 & Oficines tècniques & 600 & 1,0 &  1,0 & 11.592,0 \\ \hline
Recepció & 14,95 & Mobles de fusta & 500 & 1,0 &  1,5 & 7.475,0 \\ \hline
Arxius & 9,45 & Procés de dades, sala d'ordinador & 400 & 1,0 & 1,5 & 3.780,0 \\ \hline
\hline
Total & 301,07 & & & & 1,5 & 50.832,6 \\ \hline
Càrrega de foc & \multicolumn{5}{c|}{}& 253,3 MJ/$m^2$ \\ \hline
 \end{tabu}
 \caption{Càrrega de foc calculada per oficines i zones comunes}
\end{center}
\end{table}
%
\noindent Amb això podem calcular la càrrega de foc en $MJ/m^2$. Per fer-ho es pondera la càrrega de foc de cada zona segons l'àrea d'aquesta.
% No apareix en cursiva
%\begin{align}
%\begin{split}
%Q_s = \frac{\sum_{1}^{i}q_{i}S_iC_i}{A}R_a \\
%\end{split}\\
%\begin{split}
%Q_s = \frac{383193,4 * 1}{800} = 479,0 \ \ MJ/m^2
%\end{split}
%\end{align}
%
\begin{equation}
Q_e = \frac{\sum_{1}^{i}Q_{si}A_i}{\sum_{1}^{i}A_{i}}
\end{equation}
%\begin{equation}
%Q_s = \frac{784,89*95,93 + 581,85*386,92 + 240,42 * 317,15}{95,93 + 386,92 + 317,15} = 470,84 \ \ MJ/m^2
%\end{equation}

\noindent El resultat del càlcul indica la càrrega de foc mitjana de la nau, en aquest cas és de 474,6 MJ/$m^2$  \\
%
%
%
%
%
%
%
%
\newline Com s'ha comentat, una altra manera de calcular la càrrega de foc és considerant totes les masses existents a l'activitat i els seus poders calorífics. D'aquesta manera podem conèixer la quantitat d'energia que tenim en $MJ$. 
\begin{equation}
Q_s = \frac{\sum_{1}^{i}G_iq_{i}C_i}{A}R_a
\end{equation}
\noindent $C_i$: coeficient adimensional que pot valer 1, 1,3 o 1,6 en funció de la perillositat per combustió. S'ha consultat el catàleg de la CEA "Búsqueda y validación de parámetros de la carga de fuego en establecimientos industriales" \ per obtenir el seus valors. Aquest catàleg és vàlid fer-lo servir segons la normativa.\\
$R_a$: corregeix el grau de perillositat per l'activació. També es troba al catàleg de la CEA. La normativa no defineix de forma clara el criteri per escollir una valor o altre. S'ha decidit escollir el més desfavorable.\\
\newline Per la zona de cambres de sortida es tenen en compte les següents masses:
\begin{table}[H]
\small
\begin{center}
 \begin{tabu} to \textwidth {|X[1.5, l]|X[r]|X[r]|X[r]|X[r]|X[r]|}%{X | c c c} 
 \hline
Productes & Poder calorífic ($MJ/kg$)& Massa ($kg$) & $C_i$ & $R_a$ & $G_iq_{i}C_i$ ($MJ$) \\
 \hline \hline 
Oli mineral & 42,0 & 15 & 1,3 & 2 & 819,0 \\ \hline
Oli d'oliva & 42,0 & 15 & 1,3 & 2 & 819,0 \\ \hline
Alcohol etílic & 25,1 & 25 & 1 & 1 & 627,5 \\ \hline
Sucre & 16,7 & 10 & 1 & 2 & 167,0 \\ \hline
Cafè & 16,7 & 50 & 1 & 2 & 835,0 \\ \hline
Xocolata & 25,1 & 20 & 1 & 1,5 & 502,0 \\ \hline
Farina de blat & 16,7 & 25 & 1,3 & 2 & 542,8 \\ \hline
Llet en pols & 16,7 & 25 & 1 & 2 & 417,5 \\ \hline
Mantega & 37,2 & 20 & 1 & 1 & 744,0 \\ \hline
Tè & 16,7 & 5 & 1 & 1 & 83,5 \\ \hline
Panell d'alumini amb recobriment & 3,9 & 250 & 1 & 1 & 975,0 \\ \hline \hline
Total & & & & 2 & 6.532,3 \\ \hline
Càrrega de foc, S=39,02 $m^2$ & \multicolumn{4}{c|}{} & 334,8 MJ/$m^2$ \\ \hline
 \end{tabu}
 \caption{Càrrega de foc calculada per masses en cambres de fred de sortida}
\end{center}
\end{table}

\noindent Els aliments del magatzem i les cambres de fred d'entrada, tot i que en quantitats diferents, són els mateixos que els de l'agrupament anterior.


\begin{table}[H]
\small
\begin{center}
 \begin{tabu} to \textwidth {|X[1.5, l]|X[r]|X[r]|X[r]|X[r]|X[r]|}%{X | c c c} 
 \hline
Productes & Poder calorífic ($MJ/kg$)& Massa ($kg$) & $C_i$ & $R_a$ & $G_iq_{i}C_i$ ($MJ$) \\
 \hline \hline 
Oli mineral & 42,0 & 50 & 1,3 & 2 & 2.730,0 \\ \hline
Oli d'oliva & 42,0 & 50 & 1,3 & 2 & 2.730,0 \\ \hline
Alcohol etílic & 25,1 & 25 & 1 & 1 & 627,5 \\ \hline
Sucre & 16,7 & 10 & 1 & 2 & 167,0 \\ \hline
Cafè & 16,7 & 50 & 1 & 2 & 835,0 \\ \hline
Xocolata & 25,1 & 20 & 1 & 1,5 & 502,0 \\ \hline
Farina de blat & 16,7 & 450 & 1,3 & 2 & 9.769,5 \\ \hline
Llet en pols & 16,7 & 50 & 1 & 2 & 835,0 \\ \hline
Mantega & 37,2 & 50 & 1 & 1 & 1.860,0 \\ \hline
Tè & 16,7 & 5 & 1 & 1 & 83,50 \\ \hline
Panell d'alumini amb recobriment & 3,9 & 250 & 1 & 1 & 975,0 \\ \hline \hline
Total & & & & 2 & 21.114,5 \\ \hline
Càrrega de foc, S=56,91 $m^2$ & \multicolumn{4}{c|}{} & 742,0 MJ/$m^2$ \\ \hline
 \end{tabu}
 \caption{Càrrega de foc calculada per masses en magatzem i cambres de fred d'entrada}
\end{center}
\end{table}

\noindent A continuació la càrrega de foc calculada per masses de la zona de producció, la qual comprèn la cuina, la sala de preparació, la sala de residus i l'entrada.

\begin{table}[H]
\small
\begin{center}
 \begin{tabu} to \textwidth {|X[1.5, l]|X[r]|X[r]|X[r]|X[r]|X[r]|}%{X | c c c} 
 \hline
Productes & Poder calorífic ($MJ/kg$)& Massa ($kg$) & $C_i$ & $R_a$ & $G_iq_{i}C_i$ ($MJ$) \\
 \hline \hline 
 Oli mineral & 42,0 & 5 & 1,3 & 2,0 & 273,0 \\ \hline
 Oli d'oliva & 42,0 & 5 & 1,3 & 2,0 & 273,0 \\ \hline
 Mantega & 37,2 & 5 & 1 & 1,0 & 186,0 \\ \hline
 Farina de blat & 16,7 & 10 & 1,3 & 2 & 217,1 \\ \hline
Polièster & 25,1 & 200 & 1 & 1,0 & 5.020,0 \\ \hline
Polietilè & 42,0 & 300 & 1 & 1,0 & 12.600,0 \\ \hline
Acer inoxidable & 111,0 & 1.300 & 1 & 1,0 & 144.300,0 \\ \hline
Cartró & 15,6 & 200 & 1,3 & 1,5 & 4.056,0 \\ \hline
Panell d'alumini amb recobriment & 3,9 & 200 & 1 & 1,0 & 780,0 \\ \hline \hline
Total & & & & 2,0 & 123.305,1 \\ \hline
Càrrega de foc, S=403 $m^2$ & \multicolumn{4}{c|}{} & 611,9 MJ/$m^2$ \\ \hline
 \end{tabu}
 \caption{Càrrega de foc calculada per masses en producció}
\end{center}
\end{table}

\noindent En quart lloc s'indica la càrrega de foc calculada per masses a l'oficina i les zones comunes, com vestidors, menjador, passadissos... Algunes masses amb poca quantitat s'han assimilat amb les indicades a continuació.

\begin{table}[H]
\small
\begin{center}
 \begin{tabu} to \textwidth {|X[1.5, l]|X[r]|X[r]|X[r]|X[r]|X[r]|}%{X | c c c} 
 \hline
Productes & Poder calorífic ($MJ/kg$)& Massa ($kg$) & $C_i$ & $R_a$ & $G_iq_{i}C_i$ ($MJ$) \\
 \hline \hline 
Fusta & 16,7 & 1500 & 1,0 & 2 & 25.050,0 \\ \hline
Panell d'alumini amb recobriment & 3,9 & 500 & 1 & 1 & 1.950,0 \\ \hline
Paper & 16,7 & 200 & 1,3 & 2 & 4.320,0\\ \hline \hline
Total & & & & 2 & 31.342,0 \\ \hline
Càrrega de foc, S=301,17 $m^2$ & \multicolumn{4}{c|}{} &  208,1 MJ/$m^2$ \\ \hline
 \end{tabu}
 \caption{Càrrega de foc calculada per masses en oficina i zones comunes}
\end{center}
\end{table}

\noindent Altra vegada podem calcular la càrrega de foc ponderada de l'obrador.
\begin{equation}
Q_e = \frac{\sum_{1}^{i}Q_{si}A_i}{\sum_{1}^{i}A_{i}}
\end{equation}
\noindent El valor calculat, de 455,7 MJ/$m^2$, és molt proper a la càrrega de foc ponderada fent càlculs per superfícies i volums en el cas dels magatzems. Considerem, per tant, que els càlculs per masses són correctes.

%\clearpage
\section{\uppercase{Classificació de la insta\Lgem ació}}
La càrrega de foc calculada per superfícies i volums en el cas de magatzems és de 474,6 MJ/$m^2$. La càrrega de foc calculada per masses i el seu poder calorífic és de 455,7 MJ/$m^2$.\\
\newline Els dos valors de càrrega de foc equivalen a un nivell intrínsec baix 2, ja que aquestes càrregues de foc es troben en el rang $425<Q_s \leq  850$ MJ/$m^2$. No hi ha cap zona de l'obrador que doni un valor que indiqui un risc intrínsec més alt que baix 2.\\
\newline La nau està distanciada 4 metres de l'edifici més pròxim. Entre la nau i l'edifici més pròxim no hi ha mercaderies combustibles o elements capaços de propagar un incendi. Es desenvolupa tota l'activitat en la totalitat de la nau. L'obrador és una activitat classificada com a tipus C.\\
\newline Com que hem calculat una càrrega de foc de tipus baix 2, i la nau és de tipus C, el BOE indica que es poden construir fins a 6.000 $m^2$. En aquest cas hi ha 800 $m^2$ construïts, complim amb la normativa.\\

\section{\uppercase{Insta\Lgem ació contra incendis}}
L'obrador ha de disposar de les mesures correctes per extingir els incendis que es poden causar. Extintors, BIEs i hidrants són algunes d'aquestes mesures.\\ 
\newline Segons el reglament de seguretat contra incendis hi ha diferents tipus de focs. Els focs de classe A són focs de materials sòlids, generalment de naturalesa orgànica, la combinació dels quals s'efectua normalment amb la formació de brases. Els focs de classe B són focs de líquids o de sòlids liquables. Els focs de classe C són focs de gasos. Els focs de classe D són focs de metalls. Finalment, els focs de classe F són focs derivats de la utilització d'ingredients per cuinar (olis i greixos vegetals o animals) ens els aparells de cuina.\\
\newline A l'obrador poden haver-hi focs de classe A si per exemple es crema alguna taula o moble de fusta. Els focs de classe B es podrien donar amb algun líquid com alcohol. Un foc de classe C es podria donar per la combustió del gas que es fa servir per cuinar. Si es provoqués un incendi, el metall seria el més difícil d'encendre i es preveu que es pogués apagar amb un extintor de pols. Clarament també es poden donar focs de tipus F a la cuina. \\
\newline La instal·lació contra incendis de l'obrador està formada, majoritàriament, per extintors de pols. Aquests extintors permeten apagar els focs de tipus A, B i C i fins i tot els tipus D depenent del metall en qüestió. Com que l'element extintor és pols no hi ha risc de conduir l'electricitat. L'eficàcia d'aquests extintors és 21A, que és el mínim que marca la normativa.\\
\newline Disposem d'un extintor de $CO_2$ a tocar del quadre elèctric. Aquest extintor fa baixar la temperatura fins als -80 graus centígrads i es caracteritza per no conduir l'electricitat. Pot apagar focs dels tipus A, B i C. La seva eficàcia és de 98B.\\
\newline A la cuina els extintors són de tipus F, per poder apagar correctament el foc en olis i greixos. Aquests extintors descarreguen una fina boira a base d'acetat de potassi. Així creen una espuma que refreda l'oli i la grassa i la separen de l'aire. Aquests extintors són d'eficàcia 233B.\\
\newline Tal com marca la normativa, des de qualsevol punt de la instal·lació s'ha de poder arribar a un extintor recorrent, com a màxim, 15 m. S'han col·locat els extintors en punts estratègics, la majoria en zones comunes com els passadissos. Tots els extintors tenen marcatge CE i estan situats a una altura de 1,10 m. \\
\newline També es disposa de dues BIEs per tal de facilitar la feina als bombers en cas d'incendi, tot i que al tractar-se de tipus C i nivell baix no és obligat tenir-ne. Aquestes estan col·locades a 1 i 4 metres de les respectives entrades a l'obrador. La normativa marca un màxim de 5 metres respecte les entrades de l'obrador. Una està al costat del quadre elèctric i l'altre està al lloc per on surten els cuinats.\\
\newline Les mànegues de les BIEs són semirígides de 30 m de longitud. Al radi d'acció de cada BIE se li ha de sumar 5 metres, que és la longitud del raig d'aigua. Així, les BIEs de la instal·lació tenen un radi d'acció de 35 metres cada una. Cada zona de l'obrador està coberta per almenys una BIE.\\
\newline Les BIEs, de tipus DN 25 mm, poden donar 5 kg/$cm^2$ durant una hora i 20 minuts, compleixen amb el que marca el BOE. Amb una pressió de 10 kg/$cm^2$ durant 2 hores no experimenten fugues.\\
\newline La normativa marca que no és obligatori instal·lar alarmes de detecció automàtica d'incendis si l'edifici és de tipus C i nivell de risc intrínsec baix. Ara bé, si no s'instal·len sistemes de detecció automàtica d'incendis sí que calen sistemes manuals d'alarma d'incendi. A l'obrador, hi ha un polsador a cada sortida d'emergència així com a altres llocs de la nau. Es garanteix que la distància màxima des de qualsevol punt de la nau a un polsador és menor de 25 m.\\
\newline Es disposa de dos hidrants propers a la nau: un a 10 m i l'altre a 5, situats als laterals de la nau. Considerant que tenen un radi d'acció de 40 m, tal com marca la normativa, cobreixen tota la superfície de la nau. La pressió que poden donar és de 6 bar quan estan descarregant els cabals indicats anteriorment. Compleixen amb la normativa. Els dos tenen una boca de sortida de 100 mm.\\
\newline Ens han indicat que els hidrants poden donar fins a 1000 L/min durant 45 minuts. Els mínims que ens exigeixen són de 500 L/min durant 30 minuts.


\section{\uppercase{Tipus de materials}}
A l'obrador s'utilitzen diferents tipus de materials. Tots ells han de complir amb el Reglament de seguretat contra incendis en establiments industrials. Fer-ho dificulta el pas del foc en cas d'incendi i el malmetement dels elements físics de la nau.\\
\newline Es desenvolupa l'activitat en una planta sobre rasant, així que la protecció mínima contra el foc dels elements portants ha de ser RF-30; és la protecció que tenim.\\
\newline Per naus tipus C i amb nivell baix no s'exigeixen cobertes amb resistència al foc. Tot i així, l'obrador té una coberta de RF-15.\\
\newline Tota l'activitat de l'obrador es desenvolupa en una sola planta, la planta baixa. Compleix sense problemes amb la normativa.\\
\newline S'utilitza fusta pels mobles i per algunes portes. A la cuina les taules, estanteries i màquines són d'acer inoxidable, de classe M0. Les cambres i les portes són de panell amb una resistència al foc RF-30. Les parets, el panell d'alumini i altres elements constructius són del tipus M0, tal com marca la normativa. Els pilars de la nau tenen una protecció EF-30.\\
\newline El revestiment del terra té protecció M1, la normativa marca un mínim de M2.  Els revestiments exteriors de la façana tenen protecció contra el foc M1, compleixen amb la normativa. Les cambres de fred i de congelats tenen revestiments amb protecció M0. Els tubs dels extractors tenen una protecció M1.\\
\newline Les estanteries metàl·liques d'acer inoxidable són de classe A1, amb protecció contra el foc M0.\\
\newline Els cables elèctrics són no propagadors de flama, amb emissió de fums i opacitat reduïdes. Tenen recobriment XLPE.\\



\section{\uppercase{Condició d'evacuació i senyalització}}
En cas d'un incendi és molt important evacuar tot el personal ràpidament per a què els bombers o un grup de gent amb competències pugui solucionar el problema ocasionat. Per garantir la seguretat de les persones és important disposar de recorreguts d'emergència, amb distàncies no massa elevades i una correcta senyalització.\\
\newline El Reial decret 2267/2004 del 3 de desembre detalla aquests aspectes. El Reglament Electrotècnic de Baixa Tensió comenta com ha de ser l'enllumenat d'evacuació. La Norma Bàsica de la Edificació NBE-CPI/96 és consultada per verificar que l'amplada de les portes és correcta.\\
\newline La zona de venta al públic de l'obrador té 19,52 $m^2$ que poden ser ocupats pel públic. Segons la ITC-BT-28 del REBT la ocupació de locals es calcula com una persona per cada 0,8 $m^2$. Arrodonint, hi pot haver 25 persones a la sala de venta al públic com a clients. Es considera que hi ha uns 10 treballadors simultàniament a l'obrador. En total, es considera que l'ocupació de la nau és de 35 persones.\\
\newline Segons la normativa, per més de 25 persones i risc intrínsec baix la longitud d'emergència d'un recorregut únic ha de ser menor de 35 m. Si hi ha dues sortides alternatives, aquest número puja a 50 m. A l'obrador no hi ha cap recorregut, ja sigui únic o no, amb una distància superior a 25 m.\\
\newline Tots els extintors i BIEs estan senyalitzats amb el seu cartell corresponent de lletres blanques sobre fons vermell.\\
\newline Cada porta que forma part d'un recorregut d'evacuació està senyalitzada amb el característic cartell de lletres blanques sobre fons verd i marc de color groc pàlid. En ell s'indica amb una fletxa la direcció a seguir per evacuar l'edifici. Al seu costat o sota seu hi ha un llum d'emergència. \\
\newline El Reglament de seguretat contra incendis en establiments industrials marca que 
\begin{equation}
P = 1,10 p
\end{equation}
\noindent on p és el nombre de persones que pot ocupar l'edifici, 35 a l'obrador. Per tant, l'ocupació de l'obrador és $P=39$ persones. La normativa marca que és d'obligat compliment instal·lar llums d'emergència si P és igual o major de 25 persones.\\
\newline Els llums d'emergència són de tipus no permanent. Aquests llums disposen d'una bateria, així es carreguen. S'encenen quan la tensió de servei baixa del 70\% de tensió nominal. Encara que es talli el subministrament elèctric segueixen fent llum gràcies a les bateries prèviament carregades. Si això passa, han d'estar encesos per, com a mínim, una hora.\\
\newline S'ha determinat que a nivell de terra l'enllumenat d'evacuació dona 3 lux, el REBT marca un mínim de 1 lux. Als punts on hi ha equips manuals de protecció contra incendis, com extintors, i al Quadre General de Protecció i Comandament hi ha 10 lux; el REBT exigeix un mínim de 5 lux. El màxim a l'eix dels passos principals és de 100 lux, i el mínim de 3 lux; la relació és menor de 40, que és el màxim que marca el Reglament Electrotècnic de Baixa Tensió.\\
\newline Les portes de l'obrador d'una sola fulla mesuren 82 cm d'amplada i tenen una alçada de 2 m. Les de dues fulles també mesuren 2 metres d'alçada i tenen les fulles de 1,4 m. La normativa contra incendis remet a la NBE-CPI/96 la qual marca uns mínims de 80 cm d'amplada per les portes que siguin sortida d'evacuació. Per una sola fulla l'amplada màxima és de 1,20 m i en portes de dues fulles l'amplada mínima d'aquestes és de 0,60 m. La distància entre dues portes és menor a 25 metres.\\
\newline Aquesta mateixa normativa marca que els passadissos que formin part d'un recorregut d'evacuació han de fer, com a mínim, 1 m d'amplada. A l'obrador fan 1,5 m.

\clearpage
\chapter{\uppercase{Conclusió}}
El present document té l'objectiu de legalitzar la instal·lació contra incendis, els materials i les evacuacions d'emergència d'un obrador de plats cuinats.\\
\newline
Per desenvolupar aquesta memòria s'ha seguit el Reial decret 2267/2004 del 3 de desembre, el qual aprova el Reglament de seguretat contra incendis en establiments industrials. Amb aquest reglament es pot dir que la càrrega de foc és baixa i els materials són correctes.\\
\newline Per validar la instal·lació contra incendis s'ha seguit el Reglament d'instal·lacions de protecció contra incendis aprovat pel Reial decret 513/2017 del 22 de maig. Gràcies a aquest document s'ha pogut avaluar la instal·lació contra incendis com a correcta.\\
\newline El REBT s'ha consultat per verificar que els nivells d'il·luminació d'emergència mesurats són correctes. La Norma Básica de l'Edificació, NBE-CPI/96, ha permès considerar correctes les dimensions de les portes d'emergència i els passadissos.\\
%
\newline Amb tot l'indicat en aquesta memòria es considera que la nau té un risc baix d'incendi i que la instal·lació contra incendis, els materials i els recorreguts d'evacuació són els adequats segons la legislació present fins a dia d'avui. 

\vspace*{\fill}
\noindent Llorenç Fanals Batllori\\
Graduat en Enginyeria Electrònica Industrial i Automàtica\\
\\
\\
Girona, 26 d'octubre de 2019.

\clearpage

\begin{appendices}
%\chapter{Títol de l'annex}

%\chapter{\uppercase{Càlculs}}
Pel càlcul de les seccions dels conductors cal tenir en compte els factors de simultaneïtat d'alguns elements i els factors que marca el REBT: 1,25 pel motor elèctric de més potència de la línia, tal com es detalla a la ITC-47; i 1,8 per les lluminàries amb descàrrega, tal com s'indica a la ITC-44. A l'obrador hi ha molts motors elèctric però cap llum amb descàrrega.\\
\newline
En algunes línies es considera que el factor de potència és unitari. A la realitat mai valdrà exactament 1, però sí que es preveu que tingui un valor molt semblant. Les màquines que s'han escollit tenen un factor de potència proper a l'unitari però diferent de 1.\\
\newline Per calcular la intensitat de les línies monofàsiques es fa servir la següent fórmula:
\begin{equation}
I_{linia} = \frac{P}{V*\cos(\phi)}
\end{equation}
V = 230 V\\
P és la potència que consumeixen els elements connectats a la línia\\
$\phi$ és el factor de potència\\
\newline En trifàsic, l'equació que s'utilitza és:
\begin{equation}
I_{linia} = \frac{P}{\sqrt3*V_{linia}*\cos(\phi)}
\end{equation}
$V_{linia}$ = 400 V\\
\newline És important calcular la caiguda de tensió a les línies per tal de veure si estan dimensionades correctament. La caiguda de tensió en línies d'enllumenat no pot ser superior al 3\% i en línies de força no pot ser superior al 5\% de la tensió de subministrament. La caiguda de tensió màxima a la derivació individual és de 1,5\%.\\
\newline En monofàsic:
\begin{equation}
e(\%)=\frac{P}{V}\frac{2*l}{k*S}
\end{equation}
l és la longitud ja sigui de la fase o el neutre des del comptador a l'element més llunyà\\
$k = 56 \frac{m}{mm^{2}\si{\ohm}}$\\
S és la secció del cable en m$m^2$\\
\newline
En trifàsic, l'equació que s'utilitza és:
\begin{equation}
e(\%)=\frac{P}{V}\frac{l}{k*S}
\end{equation}
\\
El dimensionament de les línies ha de permetre que les caigudes de tensió no superin els màxims indicats prèviament. Alhora, els cables han de poder admetre les intensitats calculades, per això ens guiem amb la taula de la ITC-19 del REBT. Finalment, cal comprovar que  l'interruptor magnetotèrmic té una intensitat nominal superior a la calculada per la línia i menor a l'admissible que marca la ITC-19.\\
\newline La instal·lació és trifàsica, per tant, hi ha 3 conductors de fase i un conductor de neutre. El conductor de terra transcorre per totes les línies i té una secció igual als conductors de les línies, tal com s'indica al plànol. El neutre, que arriba per l'escomesa, també és de la mateixa secció que els conductors de fase. Les màquines trifàsiques necessiten el neutre pels seus equips electrònics.\\
\newline
Per comprovar que el valor de secció de la derivació individual és correcte quan la línia va amb una terna de cables unipolars per tub cal tenir en compte un factor d'intensitat de 0,8.
\begin{equation}
I_{DI} < 0.8 * I_{max. admissible}
\end{equation}

\noindent A continuació es mostren les diferents línies de forma detallada. La secció s'ha comprovat tenint en compte les fórmules explicades i les seccions mínimes per intensitat segons marca el REBT. S'han verificat les línies pel cas més desfavorable. Les seccions dels tubs compleixen amb la ITC-21.\\
\newline Les tensions nominals són 230 V per les línies monofàsiques i 400 V per les trifàsiques. Tots els cables de les línies són de coure de 450/750 V d'aïllament. La derivació individual és de coure amb 0,6/1 kV d'aïllament. L'aïllament de la instal·lació és de 1.000 k$\si{\ohm}$.

\begin{table}[H]
\scriptsize
\begin{center}
 \begin{tabu} to \textwidth {|X[0.5, l]|X[2, l]|X[r]|X[0.6, r]|X[r]|X[r]|X[r]|X[r]|X[r]|X[r]|X[0.5,r]|}%{X | c c c} 
 \hline
 Línia& Descripció & Potència (W) & cos($\phi$) & Intensitat (A) & Distància màxima (m) & Seccions fase, neutre, terra ($mm^{2}$) & Diàmetre tub (mm) & Caiguda de tensió (\%) & Caiguda de tensió acum. (\%)\\
 \hline \hline 
DI & Derivació individual& 87.000 \ \ \ \ & 0,96 & 131,32 & 8 &3x35 + 35 + 16& 160 & 0,23 & 0,23 \\ \hline
L1 & Enllumenat habitacions i cambra& 1.370,5 & 1 & 5,96 & 55 &2,5 + 2,5 + 2,5& 20 & 2,04 & 2,27 \\ \hline
L2 & Enllumenat cuina & 1.476 \ \ \ \  & 1 & 6,42 & 47 &2,5 + 2,5 + 2,5& 20 & 1,87 & 2,10 \\ \hline 
L3 & Força oficina, menjador, màquines de buit & 7.775 \ \ \ \  & 1 & 33,80 & 51 &10 + 10 + 10& 25 & 2,68 & 2,91 \\ \hline 
L4 & Força cuina & 7.235 \ \ \ \  & 1 & 31,46 & 33 & 6 + 6 + 6& 25 & 2,67 & 2,90 \\ \hline
L5 & Rentaplats cuina & 36.000 \ \ \ \ & 0,95 & 54,92 & 54 &3x16 + 16 + 16& 32 & 1,44 & 1,67 \\ \hline 
L6 & Extractors i cambres de fred & 19.000 \ \ \ \ & 0,9 & 30,60 & 36 &3x10 + 10 + 10& 32 & 0,86 & 1,09 \\ \hline
L7 & Abatidor sala de preparació & 16.975 \ \ \ \ & 0,95 & 25,90 & 44 &3x10 + 10 + 10& 32 & 0,84 & 1,07 \\ \hline

 \end{tabu}
 \caption{Línies detallades}
\end{center}
\end{table}



%\chapter{\uppercase{Característiques}}
L'aïllament dels cables elèctrics de les línies és EPR de 450/750 V d'aïllament. Els cables transcorren en safata perforada pel passadís i dins de tubs corrugats en muntatge superficial (B2) a la resta de zones. El diàmetre d'aquests tubs s'indica en l'anterior annex. Els càlculs s'han efectuat considerant que tota la llargada dels cables va amb el muntatge B2, que és més restrictiu que la safata.\\
\newline Es fan servir els colors gris, marró i negre per les fases, el blau pel neutre i el conductor groc i verd pel terra.\\
\newline Hi ha instal·lades caixes de derivació al llarg de la instal·lació i l'enllumenat dels vestidors, l'oficina i el menjador es controla amb interruptors de paret. Els cables dels l'enllumenats que no estan en contacte amb la paret es passen pel fals sostre.\\
\newline Les màquines trifàsiques es connecten a la xarxa mitjançant una base CETAC.\\
\newline Els extractors de la cuina de l'obrador van controlats amb variadors de freqüència. La seva línia va amb un diferencial de 100 mA de classe B degut als alts corrents de fuga que poden donar-se. Aigües amunt de tots els agrupaments hi ha instal·lat un diferencial de 300 mA de sensibilitat per protegir tota la instal·lació i alhora tenir selectivitat amb el diferencials que té aigües avall.\\
\newline El maxímetre del conjunt de protecció i mesura garanteix el subministrament elèctric tot i sobrepassar la potència contractada. Si en un moment puntual es connectés alguna màquina més i pel marge donat no saltés cap interruptor magnetotèrmic però s'estigués superant la potència contractada, hi seguiria havent subministrament elèctric i l'empresa subministradora aplicaria un recàrrec a la factura.\\
\newline 
S'agrupen les línies tenint en compte si el subministrament és trifàsic o monofàsic. S'intenta, en la mesura del possible, que tots els grups tinguin potències similars. És per això que el diferencial que agrupa les 4 línies monofàsiques és de 4 pols: els 3 conductors de fase i el neutre passaran per aquest diferencial i s'alimentaran les diferents línies monofàsiques amb diferents fases. Així es pot aconseguir una instal·lació trifàsica bastant ben equilibrada.\\
\newline Els llums d'emergència són de tipus no permanent i es considera que tenen una potència de 3 W. Al disposar de bateria i només encendre's quan hi ha una emergència, no s'han tingut en compte per la previsió de càrregues.\\
\newline Per millorar el factor de potència de la derivació individual, o sigui, de tota la instal·lació, hi ha instal·lada una bateria de condensadors de 20 kVAr la qual dona una factor de potència de 0,998.




\end{appendices}


\end{spacing}
%\cite{einstein} % per fer una cita
%\printbibliography[title=Bibliografia] %ARA BÉ

\end{document}