%Options > Configure Texmaker > Editor > Spelling Dictionary, per corrector en català

\documentclass[11pt, a4paper]{report}
\usepackage[a4paper,left=30mm,right=20mm,top=25mm,bottom=25mm]{geometry}
\sloppy %per forçar el canvi de línia si la paraula supera el marge dret
\usepackage[utf8]{inputenc}

% Per utilitzar la font Helvetica (Arial)
\renewcommand{\familydefault}{\sfdefault}
\usepackage[scaled=1]{helvet}
\usepackage[helvet]{sfmath}
\everymath={\sf}
%Equacions amb una font sans_serif, \mathrm{equació aquí, són les letres les que queden inclinades}
%\usepackage{arev} % sans-serif math font
%\usepackage{helvet} % sans-serif text font


% Per comptar imatges enlloc de mostrar 1.1, 1.2...
\usepackage{chngcntr}
\counterwithout{figure}{chapter}
\counterwithout{table}{chapter}
\counterwithout{equation}{chapter}

\usepackage{graphicx}
\graphicspath{{images/}} %directori amb les imatges que volem insertar
\usepackage{float} %per forçar imatges amb H
\usepackage[normalem]{ulem} %negreta múltiples línies
%\usepackage{soul}

\usepackage{caption}
\captionsetup[figure]{labelfont={},name={Figura},labelsep=period}
\captionsetup[table]{labelfont={},name={Taula},labelsep=period}


\usepackage{subcaption}
\usepackage{amsmath} %per fòrmules matemàtiques
\usepackage[table]{xcolor} %per colors a les taules
%\usepackage{circuitikz} %per circuits electrònics
\usepackage{siunitx} %per les labels dels components
\usepackage[american,cuteinductors,smartlabels]{circuitikz} %american/european
\usepackage{tikz} %quadrícula
\usepackage[a4paper, left=30mm, right=20mm, top=25mm, bottom=25mm]{geometry} %geometria de la pàgina, 25 però per ajustar bé
\setlength{\headsep}{20pt}
\setlength{\footskip}{25pt}
%\usepackage[a4paper, width=150mm, top=25mm, bottom=25mm]{geometry} %geometria de la pàgina
\usepackage{lipsum} %per generar dummy text
\usepackage{xpatch} %per la distància entre títol i top

%Capçaleres i peus de pàgina
\usepackage{fancyhdr}
%\pagestyle{fancy} %fancy, plain
\fancypagestyle{plain}{
  \fancyhf{}% Clear header/footer
  \fancyhead[L]{\footnotesize{Obrador de plats cuinats}}
  \fancyhead[R]{\footnotesize{Sorolls i vibracions}}
  \fancyfoot[R]{\footnotesize{\thepage}}
}
\pagestyle{plain}% Set page style to plain.

%\fancyhead{}
%\fancyhead[LO,LE]{PROJECTES}
%\fancyfoot{}
%\fancyfoot[LE,RO]{\thepage} %número de la pàgina, a la dreta
%\fancyfoot[LO, CE]{Capítol \thechapter} %nom del capítol, a l'esquerra
%\fancyfoot[CO, CE]{\href{https://github.com/LFanals}{Llorenç Fanals Batllori}} %nom de l'autor, al centre
% \renewcommand{\headrulewidth}{0.4pt}
%\renewcommand{\footrulewidth}{0.4pt}

%Per tenir el nombre de pàgina a l'inici d'un capítol
%\fancypagestyle{plain}{
%\fancyhf{}
%\renewcommand\headrulewidth{0pt}
%\fancyfoot[R]{\thepage}
%}

%Per configurar el color dels links i referències
\usepackage{color}
\usepackage{hyperref}
\hypersetup{
    colorlinks=true, %true si es volen links de colors
    linkcolor=black,  %colors de les referències internes, blue
    filecolor=magenta,      %magenta
    urlcolor=[rgb]{0,0,0}, %Color dels links d'Internet, sobre 255=2^8-1=2^0+...+2^7, {0,0.5,1}
}

%Bibliografia
\usepackage[backend=bibtex]{biblatex}
\addbibresource{bibliography.bib}

%Canviem el nom que hi ha per defecte als índex i altres, per passar-ho al català
\renewcommand{\contentsname}{Índex}
\renewcommand{\listfigurename}{Índex de figures}
\renewcommand{\chaptername}{Capítol}
\renewcommand{\appendixname}{Annex}
\renewcommand{\listtablename}{Índex de taules}
% \renewcommand{\figurename}{Figura} % ho tinc amb caption
% \captionsetup[table]{name=Taula} % ho tinc amb caption

\definecolor{color_quadricula}{HTML}{0066ff} %color per la quadrícula

% Pels circuits
%\usepackage[american]{circuitikz}
\usetikzlibrary{calc}
\ctikzset{bipoles/thickness=1}
\ctikzset{bipoles/length=1.2cm}
\ctikzset{bipoles/diode/height=.375}
\ctikzset{bipoles/diode/width=.3}
\ctikzset{tripoles/thyristor/height=.8}
\ctikzset{tripoles/thyristor/width=1}
\ctikzset{bipoles/vsourceam/height/.initial=.7}
\ctikzset{bipoles/vsourceam/width/.initial=.7}
\tikzstyle{every node}=[font=\small]
\tikzstyle{every path}=[line width=0.8pt,line cap=round,line join=round]

%Per insertar codi
\usepackage{listings}
\usepackage{color}
\definecolor{dkgreen}{rgb}{0,0.6,0}
\definecolor{gray}{rgb}{0.5,0.5,0.5}
\definecolor{mauve}{rgb}{0.58,0,0.82}

\lstset{frame=none, %tb, none
  language=Python,
  aboveskip=2mm,
  belowskip=3mm,
  showstringspaces=false,
  columns=flexible,
  basicstyle={\scriptsize\ttfamily}, %small
  numbers=none, %left
  numberstyle=\tiny\color{gray},
  keywordstyle=\color{blue},
  commentstyle=\color{dkgreen},
  stringstyle=\color{mauve},
  breaklines=true,
  breakatwhitespace=true,
  tabsize=3
}


%Per tenir el format de capítol correcte
\usepackage{titlesec}

\usepackage{etoolbox}
%\usepackage{hyperref}

%Per chapter
\titlespacing*{\chapter}{0pt}{-25pt}{11pt} %Espaiat del títol de capítol amb els altres elements
\titleformat{\chapter}[hang] %Per seguir escrivint darrera el número
{\normalfont\fontsize{11}{15}\bfseries}{\thechapter.}{0.4em}{\MakeUppercase} %\fontsize{Tamany}{Espai múltiples línies}

%Per secció
\titlespacing*{\section}{0pt}{11pt}{11pt} %Espaiat del títol de capítol amb els altres elements
\titleformat{\section}[hang] %Per seguir escrivint darrera el número
{\normalfont\fontsize{11}{15}}{\thesection.}{0.4em}{\bfseries} %\fontsize{Tamany}{Espai múltiples línies}

%Per subsecció
\titlespacing*{\subsection}{0pt}{11pt}{11pt} %Espaiat del títol de capítol amb els altres elements
\titleformat{\subsection}[hang] %Per seguir escrivint darrera el número
{\normalfont\fontsize{11}{15}}{\thesubsection.}{0.4em}{} %\fontsize{Tamany}{Espai múltiples línies}

%Per paràgraf
\titlespacing*{\paragraph}{0pt}{0pt}{22pt} %Espaiat del títol de capítol amb els altres elements
\titleformat{\paragraph}[hang] %Per seguir escrivint darrera el número
{\normalfont\fontsize{11}{15}}{}{}{} %\fontsize{Tamany}{Espai múltiples línies}


%Interlineat, 1.2*1.25=1.5
\linespread{1.25}

%Espaiat entre paràgrafs
%\setlength{\parskip}{22pt}
 

\makeatletter
\def\tagform@#1{\maketag@@@{(\ignorespaces{Eq.~#1}\unskip)}}
\makeatother



%Per no tenir negreta a l'index
\usepackage{etoolbox}% http://ctan.org/pkg/etoolbox
\makeatletter
\patchcmd{\l@chapter}{\bfseries}{}{}{}% \patchcmd{<cmd>}{<search>}{<replace>}{<success>}{<failure>}
\makeatother

%Per tenir punts a l'índex
\makeatletter
\renewcommand*\l@chapter{\@dottedtocline{0}{0em}{1.5em}}
\makeatother

%Per taula que adapta bé els espais
\usepackage{tabularx}
\usepackage{tabu} % http://mirrors.ibiblio.org/CTAN/macros/latex/contrib/tabu/tabu.pdf
\tabulinesep = 1mm
\usepackage[font=footnotesize]{caption} %Captions de les figures més petites

%Appendix
\usepackage[]{appendix} %toc, page

%Alinear al separador decimal amb espais
\usepackage{setspace}
\renewcommand*{\arraystretch}{1.25}



%-------------------------------------------------------------------------------------------------------------
%-------------------------------------------------------------------------------------------------------------
%-------------------------------------------------------------------------------------------------------------
%-------------------------------------------------------------------------------------------------------------

\begin{document}
\pagenumbering{Roman}


%\begin{titlepage}
	\begin{center}
		\vspace*{1cm}
		
		\Huge
		\textbf{Document per Projectes}
		
		\vspace{0.5cm}
		\LARGE
		Adaptat a \LaTeX
	
		\vspace{1.5cm}
		
		\textbf{Llorenç Fanals Batllori}
		
		\vfill
		
		\small
		%\uppercase{Un treball lliurat a la Universitat - en compliment dels requisits pel grau en -}\\
		% TFG
		
		\vspace{1cm}
		
		%\includegraphics[scale=width=0.4\textwidth]{images/a_graph}
	\end{center}
	
	\begin{flushright}
	\large	
	Departament o grup de recerca\\
	UdG\\
	%País\\
	28/08/2019
	\end{flushright}
	


\end{titlepage}

%\thispagestyle{plain}

\begin{center}
	\large
	\textbf{Informe}
	
	\vspace{0.4cm}
	\large
	Descripció
	
	\vspace{0.4cm}
	\textbf{Llorenç Fanals Batllori}
	
	\vspace{0.9cm}
	\textbf{Abstract}
\end{center}
\lipsum[1]




%\chapter*{Dedicacions}
%Dedico aquest treball a -

%\chapter*{Agraïments}
%Vull agraïr a \\

\cleardoublepage\pagenumbering{arabic}

\begin{spacing}{2}
\tableofcontents
\end{spacing}
%\listoffigures %No fa falta crec
%\listoftables %No fa falta crec
\begin{spacing}{1.5}

\chapter{\uppercase{Generalitats}}
L'objectiu d'aquest document és justificar que la immissió acústica de l'obrador es manté dins els nivells màxims que indiquen les normatives. Així mateix, es vol verificar que les vibracions prenen valors permesos. L'activitat està situada al carrer Ramon Serradell, número 27, a La Bisbal d'Empordà, en un polígon industrial\\
\newline Per justificar-ho s'ha consultat la Llei 16/2002 de 28 de juny que tracta sobre Protecció contra la Contaminació Acústica. Aquesta llei comenta com s'han de mesurar les immissions de sorolls i contempla una gran quantitat de casos o elements que poden causar contaminació acústica. També detalla com han de ser les mesures de vibracions i els seus valors permesos.\\
\newline S'ha tingut en compte el mapa de capacitat acústica de l'Ajuntament de la Bisbal d'Empordà. En ell es marquen clarament diferents zones segons la sensibilitat acústica màxima que admeten.\\
\newline L'Ajuntament de la Bisbal d'Empordà no té una ordenança de sorolls i vibracions, per tant, es segueix el Decret 176/2009 de 10 de novembre de Protecció contra la Contaminació Acústica, el qual s'aplica a Catalunya. Aquest decret modifica els annexos de la Llei de Protecció contra la Contaminació Acústica de la Llei 16/2002 de 28 de juny, que s'aplica a nivell nacional.


\clearpage
\chapter{\uppercase{Sorolls i vibracions}}
Les activitats industrials utilitzen màquines. Algunes d'aquestes màquines tenen parts mòbils, vibren i són sorolloses. La legislació marca uns màxims permesos de soroll i vibracions que han de ser respectats.
\section{\uppercase{Sorolls}}
El que s'entén per soroll és la superposició d'ones acústiques de diferents freqüències. Les múltiples activitats que es porten a terme en els nuclis habitats poden donar lloc a problemes de contaminació acústica que causen molèsties als ciutadans.\\
\newline L'annex 3 del Decret de Protecció contra la Contaminació Acústica de la Generalitat descriu la immissió sonora aplicable a l'ambient exterior produïda per les activitats, incloses les derivades de les relacions de veïns. Exposa els nivells màxims d'immissió segons la zona i segons l'horari. Indica com fer les mesures de forma correcta i quines equacions cal utilitzar.\\
\newline Es poden determinar els nivells d'immissió mitjançant mesures. S'han realitzat mesures contínues durant tot el període d'avaluació. La diferència entre els valors màxims obtinguts ha estat menor de 3 dB(A), per tant, s'ha pogut fer la mitjana.\\
\newline Els mesuraments s'han fet en un dia de Sol amb poca presència de vent; és el més habitual al polígon de l'obrador. Tot i això s'ha fet servir una pantalla paravent. El micròfon s'ha situat a una alçada de 1,5 metres a nivell de carrer, i a 1 metre de distància respecte la façana contigua a la cuina, que és on hi ha més soroll. S'ha calibrat l'aparell.\\
\newline S'han mesurat 45 dB(A) quan les màquines de l'obrador estaven en funcionament i 33 dB(A) quan no. La diferència és major de 10 dB(A) i per tant no cal eliminar el nivell de soroll residual.\\
\newline Per pendre mesures de soroll es fa servir la següent equació:
\begin{equation}
L_{Ar}= 10 \log\left ( \frac{1}{T} \sum_{i=1}^{n} \left ( T_{i}10^{\frac{L_{Ar,i}}{10}} \right ) \right )
\end{equation}
\noindent $L_{Ar}$: nivell d'avaluació del període.\\
 $i$: cadascuna de les fases del soroll.\\
 $T_{i}$: durada de la fase de soroll $i$ en minuts. La suma de tots el $T_i$ ha de ser $T$.\\
 $T$: temps de mesura. 180 minuts de dia, 120 minuts a l'horari de vespre, 120 minuts a l'horari de nit.\\
 $L_{Ar,i}$: nivell d'avaluació de la fase $i$. Es calcula amb:
\begin{equation}
L_{Ar,i} = L_{Aeq,Ti} + K_{f,i} + K_{t,i} + K_{i,i}
\end{equation}
\noindent $L_{Aeq,Ti}$: nivell de pressió acústica continu equivalent ponderat A, mesurat durant una fase de durada $T_i$.\\
$K_{f,i}$, $K_{t,i}$, $K_{i,i}$: correccions de nivell per a la fase $i$. Baixes freqüències, tonals i impulsius respectivament. No s'apliquen al soroll residual.\\
\newline La jornada laboral a l'obrador és de les 8:00 a les 16:00. Aquest període forma part completament del període de dia, el qual va de les 7:00 a les 21:00. Es pren $T=180$ minuts. Es fa una mesura contínua.\\
%
%
%
%
\newline Per determinar si hi ha components de baixa freqüència es miren les octaves de 20 a 160 Hz i es calcula la diferència entre els valors obtinguts.
\begin{equation}
L_{f} = L_{Ceq,T} - L_{Aeq,T}
\end{equation}
\noindent $L_{Ceq,T}, L_{Aeq,T}$: resultat de la mitjana dels tres mesuraments vàlids, com ja s'ha comentat anteriorment.\\
\newline Si la diferència $L_{Ceq,T} - L_{Aeq,T}$ és menor a 20 dB(A) es considera que no hi ha components de baixa freqüència significatius. A l'obrador, la diferència ha estat de 9 dB(A). Es pot dir que $K_f = 0$.\\
\newline Per determinar si hi ha presència de components totals emergents es fa una anàlisi espectral cada terç d'octava, entre 20 i 10.000 Hz. Es calcula la diferència com:
\begin{equation}
L_t = L_f - L_s
\end{equation}
\noindent $L_f$: nivell de pressió acústica de la banda f que conté el to emergent.\\
$L_s$: mitjana aritmètica dels nivells de la banda situada per immediatament per sobre i per sota de $f$.\\
$L_f$ i $L_s$: mitjana energètica de les tres mesures preses com a vàlides.\\
\newline S'ha calculat que \emph{$L_f$} a 50 Hz val 45 dB i que \emph{$L_s$} val 41 dB; la diferència és de 4 dB. Es determina, amb la taula del Decret, que es sobrepassa el nivell mínim audible $T_f$ a 50 Hz que és de 44,0 dB. Per tant, s'ha de considerar. Com que 4 està entre el rang $3 \leq  L_i \leq  6$,  $K_t = 3$.\\
\newline Per determinar si el soroll té components impulsius en primer lloc es mesura simultàniament el nivell de pressió acústica contínua equivalent ponderat $A$, $L_{Aeq,Ti}$, amb la constant temporal d'impuls $I$, $L_{AIeq,Ti}$, durant un temps $T_i$. En el nostre cas, en què es fa continu, no té sentit fer el càlcul, el resultat és 0:
\begin{equation}
L_i = L_{AIeq,T} - L_{Aeq,T}
\end{equation}
\noindent $L_{AIeq,T}, L_{Aeq,T}$: resultat de la mitjana energètica dels tres mesuraments considerats vàlids.\\
\newline En aquest cas $K_i = 0$.\\
\newline Amb aquesta informació podem determinar que 
\begin{equation}
L_{Ar} = L_{Aeq} + K_{f} + K_{t} + K_{i}
\end{equation}
\noindent \emph{$L_{Aeq}$} val 45 dB(A) i \emph{$K_{f}$} val 3 dB(A); sumen 48 dB(A).
\noindent Així, es conclou que la immissió a fora la nau causada per l'obrador és de 48 dB (A).\\
\newline Per deferència s'inclou una taula amb el nivell sonor en dB(A) de les màquines que hi ha instal·lades dins la nau. Els valors són donats pels fabricants.
\begin{table}[H]
\small
\begin{center}
 \begin{tabu} to \textwidth {|X[l]|X[2, l]|X[0.6, r]|X[0.5, r]|}
 \hline
Màquina & Model & Quantitat & Soroll (dBA) \\
 \hline \hline 
Rentaplats & AD-125 SOFT HRS 400/230/230V 3N/3/1N 50H & 3 & 72,0 \\ \hline
Forn & APE-201 400/230V 3N/3 50/60Hz & 4 & 60,0 \\ \hline
Extractor & SP CRMT/4-315/130-4 & 2 & 78,0 \\ \hline
Màquina de buit & SV-2-850L/100 230-400V 3N 50Hz & 3 & 63,0 \\ \hline
Abatidor & CMKP-202D PAS S.P.CAL SUELO 400V 3N 50Hz , UCC-1052 No 400V 3N 50Hz & 1 & 60,0 \\ \hline
Congelador & Polar CD085 & 2 & 40,0 \\ \hline
Nevera & Polar CD084 & 2 & 40,0 \\ \hline
Màquina de fred per congelador & BSB330DB11XX & 2 & 44,0 \\ \hline
Màquina de fred per nevera & BSB220DA11XX & 2 & 40,0 \\ \hline
 \end{tabu}
 \caption{Nivells d'emissió de soroll donats pels fabricants}
\end{center}
\end{table}
\noindent Com s'observa, bastantes màquines emeten més de 48 dB(A), que és el valor mesurat, però totes menys les màquines de fred se situen a l'interior de la nau. Cal tenir en compte que les parets de formigó de la nau, de 20 cm, tenen un aïllament acústic de 57 dB(A). La fórmula que relaciona l'aïllament amb els nivells de pressió sonora és:
\begin{equation}
D = L_1 - L_2
\end{equation}
\noindent D: aïllament acústic.\\
$L_1$: nivell de pressió sonora a l'emissor.\\
$L_2$: nivell de pressió sonora al local receptor.\\
%
%
%
%
%
%
\newline L'Ajuntament de la Bisbal d'Empordà té un mapa de zones de sensibilitat acústica.\\
\newline La nau està classificada com a tipus C amb un nivell de risc d'incendi intrínsec baix 2. Està situada en un polígon industrial, el mapa indica que la seva zona de sensibilitat acústica és C2. Els nivells màxims permesos a aquesta zona són:
\begin{table}[H]
\small
\begin{center}
 \begin{tabular} {|l|r|r|r|}
     \hline
   Zona & \multicolumn{3}{| c |}{Valors límits d'immissió en dB(A)}\\
    \hline
    & $L_{d (7h-21h)}$ & $L_{d (21h-23h)}$ & $L_{d (23h-7h)}$ \\ \hline
C2 - Predomini del sòl industrial & 70 & 70 & 60 \\ \hline

 \end{tabular}
 \caption{Valors d'immissió acústica en zona C a La Bisbal d'Empordà}
\end{center}
\end{table}
%
\noindent La jornada laboral a l'obrador és de 8:00 a 16:00, el límit d'immissió és de 70 dB(A). Els 48 dB(A), que es donen en el primer període únicament, compleixen perfectament amb els límits marcats per l'Ajuntament de La Bisbal d'Empordà.\\
\newline Fora de la jornada laboral els equips de fred segueixen funcionant, però la resta de màquines estan apagades. Durant els tres períodes es compleix amb els límits d'immissió acústica marcats per l'Ajuntament de la Bisbal d'Empordà.


\section{\uppercase{Vibracions}}
%comentar silent blocks de les màquines i la campana.
Les vibracions són moviments periòdics dels punts materials que componen un cos. Són originades per una font d'energia mecànica que provoca deformacions elàstiques i l'aparició de forces externes o internes que tendeixen a frenar el cos.\\
\newline Es segueix la Llei 16/2002 de 28 de juny de Protecció contra la Contaminació acústica per validar que les acceleracions prenen valors correctes.\\
\newline Per reduir les vibracions les màquines porten incorporats blocs silenciosos, anomenats vulgarment silentblocks. Un bloc silenciós o bloc antivibratori està fet d'un material flexible. Solen estar fabricats de cautxú o teixit d'acer inoxidable. Aquests materials tenen un mòdul de Young relativament petit, un rang de deformació elàstica generós i destaquen per la facilitat d'absorbir vibracions. Aquestes característiques els permeten deformar-se elàsticament sota l'acció d'una força i mantenir la forma. Solen treballar a compressió.\\
\newline A l'obrador disposem de dos extractors situats sobre el fals sostre de 3,5 m d'altura. Com s'aprecia als plànols, les canonades d'extracció estan unides amb juntes antivibratòries i la caixa que conté la turbina es recolza sobre el fals sostre amb 2 blocs silenciosos per pota, treballant a compressió. A més, els tubs extractors s'uneixen mitjançant una brida i un bloc silenciós a les bigues que tenen al seu costat.\\
\newline Com s'indica anteriorment, la turbina de l'extractor arriba a emetre 78 dB(A) de soroll, cosa que porta a pensar que les vibracions són molt considerables. És per això que a l'obrador hi ha instal·lats blocs silenciosos i juntes antivibratòries. La màquina, segons el fabricant, ja porta blocs silenciosos incorporats, però el fet d'haver-ne instal·lat d'addicionals fa disminuir encara més les vibracions.\\
\newline Les màquines de buit, els rentaplats, les neveres i els congeladors disposen de potes i es recolzen sobre el paviment de la nau. Aquestes potes tenen blocs silenciosos de cautxú o de nylon segons la màquina. Ambdós materials fan disminuir les vibracions. A més, algunes d'aquestes màquines porten blocs silenciosos que el fabricant ha incorporat a l'interior.\\
\newline Els equips de fred situats a l'exterior de la nau recolzen el seu pes sobre l'estructura d'aquesta mitjançant uns suports horitzontals. Entre aquests suports i les màquines hi ha blocs silenciosos treballant a compressió.\\
%
%
%
%
%
%
%
\newline S'han efectuat mesures d'acceleracions estacionàries durant 2 minuts. S'ha situat l'acceleròmetre al terra, al centre de l'oficina. S'han mesurat acceleracions en els tres eixos. S'ha realitzat una verificació acústica i s'ha assegurat que la desviació és menor de 0,5 dB respecte el valor de referència actual.\\
\newline S'ha recollit tot l'espectre freqüencial de 1 Hz a 80 Hz amb resolució gràfica i física d'un terç d'octava. A continuació s'ha atenuat cada freqüència segons la corba d'atenuació:
\begin{equation}
\sqrt{1+\left ( \frac{f}{5,6} \right )^2}
\end{equation}
\noindent S'ha fet una suma quadràtica del mòdul de cada acceleració.
\begin{equation}
a_w = \sqrt{a_{1 \  Hz}^2 + a_{1,25 \  Hz}^2  + ... + a_{80 \  Hz}^2}
\end{equation}
\noindent El resultat ha estat de $ 5*10^{-4}$ m/$s^2$.\\
\newline Finalment s'ha calculat el nivell d'avaluació $L_{aw}$:
\begin{equation}
L_{aw} = 20 \log \left( \frac{a_w}{a_0} \right)
\end{equation}
\noindent $a_0$: acceleració de referència; $a_0 = 10^{-6} m/s^2$.\\
\newline El valor de nivell d'avaluació calculat ha estat de 54 dB, compleix sense problemes el màxim marcat per la Llei de Protecció contra la Contaminació Acústica. Aquesta llei indica que en una zona de sensibilitat baixa, com és el cas del polígon on està situat l'obrador, aquest màxim és de 80 dB. 




\clearpage
\chapter{\uppercase{Conclusió}}
El present document té l'objectiu de legalitzar la instal·lació contra incendis, els materials i les evacuacions d'emergència d'un obrador de plats cuinats.\\
\newline
Per desenvolupar aquesta memòria s'ha seguit el Reial decret 2267/2004 del 3 de desembre, el qual aprova el Reglament de seguretat contra incendis en establiments industrials. Amb aquest reglament es pot dir que la càrrega de foc és baixa i els materials són correctes.\\
\newline Per validar la instal·lació contra incendis s'ha seguit el Reglament d'instal·lacions de protecció contra incendis aprovat pel Reial decret 513/2017 del 22 de maig. Gràcies a aquest document s'ha pogut avaluar la instal·lació contra incendis com a correcta.\\
\newline El REBT s'ha consultat per verificar que els nivells d'il·luminació d'emergència mesurats són correctes. La Norma Básica de l'Edificació, NBE-CPI/96, ha permès considerar correctes les dimensions de les portes d'emergència i els passadissos.\\
%
\newline Amb tot l'indicat en aquesta memòria es considera que la nau té un risc baix d'incendi i que la instal·lació contra incendis, els materials i els recorreguts d'evacuació són els adequats segons la legislació present fins a dia d'avui. 

\vspace*{\fill}
\noindent Llorenç Fanals Batllori\\
Graduat en Enginyeria Electrònica Industrial i Automàtica\\
\\
\\
Girona, 26 d'octubre de 2019.

\clearpage

\begin{appendices}
%\chapter{Títol de l'annex}

%\chapter{\uppercase{Càlculs}}
Pel càlcul de les seccions dels conductors cal tenir en compte els factors de simultaneïtat d'alguns elements i els factors que marca el REBT: 1,25 pel motor elèctric de més potència de la línia, tal com es detalla a la ITC-47; i 1,8 per les lluminàries amb descàrrega, tal com s'indica a la ITC-44. A l'obrador hi ha molts motors elèctric però cap llum amb descàrrega.\\
\newline
En algunes línies es considera que el factor de potència és unitari. A la realitat mai valdrà exactament 1, però sí que es preveu que tingui un valor molt semblant. Les màquines que s'han escollit tenen un factor de potència proper a l'unitari però diferent de 1.\\
\newline Per calcular la intensitat de les línies monofàsiques es fa servir la següent fórmula:
\begin{equation}
I_{linia} = \frac{P}{V*\cos(\phi)}
\end{equation}
V = 230 V\\
P és la potència que consumeixen els elements connectats a la línia\\
$\phi$ és el factor de potència\\
\newline En trifàsic, l'equació que s'utilitza és:
\begin{equation}
I_{linia} = \frac{P}{\sqrt3*V_{linia}*\cos(\phi)}
\end{equation}
$V_{linia}$ = 400 V\\
\newline És important calcular la caiguda de tensió a les línies per tal de veure si estan dimensionades correctament. La caiguda de tensió en línies d'enllumenat no pot ser superior al 3\% i en línies de força no pot ser superior al 5\% de la tensió de subministrament. La caiguda de tensió màxima a la derivació individual és de 1,5\%.\\
\newline En monofàsic:
\begin{equation}
e(\%)=\frac{P}{V}\frac{2*l}{k*S}
\end{equation}
l és la longitud ja sigui de la fase o el neutre des del comptador a l'element més llunyà\\
$k = 56 \frac{m}{mm^{2}\si{\ohm}}$\\
S és la secció del cable en m$m^2$\\
\newline
En trifàsic, l'equació que s'utilitza és:
\begin{equation}
e(\%)=\frac{P}{V}\frac{l}{k*S}
\end{equation}
\\
El dimensionament de les línies ha de permetre que les caigudes de tensió no superin els màxims indicats prèviament. Alhora, els cables han de poder admetre les intensitats calculades, per això ens guiem amb la taula de la ITC-19 del REBT. Finalment, cal comprovar que  l'interruptor magnetotèrmic té una intensitat nominal superior a la calculada per la línia i menor a l'admissible que marca la ITC-19.\\
\newline La instal·lació és trifàsica, per tant, hi ha 3 conductors de fase i un conductor de neutre. El conductor de terra transcorre per totes les línies i té una secció igual als conductors de les línies, tal com s'indica al plànol. El neutre, que arriba per l'escomesa, també és de la mateixa secció que els conductors de fase. Les màquines trifàsiques necessiten el neutre pels seus equips electrònics.\\
\newline
Per comprovar que el valor de secció de la derivació individual és correcte quan la línia va amb una terna de cables unipolars per tub cal tenir en compte un factor d'intensitat de 0,8.
\begin{equation}
I_{DI} < 0.8 * I_{max. admissible}
\end{equation}

\noindent A continuació es mostren les diferents línies de forma detallada. La secció s'ha comprovat tenint en compte les fórmules explicades i les seccions mínimes per intensitat segons marca el REBT. S'han verificat les línies pel cas més desfavorable. Les seccions dels tubs compleixen amb la ITC-21.\\
\newline Les tensions nominals són 230 V per les línies monofàsiques i 400 V per les trifàsiques. Tots els cables de les línies són de coure de 450/750 V d'aïllament. La derivació individual és de coure amb 0,6/1 kV d'aïllament. L'aïllament de la instal·lació és de 1.000 k$\si{\ohm}$.

\begin{table}[H]
\scriptsize
\begin{center}
 \begin{tabu} to \textwidth {|X[0.5, l]|X[2, l]|X[r]|X[0.6, r]|X[r]|X[r]|X[r]|X[r]|X[r]|X[r]|X[0.5,r]|}%{X | c c c} 
 \hline
 Línia& Descripció & Potència (W) & cos($\phi$) & Intensitat (A) & Distància màxima (m) & Seccions fase, neutre, terra ($mm^{2}$) & Diàmetre tub (mm) & Caiguda de tensió (\%) & Caiguda de tensió acum. (\%)\\
 \hline \hline 
DI & Derivació individual& 87.000 \ \ \ \ & 0,96 & 131,32 & 8 &3x35 + 35 + 16& 160 & 0,23 & 0,23 \\ \hline
L1 & Enllumenat habitacions i cambra& 1.370,5 & 1 & 5,96 & 55 &2,5 + 2,5 + 2,5& 20 & 2,04 & 2,27 \\ \hline
L2 & Enllumenat cuina & 1.476 \ \ \ \  & 1 & 6,42 & 47 &2,5 + 2,5 + 2,5& 20 & 1,87 & 2,10 \\ \hline 
L3 & Força oficina, menjador, màquines de buit & 7.775 \ \ \ \  & 1 & 33,80 & 51 &10 + 10 + 10& 25 & 2,68 & 2,91 \\ \hline 
L4 & Força cuina & 7.235 \ \ \ \  & 1 & 31,46 & 33 & 6 + 6 + 6& 25 & 2,67 & 2,90 \\ \hline
L5 & Rentaplats cuina & 36.000 \ \ \ \ & 0,95 & 54,92 & 54 &3x16 + 16 + 16& 32 & 1,44 & 1,67 \\ \hline 
L6 & Extractors i cambres de fred & 19.000 \ \ \ \ & 0,9 & 30,60 & 36 &3x10 + 10 + 10& 32 & 0,86 & 1,09 \\ \hline
L7 & Abatidor sala de preparació & 16.975 \ \ \ \ & 0,95 & 25,90 & 44 &3x10 + 10 + 10& 32 & 0,84 & 1,07 \\ \hline

 \end{tabu}
 \caption{Línies detallades}
\end{center}
\end{table}



%\chapter{\uppercase{Característiques}}
L'aïllament dels cables elèctrics de les línies és EPR de 450/750 V d'aïllament. Els cables transcorren en safata perforada pel passadís i dins de tubs corrugats en muntatge superficial (B2) a la resta de zones. El diàmetre d'aquests tubs s'indica en l'anterior annex. Els càlculs s'han efectuat considerant que tota la llargada dels cables va amb el muntatge B2, que és més restrictiu que la safata.\\
\newline Es fan servir els colors gris, marró i negre per les fases, el blau pel neutre i el conductor groc i verd pel terra.\\
\newline Hi ha instal·lades caixes de derivació al llarg de la instal·lació i l'enllumenat dels vestidors, l'oficina i el menjador es controla amb interruptors de paret. Els cables dels l'enllumenats que no estan en contacte amb la paret es passen pel fals sostre.\\
\newline Les màquines trifàsiques es connecten a la xarxa mitjançant una base CETAC.\\
\newline Els extractors de la cuina de l'obrador van controlats amb variadors de freqüència. La seva línia va amb un diferencial de 100 mA de classe B degut als alts corrents de fuga que poden donar-se. Aigües amunt de tots els agrupaments hi ha instal·lat un diferencial de 300 mA de sensibilitat per protegir tota la instal·lació i alhora tenir selectivitat amb el diferencials que té aigües avall.\\
\newline El maxímetre del conjunt de protecció i mesura garanteix el subministrament elèctric tot i sobrepassar la potència contractada. Si en un moment puntual es connectés alguna màquina més i pel marge donat no saltés cap interruptor magnetotèrmic però s'estigués superant la potència contractada, hi seguiria havent subministrament elèctric i l'empresa subministradora aplicaria un recàrrec a la factura.\\
\newline 
S'agrupen les línies tenint en compte si el subministrament és trifàsic o monofàsic. S'intenta, en la mesura del possible, que tots els grups tinguin potències similars. És per això que el diferencial que agrupa les 4 línies monofàsiques és de 4 pols: els 3 conductors de fase i el neutre passaran per aquest diferencial i s'alimentaran les diferents línies monofàsiques amb diferents fases. Així es pot aconseguir una instal·lació trifàsica bastant ben equilibrada.\\
\newline Els llums d'emergència són de tipus no permanent i es considera que tenen una potència de 3 W. Al disposar de bateria i només encendre's quan hi ha una emergència, no s'han tingut en compte per la previsió de càrregues.\\
\newline Per millorar el factor de potència de la derivació individual, o sigui, de tota la instal·lació, hi ha instal·lada una bateria de condensadors de 20 kVAr la qual dona una factor de potència de 0,998.




\end{appendices}


\end{spacing}
%\cite{einstein} % per fer una cita
%\printbibliography[title=Bibliografia] %ARA BÉ

\end{document}