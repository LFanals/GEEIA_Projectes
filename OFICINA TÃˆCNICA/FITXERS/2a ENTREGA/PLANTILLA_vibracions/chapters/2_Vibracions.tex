\section{\uppercase{Vibracions}}
%comentar silent blocks de les màquines i la campana.
Les vibracions són moviments periòdics dels punts materials que componen un cos. Són originades per una font d'energia mecànica que provoca deformacions elàstiques i l'aparició de forces externes o internes que tendeixen a frenar el cos.\\
\newline Es segueix la Llei 16/2002 de 28 de juny de Protecció contra la Contaminació acústica per validar que les acceleracions prenen valors correctes.\\
\newline Per reduir les vibracions les màquines porten incorporats blocs silenciosos, anomenats vulgarment silentblocks. Un bloc silenciós o bloc antivibratori està fet d'un material flexible. Solen estar fabricats de cautxú o teixit d'acer inoxidable. Aquests materials tenen un mòdul de Young relativament petit, un rang de deformació elàstica generós i destaquen per la facilitat d'absorbir vibracions. Aquestes característiques els permeten deformar-se elàsticament sota l'acció d'una força i mantenir la forma. Solen treballar a compressió.\\
\newline A l'obrador disposem de dos extractors situats sobre el fals sostre de 3,5 m d'altura. Com s'aprecia als plànols, les canonades d'extracció estan unides amb juntes antivibratòries i la caixa que conté la turbina es recolza sobre el fals sostre amb 2 blocs silenciosos per pota, treballant a compressió. A més, els tubs extractors s'uneixen mitjançant una brida i un bloc silenciós a les bigues que tenen al seu costat.\\
\newline Com s'indica anteriorment, la turbina de l'extractor arriba a emetre 78 dB(A) de soroll, cosa que porta a pensar que les vibracions són molt considerables. És per això que a l'obrador hi ha instal·lats blocs silenciosos i juntes antivibratòries. La màquina, segons el fabricant, ja porta blocs silenciosos incorporats, però el fet d'haver-ne instal·lat d'addicionals fa disminuir encara més les vibracions.\\
\newline Les màquines de buit, els rentaplats, les neveres i els congeladors disposen de potes i es recolzen sobre el paviment de la nau. Aquestes potes tenen blocs silenciosos de cautxú o de nylon segons la màquina. Ambdós materials fan disminuir les vibracions. A més, algunes d'aquestes màquines porten blocs silenciosos que el fabricant ha incorporat a l'interior.\\
\newline Els equips de fred situats a l'exterior de la nau recolzen el seu pes sobre l'estructura d'aquesta mitjançant uns suports horitzontals. Entre aquests suports i les màquines hi ha blocs silenciosos treballant a compressió.\\
%
%
%
%
%
%
%
\newline S'han efectuat mesures d'acceleracions estacionàries durant 2 minuts. S'ha situat l'acceleròmetre al terra, al centre de l'oficina. S'han mesurat acceleracions en els tres eixos. S'ha realitzat una verificació acústica i s'ha assegurat que la desviació és menor de 0,5 dB respecte el valor de referència actual.\\
\newline S'ha recollit tot l'espectre freqüencial de 1 Hz a 80 Hz amb resolució gràfica i física d'un terç d'octava. A continuació s'ha atenuat cada freqüència segons la corba d'atenuació:
\begin{equation}
\sqrt{1+\left ( \frac{f}{5,6} \right )^2}
\end{equation}
\noindent S'ha fet una suma quadràtica del mòdul de cada acceleració.
\begin{equation}
a_w = \sqrt{a_{1 \  Hz}^2 + a_{1,25 \  Hz}^2  + ... + a_{80 \  Hz}^2}
\end{equation}
\noindent El resultat ha estat de $ 5*10^{-4}$ m/$s^2$.\\
\newline Finalment s'ha calculat el nivell d'avaluació $L_{aw}$:
\begin{equation}
L_{aw} = 20 \log \left( \frac{a_w}{a_0} \right)
\end{equation}
\noindent $a_0$: acceleració de referència; $a_0 = 10^{-6} m/s^2$.\\
\newline El valor de nivell d'avaluació calculat ha estat de 54 dB, compleix sense problemes el màxim marcat per la Llei de Protecció contra la Contaminació Acústica. Aquesta llei indica que en una zona de sensibilitat baixa, com és el cas del polígon on està situat l'obrador, aquest màxim és de 80 dB. 




\clearpage