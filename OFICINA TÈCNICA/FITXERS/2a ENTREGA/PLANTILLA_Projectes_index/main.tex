\documentclass[11pt, a4paper]{report}
\sloppy %per forçar el canvi de línia si la paraula supera el marge dret
\usepackage[utf8]{inputenc}
\usepackage[T1]{fontenc}
\usepackage[utf8]{inputenc}
\usepackage[catalan]{babel}
\usepackage{newunicodechar}
\newunicodechar{Ŀ}{\L.}
\newunicodechar{ŀ}{\l.}


%Options > Configure Texmaker > Editor > Spelling Dictionary, per corrector en català

% Per utilitzar la font Helvetica (Arial)
\renewcommand{\familydefault}{\sfdefault}
\usepackage[scaled=1]{helvet}
\usepackage[helvet]{sfmath}
\everymath={\sf}
%Equacions amb una font sans_serif, \mathrm{equació aquí, són les letres les que queden inclinades}
%\usepackage{arev} % sans-serif math font
%\usepackage{helvet} % sans-serif text font


% Per comptar imatges enlloc de mostrar 1.1, 1.2...
\usepackage{chngcntr}
\counterwithout{figure}{chapter}
\counterwithout{table}{chapter}
\counterwithout{equation}{chapter}

\usepackage{graphicx}
\graphicspath{{images/}} %directori amb les imatges que volem insertar
\usepackage{float} %per forçar imatges amb H
\usepackage[normalem]{ulem} %negreta múltiples línies
%\usepackage{soul}

\usepackage{caption}
\captionsetup[figure]{labelfont={},name={Figura},labelsep=period}
\captionsetup[table]{labelfont={},name={Taula},labelsep=period}


\usepackage{subcaption}
\usepackage{amsmath} %per fòrmules matemàtiques
\usepackage[table]{xcolor} %per colors a les taules
%\usepackage{circuitikz} %per circuits electrònics
\usepackage{siunitx} %per les labels dels components
\usepackage[american,cuteinductors,smartlabels]{circuitikz} %american/european
\usepackage{tikz} %quadrícula
\usepackage[a4paper, left=30mm, right=20mm, top=25mm, bottom=25mm]{geometry} %geometria de la pàgina
\setlength{\headsep}{20pt}
%\usepackage[a4paper, width=150mm, top=25mm, bottom=25mm]{geometry} %geometria de la pàgina
\usepackage{lipsum} %per generar dummy text
\usepackage{xpatch} %per la distància entre títol i top

%Capçaleres i peus de pàgina
\usepackage{fancyhdr}
%\pagestyle{fancy} %fancy, plain
\fancypagestyle{plain}{
  \fancyhf{}% Clear header/footer
  \fancyhead[L]{\footnotesize{Obrador de plats cuinats}}
  \fancyhead[R]{\footnotesize{Plànols contraincendis}}
  \fancyfoot[R]{\footnotesize{\thepage}}
}
\pagestyle{plain}% Set page style to plain.

%\fancyhead{}
%\fancyhead[LO,LE]{PROJECTES}
%\fancyfoot{}
%\fancyfoot[LE,RO]{\thepage} %número de la pàgina, a la dreta
%\fancyfoot[LO, CE]{Capítol \thechapter} %nom del capítol, a l'esquerra
%\fancyfoot[CO, CE]{\href{https://github.com/LFanals}{Llorenç Fanals Batllori}} %nom de l'autor, al centre
% \renewcommand{\headrulewidth}{0.4pt}
%\renewcommand{\footrulewidth}{0.4pt}

%Per tenir el nombre de pàgina a l'inici d'un capítol
%\fancypagestyle{plain}{
%\fancyhf{}
%\renewcommand\headrulewidth{0pt}
%\fancyfoot[R]{\thepage}
%}

%Per configurar el color dels links i referències
\usepackage{color}
\usepackage{hyperref}
\hypersetup{
    colorlinks=true, %true si es volen links de colors
    linkcolor=black,  %colors de les referències internes, blue
    filecolor=magenta,      %magenta
    urlcolor=[rgb]{0,0,0}, %Color dels links d'Internet, sobre 255=2^8-1=2^0+...+2^7, {0,0.5,1}
}

%Bibliografia
\usepackage[backend=bibtex]{biblatex}
\addbibresource{bibliography.bib}

%Canviem el nom que hi ha per defecte als índex i altres, per passar-ho al català
\renewcommand{\contentsname}{Índex}
\renewcommand{\listfigurename}{Índex de figures}
\renewcommand{\chaptername}{Capítol}
\renewcommand{\appendixname}{Annex}
\renewcommand{\listtablename}{Índex de taules}
% \renewcommand{\figurename}{Figura} % ho tinc amb caption
% \captionsetup[table]{name=Taula} % ho tinc amb caption

\definecolor{color_quadricula}{HTML}{0066ff} %color per la quadrícula

% Pels circuits
%\usepackage[american]{circuitikz}
\usetikzlibrary{calc}
\ctikzset{bipoles/thickness=1}
\ctikzset{bipoles/length=1.2cm}
\ctikzset{bipoles/diode/height=.375}
\ctikzset{bipoles/diode/width=.3}
\ctikzset{tripoles/thyristor/height=.8}
\ctikzset{tripoles/thyristor/width=1}
\ctikzset{bipoles/vsourceam/height/.initial=.7}
\ctikzset{bipoles/vsourceam/width/.initial=.7}
\tikzstyle{every node}=[font=\small]
\tikzstyle{every path}=[line width=0.8pt,line cap=round,line join=round]

%Per insertar codi
\usepackage{listings}
\usepackage{color}
\definecolor{dkgreen}{rgb}{0,0.6,0}
\definecolor{gray}{rgb}{0.5,0.5,0.5}
\definecolor{mauve}{rgb}{0.58,0,0.82}

\lstset{frame=none, %tb, none
  language=Python,
  aboveskip=2mm,
  belowskip=3mm,
  showstringspaces=false,
  columns=flexible,
  basicstyle={\scriptsize\ttfamily}, %small
  numbers=none, %left
  numberstyle=\tiny\color{gray},
  keywordstyle=\color{blue},
  commentstyle=\color{dkgreen},
  stringstyle=\color{mauve},
  breaklines=true,
  breakatwhitespace=true,
  tabsize=3
}


%Per tenir el format de capítol correcte
\usepackage{titlesec}

\usepackage{etoolbox}
%\usepackage{hyperref}

%Per chapter
\titlespacing*{\chapter}{0pt}{-11pt}{11pt} %Espaiat del títol de capítol amb els altres elements
\titleformat{\chapter}[hang] %Per seguir escrivint darrera el número
{\normalfont\fontsize{11}{15}\bfseries}{\thechapter.}{0.4em}{\MakeUppercase} %\fontsize{Tamany}{Espai múltiples línies}

%Per secció
\titlespacing*{\section}{0pt}{11pt}{11pt} %Espaiat del títol de capítol amb els altres elements
\titleformat{\section}[hang] %Per seguir escrivint darrera el número
{\normalfont\fontsize{11}{15}}{\thesection.}{0.4em}{\bfseries} %\fontsize{Tamany}{Espai múltiples línies}

%Per subsecció
\titlespacing*{\subsection}{0pt}{11pt}{11pt} %Espaiat del títol de capítol amb els altres elements
\titleformat{\subsection}[hang] %Per seguir escrivint darrera el número
{\normalfont\fontsize{11}{15}}{\thesubsection.}{0.4em}{} %\fontsize{Tamany}{Espai múltiples línies}

%Per paràgraf
\titlespacing*{\paragraph}{0pt}{0pt}{22pt} %Espaiat del títol de capítol amb els altres elements
\titleformat{\paragraph}[hang] %Per seguir escrivint darrera el número
{\normalfont\fontsize{11}{15}}{}{}{} %\fontsize{Tamany}{Espai múltiples línies}


%Interlineat, 1.2*1.25=1.5
\linespread{1.25}

%Espaiat entre paràgrafs
%\setlength{\parskip}{22pt}
 

\makeatletter
\def\tagform@#1{\maketag@@@{(\ignorespaces{Eq.~#1}\unskip)}}
\makeatother



%Per no tenir negreta a l'index
\usepackage{etoolbox}% http://ctan.org/pkg/etoolbox
\makeatletter
\patchcmd{\l@chapter}{\bfseries}{}{}{}% \patchcmd{<cmd>}{<search>}{<replace>}{<success>}{<failure>}
\makeatother

%Per tenir punts a l'índex
\makeatletter
\renewcommand*\l@chapter{\@dottedtocline{0}{0em}{1.5em}}
\makeatother

%Per taula que adapta bé els espais
\usepackage{tabularx}
\usepackage{tabu} % http://mirrors.ibiblio.org/CTAN/macros/latex/contrib/tabu/tabu.pdf
\tabulinesep = 1mm
\usepackage[font=footnotesize]{caption} %Captions de les figures més petites

%Appendix
\usepackage[]{appendix} %toc, page

%Alinear al separador decimal amb espais
\usepackage{setspace}
\renewcommand*{\arraystretch}{1.25}


%-------------------------------------------------------------------------------------------------------------
%-------------------------------------------------------------------------------------------------------------
%-------------------------------------------------------------------------------------------------------------
%-------------------------------------------------------------------------------------------------------------

\begin{document}
\pagenumbering{Roman}


%\begin{titlepage}
	\begin{center}
		\vspace*{1cm}
		
		\Huge
		\textbf{Document per Projectes}
		
		\vspace{0.5cm}
		\LARGE
		Adaptat a \LaTeX
	
		\vspace{1.5cm}
		
		\textbf{Llorenç Fanals Batllori}
		
		\vfill
		
		\small
		%\uppercase{Un treball lliurat a la Universitat - en compliment dels requisits pel grau en -}\\
		% TFG
		
		\vspace{1cm}
		
		%\includegraphics[scale=width=0.4\textwidth]{images/a_graph}
	\end{center}
	
	\begin{flushright}
	\large	
	Departament o grup de recerca\\
	UdG\\
	%País\\
	28/08/2019
	\end{flushright}
	


\end{titlepage}

%\thispagestyle{plain}

\begin{center}
	\large
	\textbf{Informe}
	
	\vspace{0.4cm}
	\large
	Descripció
	
	\vspace{0.4cm}
	\textbf{Llorenç Fanals Batllori}
	
	\vspace{0.9cm}
	\textbf{Abstract}
\end{center}
\lipsum[1]




%\chapter*{Dedicacions}
%Dedico aquest treball a -

%\chapter*{Agraïments}
%Vull agraïr a \\



\begin{spacing}{1.5}
%\tableofcontents
\end{spacing}

\cleardoublepage\pagenumbering{arabic}

\noindent \textbf{ÍNDEX} \\
\newline \noindent 1. PLANTA\\
\newline \noindent 2. CONTRAINCENDIS\\
\newline \noindent 3. SECCIÓ A-A'\\
%\section{Característiques específiques}
%CE\Lgem A -- ce\lgem a


\clearpage

\cleardoublepage\pagenumbering{arabic}
%\listoffigures %No fa falta crec
%\listoftables %No fa falta crec


%\chapter{\uppercase{Generalitats}}

Amb aquest projectes es vol legalitzar la instal·lació elèctrica d'un obrador de menjar preparat situat en una nau industrial de la Bisbal d'Empordà, al Carrer Ramon Serradell número 27. Un gran volum de la producció es destina a càterings tot i que també es ven al detall a la sala de venta al públic. Es preveu que hi hagi uns 10 treballadors alhora durant la jornada laboral.\\
\newline
El local dins el qual es preveu que es desenvolupi l'activitat és una nau industrial de 40 m x 20 m, o sigui 800 $m^{2}$ de superfície. S'utilitza únicament la planta baixa, es cobreix totalment amb un fals sostre de 3,5 m en algunes zones i 2,5 m en altres. La nau disposa d'una cuina, cambres de fred per entrada d'aliments i sortida de menjar, una oficina, una recepció, uns vestidors amb dutxes i lavabo i una sala de venta al públic, principalment.\\
\newline
La venda al detall de menjar és en una zona de la qual el públic només té accés a 19.43 $m^{2}$. El Reglament Electrotècnic de Baixa Tensió, a la ITC-28, diu que es comptabilitza una persona per 0,8 $m^{2}$, i que a partir de 50 persones el local es considera de pública concurrència. Com que en l'activitat la superfície que es preveu accessible al públic general és de menys de 40 $m^{2}$, el local no és de pública concurrència.\\
\newline La cuina de l'obrador es considera una zona amb presència d'aigua. Les cambres de fred, on es produeixen condensacions de vapor d'aigua, també es poden considerar zones humides. La ITC-30 detalla els punts que s'han de tenir en compte per aquestes zones.\\
\newline
Per verificar la instal·lació es seguirà el Reglament Electrotècnic de Baixa Tensió, aprovat el 2 d'agost pel Real Decreto 842/2002, amb les corresponents modificacions i ampliacions aplicades fins a la data d'aquest projecte. Es tenen en compte les normes UNE que afecten a la instal·lació. S'ha consultat el Vademècum d'Endesa per dimensionar correctament la caixa de protecció i mesura.

%\section{Característiques específiques}



\clearpage


%\begin{appendices}
%\chapter{Títol de l'annex}
%\chapter{\uppercase{Característiques}}
La instal·lació compleix amb el REBT.\\
\newline S'utilitzen interruptors magnetotèrmics amb corba C amb la intensitat indicada a les taules i a l'esquema unifilar. S'han dimensionat aquests magnetotèrmics per protegir la instal·lació de curtcircuits i sobreintensitats. El valor d'intensitat pel qual el magnetotèrmic pot començar a actuar és major que el valor d'intensitat previst a la línia i alhora menor que la intensitat admissible pel cable.\\
\newline Es preveu utilitzar interruptors diferencials de tipus A amb valors de 30 mA, 100 mA i 300 mA aigües amunt de cada agrupament de línies. Els rentaplats i els extractors, que es preveu que disposaran de variadors de freqüència, aniran amb diferencials de 100 mA degut als alts corrents de fuga que poden donar-se en aquestes línies. Aigües amunt de tots els agrupaments s'instal·la un diferencial de 300 mA de sensibilitat per protegir tota la instal·lació. Tots els interruptors diferencials tenen una intensitat nominal suficient per poder actuar quan la instal·lació està funcionant correctament. Es garanteix amb la posada a terra que cap contacte indirecte superi els 24 V que marca el REBT per locals humits, com és el cas. Així, es garanteix el correcte funcionament de la instal·lació.\\
\newline El QGPM també disposa d'un interruptor contra sobretensions permanents i transitòries, el qual ha d'actuar quan la tensió de servei excedeix de forma considerable la tensió de servei indicada al punt "Condicions del subministrament".\\
\newline La instal·lació disposa d'un model comercial de Conjunt de Protecció i Mesura (CPM) anomenat TMF10 el qual té l'escomesa com a entrada i la derivació individual com a sortida. El TMF10 disposa de maxímetre, mòdul de comunicacions, fusibles, embarrat i interruptor general regulable (ICP-M).\\
\newline No s'instal·la cap CGP perquè s'entén que va incorporada al TMF10. Com que el TMF10 té un maxímetre, no s'instal·la cap ICP. La instal·lació segueix estant protegida contra sobreintensitats a aigües amunt gràcies a l'interruptor ICP-M i al magnetotèrmic general del Quadre General de Protecció i Comandament (QGPM). El maxímetre garanteix el subministrament elèctric tot i sobrepassar la potència contractada. Si en un moment puntual es connectés alguna màquina més i pel marge donat no saltés cap interruptor magnetotèrmic però s'estigués superant la potència contractada, hi seguiria havent subministrament elèctric i l'empresa subministradora aplicaria un recàrrec a la factura.\\

%\chapter{\uppercase{Càlculs}}
En algunes línies es considera que el factor de potència és unitari. A la realitat mai valdrà exactament 1, però sí que es preveu que tingui un valor molt semblant. Les màquines que s'han escollit tenen un factor de potència proper a l'unitari però diferent de 1.\\
\newline Per calcular la intensitat de les línies monofàsiques es fa servir la següent fórmula:
\begin{equation}
I_{linia} = \frac{P}{V*\cos(\phi)}
\end{equation}
En trifàsic, l'equació que s'utilitza és:
\begin{equation}
I_{linia} = \frac{P}{\sqrt3*V*\cos(\phi)}
\end{equation}
És important calcular la caiguda de tensió a les línies per tal de dimensionar-les. La caiguda de tensió en línies d'enllumenat no pot ser superior al 3\% i en línies de força no pot ser superior al 5\% de la tensió de subministrament.\\
\newline En monofàsic:
\begin{equation}
e(\%)=\frac{P}{V}\frac{2*l}{k*S}
\end{equation}
En trifàsic, l'equació que s'utilitza és:
\begin{equation}
e(\%)=\frac{P}{V}\frac{l}{k*S}
\end{equation}
S'utilitzen cables de coure, per tant, $k = 56 \frac{m}{mm^{2}\si{\ohm}}$.\\
\newline El dimensionament de les línies ha de permetre que les caigudes de tensió no superin els màxims indicats prèviament. Alhora, els cables han de poder admetre les intensitats calculades, per això ens guiem amb la taula de la ITC-19 del REBT. Finalment, cal escollir un interruptor magnetotèrmic d'intensitat nominal superior a la calculada per la línia i menor a l'admissible que marca la ITC-19.\\
\newline Es combina el muntatge amb safata perforada al llarg del passadís principal de l'obrador amb muntatge superficial (B2) a la resta de zones. Per alimentar els punts de llum que no es troben tocant cap paret es fan passar els cables pel fals sostre. Els càlculs estan fets amb el cas més restrictiu: el muntatge superficial B2.\\
\newline La instal·lació és trifàsica, per tant, hi ha 3 conductors de fase i un conductor de neutre. El conductor de terra transcorre per totes les línies i té una secció igual als conductors de les línies, tal com s'indica al plànol. El neutre, que arriba per l'escomesa, també s'instal·la de la mateixa secció que els conductors de fase.
%\end{appendices}

%\cite{einstein} % per fer una cita
%\printbibliography[title=Bibliografia] %ARA BÉ

\end{document}