\chapter{\uppercase{Generalitats}}
L'objectiu d'aquesta memòria és legalitzar la instal·lació contra incendis, els tipus de materials i els recorreguts d'evacuació d'una nau industrial de plats cuinats situada al carrer Ramon Serradell número 27, a La Bisbal d'Empordà. \\
\newline Per desenvolupar el present document s'ha consultat el Reial Decret 2267/2004 del 3 de desembre, el qual aprova el Reglament de seguretat contra incendis en establiments industrials. Aquest reglament ha servit per calcular la càrrega de foc, classificar la instal·lació en funció del seu risc de foc i el seu emplaçament, i comprovar si els materials utilitzats i els recorreguts d'evacuació són correctes.\\
\newline També s'ha consultat el Reial Decret 513/2017 del 22 de maig, el qual aprova el Reglament d'instal·lacions de protecció contra incendis. Aquest document és complementari a l'anterior, detalla com ha de ser la instal·lació contra incendis.\\
\newline El REBT aprovat pel Reial Decret 842/2002, del 2 d'agost, s'ha consultat per verificar la correcció de l'enllumenat d'emergència.\\
\newline La Norma Bàsica de l'Edificació NBE-CPI/96 s'ha utilitzat per verificar que els recorreguts d'evacuació i les sortides són legals.

\clearpage