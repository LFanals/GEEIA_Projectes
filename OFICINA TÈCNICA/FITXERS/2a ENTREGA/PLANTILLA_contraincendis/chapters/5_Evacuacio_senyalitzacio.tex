\section{\uppercase{Condició d'evacuació i senyalització}}
En cas d'un incendi és molt important evacuar tot el personal ràpidament per a què els bombers o un grup de gent amb competències pugui solucionar el problema ocasionat. Per garantir la seguretat de les persones és important disposar de recorreguts d'emergència, amb distàncies no massa elevades i una correcta senyalització.\\
\newline El Reial decret 2267/2004 del 3 de desembre detalla aquests aspectes. El Reglament Electrotècnic de Baixa Tensió comenta com ha de ser l'enllumenat d'evacuació. La Norma Bàsica de la Edificació NBE-CPI/96 és consultada per verificar que l'amplada de les portes és correcta.\\
\newline La zona de venta al públic de l'obrador té 19,52 $m^2$ que poden ser ocupats pel públic. Segons la ITC-BT-28 del REBT la ocupació de locals es calcula com una persona per cada 0,8 $m^2$. Arrodonint, hi pot haver 25 persones a la sala de venta al públic com a clients. Es considera que hi ha uns 10 treballadors simultàniament a l'obrador. En total, es considera que l'ocupació de la nau és de 35 persones.\\
\newline Segons la normativa, per més de 25 persones i risc intrínsec baix la longitud d'emergència d'un recorregut únic ha de ser menor de 35 m. Si hi ha dues sortides alternatives, aquest número puja a 50 m. A l'obrador no hi ha cap recorregut, ja sigui únic o no, amb una distància superior a 25 m.\\
\newline Tots els extintors i BIEs estan senyalitzats amb el seu cartell corresponent de lletres blanques sobre fons vermell.\\
\newline Cada porta que forma part d'un recorregut d'evacuació està senyalitzada amb el característic cartell de lletres blanques sobre fons verd i marc de color groc pàlid. En ell s'indica amb una fletxa la direcció a seguir per evacuar l'edifici. Al seu costat o sota seu hi ha un llum d'emergència. \\
\newline El Reglament de seguretat contra incendis en establiments industrials marca que 
\begin{equation}
P = 1,10 p
\end{equation}
\noindent on p és el nombre de persones que pot ocupar l'edifici, 35 a l'obrador. Per tant, l'ocupació de l'obrador és $P=39$ persones. La normativa marca que és d'obligat compliment instal·lar llums d'emergència si P és igual o major de 25 persones.\\
\newline Els llums d'emergència són de tipus no permanent. Aquests llums disposen d'una bateria, així es carreguen. S'encenen quan la tensió de servei baixa del 70\% de tensió nominal. Encara que es talli el subministrament elèctric segueixen fent llum gràcies a les bateries prèviament carregades. Si això passa, han d'estar encesos per, com a mínim, una hora.\\
\newline S'ha determinat que a nivell de terra l'enllumenat d'evacuació dona 3 lux, el REBT marca un mínim de 1 lux. Als punts on hi ha equips manuals de protecció contra incendis, com extintors, i al Quadre General de Protecció i Comandament hi ha 10 lux; el REBT exigeix un mínim de 5 lux. El màxim a l'eix dels passos principals és de 100 lux, i el mínim de 3 lux; la relació és menor de 40, que és el màxim que marca el Reglament Electrotècnic de Baixa Tensió.\\
\newline Les portes de l'obrador d'una sola fulla mesuren 82 cm d'amplada i tenen una alçada de 2 m. Les de dues fulles també mesuren 2 metres d'alçada i tenen les fulles de 1,4 m. La normativa contra incendis remet a la NBE-CPI/96 la qual marca uns mínims de 80 cm d'amplada per les portes que siguin sortida d'evacuació. Per una sola fulla l'amplada màxima és de 1,20 m i en portes de dues fulles l'amplada mínima d'aquestes és de 0,60 m. La distància entre dues portes és menor a 25 metres.\\
\newline Aquesta mateixa normativa marca que els passadissos que formin part d'un recorregut d'evacuació han de fer, com a mínim, 1 m d'amplada. A l'obrador fan 1,5 m.

\clearpage