\chapter{\uppercase{Generalitats}}

Amb aquest projectes es vol legalitzar la instal·lació elèctrica d'un obrador de menjar preparat situat en una nau industrial de la Bisbal d'Empordà, al Carrer Ramon Serradell número 27. Un gran volum de la producció es destina a càterings tot i que també es ven al detall a la sala de venta al públic. Es preveu que hi hagi uns 10 treballadors alhora durant la jornada laboral.\\
\newline
El local dins el qual es preveu que es desenvolupi l'activitat és una nau industrial de 40 m x 20 m, o sigui 800 $m^{2}$ de superfície. S'utilitza únicament la planta baixa, es cobreix totalment amb un fals sostre de 3,5 m en algunes zones i 2,5 m en altres. La nau disposa d'una cuina, cambres de fred per entrada d'aliments i sortida de menjar, una oficina, una recepció, uns vestidors amb dutxes i lavabo i una sala de venta al públic, principalment.\\
\newline
La venda al detall de menjar és en una zona de la qual el públic només té accés a 19.43 $m^{2}$. El Reglament Electrotècnic de Baixa Tensió, a la ITC-28, diu que es comptabilitza una persona per 0,8 $m^{2}$, i que a partir de 50 persones el local es considera de pública concurrència. Com que en l'activitat la superfície que es preveu accessible al públic general és de menys de 40 $m^{2}$, el local no és de pública concurrència.\\
\newline La cuina de l'obrador es considera una zona amb presència d'aigua. Les cambres de fred, on es produeixen condensacions de vapor d'aigua, també es poden considerar zones humides. La ITC-30 detalla els punts que s'han de tenir en compte per aquestes zones.\\
\newline
Per verificar la instal·lació es seguirà el Reglament Electrotècnic de Baixa Tensió, aprovat el 2 d'agost pel Real Decreto 842/2002, amb les corresponents modificacions i ampliacions aplicades fins a la data d'aquest projecte. Es tenen en compte les normes UNE que afecten a la instal·lació. S'ha consultat el Vademècum d'Endesa per dimensionar correctament la caixa de protecció i mesura.

%\section{Característiques específiques}



\clearpage