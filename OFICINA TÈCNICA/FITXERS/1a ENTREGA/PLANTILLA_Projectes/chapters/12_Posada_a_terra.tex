\chapter{\uppercase{Posada a terra}}
La ITC-18 del REBT comenta que les posades a terra s'estableixen, principalment, per limitar la tensió que, respecte el terra, poden presentar les carcasses metàl·liques dels equips. Això fa disminuir el risc que poden suposar aquest tipus d'avaries i amb l'actuació de les proteccions adients es pot eliminar aquest risc.\\
\newline Seguint el criteri de la ITC-30, la cuina de l'obrador és una zona humida, on, a part de l'aigua utilitzada per cuinar, serà molt habitual netejar amb aigua abundant. No totes les zones de l'obrador seran humides, però només que una ho sigui, ja es pot dir que la tensió màxima de defecte és de 24 V, tal com marca la ITC-18.\\
\newline Tenim tres diferencials de sensibilitats diferents en la nostra instal·lació: 30 mA, 100 mA i 300 mA. El més restrictiu és el de 300 mA, per tant, el que escollim per fel el càlcul. 
\begin{equation}
R_{terra} < 80 \ \si{\ohm}
\end{equation}
El règim de neutre de la instal·lació és TT.\\
\newline La instal·lació de posada a terra està feta amb 15 planxes metàl·liques de 1 m per 0,5 m enterrades sota els terrenys de la parcel·la. \\
\newline La ITC-18 diu que la posada a terra es revisarà com a mínim un cop a l'any en l'època en què el terreny estigui més sec per tal d'eliminar els possibles defectes que hi hagi. L'última mesura efectuada de la posada a terra es va fer el dia 2 d'agost de 2019 (02/08/2019) i indicava un valor de resistència de $15$ \si{\ohm}.

\clearpage