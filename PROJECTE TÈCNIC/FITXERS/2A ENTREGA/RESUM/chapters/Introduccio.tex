\chapter{\uppercase{Introducció}}


%La instal·lació de plaques solars era una raresa fins fa pocs anys. Recentment es considera una possiblitat molt atractiva gràcies a l'abaratiment dels costos d'aquestes i al nou Decret d'Autoconsum. Aquestes són algunes de les raons per les quals un client m'ha demanat la creació d'un projecte que contemplés la instal·lació de plaques solars fotovoltaiques a una casa unifamiliar situada al Carrer Canigó, número 27, a Vulpellac.\\


L'objectiu del projecte és calcular i detallar la instal·lació elèctrica fotovoltaica d'un habitatge unifamiliar i dissenyar una placa electrònica per tal de captar dades de les plaques de forma periòdica. Aquestes dades seran accessibles pels usuaris de la instal·lació.\\
\newline El dimensionament de la instal·lació fotovoltaica s'ha fet a partir del consum elèctric del darrer any i amb la consigna de tenir una generació similar a aquest al cap de l'any. S'han calculat els angles més òptims d'inclinació i orientació de les plaques. L'inversor s'ha escollit d'acord amb les especificacions de les plaques. La instal·lació elèctrica s'ha calculat considerant les condicions més adverses i seguint el REBT.\\
\newline La placa electrònica disposa de la part de potència per adaptar les tensions als nivells correctes, la part d'instrumentació per reduir les tensions d'entrada, la part del multiplexor per passar diferents senyals a l'entrada analògica i finalment la part de comunicació per connectar-se via Wi-Fi i transmetre informació. \\
\newline S'ha dissenyat una pàgina web on es visualitzen les dades més recents de tensió de cada panell fotovoltaic. El fet de tenir a disposició aquesta informació es preveu que faciliti al client la coneixença de l'estat de les plaques.




\clearpage


