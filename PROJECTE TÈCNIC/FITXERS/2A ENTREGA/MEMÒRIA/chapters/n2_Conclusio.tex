\chapter{\uppercase{Conclusió}}
Per desenvolupar el projecte s'ha consultat l'IDAE de novembre de 2019 per tal de determinar les pèrdues generades per les ombres, els angles d'orientació i inclinació, la connexió dels panells entre sí i les seves proteccions. El REBT ha servit per dimensionar correctament els cables conductors segons caigudes de tensió i criteri tèrmic. També ha estat d'utilitat per dimensionar les proteccions.\\
\newline Amb aquest projecte es detalla la instal·lació elèctrica fotovoltaica d'una casa unifamiliar connectada a la xarxa amb modalitat d'autoconsum amb compensació d'excedents. S'ha decidit la potència dels panells, passant pel dimensionament de la instal·lació i de l'inversor. Al cap de l'any, l'energia generada per les plaques fotovoltaiques és propera a la consumida.\\
\newline Per la banda de l'electrònica s'ha dissenyat el hardware que es fa servir i com les diferents parts del circuit s'adapten entre elles per donar lloc a una placa electrònica capaç de llegir les tensions de cada placa de forma correcta i publicar-les en una web.\\
\newline El client podrà visualitzar els perfils de tensió de les últimes 24 hores de cada placa. Així sabrà com han afectat les ombres i podrà detectar un possible malmetement d'algun panell. Es preveu que la realització d'aquest projecte ha de permetre al client generar energia elèctrica de forma més sostenible amb el medi ambient. \\

\vspace*{\fill}
\noindent Llorenç Fanals Batllori\\
Graduat en Enginyeria Electrònica Industrial i Automàtica\\
\\
\\
Girona, 28 de novembre de 2019.

\clearpage