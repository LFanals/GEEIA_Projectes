\chapter{\uppercase{Generalitats}}
L'objectiu d'aquest document és justificar que la immissió acústica de l'obrador es manté dins els nivells màxims que indiquen les normatives. Així mateix, es vol verificar que les vibracions prenen valors permesos. L'activitat està situada al carrer Ramon Serradell, número 27, a La Bisbal d'Empordà, en un polígon industrial\\
\newline Per justificar-ho s'ha consultat la Llei 16/2002 de 28 de juny que tracta sobre Protecció contra la Contaminació Acústica. Aquesta llei comenta com s'han de mesurar les immissions de sorolls i contempla una gran quantitat de casos o elements que poden causar contaminació acústica. També detalla com han de ser les mesures de vibracions i els seus valors permesos.\\
\newline S'ha tingut en compte el mapa de capacitat acústica de l'Ajuntament de la Bisbal d'Empordà. En ell es marquen clarament diferents zones segons la sensibilitat acústica màxima que admeten.\\
\newline L'Ajuntament de la Bisbal d'Empordà no té una ordenança de sorolls i vibracions, per tant, es segueix el Decret 176/2009 de 10 de novembre de Protecció contra la Contaminació Acústica, el qual s'aplica a Catalunya. Aquest decret modifica els annexos de la Llei de Protecció contra la Contaminació Acústica de la Llei 16/2002 de 28 de juny, que s'aplica a nivell nacional.


\clearpage