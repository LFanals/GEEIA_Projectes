\chapter{\uppercase{Conclusió}}
L'objectiu d'aquest document és legalitzar la immissió sonora i el nivell de vibracions de l'obrador de plats cuinats.\\
\newline Per desenvolupar aquest document s'ha seguit la llei 16/2002 de 28 de juny i el Decret 176/2009 de 10 de novembre de Protecció contra la Contaminació Acústica, el qual s'aplica a Catalunya.\\
\newline També s'ha consultat el mapa de contaminació acústica de l'Ajuntament de La Bisbal d'Empordà, el qual indica les sensibilitats sonores a les diferents zones del municipi.\\
\newline Amb tot l'indicat en aquest document es considera que l'activitat emet uns nivells de soroll i vibracions legals per la seva situació geogràfica i que fins a dia d'avui compleix la normativa al respecte.

\vspace*{\fill}
\noindent Llorenç Fanals Batllori\\
Graduat en Enginyeria Electrònica Industrial i Automàtica\\
\\
\\
Girona, 14 de novembre de 2019.

\clearpage