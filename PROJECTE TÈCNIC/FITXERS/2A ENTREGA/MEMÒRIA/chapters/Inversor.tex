\chapter{\uppercase{Inversor}}
%Fronius primo 3.0-1, %https://www.fronius.com/es-es/spain/energia-solar/productos/sector-dom%C3%A9stico/inversor/fronius-primo/fronius-primo-3-0-1

L'inversor escollit és un FRONIUS Primo 3.0-1. És costós en comparació a altres inversors però alhora molt fiable. Permet connectar-lo per TCP-IP a la xarxa d'Internet de la casa i així poder consultar en tot moment la quantitat d'energia que la instal·lació de panells solars fotovoltaics genera.\\
Seguidament s'adjunta una taula amb algunes de les característiques més rellevants de l'inversor.
\begin{table}[H]
  \centering
    \begin{tabular} {|l|r|} \hline
  \multicolumn{1}{|c|}{Característica} &  \multicolumn{1}{c|}{Valor}\\ \hline \hline
	Màxima corrent d'entrada ($I_{dc \ max}$) & 12 / 12 A \\ \hline
	Màxima corrent de curtcircuit & 18 / 18 A \\ \hline
	Rang màxim de tensió d'entrada CC ($U_{cc \ min}$ - $U_{cc \ max}$) & 80 - 1.000 V \\ \hline
	Tensió mínima de posada en marxa ($U_{dc \ arranc}$) & 80 V \\ \hline
	Nombre d'entrades CC & 2 + 2 \\ \hline
	Potència nominal AC ($P_{ac,r}$) & 3.000 W \\ \hline
	Acoblament a la xarxa ($U_{ac,r}$) & 1 NPE 230 V \\ \hline
	
    \end{tabular}%
  \label{tab:addlabel}%
  \caption{Característiques de l'inversor FRONIUS Primo 3.0-1}
 \end{table}%

\noindent L'inversor és adient per la instal·lació de plaques i l'associació d'aquestes que es proposa. En cap cas se superen els màxims que fixa el fabricant de l'inversor. La potència de l'inversor és correcta per la potència màxima que ens poden donar els panells fotovoltaics. Per començar a funcionar l'inversor necessita un mínim de 80 V. Si de dia un o dos panells estan ombrejats i la resta no l'inversor pot seguir funcionant i lliurant energia.\\
\newline Els rendiments de l'inversor són molt elevats, per un gran rang de treball el rendiment és del 95 \% o més. En cap cas el rendiment baixa del 80 \%.\\
\newline El fabricant indica que l'inversor intenta donar sempre la màxima potència mitjançant el seu sistema de Dynamic Peak Manager.\\
\newline Fronius permet connectar els seus inversors a Internet amb cable d'Ethernet i amb una aplicació web poder consultar l'energia que han generat i que estan generant les plaques. La interfície és simple i entenedora.\\
\newline Aquest inversor és fàcil de muntar gràcies a què es pot desmuntar en dues parts: la fixa que va a la paret i és on es realitzen les connexions i la de potència, que pesa bastant. Primer es munta la part fixa, es realitzen les connexions, i després s'acobla la part de potència.\\
\newline Fronius indica que el Primo 3.0-1 és un equip pensat pel futur, per xarxes elèctriques intel·ligents. Hi ha la possibilitat de comunicar l'inversor per interfícies molt diverses com Modbus RTU, Fronius Solar API, Ethernet...



\clearpage


% Table generated by Excel2LaTeX from sheet 'Hoja1'
%\begin{table}[H]
%  \centering
%    \begin{tabularx} {\textwidth} {|X|r|} \hline
%  \multicolumn{1}{|c|}{Descripció} &  \multicolumn{1}{c|}{Quantitat}\\ \hline \hline
%
 %   Placa GLC 330 W & 10 \\ \hline
%    Inversor FRONIUS Primo 3.0-1 Light 3kW & 1 \\ \hline
%    Metres cable Ethernet RJ-45 CAT 8 & 10 \\ \hline
%    Metres cable 4 m$m^2$ PVC & 45 \\ \hline
 %   Metres cable 1,5 m$m^2$ PVC & 100 \\ \hline
 %   Punteres Enghofer E 4-10, 4 m$m^2$, 10 mm & 20 \\ \hline
 %   Punteres Enghofer E 1.5-10 1,5 m$m^2$ 10 mm & 12 \\ \hline
 %   Cinta aïllant 10 m 1,6 cm & 3 \\ \hline
 %   Caixa estanca Solera CONS 100x100x55 mm & 2 \\ \hline
  %  Canal Euroquint 25,16 mm 1,5 metres & 20 \\ \hline
%    Curva canal VECAMCO & 10 \\ \hline
%    Paquet de 50 brides 200x2,6  mm & 2 \\ \hline
%    Regleta nylon 12 pols 16 mm & 4 \\ \hline
%    Premsaestopes M12 & 10 \\ \hline
%    Cargol autoroscant M4 16 mm & 12 \\ \hline
%    Tacs Fischer 072095 nylon 6x50 mm & 50 \\ \hline
%    Díode SM74611KTTR & 10 \\ \hline
%            Hores enginyer & 1 \\ \hline
%    Hores oficial de primera & 12 \\ \hline
%    Hores oficial de segona & 12 \\ \hline
%    \end{tabularx}%
%  \label{tab:addlabel}%
% \end{table}%
