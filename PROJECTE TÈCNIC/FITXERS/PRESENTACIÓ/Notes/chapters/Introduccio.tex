%\chapter{\uppercase{Introducció}}

\section*{1. Portada}
Breu presentació personal de mi i del títol del treball.

\section*{2. Introducció}
L'ordre escollit és lògic, primer vull exposar el dimensionament de potència, amb això quins panells i inversor escullo i finalment els aspectes que s'han tingut en compte pels càlculs de la instal·lació elèctrica. Per altra banda, per l'electrònica, primer parlar de hardware i després de la programació i els resultats que podria visualitzar el client.

\section*{3. Característiques de l'habitatge}
Citar que la casa unifamiliar està situada a Vulpellac amb latitud molt propera als 42 graus. Coneixent la latitud es pot calcular l'angle òptim d'inclinació dels panells i amb aquest la irradiació global.\\
\newline El criteri utilitzat és el d'aconseguir una generació semblant però menor al consum, anualment parlant. Modalitat d'autoconsum, connectat a la xarxa amb excedents.

\section*{4. Generador fotovoltaic i inversor}
Un cop se sap la potència dir que s'ha escollit un panell amb bones prestacions i quants se'n posaran.\\
\newline Explicar molt breument que un panell pot treballar com a receptor quan té altres panells en sèrie que generen més. Per això es decideix col·locar díodes en paral·lel amb cada fila de cel·les. Avançar que això farà disminuir la tensió als terminals del panell.\\
\newline Citar que s'escull un inversor amb 2 entrades i correcte per l'aplicació.

\section*{5. Instal·lació elèctrica}
Anomenar els factors d'escalfament, agrupament, radiació solar, el 125\% de la ITC-BT-40, el fet de què un panell solar té la intensitat limitada, que el magnetotèrmic que poso a l'entrada de l'inversor només serveix com a interruptor i no com a protecció.\\
\newline Diferenciar que per cable s'han tingut en compte tots els factors, en canvi per dimensionar els magnetotèrmics, que només fan de simple interruptor, es té en compte la intensitat de curtcircuit, que és la màxima que pot passar.\\
\newline A la sortida de l'inversor comentar que hi ha un magnetotèrmic que deixés passar la intensitat equivalent als 3.300 W i un diferencial.

\section*{6. Placa electrònica d'adquisició de dades i comunicació}
Explicar que la placa s'alimenta amb 5 V que es converteixen a 3,3 V per alguns integrats. Citar que s'han mirat intensitats i que la que ens pot donar el regulador és suficient.\\
\newline L'atenuador s'encarrega de baixar els nivells de tensions per treballar amb operacionals, que s'alimenten a 5 V, i després es fa la resta de tensions. Deixar clar que tenim 10 diferències de tensions que s'entren al multiplexor. Es programa l'ESP-12E. Hi ha adaptació correcta d'impedàncies. 

\section*{7. Programació i web}
Explicar que es programa l'ESP-12E. Remarcar que s'ha de configurar en el programa la xarxa Wi-Fi i la contrasenya d'aquesta. Comentar de forma superficial l'organigrama del programa, les condicions que hi ha i quines accions pot fer.\\
\newline Citar que és al programa on es defineix la web i que s'incorporen gràfiques.

\section*{8. Conclusió}
Concloure que amb aquest treball s'ha dimensionat una instal·lació fotovoltaica tenint en compte les particularitats d'aquests generadors. S'ha dissenyat una placa electrònica per adquirir dades i comunicar-les tan a nivell de hardware com a nivell de software. Es publiquen les dades en una pàgina web.

\section*{9. Preguntes}
Torn de preguntes. No tenir excessiva pressa al contestar. Si es dona el cas, citar que la resposta es troba a la memòria o a algun altre document. Mostrar confiança i argumentar les respostes.





















\clearpage


